\documentclass[twocolumn]{ieeetran}

% =========================
% XeLaTeX 必須
% =========================
\usepackage{iftex}
\ifXeTeX\else
  \errmessage{This template requires XeLaTeX. Set latexmk_use_xelatex:true in your CI.}
\fi

% =========================
% 日本語フォント(内蔵のみ)
% =========================
\usepackage{fontspec}
\usepackage{xeCJK}
\setmainfont{TeX Gyre Termes}
\setsansfont{TeX Gyre Heros}
\setmonofont{DejaVu Sans Mono}
\setCJKmainfont[
  BoldFont   = IPAexGothic,
  ItalicFont = IPAexMincho
]{IPAexMincho}
\setCJKsansfont{IPAexGothic}
\setCJKmonofont{IPAexGothic}

% =========================
% 数式・図・表
% =========================
% (図未投入でも停止しない:本提出時はこの1行をコメントアウト)
\PassOptionsToPackage{draft}{graphicx}
\usepackage{graphicx}
\graphicspath{{./}{figures/}}
\DeclareGraphicsExtensions{.pdf,.png,.jpg}

% 図欠落でもビルド継続(\includegraphics を安全版に置換)
\newcommand{\safeincludegraphics}[2][]{%
  \IfFileExists{#2}{\includegraphics[#1]{#2}}{%
    \fbox{\parbox[c][0.45\linewidth][c]{0.9\linewidth}{\centering
    \textit{[Missing image: #2]}}}}}
\let\includegraphics\safeincludegraphics

\usepackage{amsmath,amssymb,bm}
\usepackage{booktabs}
\usepackage{multirow}

% --- TikZ / pgfplots ---
\usepackage{tikz}
\usetikzlibrary{plotmarks,positioning,calc,decorations.pathmorphing}
\usepackage{pgfplots}
\pgfplotsset{compat=1.18}
\usepgfplotslibrary{groupplots} % \begin{groupplot} 用

% =========================
% URL/ハイパーリンク と '_' 対策
% =========================
\usepackage{url}
\def\UrlBreaks{\do\/\do-\do\_\do\&\do\?}
\usepackage[strings]{underscore} % ← 必ず hyperref より前

% hyperref は最後に
\usepackage[unicode,breaklinks=true]{hyperref}
\hypersetup{
  pdfauthor={Shinichi Samizo},
  pdftitle={uTEPヘッド用ドライバICにおけるウエハ酸素濃度変動起因の品質問題対応:酸素規格・ベンダー管理・スクリーニング工程再設計},
  pdfborder={0 0 0}
}

% \usepackage[top=19mm,bottom=19mm,left=19mm,right=19mm]{geometry} % 必要時のみ

% =========================
% タイトル/著者
% =========================
\title{uTEPヘッド用ドライバICにおけるウエハ酸素濃度変動起因の品質問題対応:\\
酸素規格・ベンダー管理・スクリーニング工程再設計}

\author{%
  \IEEEauthorblockN{三溝 真一 (Shinichi Samizo)}%
  \IEEEauthorblockA{%
    独立系半導体研究者(元セイコーエプソン株式会社)\\%
    Independent Semiconductor Researcher (ex-Seiko Epson Corporation)\\[3pt]%
    Email:~\href{mailto:shin3t72@gmail.com}{shin3t72@gmail.com}\quad
    GitHub:~\url{https://github.com/Samizo-AITL}%
  }%
}

% ============================================================
% アブストラクトとキーワードをタイトル出力前に予約
% ============================================================
\IEEEtitleabstractindextext{%
\begin{abstract}
\textbf{和文概要}—
本報告は、uTEPプリントヘッドを駆動する高耐圧ドライバICにおいて、
ウエハ酸素濃度変動を起因とする品質問題に対し、
量産スクリーニング条件を再設計した事例を示す。
ウエハ供給元変更により酸素濃度が上昇し、
結晶起因欠陥(COP: Crystal-Originated Particle)を起点とする
ゲート酸化膜リークが顕在化した。
従来の単回スクリーニングでは潜在欠陥の取り残しが発生し、
ヘッド工程およびプリンタ組立段階で不良が顕在化した。
本報告では、電圧・温度・印加回数を変数とする工程条件を再設計し、
動作保証範囲内(85℃以下)で不良収束を実現した。
あわせて、ウエハ酸素濃度規格化およびベンダー管理強化を提言する。

\bigskip
\noindent
\textbf{Abstract}—
This report presents a case study on redesigning the mass-production screening process
for a high-voltage driver IC used in uTEP printhead applications,
addressing quality issues induced by wafer oxygen concentration variation.
Following a wafer supplier change, an increase in oxygen concentration
led to the emergence of crystal-originated particle (COP) defects,
causing localized gate oxide leakage.
The conventional single-pass screening failed to capture latent defects,
resulting in escapes that propagated through head assembly
and printer integration stages.
By redesigning the screening parameters—voltage, temperature, and repetition count—
the revised process achieved defect convergence within the guaranteed temperature limit (below 85 °C).
Additionally, establishing wafer oxygen concentration specifications
and strengthening vendor management were found essential for recurrence prevention.
\end{abstract}

\begin{IEEEkeywords}
uTEP head, Inkjet, Driver IC, Gate Oxide Leakage, Wafer Oxygen, COP Defect, Quality Control, Screening Process Redesign, Reliability Engineering, Process Optimization
\end{IEEEkeywords}
}

% ============================================================
\begin{document}
\maketitle
\IEEEdisplaynontitleabstractindextext % ← 図がアブストラクトより上に出るのを防止
% ============================================================

%------------------------------------------------------
\section{はじめに}
uTEP(\textit{micro Thin-film Electrostatic / Piezo} 系列を含む統合アーキテクチャ)プリントヘッドは、
高電圧(HV)駆動によりインク液柱を微細ノズルから高精度に射出する構造を有している。
これを駆動するドライバICは、数百~数千チャネルのアクチュエータ電極を並列制御するため、
高集積化と高信頼性を同時に満たす必要がある重要部品である。

当該ドライバICは、CMOS 0.35\,$\mu$mプロセスをベースとし、
3.3\,Vロジック動作と45\,Vクラスの高耐圧出力を単一チップ上で統合している。
しかし、ウエハ供給元の追加・切替後に一部ロットでウエハ中酸素濃度が上昇し、
結晶起因欠陥(COP: Crystal-Originated Particle)由来の微小ボイドが活性層内に析出した。
この結晶欠陥がゲート酸化膜下に残存することで局所的な電界集中が生じ、
高電圧印加時にゲート酸化膜リーク(局所絶縁破壊)を誘発することが確認された。

従来の量産スクリーニング条件(48\,V単回印加)では、
これらの潜在欠陥を十分に顕在化させることができず、
ヘッド電気特性検査やプリンタ組立後の最終検査で不良が顕在化する事例が発生した。
一部はプリンタ本体への組込み後に検出されたため、
工程内回収によって市場流出は防止したものの、歩留低下および損害は甚大であった。

本報告では、ウエハ酸素濃度変動を前提としたデバイス特性解析結果をもとに、
電圧・温度・印加回数を変数とするスクリーニング条件を再設計し、
量産工程での不良収束性を改善するとともに、
ウエハ酸素濃度の規格化およびベンダー管理の強化によって、
再発防止と品質安定化を図った取り組みについて述べる。

%------------------------------------------------------
\section{問題の概要(uTEP適用での観測)}

図\ref{fig_lpd}にウエハLPD(Laser Particle Defect)分布の代表例を示す。
高酸素ロットではLPD密度が著しく高く、結晶起因欠陥(COP: Crystal-Originated Particle)に由来する
局所欠陥の増加が確認された。
これらの欠陥はHVデバイスのゲート酸化膜下部に微小ボイドとして存在し、
通電時に局所的な電界集中を生じてゲート酸化膜リークを誘発する。

従来のスクリーニング条件(48\,V単回印加)では、
これら潜在欠陥の顕在化が不十分であり、取り残しが発生した。
その結果、uTEPヘッド組立後の電気特性検査や、
プリンタ組込み後の量産信頼性試験においてゲートリーク起因の不良が再現した。
特に、HVゲートの初期劣化モードは低電圧印加では顕在化しにくく、
スクリーニング初期段階での除去が不完全であったことが確認された。

本事象は、ウエハ酸素濃度の上昇により結晶欠陥サイズおよび分布密度が変化し、
結果としてHV絶縁膜の電界分布特性が不均一化したことに起因する。
単純な電圧マージン拡大では再現性が得られず、
酸素濃度を変動要因として考慮した電圧・温度・印加回数の最適化が必要である。

\begin{figure}[t]
  \centering
  \begin{tikzpicture}
    % --- 2枚並列のグループプロット(列幅内フィット済み)---
    \begin{groupplot}[
      group style={group size=2 by 1, horizontal sep=10mm},
      width=0.45\columnwidth, height=0.45\columnwidth,
      ymin=0,
      ymajorgrids,
      xmajorgrids,
      tick align=outside,
      tick style={thin},
      xlabel={LPDサイズ [$\mu$m]},
      ylabel={密度 [cm$^{-2}$]},
      xtick={0,1,2,3,4,5},
      xticklabel style={/pgf/number format/fixed},
      yticklabel style={/pgf/number format/fixed},
      title style={yshift=-2pt,font=\scriptsize},
      label style={font=\scriptsize},
      tick label style={font=\scriptsize},
      legend style={font=\scriptsize, draw=none, fill=white, fill opacity=0.8},
      legend cell align=left,
      enlarge x limits=0.05
    ]

    % --- 左:通常ロット ---
    \nextgroupplot[
      title={通常ロット(代表)},
      legend pos=north east,
    ]
      \addplot[ybar, bar width=6pt]
        coordinates {
          (0.5,  15)
          (1.0,  10)
          (1.5,   7)
          (2.0,   5)
          (2.5,   3)
          (3.0,   2)
          (3.5,   1)
          (4.0, 0.5)
          (4.5, 0.2)
          (5.0, 0.1)
        };
      \addlegendentry{LPD密度}

    % --- 右:高酸素ロット ---
    \nextgroupplot[
      title={高酸素ロット(代表)},
      legend pos=north east,
    ]
      \addplot[ybar, bar width=6pt]
        coordinates {
          (0.5,  40)
          (1.0,  32)
          (1.5,  25)
          (2.0,  18)
          (2.5,  12)
          (3.0,   9)
          (3.5,   6)
          (4.0,   4)
          (4.5,   2)
          (5.0,   1)
        };
      \addlegendentry{LPD密度}

    \end{groupplot}
  \end{tikzpicture}
  \caption{ウエハLPD分布比較(通常ロットと高酸素ロットの代表例)}
  \label{fig_lpd}
\end{figure}

%------------------------------------------------------
\section{スクリーニング条件最適化(量産仕様内)}

本章では、uTEPヘッド適用を前提とした高耐圧ドライバICにおける
量産スクリーニング条件の最適化検討について述べる。
目的は、\textbf{仕様範囲内(温度85\,℃以下)}において
不良除去率を最大化しつつ、良品への影響を最小化する工程条件を確立することである。

\subsection{評価パラメータ}

表\ref{tab_param}に評価に用いた主要パラメータを示す。
試験は、量産実績を持つ通常ロットおよび高酸素ロットを対象とした。
印加条件の主変数は電圧・温度・印加回数の3因子とし、
各回の通電後にゲートリーク電流および絶縁破壊電圧を常温下で評価した。

\begin{table}[t]
\centering
\caption{スクリーニング評価パラメータ(HVドライバIC)}
\label{tab_param}
\begin{tabular}{ll}
\toprule
項目 & 条件設定 \\
\midrule
印加電圧 & 48\,V(絶対最大定格内、電流コンプライアンス付) \\
印加温度 & 85\,℃(動作保証上限) \\
印加回数 & 1~8回(繰返し印加) \\
印加時間 & 各回30\,分 \\
測定項目 & $\Delta I_\mathrm{GATE}$, $\Delta BV_\mathrm{G}$, d$I$/d$V$(常温測定) \\
\bottomrule
\end{tabular}
\end{table}

\subsection{結果と収束挙動(代表ロット)}

表\ref{tab_results}に、スクリーニング回数ごとの新規検出率および残存不良率を示す。
初回印加で約60\,\%の不良を除去し、2~3回目で大部分の潜在欠陥が顕在化した。
4回目以降では新規検出がほぼ停止し、高酸素ロットにおいても残存不良率は30\,ppm以下に収束した。

この結果より、48\,V・85\,℃条件下で4回印加を行うことで、
良品への劣化影響を伴わずに不良収束性を確保できることが確認された。
図\ref{fig_convergence}に代表ロットでの不良収束曲線を示す。

\begin{table}[t]
\centering
\caption{スクリーニング回数と不良収束率(代表ロット)}
\label{tab_results}
\begin{tabular}{ccc}
\toprule
回数 & 新規検出率[\%] & 残存不良率[ppm] \\
\midrule
1 & 58.9 & 420 \\
2 & 25.1 & 125 \\
3 & 11.0 & 52 \\
4 & 4.0  & 30 \\
5以降 & $\approx$0 & $<$30 \\
\bottomrule
\end{tabular}
\end{table}

\begin{figure}[t]
  \centering
  \begin{tikzpicture}
    \begin{axis}[
      width=\columnwidth,
      height=0.58\columnwidth,
      xmin=1, xmax=8.02,
      ymin=0, ymax=500,
      xtick={1,2,3,4,5,6,7,8},
      ytick={0,100,200,300,400,500},
      xlabel={スクリーニング回数},
      ylabel={残存不良率 [ppm]},
      ymajorgrids,
      xmajorgrids,
      grid style={dashed,gray!40},
      line width=0.7pt,
      mark size=2.0pt,
      enlargelimits=false,
      clip=false,
      legend style={font=\scriptsize,draw=none,fill=white,fill opacity=0.8},
      legend cell align=left,
      tick label style={font=\scriptsize},
      label style={font=\scriptsize},
    ]
      \addplot[mark=*, color=blue]
        coordinates {
          (1,420)
          (2,125)
          (3,52)
          (4,30)
          (5,25)
          (6,24)
          (7,23)
          (8,23)
        };
      \addlegendentry{残存不良率(代表ロット)}
    \end{axis}
  \end{tikzpicture}
  \caption{回数に対する不良収束曲線(uTEP適用HVドライバIC、代表ロット)}
  \label{fig_convergence}
\end{figure}

%------------------------------------------------------
\section{考察}

uTEPヘッド適用においては、温度上限を超えることなく、
\textbf{繰返し印加回数の最適化}によって
COP(Crystal-Originated Particle)起因の潜在欠陥を早期に顕在化させることが可能である。
これは、通電応力によりゲート絶縁膜内部の電界集中部が段階的に活性化し、
微小リーク経路が成長・開通する過程を利用しているためである。

各回の印加後に常温で$\Delta I_\mathrm{GATE}$および$\Delta BV_\mathrm{G}$を逐次計測することにより、
欠陥進展挙動をモニタしつつ、良品の酸化膜疲労を抑制できる。
特に、$\Delta BV_\mathrm{G}$の漸減と$\Delta I_\mathrm{GATE}$の急増が同時に観測されたデバイスは、
初期リーク経路の形成段階にあると判断でき、
量産スクリーニング工程において高い再現性で不良抽出が可能であった。

本手法は、電圧・温度をいずれも仕様範囲内に保持したまま欠陥顕在化を促進できる点で、
従来の単回スクリーニングと比較して良品影響が極めて小さい。
また、ウエハ酸素濃度や結晶成長履歴といった材料起因ばらつきを吸収できるため、
ロット間での品質安定性向上にも寄与する。

さらに、本設計手法はHVデバイス一般に共通する
酸化膜ストレス緩和挙動と電界劣化モデルに基づいており、
uTEP専用ICに限らず、インクジェット駆動系の他高耐圧品種や
温度制約の厳しい小型プリンタ向けデバイスにも水平展開が可能である。
特に、高信頼性を要求される産業機器・車載用途においても、
本スクリーニング設計指針は有効であると考えられる。

%------------------------------------------------------
\section{結論}

本研究では、uTEPヘッド用高耐圧ドライバICを対象として、
ウエハ酸素濃度変動を考慮した量産スクリーニング条件の最適化を実施した。
電圧48\,V・温度85\,℃という仕様範囲内で複数回印加を行う逐次スクリーニング方式を導入した結果、
高酸素ロットにおいても残存不良率を30\,ppm以下に収束させることができた。
同時に、良品デバイスへの影響は測定誤差レベル以下に抑えられ、
信頼性劣化は認められなかった。

本手法は、COP(Crystal-Originated Particle)起因の潜在欠陥を
初期段階で顕在化させることで、ウエハ材料ばらつきを工程内で吸収し、
量産品質の安定化に寄与するものである。
また、温度上限を超えない条件で高い除去効率を達成できるため、
他の高耐圧デバイス(アクチュエータ駆動用IC、車載用IC等)への適用拡張が期待できる。

今後は、$\Delta I_\mathrm{GATE}$および$\Delta BV_\mathrm{G}$の時間依存解析を組み合わせ、
デバイス個体ごとの劣化進展モデルを構築することで、
より短時間かつ自動化されたスクリーニングプロセスの実現を目指す。
さらに、ウエハベンダー追加時の酸素濃度規格化や材料認定手順と組み合わせることで、
品質変動の早期検知および再発防止に資する包括的な管理体系の確立を進める。

%------------------------------------------------------
\section*{謝辞}
本研究の遂行にあたり、ご協力を賜ったデバイス技術部ならびに信頼性評価チーム、
およびuTEPヘッド量産に携わる関係各位に深く感謝の意を表する。

%------------------------------------------------------
\begin{thebibliography}{99}

\bibitem{samizo_bom}
三溝 真一, 「設計から量産部品発注に至る実務フローとBOM運用ルールの体系化」, 2025.

\bibitem{samizo_mach}
三溝 真一, 「Sn–Bi代替による接合方式移行(Mach世代)」, 2025.

\bibitem{samizo_tfp}
三溝 真一, 「薄膜PZTアクチュエータの信頼性解析と対策」, 2025.

\end{thebibliography}

% =====================================================
% 著者略歴(日本語・IEEE調の簡潔版)
% =====================================================
\section*{著者略歴}
\noindent\textbf{三溝 真一(Shinichi Samizo)}:
信州大学大学院 工学系研究科 電気電子工学専攻修了(修士)。セイコーエプソン株式会社にて、半導体デバイス(ロジック/メモリ/高耐圧インテグレーション)およびインクジェット薄膜ピエゾアクチュエータ、PrecisionCoreプリントヘッドの製品化に従事。現在は独立系半導体研究者として、プロセス/デバイス教育、メモリアーキテクチャ、高耐圧CMOS統合設計、AIシステム統合に取り組む。連絡先:\href{mailto:shin3t72@gmail.com}{shin3t72@gmail.com}.

\end{document}
