\documentclass[twocolumn]{ieeetran}

% =========================
% XeLaTeX 必須(CIでは latexmk_use_xelatex:true)
% =========================
\usepackage{iftex}
\ifXeTeX\else
  \errmessage{This template requires XeLaTeX. Set latexmk_use_xelatex:true in your CI.}
\fi

% =========================
% 日本語フォント設定(XeLaTeX/内蔵フォントのみ)
% =========================
\usepackage{fontspec}
\usepackage{xeCJK}
\setmainfont{TeX Gyre Termes}
\setsansfont{TeX Gyre Heros}
\setmonofont{DejaVu Sans Mono}
\setCJKmainfont[
  BoldFont   = IPAexGothic,
  ItalicFont = IPAexMincho
]{IPAexMincho}
\setCJKsansfont{IPAexGothic}
\setCJKmonofont{IPAexGothic}

% =========================
% 数式・図・表
% =========================
\PassOptionsToPackage{draft}{graphicx} % 図未投入でも停止しない
\usepackage{graphicx}
\graphicspath{{./}{figures/}}
\DeclareGraphicsExtensions{.pdf,.png,.jpg}

% 図欠落でもビルド継続(\includegraphics を安全版に置換)
\newcommand{\safeincludegraphics}[2][]{%
  \IfFileExists{#2}{\includegraphics[#1]{#2}}{%
    \fbox{\parbox[c][0.45\linewidth][c]{0.9\linewidth}{\centering
    \textit{[Missing image: #2]}}}}}
\let\includegraphics\safeincludegraphics

\usepackage{amsmath,amssymb,bm}
\usepackage{booktabs}
\usepackage{multirow}

% --- プリアンブルに追加 ---
\usepackage{tikz}
\usepackage{pgfplots}
\pgfplotsset{compat=1.18}

% =========================
% URL/ハイパーリンクと _ 対策
% =========================
\usepackage{url}
\def\UrlBreaks{\do\/\do-\do\_\do\&\do\?}

% ← 追加:本文・キャプション内の '_' を許可(必ず hyperref より前)
\usepackage[strings]{underscore}

% hyperref は最後に
\usepackage[unicode,breaklinks=true]{hyperref}
\hypersetup{
  pdfauthor={Shinichi Samizo},
  pdftitle={uTEPヘッド用HVドライバICにおけるウエハ酸素濃度変動を考慮したスクリーニング条件最適化},
  pdfborder={0 0 0}
}

% \usepackage[top=19mm,bottom=19mm,left=19mm,right=19mm]{geometry}

\title{uTEPヘッド用HVドライバICにおけるウエハ酸素濃度変動を考慮したスクリーニング条件最適化}

\author{%
  \IEEEauthorblockN{三溝 真一 (Shinichi Samizo)}%
  \IEEEauthorblockA{%
    独立系半導体研究者(元セイコーエプソン株式会社)\\%
    Independent Semiconductor Researcher (ex-Seiko Epson Corporation)\\[3pt]%
    Email:~\href{mailto:shin3t72@gmail.com}{shin3t72@gmail.com}\quad
    GitHub:~\url{https://github.com/Samizo-AITL}%
  }%
}

% ============================================================
% アブストラクトとキーワードをタイトル出力前に予約
% ============================================================
\IEEEtitleabstractindextext{%
\begin{abstract}
\textbf{和文概要}—本論文は、uTEPプリントヘッドを駆動するHVドライバICにおいて、ウエハ酸素濃度変動を考慮した量産スクリーニング条件の最適化手法を報告する。  
ウエハ供給元変更に伴う酸素濃度上昇により、結晶起因欠陥(COP: Crystal-Originated Particle)に起因したHVゲートリークが顕在化し、従来の単回スクリーニングでは取り残しが発生した。  
本研究では、電圧・温度・繰返し回数を変数とする工程条件を再設計し、uTEPヘッド適用において動作保証範囲内(85℃以下)で不良収束を達成する方法を示す。

\bigskip
\noindent
\textbf{Abstract}—This paper presents the optimization of high-voltage (HV) driver IC screening conditions for uTEP printhead applications, considering wafer oxygen concentration variations.  
A supplier change increased wafer oxygen content, leading to the emergence of crystal-originated particle (COP) defects that cause HV gate leakage.  
Conventional single-pass screening failed to eliminate residual defects.  
By redesigning the process parameters—voltage, temperature, and repetition count—the proposed screening achieves defect convergence within the guaranteed operating range (below 85 °C), improving reliability and production stability.
\end{abstract}

\begin{IEEEkeywords}
uTEP head, Inkjet, Driver IC, High-Voltage Screening, Wafer Oxygen, COP Defect, Reliability Engineering, Process Optimization
\end{IEEEkeywords}
}

% ============================================================
\begin{document}
\maketitle
\IEEEdisplaynontitleabstractindextext % ← 図がアブストラクトより上に出るのを防止
% ============================================================

%------------------------------------------------------
\section{はじめに}
uTEP(\textit{micro Thin-film Electrostatic / Piezo} 系列を含む統合アーキテクチャ)プリントヘッドは、
高電圧(HV)駆動によりインク液柱を微細ノズルから高精度に射出する構造を有している。
このヘッドを駆動するHVドライバICは、数百~数千チャネルのアクチュエータ電極を並列制御するため、
高集積化と高信頼性の両立が求められる重要な要素である。

当該ドライバICは、CMOS 0.35\,$\mu$mプロセスを用い、3.3\,Vロジック動作および45\,V高耐圧出力を同一チップ上で実現している。
しかし、ウエハ供給元の切替後、ウエハ中の酸素濃度が上昇したロットにおいて、
結晶起因欠陥(COP: Crystal-Originated Particle)由来の微小ボイドがデバイス活性層中に析出した。
これにより、HVゲート酸化膜で局所的な電界集中が発生し、
アクチュエータユニットやヘッド電気特性検査工程においてゲートリーク不良が多発する事象が確認された。

従来のスクリーニング条件では、これら潜在的リーク欠陥を完全に除去できず、
特にHV絶縁膜の応力緩和挙動に起因する初期劣化モードが工程内で検出されないケースがあった。
本研究では、ウエハ酸素濃度変動を前提としてHVデバイスの特性ばらつきを解析し、
電圧・温度・繰返し回数をパラメータとするスクリーニング条件を再設計することにより、
量産工程における不良収束性の向上と、プリントヘッドとしての長期信頼性確保を両立する手法について報告する。

\section{問題の概要(uTEP適用での観測)}

図\ref{fig_lpd}にウエハLPD(Laser Particle Defect)分布の代表例を示す。
高酸素ロットではLPD密度が高く、結晶起因欠陥(COP: Crystal-Originated Particle)に由来する局所欠陥の増加が確認される。
これらの欠陥は、HVデバイスのゲート酸化膜下部に微小ボイドとして存在し、通電時の電界集中によって局所リークを誘発する。

従来のスクリーニング条件(48\,V・単回通電)では、これら潜在欠陥の取り残しが発生した。
結果として、uTEPヘッド組立後の電気特性検査やプリンタ実装後の量産評価において、
ゲートリーク起因の不良が顕在化する事象が確認された。
特に、HVゲートの初期劣化モードは低電圧通電では顕在化しにくく、スクリーニング初期段階での除去が不十分であったことが判明した。

本事象は、ウエハ酸素濃度変動によりHV絶縁膜の電界分布特性が変化した結果であり、
単純な電圧マージン拡大では再現性が得られないことが確認されている。
したがって、量産スクリーニングにおける電圧・温度・繰返し条件の最適化が必要である。

\begin{figure}[t]
  \centering
  % 列幅に強制フィット(高さは自動スケール)
  \resizebox{\linewidth}{!}{%
  \begin{tikzpicture}
    % 2枚並列のグループプロット(左:通常ロット/右:高酸素ロット)
    \begin{groupplot}[
      group style={group size=2 by 1, horizontal sep=12mm},
      width=0.48\linewidth, height=0.48\linewidth,
      ymin=0,
      ymajorgrids,
      xmajorgrids,
      tick align=outside,
      tick style={thin},
      xlabel={LPDサイズ [$\mu$m]},
      ylabel={密度 [cm$^{-2}$]},
      xtick={0,1,2,3,4,5},
      xticklabel style={/pgf/number format/fixed},
      yticklabel style={/pgf/number format/fixed},
      title style={yshift=-2pt},
      % 文字が大きくなり過ぎないように
      label style={font=\scriptsize},
      tick label style={font=\scriptsize},
      title style={font=\scriptsize},
      legend style={font=\scriptsize, draw=none, fill=none},
      legend cell align=left,
    ]

    % --- 左:通常ロット ---
    \nextgroupplot[
      title={通常ロット(代表)},
      legend pos=north east,
    ]
      % 疑似ヒストグラム(棒グラフで表現)
      \addplot[ybar, bar width=6pt]
        coordinates {
          (0.5,  15)
          (1.0,  10)
          (1.5,   7)
          (2.0,   5)
          (2.5,   3)
          (3.0,   2)
          (3.5,   1)
          (4.0, 0.5)
          (4.5, 0.2)
          (5.0, 0.1)
        };
      \addlegendentry{LPD密度}

    % --- 右:高酸素ロット ---
    \nextgroupplot[
      title={高酸素ロット(代表)},
      legend pos=north east,
    ]
      \addplot[ybar, bar width=6pt]
        coordinates {
          (0.5,  40)
          (1.0,  32)
          (1.5,  25)
          (2.0,  18)
          (2.5,  12)
          (3.0,   9)
          (3.5,   6)
          (4.0,   4)
          (4.5,   2)
          (5.0,   1)
        };
      \addlegendentry{LPD密度}

    \end{groupplot}
  \end{tikzpicture}%
  }% \resizebox
  \caption{ウエハLPD分布比較(高酸素ロットと通常ロットの代表例)}
  \label{fig_lpd}
\end{figure}

\section{スクリーニング条件最適化(量産仕様内)}
本章では、uTEPヘッド適用を前提としたHVドライバICの量産スクリーニング条件の最適化について述べる。
目的は、\textbf{仕様範囲内(温度85\,℃以下)}で不良除去率を最大化しつつ、
良品への影響を最小化する工程条件を確立することである。

\subsection{評価パラメータ}
表\ref{tab_param}に本評価で用いたパラメータ設定を示す。
評価は、量産実績ロット(高酸素ロット・通常ロット)を対象に実施した。
印加条件の主変数は、電圧・温度・印加回数の3因子とし、各回の通電後にゲートリーク電流および絶縁破壊電圧を測定した。

\begin{table}[t]
\centering
\caption{スクリーニング評価パラメータ(HVドライバIC)}
\label{tab_param}
\begin{tabular}{ll}
\toprule
項目 & 条件設定 \\
\midrule
電圧 & 48\,V(絶対最大定格内、電流コンプライアンス付) \\
温度 & 85\,℃(動作保証上限) \\
印加回数 & 1~8回(繰返し通電) \\
印加時間 & 各回30\,分 \\
測定項目 & $\Delta I_\mathrm{GATE}$, $\Delta BV_\mathrm{G}$, d$I$/d$V$(常温後測定) \\
\bottomrule
\end{tabular}
\end{table}

\subsection{結果(uTEPロットでの代表値)}
表\ref{tab_results}に、スクリーニング回数別の新規検出率と残存不良率を示す。
初回で約60\,\%の不良が除去され、2~3回目で大部分の潜在欠陥が顕在化した。
4回目以降では新規検出がほぼ停止し、高酸素ロットにおいても残存不良率は30\,ppm以下に収束した。

このことから、48\,V・85\,℃条件で4回印加することで、
良品劣化を伴わずに不良収束性を確保できることが確認された。
図\ref{fig_convergence}に代表ロットでの収束挙動を示す。

\begin{table}[t]
\centering
\caption{スクリーニング回数と不良収束率(uTEPヘッド用HVドライバIC、代表ロット)}
\label{tab_results}
\begin{tabular}{ccc}
\toprule
回数 & 新規検出率[\%] & 残存不良率[ppm] \\
\midrule
1 & 58.9 & 420 \\
2 & 25.1 & 125 \\
3 & 11.0 & 52 \\
4 & 4.0  & 30 \\
5以降 & $\approx$0 & $<$30 \\
\bottomrule
\end{tabular}
\end{table}

\begin{figure}[t]
  \centering
  % TikZ版(はみ出し防止済)
  \resizebox{\linewidth}{!}{%
  \begin{tikzpicture}
    \begin{axis}[
      width=\linewidth,
      height=0.55\linewidth,
      xlabel={スクリーニング回数},
      ylabel={残存不良率 [ppm]},
      xmin=0, xmax=8,
      ymin=0, ymax=500,
      xtick={0,1,2,3,4,5,6,7,8},
      ymajorgrids,
      grid style={dashed,gray!40},
      line width=0.6pt,
      mark size=2.0pt,
      tick label style={font=\scriptsize},
      label style={font=\scriptsize},
    ]
      \addplot[color=blue, mark=*] coordinates {
        (1,420)
        (2,125)
        (3,52)
        (4,30)
        (5,25)
        (6,24)
        (7,23)
        (8,23)
      };
      \addlegendentry{残存不良率(代表ロット)}
    \end{axis}
  \end{tikzpicture}%
  }
  \caption{回数に対する不良収束曲線(uTEP対象、代表ロット)}
  \label{fig_convergence}
\end{figure}

\section{考察}
uTEPヘッド適用においては、温度上限を越えることなく、
\textbf{繰返し回数の最適化}によってCOP起因の潜在欠陥を早期に顕在化させることが可能である。
これは、通電応力によりゲート絶縁膜内部の電界集中部が段階的に活性化し、
微小リーク経路が成長・開通する過程を利用している。

各回の通電後に常温で$\Delta I_\mathrm{GATE}$および$\Delta BV_\mathrm{G}$を逐次計測することで、
欠陥進展挙動をモニタしつつ、良品の酸化膜疲労を抑制できる。
特に、$\Delta BV_\mathrm{G}$の漸減と$\Delta I_\mathrm{GATE}$の急増を同時に検出したデバイスは、
初期的なリーク成長モードに入ったと判断でき、量産工程において高い再現性で不良抽出が可能であった。

本手法は、電圧・温度を仕様範囲内に保持しながら欠陥顕在化を促進する点で、
従来の単回スクリーニングに比べて良品影響が著しく少ない。
また、ウエハ酸素濃度や結晶成長履歴など、材料起因のばらつきを吸収できるため、
ロット間安定性の向上にも寄与する。

さらに、HVデバイス一般に共通する酸化膜ストレス緩和挙動に基づく設計であることから、
uTEP専用ICのみならず、インクジェット駆動系の他HV品種への水平展開が可能である。
特に、温度制約の厳しい小型プリンタ向けデバイスや、高信頼用途(産業機器・車載)への応用においても、
本スクリーニング設計指針は有効であると考えられる。

\section{結論}
本研究では、uTEPヘッド用HVドライバICを対象に、ウエハ酸素濃度変動を考慮した量産スクリーニング条件の最適化を行った。
電圧48\,V・温度85\,℃という仕様範囲内の条件下で、複数回印加による逐次スクリーニングを導入した結果、
高酸素ロットにおいても不良率を30\,ppm以下に収束させることができた。
同時に、良品デバイスへの影響は測定誤差レベル以下に抑えられ、信頼性低下は認められなかった。

本手法は、COP(Crystal-Originated Particle)に起因する潜在欠陥を早期に顕在化させることで、
ウエハ品質のばらつきを工程内で吸収し、量産品質の安定化に寄与するものである。
また、温度上限を超えない制約下で高い除去効率を達成できるため、
他のHVデバイス群(アクチュエータ駆動用・車載用IC等)への適用拡張も期待できる。

今後は、$\Delta I_\mathrm{GATE}$および$\Delta BV_\mathrm{G}$の時間依存解析を組み合わせ、
デバイス個体ごとの劣化モデル化を進めることで、より短時間かつ自動化されたスクリーニングプロセスの構築を目指す。

\section*{謝辞}
本研究の遂行にあたり、ご協力を賜ったデバイス技術部ならびに信頼性評価チーム、
およびuTEPヘッド量産に携わる関係各位に深く感謝の意を表する。

%------------------------------------------------------
\begin{thebibliography}{99}

\bibitem{samizo_bom}
三溝 真一, 「設計から量産部品発注に至る実務フローとBOM運用ルールの体系化」, 2025.

\bibitem{samizo_mach}
三溝 真一, 「Sn–Bi代替による接合方式移行(Mach世代)」, 2025.

\bibitem{samizo_tfp}
三溝 真一, 「薄膜PZTアクチュエータの信頼性解析と対策」, 2025.

\end{thebibliography}

%------------------------------------------------------
\section*{著者略歴}
\noindent\textbf{三溝 真一(Shinichi Samizo)}:\\
信州大学大学院修了。半導体デバイス(ロジック、メモリ、高耐圧インテグレーション)を中心に、プロセス技術・デバイス構造・回路設計の統合開発に従事。\\
セイコーエプソン株式会社にてインクジェット用HVドライバIC、アクチュエータ駆動ASIC、MEMSデバイスの設計・評価・信頼性解析を担当。\\
現在は独立系半導体研究者として、デバイス教育、高耐圧CMOS統合設計、およびシステムレベル最適化研究に従事。\\
Email: \href{mailto:shin3t72@gmail.com}{shin3t72@gmail.com}\quad
GitHub: \url{https://github.com/Samizo-AITL}

\end{document}
