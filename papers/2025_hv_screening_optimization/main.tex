\documentclass[twocolumn]{ieeetran}
\usepackage[dvipdfmx]{graphicx}
\usepackage{amsmath}
\usepackage{url}
\usepackage{hyperref}

\title{インクジェット用HVドライバICにおけるウエハ酸素濃度変動を考慮したスクリーニング条件最適化}
\author{三溝 真一(Shinichi Samizo)\\
独立半導体研究者 / AITL Project}
\date{}

\begin{document}
\maketitle

\begin{abstract}
本論文では、インクジェットヘッド駆動用HVドライバICにおけるスクリーニング条件最適化について報告する。
ウエハ供給元の変更により酸素濃度が上昇し、結晶起因欠陥(COP)によるHVゲートリーク不良が顕在化した。
従来のスクリーニング条件では潜在欠陥を完全に検出できず、後工程で不良が発生した。
本研究では、電圧・温度・繰返し回数を変数としてスクリーニング条件を再設計し、
保証範囲内で欠陥除去率を最大化する手法を確立した。
その結果、良品デバイスへの影響を抑えつつ、ウエハロット差起因の不良流出を防止できることを確認した。
\end{abstract}

\section{はじめに}
インクジェットプリンタ用ヘッドを駆動するHVドライバICは、
CMOS 0.35\,$\mu$mプロセスを用い、3.3\,V/45\,V動作を実現している。
近年、ウエハ供給元の変更により酸素濃度が高いロットが混在し、
COP(Crystal-Originated Particle)由来の局所欠陥がHVデバイスに発生した。
従来のスクリーニングではこれらの微小欠陥を完全に除去できず、
アクチュエータおよびヘッド工程で不良が多発した。

\section{問題の概要}
高酸素ロットでは、レーザ粒子検査(LPD)での欠陥密度が上昇し、
HVゲート膜界面での局所電界集中によりリーク電流が増加する傾向が確認された。
従来条件(48\,V、単回ストレス)では、COP起因の早期TDDBモードを取り切れず、
後工程での歩留まり低下や市場流出リスクが顕在化した。

\section{スクリーニング条件の最適化}
本検討では、スクリーニング条件を電圧・温度・ストレス回数の3要素で体系的に再設計した。
目標は「仕様範囲内で不良収束を確認し、良品劣化を防止する」ことである。

\subsection{評価パラメータ}
\begin{itemize}
\item 電圧:48\,V(絶対最大定格以内)
\item 温度:85\,℃(動作保証上限)
\item 繰返し回数:1~8回
\item 印加時間:各30分
\item 測定項目:$\Delta I_\mathrm{GATE}$、$\Delta BV_\mathrm{G}$、d$I$/d$V$
\end{itemize}

\subsection{結果}
繰返し回数を増やすことで不良が指数的に収束することを確認した。
特に4回目以降では新規不良の発生がほぼ停止し、
ウエハ酸素濃度の高いロットでも残存不良率が30\,ppm以下に低下した。
温度上限を超えない範囲でも、十分なスクリーニング強度が得られることを実証した。

\section{考察}
スクリーニング強度は温度よりも繰返し回数への依存性が高く、
複数回の通電によりCOP起因欠陥を早期に顕在化できる。
また、各回後の常温測定でリーク変化を逐次監視することで、
良品デバイスの疲労を防止できる。
本最適化手法は、HVデバイスにおける材料ロット差吸収にも有効である。

\section{結論}
インクジェット用HVドライバICにおいて、ウエハ酸素濃度変動に起因する不良を抑制するため、
スクリーニング条件の最適化を実施した。
電圧48\,V・温度85\,℃の仕様範囲内で、繰返し回数を複数化することで不良収束を確認し、
良品デバイスの信頼性を維持した。
本手法は他のHVプロセス製品にも適用可能である。

\section*{謝辞}
本検討に協力頂いたデバイス技術部、信頼性評価チーム各位に深謝する。

\section*{著者略歴}
\noindent\textbf{三溝 真一(Shinichi Samizo)}:\\
信州大学大学院修了。セイコーエプソンにて半導体およびインクジェット開発に従事。\\
現在は独立系半導体研究者としてデバイス教育とシステム統合研究に従事。\\
連絡先:\href{mailto:shin3t72@gmail.com}{shin3t72@gmail.com}

\end{document}
