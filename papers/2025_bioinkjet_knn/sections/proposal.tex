\section{Proposed Architecture}
The proposed Bio-Inkjet (Bio-IJ) actuator system is designed to balance
biocompatibility, manufacturability, and sufficient actuation
performance for biomedical applications. Its main components are as
follows:
\begin{itemize}
  \item \textbf{Bulk KNN multilayer stack actuator}: 
        A 200--500~\si{\micro\meter} thick piezoelectric stack providing
        moderate displacement suitable for picoliter-scale droplet
        ejection.
  \item \textbf{COF driver IC}: 
        A chip-on-film high-voltage driver with 16--32 channels, 
        operating at up to $\pm 50$~V, and incorporating waveform RAM 
        with SPI control for flexible pulse shaping.
  \item \textbf{Si cavity and nozzle array}: 
        A silicon-etched cavity directly bonded to the actuator, with
        nozzles of $\varphi$\SI{8}{\micro\meter} diameter producing
        droplets in the 3--5~\si{\pico\liter} range.
  \item \textbf{Reservoir and back pressure control}: 
        A fluidic supply stabilized at approximately $-2.0$~kPa using
        a PID-controlled regulator to ensure consistent meniscus
        positioning.
  \item \textbf{PI membrane damper}: 
        A polyimide-based damping layer integrated in the reservoir to
        suppress pressure fluctuations and prevent satellite droplets.
\end{itemize}

The associated \textit{process flow} begins with fabrication of the bulk
KNN stack and electrode finishing, followed by terminal cutting and COF
assembly. The actuator is then mounted with a heat spreader for thermal
management and finally bonded to the silicon cavity structure,
resulting in an integrated Bio-IJ printhead module.
