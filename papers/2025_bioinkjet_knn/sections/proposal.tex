\section{Proposed Architecture}
The proposed Bio-Inkjet (Bio-IJ) actuator system is designed to balance
biocompatibility, manufacturability, and sufficient actuation
performance for biomedical applications. Its main components are as
follows:
\begin{itemize}
  \item \textbf{Bulk KNN multilayer stack actuator}:
        A 200--500~\si{\micro\meter} thick piezoelectric stack providing
        moderate displacement suitable for picoliter-scale droplet
        ejection.
  \item \textbf{COF driver IC}:
        A chip-on-film high-voltage driver with 16--32 channels,
        operating at up to $\pm 50$~\si{\volt}, and incorporating waveform RAM
        with SPI control for flexible pulse shaping.
  \item \textbf{Si cavity and nozzle array}:
        A silicon-etched cavity directly bonded to the actuator, with
        nozzles of $\varphi$\SI{8}{\micro\meter} diameter producing
        droplets in the 3--5~\si{\pico\liter} range.
  \item \textbf{Reservoir and back-pressure control}:
        A fluidic supply stabilized at approximately \SI{-2.0}{\kilo\pascal}
        using a PID-controlled regulator to ensure consistent meniscus
        positioning.
  \item \textbf{PI membrane damper}:
        A polyimide-based damping layer integrated in the reservoir to
        suppress pressure fluctuations and prevent satellite droplets.
\end{itemize}

The associated \textit{process flow} begins with fabrication of the bulk
KNN stack and electrode finishing, followed by terminal cutting and COF
assembly. The actuator is then mounted with a heat spreader for thermal
management and finally bonded to the silicon cavity structure,
resulting in an integrated Bio-IJ printhead module.

% ===== Figures =====

% --- Fig.1: Block diagram (two-column width) ---
\begin{figure*}[t]
  \centering
  \resizebox{\textwidth}{!}{%
  \begin{tikzpicture}[
      node distance=18mm,
      box/.style={draw, rounded corners, align=center, minimum width=36mm, minimum height=10mm},
      arrow/.style={->, thick},
      every node/.style={font=\footnotesize}
    ]
    % Nodes
    \node[box] (knn) {Bulk KNN\\multilayer actuator\\(200--500\,\si{\micro\meter})};
    \node[box, right=30mm of knn] (cof) {COF driver IC\\(16--32 ch, $\pm 50$\,\si{\volt})\\Waveform RAM / SPI};
    \node[box, right=30mm of cof] (noz) {Si cavity \&\\nozzle array\\($\varphi$\,8\,\si{\micro\meter}, 3--5\,\si{\pico\liter})};
    \node[box, above=14mm of cof] (bp) {Reservoir \& Back Pressure\\($\approx -2.0$\,\si{\kilo\pascal}) + PI damper};

    % Arrows
    \draw[arrow] (cof) -- node[above]{HV pulses} (knn);
    \draw[arrow] (knn) -- (noz);
    \draw[arrow] (bp)  -- (noz);
  \end{tikzpicture}%
  }
  \caption{System architecture of the proposed Bio-Inkjet (Bio-IJ). A bulk KNN actuator, COF high-voltage driver, silicon cavity/nozzles, and fluidics (back-pressure + PI damper) are integrated.}
  \label{fig:block}
\end{figure*}

% --- Fig.2: Cross section (single-column width) ---
\begin{figure}[t]
  \centering
  \resizebox{\columnwidth}{!}{%
  \begin{tikzpicture}[
      every node/.style={font=\footnotesize},
      box/.style={draw, rounded corners, align=center, minimum height=6mm}
    ]
    % Layers
    \node[box, minimum width=75mm] (hs) {Heat spreader};
    \node[box, below=1mm of hs,  minimum width=75mm] (act) {Bulk KNN multilayer stack (200--500\,\si{\micro\meter})};
    \node[box, below=1mm of act, minimum width=75mm] (cav) {Etched Si cavity};
    \node[box, below=1mm of cav, minimum width=75mm] (sub) {Si substrate / nozzle outlet ($\varphi$\,8\,\si{\micro\meter})};

    % Callouts
    \node[align=left, left=0mm of act, xshift=-3mm] (lefttxt) {Back pressure $\approx -2.0$\,\si{\kilo\pascal}\\PI membrane damper\\Supply};
    \draw[-] (lefttxt.east) -- ++(3mm,0) |- (cav.west);

    \node[align=left, right=0mm of act, xshift=3mm] (righttxt) {Target: 3--5\,\si{\pico\liter} droplets\\$\sim 10^6$ shots class};
    \draw[-] (righttxt.west) -- ++(-3mm,0) |- (sub.east);
  \end{tikzpicture}%
  }
  \caption{Cross-sectional schematic of the Bio-IJ printhead. The KNN stack is bonded to a silicon cavity with $\varphi$8\,\si{\micro\meter} nozzles; fluidics provide back-pressure stabilization and damping.}
  \label{fig:section}
\end{figure}
