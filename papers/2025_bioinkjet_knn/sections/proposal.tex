\section{Proposed Architecture}
The proposed Bio-Inkjet (Bio-IJ) actuator system is designed to balance
biocompatibility, manufacturability, and sufficient actuation
performance for biomedical applications. Its main components are as
follows:
\begin{itemize}
  \item \textbf{Bulk KNN multilayer stack actuator}:
        A 200--500~\si{\micro\meter} thick piezoelectric stack providing
        moderate displacement suitable for picoliter-scale droplet
        ejection.
  \item \textbf{COF driver IC}:
        A chip-on-film high-voltage driver with 16--32 channels,
        operating at up to $\pm 50$~\si{\volt}, and incorporating waveform RAM
        with SPI control for flexible pulse shaping.
  \item \textbf{Si cavity and nozzle array}:
        A silicon-etched cavity directly bonded to the actuator, with
        nozzles of $\varphi$\SI{8}{\micro\meter} diameter producing
        droplets in the 3--5~\si{\pico\liter} range.
  \item \textbf{Reservoir and back pressure control}:
        A fluidic supply stabilized at approximately $-2.0$~kPa using
        a PID-controlled regulator to ensure consistent meniscus
        positioning.
  \item \textbf{PI membrane damper}:
        A polyimide-based damping layer integrated in the reservoir to
        suppress pressure fluctuations and prevent satellite droplets.
\end{itemize}

The associated \textit{process flow} begins with fabrication of the bulk
KNN stack and electrode finishing, followed by terminal cutting and COF
assembly. The actuator is then mounted with a heat spreader for thermal
management and finally bonded to the silicon cavity structure,
resulting in an integrated Bio-IJ printhead module.

% ===== Figures (1-column) =====

% Fig.1: Block diagram (vertical layout to fit one column)
\begin{figure}[t]
  \centering
  \tikzset{font=\footnotesize}
  \begin{tikzpicture}[
      node distance=4mm and 6mm,
      box/.style={draw, rounded corners, align=center, minimum width=34mm, minimum height=7mm},
      arrow/.style={-Latex, thick}
    ]
    % Nodes (vertical stacking)
    \node[box] (cof) {COF driver IC\\(16--32 ch, $\pm 50$~\si{\volt})\\Waveform RAM / SPI};
    \node[box, below=of cof] (knn) {Bulk KNN multilayer actuator\\(200--500~\si{\micro\meter})};
    \node[box, below=of knn] (noz) {Si cavity \& nozzle array\\($\varphi$\,8~\si{\micro\meter}, 3--5~\si{\pico\liter})};
    \node[box, right=16mm of knn] (bp) {Reservoir \& back pressure\\($\approx -2.0$~kPa) + PI damper};

    % Arrows
    \draw[arrow] (cof) -- node[right]{HV pulses} (knn);
    \draw[arrow] (knn) -- (noz);
    \draw[arrow] (bp.west) -- (noz.east);
  \end{tikzpicture}
  \caption{System architecture of the proposed Bio-Inkjet (Bio-IJ). A bulk KNN actuator, COF high-voltage driver, silicon cavity/nozzles, and fluidics (back-pressure with PI damper) are integrated.}
  \label{fig:block}
\end{figure}

% --- Fig.2: Cross section (single-column width, no overflow) ---
\begin{figure}[!t]
  \centering
  \begin{tikzpicture}[
      every node/.style={font=\scriptsize},
      box/.style={draw, rounded corners, align=center,
                  minimum height=5mm, text width=.9\linewidth},
      line width=0.3pt
    ]
    % 本体レイヤ(幅は列幅の 0.9 倍に固定)
    \node[box] (hs)  {Heat spreader};
    \node[box, below=1mm of hs]  (act) {Bulk KNN multilayer stack (200--500\,\si{\micro\meter})};
    \node[box, below=1mm of act] (cav) {Etched Si cavity};
    \node[box, below=1mm of cav] (sub) {Si substrate / nozzle outlet ($\varphi$\,8\,\si{\micro\meter})};

    % 説明(図の内側・上部に収める)
    \node[align=left, anchor=west]  at ([xshift=-.45\linewidth,yshift=2.5mm]hs.north)
      {Back pressure $\approx -2.0$\,kPa;\quad PI membrane damper;\quad Supply};
    \node[align=right, anchor=east] at ([xshift=.45\linewidth,yshift=2.5mm]hs.north)
      {Target: 3--5\,\si{\pico\liter} droplets;\quad $\sim 10^6$ shots class};

    % (必要なら上の2行から矢印を足せますが、幅外へ出ないように線は内側で完結させます)
  \end{tikzpicture}
  \caption{Cross-sectional schematic of the Bio-IJ printhead.
  The KNN stack is bonded to a silicon cavity with $\varphi$\,8\,\si{\micro\meter}
  nozzles; fluidics provide back-pressure stabilization and damping.}
  \label{fig:section}
\end{figure}
