\section{Conclusion}
This paper has proposed a Bio-Inkjet (Bio-IJ) architecture based on
bulk KNN actuators as a lead-free alternative to conventional PZT-based
printheads.
By combining multilayer KNN stacks, COF driver ICs, and silicon cavity
integration, the system achieves picoliter-scale droplet generation
under moderate voltages while ensuring material biocompatibility.

Unlike industrial printing, where full PZT compatibility in terms of
maximum $d_{33}$, billion-cycle endurance, and cost efficiency is
required, biomedical printing places emphasis on safety, controlled
droplet volume, and operational reliability over shorter lifetimes.
The proposed approach aligns well with these requirements, providing
sufficient performance for applications such as cell patterning,
protein microarrays, and hydrogel 3D fabrication.

These findings highlight bulk KNN as a practical foundation for
standardizing lead-free Bio-IJ systems in research, clinical, and
educational domains.
Future work will involve experimental validation of droplet formation,
long-term reliability testing under bio-relevant conditions, and system
integration with existing bioprinting workflows.

\textbf{Most importantly, this work emphasizes that in biomedical inkjet
printing, safety and biocompatibility must take precedence over extreme
performance metrics, positioning KNN-based Bio-IJ as a safe and viable
path toward Pb-free bioprinting.}
