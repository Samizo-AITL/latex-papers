% 本文では \subsection 扱いにする
\subsection{Self-Diagnosis and AITL Control}

The electromechanical response of the bulk KNN actuator can be exploited 
for self-diagnosis and adaptive control. 
Surrogate signals derived from drive current and charge increments 
correlate with droplet volume and viscosity, enabling online monitoring.

Two representative strategies are highlighted:

\begin{itemize}
  \item \textbf{Self-diagnosis:} Detecting missing droplets 
  by charge/displacement thresholds, estimating viscosity 
  from relaxation dynamics.
  \item \textbf{Adaptive control:} Updating PID gains and FSM transitions 
  in real time based on surrogate measurements, while keeping 
  safety constraints (drive voltage $\leq \pm 50$ V).
\end{itemize}

Detailed mathematical models, including recursive least squares (RLS) 
identification, MIT-rule adaptation, and viscosity mapping, 
are provided in the Appendix.
