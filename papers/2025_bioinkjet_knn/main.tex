\documentclass[conference]{IEEEtran}

% ===== Packages =====
\usepackage{graphicx}
\usepackage{amsmath,amssymb} % triangleq 等に必要
\usepackage{siunitx}
\usepackage{cite}
\usepackage{tikz}
\usetikzlibrary{arrows.meta,positioning,fit,calc,shapes.geometric,decorations.pathreplacing}
\usepackage[hidelinks]{hyperref}
\urlstyle{same}

% ===== Title & Author =====
\title{Lead-free Bio-Inkjet Printing with Bulk KNN Actuators and AITL-based Adaptive Control}

\author{%
  \IEEEauthorblockN{Shinichi Samizo}
  \IEEEauthorblockA{Independent Semiconductor Researcher\\
  Former Engineer at Seiko Epson Corporation\\
  Email: \href{mailto:shin3t72@gmail.com}{shin3t72@gmail.com}\\
  GitHub: \url{https://github.com/Samizo-AITL}}%
}

\begin{document}
\maketitle

% ===== Abstract =====
\begin{abstract}
This paper proposes a biocompatible inkjet printing architecture based on
lead-free piezoelectric actuators, specifically bulk KNN (K,Na)NbO$_3$, combined
with chip-on-film (COF) driver ICs, silicon cavity integration, and
an AITL (Adaptive Intelligent Three-Layer) supervisory control framework.
Unlike conventional PZT-driven industrial heads, the KNN approach supports
biocompatibility and Pb-free compliance, while moderate performance
(3--5 pL droplets, $10^6$ shots, $\pm50$ V operation) satisfies bioprinting needs.
Beyond actuation, the electromechanical response of KNN is leveraged for
\emph{self-diagnosis}: detecting missing droplets and estimating ink viscosity.
This diagnostic data feeds into an AITL architecture---an inner PID loop for
real-time stability, an FSM layer for mode transitions and compensation,
and an LLM-driven supervisory layer for adaptive redesign of control rules.
We present the architecture, process flow, and representative biomedical
applications, highlighting the potential of KNN as both actuator and sensor.
\end{abstract}

% ===== Body =====
% 章タイトルは各 sections/*.tex の先頭の \section に任せる(main側では宣言しない)
\section{Introduction}

In the late 1990s, Japan's semiconductor industry was in transition. At Epson's Sakata 8-inch fab, DRAM was \emph{not} pursued as an end business; rather, DRAM technology transfer was used as a \textbf{strategic vehicle} to absorb submicron process technologies at and beyond 0.35~\si{\micro\meter} and redeploy them into Epson's core devices (ASICs, logic ICs, display drivers, and inkjet driver ICs).

The technology transfer from Mitsubishi covered three nodes, each with a clear role:
(1) 0.5~\si{\micro\meter} 16~Mbit DRAM --- to establish mass-production capability and stabilize fab operation; 
(2) 0.35~\si{\micro\meter} 64~Mbit DRAM (2nd gen) --- to introduce a scaled process while tackling yield window narrowing; 
(3) 0.25~\si{\micro\meter} 64~Mbit DRAM (3rd gen) --- as the next-stage validation bed and the basis for in-house deployment.

This paper focuses on the 0.25~\si{\micro\meter} (3rd gen) ramp-up in 1998: a process overview, the ramp-up method, and a failure-analysis–driven yield-improvement cycle. We also trace how these results enabled the 0.25~\si{\micro\meter} mobile pseudo-SRAM (VSRAM) in 2001 and why trench-based 0.18~\si{\micro\meter} VSRAM was abandoned.

We summarize cell/circuit features of the 0.25-\textmu m 64-Mbit DRAM (stack capacitor, divided word-line, WSi stacks), and the wafer-test binning (Bin1--Bin7). For VSRAM, SRAM-like access is achieved by internal refresh logic while keeping DRAM cells; mobile-grade operation extended the temperature guarantee from 80~\si{\celsius} to 90~\si{\celsius}.

The associated \textit{process flow} begins with fabrication of the bulk
KNN stack and electrode finishing, followed by terminal cutting and COF
assembly. The actuator is then mounted with a heat spreader for thermal
management and finally bonded to the silicon cavity structure,
resulting in an integrated Bio-IJ printhead module.

% ===== Figures =====
\begin{figure}[t]
  \centering
  % TODO: Replace with TikZ or external PDF/PNG
  \includegraphics[width=0.9\linewidth]{figures/block_diagram.pdf}
  \caption{System architecture of the proposed Bio-Inkjet (Bio-IJ).
  The design integrates a bulk KNN multilayer actuator, COF driver IC,
  silicon cavity/nozzle array, and fluidic components (reservoir with
  PI damper and back-pressure control).}
  \label{fig:block}
\end{figure}

\begin{figure}[t]
  \centering
  % TODO: Replace with TikZ or external PDF/PNG
  \includegraphics[width=0.9\linewidth]{figures/cross_section.pdf}
  \caption{Cross-sectional schematic of the Bio-IJ printhead.
  The bulk KNN actuator is bonded onto a silicon cavity with outlet
  nozzles of $\varphi$\SI{8}{\micro\meter}, supported by reservoir
  fluidics, damping, and back-pressure stabilization.}
  \label{fig:section}
\end{figure}

% sections/aitl.tex
\section{Self-Diagnosis and AITL Control: Models and Examples}
This section formalizes (i) a surrogate sensing model that lets the bulk KNN actuator
serve as a proxy sensor and (ii) an AITL (PID--FSM--LLM) control scheme including
concrete PID gain update rules and FSM transition conditions.

\subsection{Surrogate Sensing and Plant Model}
During a drive pulse $u(t)$, the KNN stack current $i(t)$ and charge $q(t)=\int i(t)\,dt$
reflect the electromechanical response (displacement/pressure).
We define a scalar surrogate measurement per shot:
\begin{equation}
y_k \triangleq \alpha_1 q_k^{\text{peak}} + \alpha_2 \Delta q_k + \alpha_3 t_{r,k},
\end{equation}
where $q_k^{\text{peak}}$ is the peak charge, $\Delta q_k$ the charge increment over a fixed
window, and $t_{r,k}$ the rise time; $\alpha_i$ are calibration coefficients.
$y_k$ correlates monotonically with ejected droplet volume $v_k$ for a fixed nozzle geometry.

For inner-loop design we use a discrete-time, second-order, input-output model that captures
fluid loading and viscosity effects:
\begin{equation}
y_{k+1} = a_1(\eta)\,y_k + a_2(\eta)\,y_{k-1} + b_0(\eta)\,u_k + b_1(\eta)\,u_{k-1} + d_k,
\label{eq:plant}
\end{equation}
with viscosity $\eta$ affecting $\{a_i(\eta), b_i(\eta)\}$ mainly through the effective time constant.
A one-step-ahead predictor identifies $\hat{\theta}_k=[\hat{a}_1,\hat{a}_2,\hat{b}_0,\hat{b}_1]^T$ online
via recursive least squares (RLS):
\begin{align}
\phi_k &= \begin{bmatrix} y_k & y_{k-1} & u_k & u_{k-1} \end{bmatrix}^T,\\
\hat{\theta}_{k} &= \hat{\theta}_{k-1} + K_k\big(y_{k}-\phi_{k-1}^T\hat{\theta}_{k-1}\big),\\
K_k &= \frac{P_{k-1}\phi_{k-1}}{\lambda + \phi_{k-1}^T P_{k-1}\phi_{k-1}},\quad
P_k=\lambda^{-1}\!\left(P_{k-1}-K_k\phi_{k-1}^T P_{k-1}\right),
\end{align}
with forgetting factor $\lambda\in(0,1]$.

\subsection{Inner PID (Velocity Form) with Anti-Windup}
Let $e_k = y_k^{\star} - y_k$ be the tracking error for a reference $y_k^{\star}$ (mapped from the
target droplet volume). The PID in velocity form is
\begin{equation}
u_k = \mathrm{sat}_{[u_{\min},u_{\max}]}\Big\{u_{k-1}
+ K_p (e_k - e_{k-1}) + K_i T_s e_k
+ K_d \tfrac{1}{T_s}(e_k - 2e_{k-1} + e_{k-2}) \Big\},
\label{eq:pid}
\end{equation}
with sampling period $T_s$ and $u_{\min},u_{\max}=\pm 50\,\mathrm{V}$.
Anti-windup uses a back-calculation term:
\begin{equation}
I_k = I_{k-1} + K_i T_s e_k + K_{aw}(u_k - u_k^{\text{unsat}}),
\end{equation}
where $u_k^{\text{unsat}}$ is the RHS before saturation and $K_{aw}>0$.

\subsection{Adaptive PID Gain Update: Two Examples}
We provide two practical gain-update options driven by the online model
or directly by performance gradients.

\paragraph*{(A) IMC-PID Retuning from Identified FOPDT}
Fit the predictor \eqref{eq:plant} locally by a first-order-plus-dead-time (FOPDT) surrogate
$G(s)\approx \frac{k}{\tau s+1}e^{-\theta s}$. With a chosen closed-loop time constant $\lambda$,
the IMC--PID mapping gives
\begin{equation}
K_p = \frac{\tau+\tfrac{\theta}{2}}{k(\lambda+\theta)},\quad
K_i = \frac{K_p}{\tau+\tfrac{\theta}{2}},\quad
K_d = \frac{\tau\theta}{2(\tau+\tfrac{\theta}{2})}.
\label{eq:imc}
\end{equation}
As viscosity increases ($\eta\uparrow$), $\tau$ grows; \eqref{eq:imc} automatically increases $K_p$
moderately and adjusts $K_i,K_d$ to maintain rise time and damping.

\paragraph*{(B) MIT-Rule (Gradient) Adaptation on Shot Cost}
Define a per-shot cost $J_k=\tfrac{1}{2}e_k^2 + \alpha\,\Delta u_k^2$, $\Delta u_k=u_k-u_{k-1}$.
Let $\theta=[K_p,K_i,K_d]^T$. The discrete MIT rule updates
\begin{equation}
\theta_{k+1}=\theta_k - \Gamma\,\psi_k\,e_k,\qquad
\psi_k \approx \frac{\partial y_k}{\partial \theta}\Big|_{\text{filtered}},
\end{equation}
where $\Gamma=\mathrm{diag}(\gamma_p,\gamma_i,\gamma_d)$ and $\psi_k$ is a filtered regressor
computed from \eqref{eq:pid} and the predictor \eqref{eq:plant}.
Projection enforces bounds $\theta\in[\theta_{\min},\theta_{\max}]$ for safety.

\subsection{Viscosity Estimation and Mapping}
From the identified $\hat{\tau}$ (e.g., by fitting the step response of \eqref{eq:plant}),
a power-law mapping approximates viscosity:
\begin{equation}
\hat{\eta} = c_\eta\,\hat{\tau}^{\,\beta},\qquad \beta>0,
\end{equation}
calibrated off-line with standard fluids.
Drive adaptation then uses $\hat{\eta}$ to select either \eqref{eq:imc} or to adjust
the target closed-loop bandwidth $\lambda(\hat{\eta})$.

\subsection{FSM Layer: States, Guards, and Actions}
We define states $\mathcal{S}=\{\textsf{NORMAL},\textsf{COMPENSATE},\textsf{CLEAN},\textsf{PAUSE}\}$
with guards derived from surrogate features.
Let $A_k$ be charge- or displacement-derived amplitude, $t_{r,k}$ the rise time,
and $\rho_k$ a dot-missing score.
\begin{align}
&\text{Good shot:}\quad A_k \ge A_{\min}\ \land\ t_{r,k}\le t_{r,\max}\\
&\text{Miss score:}\quad \rho_k = w_1\frac{A_{\min}-A_k}{A_{\min}} + w_2\frac{t_{r,k}-t_{r,\max}}{t_{r,\max}}.
\end{align}
Transitions (with $N_{\text{good}}$ consecutive good shots and integer counters $c$):
\begin{align}
\textsf{NORMAL}\ \xrightarrow{\ \rho_k>\tau_{\text{miss}}\ }\ \textsf{COMPENSATE},\qquad
&\text{Action: fire neighbor nozzle(s), raise $u$ by }\Delta u_{\text{comp}}.\\
\textsf{COMPENSATE}\ \xrightarrow{\ N_{\text{good}}\ }\ \textsf{NORMAL},\qquad
&\text{Action: restore nominal $u$, reset $c$.}\\
\textsf{COMPENSATE}\ \xrightarrow{\ c\ge c_{\text{clean}}\ }\ \textsf{CLEAN},\qquad
&\text{Action: cleaning pulse train, then re-test.}\\
\textsf{ANY}\ \xrightarrow{\ \hat{\eta}>\eta_{\max}\ }\ \textsf{PAUSE},\qquad
&\text{Action: hold firing, raise temperature or dilute per recipe.}
\end{align}
Thresholds $\tau_{\text{miss}},A_{\min},t_{r,\max},\eta_{\max}$ are set by calibration and
can be adjusted by the LLM layer based on aggregated logs.

\subsection{Outer LLM Layer: Safe Policy Updates}
The LLM aggregates run logs $\{y_k,u_k,\hat{\eta},\rho_k\}$ and proposes parameter updates
$\{\lambda,\tau_{\text{miss}},\Delta u_{\text{comp}},c_{\text{clean}},N_{\text{good}}\}$ subject to
\emph{safety constraints}: $|u_k|\le 50\,\mathrm{V}$, thermal budget, maximum duty, and
nozzle health metrics. Updates are rolled out in A/B shots with automatic rollback if
the shot cost $J_k$ regresses beyond a tolerance.

\subsection{Implementation Notes}
(1) Rate limiting on $u_k$ ensures acoustic stability; (2) integrate an anti-windup
observer for $I_k$; (3) log features at \SI{10}{kHz} equivalent bandwidth for robust
RLS; (4) all guards should include hysteresis to prevent chattering.
          % ← ここは本文では \subsection にしておく(下記参照)
\section{Applications}
Potential applications include:
\begin{itemize}
  \item \textbf{Cell patterning}: deposition of living cells with survival rates above 80\%.
  \item \textbf{Protein microarrays}: picoliter deposition of antibodies or DNA spots.
  \item \textbf{Hydrogel 3D printing}: pL-scale deposition followed by UV or thermal curing.
\end{itemize}

The moderate performance of KNN actuators is sufficient for these
applications, where droplet volume control and biocompatibility are
more critical than long-term durability.

\section{Conclusion}
This paper has proposed a Bio-Inkjet (Bio-IJ) architecture based on
bulk KNN actuators as a lead-free alternative to conventional PZT-based
printheads.
By combining multilayer KNN stacks, COF driver ICs, and silicon cavity
integration, the system achieves picoliter-scale droplet generation
under moderate voltages while ensuring material biocompatibility.

Unlike industrial printing, where full PZT compatibility in terms of
maximum $d_{33}$, billion-cycle endurance, and cost efficiency is
required, biomedical printing places emphasis on safety, controlled
droplet volume, and operational reliability over shorter lifetimes.
The proposed approach aligns well with these requirements, providing
sufficient performance for applications such as cell patterning,
protein microarrays, and hydrogel 3D fabrication.

These findings highlight bulk KNN as a practical foundation for
standardizing lead-free Bio-IJ systems in research, clinical, and
educational domains.
Future work will involve experimental validation of droplet formation,
long-term reliability testing under bio-relevant conditions, and system
integration with existing bioprinting workflows.

\textbf{Most importantly, this work emphasizes that in biomedical inkjet
printing, safety and biocompatibility must take precedence over extreme
performance metrics, positioning KNN-based Bio-IJ as a safe and viable
path toward Pb-free bioprinting.}


% 少し余白を追加
\vspace{1em}

% ===== Appendix for math details =====
\appendices
\section{AITL Control Derivations}
% Appendix用: 数式展開を移動
\subsection{Surrogate Sensing and Plant Model}
During a drive pulse $u_k$, the KNN stack current $i(t)$ and 
charge $q(t)=\int i(t) dt$ define surrogate features:
\begin{equation}
y_k \triangleq \alpha_1 q_k^{\text{peak}} + \alpha_2 \Delta q_k + \alpha_3 t_{r,k},
\end{equation}
with calibration coefficients $\alpha_i$.

\subsection{Inner-Loop Dynamics}
We adopt a discrete 2nd-order model:
\begin{equation}
y_{k+1} = a_1(\eta) y_k + a_2(\eta) y_{k-1} + b_0(\eta) u_k + b_1(\eta) u_{k-1} + d_k,
\end{equation}
where viscosity $\eta$ influences $\{a_i, b_i\}$.

\subsection{Recursive Least Squares (RLS)}
The parameter update follows:
\begin{align}
\theta_k &= \theta_{k-1} + K_k (y_k - \phi_k^\top \theta_{k-1}), \\
K_k &= \frac{P_{k-1}\phi_k}{\lambda + \phi_k^\top P_{k-1}\phi_k}, \\
P_k &= \lambda^{-1}(P_{k-1} - K_k \phi_k^\top P_{k-1}),
\end{align}
with forgetting factor $\lambda \in (0,1]$.

\subsection{Adaptive PID Examples}
\textbf{IMC-PID Retuning:}
\begin{equation}
K_p = \frac{\tau + \theta}{K(\lambda + \theta)}, \quad 
K_i = \frac{K_p}{\tau + \theta}, \quad 
K_d = \frac{\tau\theta}{2(\tau + \theta)}.
\end{equation}

\textbf{MIT Rule:}
\begin{equation}
\theta_{k+1} = \theta_k - \Gamma \psi_k e_k, \quad 
\psi_k = \frac{\partial \hat{y}_k}{\partial \theta}.
\end{equation}


% 少し余白を追加
\vspace{1em}

% ===== References =====
\nocite{Saito2004KNN,Dubois2019BioIJ}
\bibliographystyle{IEEEtran}
\bibliography{refs}

% ===== Biography =====
\section*{Author Biography}
\textbf{Shinichi Samizo} received the M.S. degree in Electrical and Electronic
Engineering from Shinshu University, Japan. He worked at Seiko Epson
Corporation as an engineer in semiconductor memory and mixed-signal
device development, and contributed to inkjet MEMS actuators and
PrecisionCore printhead technology. He is currently an independent
semiconductor researcher focusing on process/device education, memory
architecture, and AI system integration. Contact:
\href{mailto:shin3t72@gmail.com}{shin3t72@gmail.com}.
\end{document}
