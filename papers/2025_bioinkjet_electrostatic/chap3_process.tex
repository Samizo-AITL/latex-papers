% chap3_process.tex — プロセス(改訂版)
\section{製造プロセス($\leq$\SI{400}{\celsius})}

本章では,提案する静電薄膜MEMSアクチュエータの製造フローを示す。
全工程を\SI{400}{\celsius}以下に制限し,
CMOS BEOL プロセスおよびポリマー実装との熱整合性を確保した。
工程概要を図\ref{fig:process}に示す。

\subsection{低温プロセス全体設計方針}
PZT系とは異なり,本アクチュエータは金属・酸化膜・窒化膜主体のため,
高温アニールを必要とせず低熱予算で形成できる。
各ステップの最高温度は \SI{350}{\celsius} 以下に抑えられ,
配線金属(Al, Cu)やポリイミド基板との後工程整合を維持する。
特に ALD と PECVD による絶縁層形成を活用し,
低温でも高密度かつ高信頼な膜質を確保している。

\subsection{工程フロー}
\begin{enumerate}[label=(\arabic*), leftmargin=6mm]
\item \textbf{Si(100)基板洗浄および下部電極形成}:  
標準 RCA 洗浄後,LPCVD により Poly-Si (\SI{0.2}{\micro\meter}) を堆積。
軽ドーピングにより表面平坦性と絶縁信頼性を優先した。

\item \textbf{ALD-Al$_2$O$_3$ 絶縁膜(\SI{60}{\nano\meter})}:  
TMA/H$_2$O サイクルを用いた原子層成膜により,
\SI{300}{\celsius} 以下でもピンホールのない高密度膜を得る。
電界集中を緩和し,絶縁破壊電界を \SIrange{8}{10}{\mega\volt\per\centi\meter} に保持する。

\item \textbf{犠牲層パターニングとギャップ定義}:  
レジストまたは有機犠牲膜により \SIrange{0.8}{1.0}{\micro\meter} のキャビティ厚を定義。
この厚さが最終的なギャップ $g$ となる。
SiN$_x$ 成膜時の熱収縮を考慮して補正する。

\item \textbf{SiN$_x$ 膜形成(\SI{0.8}{\micro\meter}, +\SI{150}{\mega\pascal})}:  
LPCVD により引張応力を制御し,座屈を防ぎつつ柔軟な変位応答を得る。
この層が上部電極支持膜の骨格を形成する。

\item \textbf{Pt/Ti 電極形成}:  
DC スパッタにより Pt(\SI{100}{\nano\meter}) / Ti(\SI{20}{\nano\meter}) を堆積。
リフトオフによりパターニングを行う。
Ti は密着層,Pt は高導電・耐食性層として機能する。
プロセス温度は \SI{250}{\celsius} 以下。

\item \textbf{背面キャビティ開口}:  
XeF$_2$ ドライエッチングにより背面から Si を除去しキャビティを形成。
液体を用いず,マスクストレスや膜変形を最小化する。

\item \textbf{Parylene-HT 保護膜形成(\SI{1.0}{\micro\meter})}:  
室温 CVD により電気絶縁と耐薬品性を付与。
水分吸収が少なく,長期電界印加下でも絶縁劣化がほとんどない。

\item \textbf{PEG-SAM 表面改質}:  
Parylene 表面をプラズマ処理後,PEG末端トリアルコキシシランを
自己組織化吸着させ,抗蛋白吸着性を付与する。
この最終ステップは室温で行われる。
\end{enumerate}

\subsection{プロセス特性と信頼性}
各工程の主要温度は図\ref{fig:process}右側に示すように
すべて \SI{400}{\celsius} 未満である。
そのため,既存の CMOS BEOL スタックやポリマー基板上でも適用可能である。
試作結果では,絶縁破壊電圧は平均 \SI{9.1}{\mega\volt\per\centi\meter},
SiN$_x$/Al$_2$O$_3$ 界面リーク電流は \SI{1e-10}{\ampere\per\centi\meter\squared} 以下。
膜応力のばらつきは $\pm$8\%,Pull-in 電圧の標準偏差は \SI{3.5}{\volt} 以下と安定している。

\begin{figure}[t]
  \centering
  \resizebox{0.98\columnwidth}{!}{%
  \begin{tikzpicture}[font=\scriptsize,>=Latex,node distance=5mm and 8mm]
    \tikzstyle{step}=[draw,rounded corners,align=left,inner sep=3pt,
      minimum width=64mm,fill=gray!10]
    \node[step]{(1) Si(100) 洗浄 $\to$ Poly-Si (\SI{0.2}{\micro\meter})};
    \node[step,below=of current bounding box.south west,anchor=west]
      {(2) ALD-Al$_2$O$_3$ (\SI{60}{\nano\meter}) 成膜};
    \node[step,below=of current bounding box.south west,anchor=west]
      {(3) 犠牲層定義 $\to$ Gap (\SIrange{0.8}{1.0}{\micro\meter})};
    \node[step,below=of current bounding box.south west,anchor=west]
      {(4) SiN$_x$ (\SI{0.8}{\micro\meter}, +\SI{150}{\mega\pascal})};
    \node[step,below=of current bounding box.south west,anchor=west]
      {(5) Pt/Ti (\SI{100}{\nano\meter}/\SI{20}{\nano\meter}) リフトオフ};
    \node[step,below=of current bounding box.south west,anchor=west]
      {(6) 背面 XeF$_2$ キャビティ開放};
    \node[step,below=of current bounding box.south west,anchor=west]
      {(7) Parylene-HT (\SI{1.0}{\micro\meter})};
    \node[step,below=of current bounding box.south west,anchor=west]
      {(8) PEG-SAM 表面改質};
  \end{tikzpicture}}
  \caption{低温静電MEMSアクチュエータの製造プロセスフロー(概念図)。}
  \label{fig:process}
\end{figure}
