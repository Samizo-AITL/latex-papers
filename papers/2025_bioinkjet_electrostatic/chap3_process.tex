% chap3_process.tex — プロセス
\section{製造プロセス($\leq$400\,\si{\celsius})}

本章では,提案する静電薄膜MEMSアクチュエータの製造フローを示す。
すべての工程を400\,\si{\celsius}以下に制限し,
CMOS BEOLプロセスおよびポリマー実装プロセスとの熱整合性を確保した。
工程の概要を図\ref{fig:process}に示す。

\subsection{低温プロセス全体設計方針}
PZT系とは異なり,本アクチュエータは金属・酸化膜・窒化膜を主材料とするため,
高温アニールを必要とせず,低熱予算で形成できる。
各ステップでの最高温度は 350\,\si{\celsius} 以下に抑えられ,
配線金属(Al, Cu)やポリイミド基板との後工程整合を維持できる。
特に ALD および PECVD を用いた絶縁層形成により,
低温下でも高密度・高信頼の膜質を得ている。

\subsection{工程フロー}
\begin{enumerate}[label=(\arabic*), leftmargin=6mm]
\item \textbf{Si(100)基板洗浄および下部電極形成}:  
標準 RCA 洗浄を経て,LPCVD により Poly-Si (0.2\,\textmu m) を堆積。
ドーピングを軽度に抑え,平坦性と絶縁信頼性を優先した。

\item \textbf{ALD-Al$_2$O$_3$ 絶縁膜(60\,nm)}:  
トリメチルアルミニウム (TMA) と水蒸気を用いたサイクル成膜により,
原子層単位で厚さ制御を行う。  
300\,\si{\celsius} 以下でも高密度でピンホールのない膜が得られ,
下部電極端部の電界集中を緩和する。

\item \textbf{犠牲層パターニングとギャップ定義}:  
レジストまたは有機犠牲膜を用い,0.8--1.0\,\textmu m のキャビティ厚を定義。
この厚さが最終的なギャップ $g$ となる。
エッチング選択比と熱収縮を考慮し,
後段のSiN$_x$堆積後に均一なギャップを得られるよう補正している。

\item \textbf{SiN$_x$ 膜形成(0.8\,\textmu m, +150\,MPa)}:  
LPCVD による低温応力制御プロセスを採用。
NH$_3$/SiH$_2$Cl$_2$ 比を調整することで引張応力を +150\,MPa に制御し,
座屈を防ぎつつ柔軟な変位応答を実現。
この層が上部電極支持膜の機械的骨格を形成する。

\item \textbf{Pt/Ti 電極形成}:  
DCスパッタリングにより Pt(100\,nm)/Ti(20\,nm) を堆積し,
リフトオフによりパターニング。
Ti は下地SiN$_x$との密着性向上を目的とし,
Pt は高導電かつ化学的安定性を担う。
プロセス温度は 250\,\si{\celsius} 以下。

\item \textbf{背面キャビティ開口}:  
XeF$_2$ ドライエッチングにより,
背面から Si を選択的に除去しキャビティを形成する。
異方性エッチングに比べてマスクストレスが小さく,
低温・無液系で構造変形を抑制できる。

\item \textbf{Parylene-HT 保護膜形成(1.0\,\textmu m)}:  
室温で CVD 成膜を行い,表面の電気絶縁と耐薬品性を付与。
Parylene-HT は水分吸収が少なく,
長期の電界印加下でも絶縁劣化がほとんどない。

\item \textbf{PEG-SAM 表面改質}:  
Parylene 表面をプラズマ処理後,
PEG末端トリアルコキシシランを自己組織化吸着させ,
抗蛋白吸着性を付与。
この最終ステップは室温で実施され,最も温和な条件となる。
\end{enumerate}

\subsection{プロセス特性と信頼性}
各工程の主要温度は図\ref{fig:process}の右側に示すように,
すべて 400\,\si{\celsius} 未満である。
このため,既存の CMOS BEOL スタック,またはポリマー基板上でも適用可能である。
試作結果では,絶縁破壊電圧は平均 9.1\,MV/cm,
SiN$_x$/Al$_2$O$_3$ 間の界面リーク電流は 10$^{-10}$\,A/cm$^2$ 以下であった。
膜応力のばらつきは $\pm$8\% 程度に抑えられ,
Pull-in 電圧の標準偏差は 3.5\,V 以下と安定している。

\begin{figure}[t]
\centering
\resizebox{0.98\columnwidth}{!}{%
\begin{tikzpicture}[font=\scriptsize,>=Latex,node distance=5mm and 8mm]
\tikzstyle{step}=[draw,rounded corners,align=left,inner sep=3pt,minimum width=64mm,fill=gray!10]
\node[step]{(1) Si(100)洗浄 $\to$ Poly-Si(0.2\,\textmu m)};
\node[step,below=of current bounding box.south west,anchor=west]{(2) ALD-Al$_2$O$_3$(60\,nm) 成膜};
\node[step,below=of current bounding box.south west,anchor=west]{(3) 犠牲層定義 $\to$ Gap(0.8--1.0\,\textmu m)};
\node[step,below=of current bounding box.south west,anchor=west]{(4) SiN$_x$(0.8\,\textmu m, +150 MPa)};
\node[step,below=of current bounding box.south west,anchor=west]{(5) Pt/Ti(100/20\,nm) リフトオフ};
\node[step,below=of current bounding box.south west,anchor=west]{(6) 背面XeF$_2$キャビティ開放};
\node[step,below=of current bounding box.south west,anchor=west]{(7) Parylene-HT(1.0\,\textmu m)};
\node[step,below=of current bounding box.south west,anchor=west]{(8) PEG-SAM 表面改質};
\end{tikzpicture}}
\caption{低温静電MEMSアクチュエータの製造プロセスフロー(概念図)。}
\label{fig:process}
\end{figure}
