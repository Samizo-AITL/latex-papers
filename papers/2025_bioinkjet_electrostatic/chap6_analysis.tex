% chap6_analysis.tex — 解析
\section{解析と設計合理性}

本章では,静電・機械・流体の各設計要素を統合し,
自然に導かれる最適設計点($\sim$800\,dpi, 31.75\,µm ピッチ)の
物理的合理性を明らかにする。

\subsection{静電-機械整合(電界と膜応力)}
静電アクチュエータの Pull-in 電圧は
\begin{equation}
V_{\mathrm{PI}} = \sqrt{\frac{8 k g_0^3}{27 \varepsilon_0 \varepsilon_r A}},
\label{eq:vpi}
\end{equation}
で表される。設計値 $A=(25\,\textmu\mathrm{m})^2$, $g_0=0.8\,\textmu$m,
$k=150\,\mathrm{N/m}$ から $V_{\mathrm{PI}}\approx100$\,V が得られる。
駆動電圧 45\,V はこの約 45\% に相当し,安全率は 2.2。
したがって変位は線形域($x/g_0 < 0.15$)内にあり,座屈・吸引崩壊を起こさない。

有限要素解析(FEM)による静電-機械連成解析では,
最大応力 $\sigma_\mathrm{max}=185$\,MPa(SiN$_x$膜内)で,
引張破壊応力($\sim$700\,MPa)に対し十分な余裕を持つ。
応力分布は中央ドーム形状を示し,周縁部に局所集中は見られなかった。

\subsection{電気-流体整合(寄生容量と圧力変換)}
配線および電極周辺の寄生容量 $C_p$ は FEM 電界解析により 5--8\,pF/ch。
アクチュエータ本体容量 $C_\mathrm{a}=30$\,pF と直列に見なすと,
有効容量 $C_\mathrm{eq}=C_\mathrm{a}C_p/(C_\mathrm{a}+C_p)\approx24$\,pF。
これにより電界強度は $E=V/(g_0-x)\approx56$\,MV/m(45\,V時)で,
SiN$_x$/Al$_2$O$_3$ 積層の絶縁耐力(9\,MV/cm)を十分下回る。

膜変位 0.1\,µm はキャビティ体積変化 $\Delta V\approx0.3$\,pL を与え,
式(\ref{eq:pressure})より $\Delta P\approx50$\,kPa。
この圧力は前章で示した穏やかな吐出条件を満たし,
また電気エネルギー $\tfrac{1}{2}CV^2\simeq0.1$\,µJ が
ほぼ全て液体圧縮仕事に転換される効率点にある。

\subsection{熱-生体整合(温度上昇と安全性)}
連続駆動時の平均消費電力 $P_\mathrm{avg}=E f_\mathrm{rep}$ は
0.5--1.0\,mW/ch 程度であり,チップ熱抵抗 150\,K/W として
温度上昇 $\Delta T<0.2$\,K/ch。
1000ノズル並列でも $\Delta T_\mathrm{array}\le2\,^\circ$C に留まる。
生体液(DNA, BSA, 細胞懸濁液)では,この温度上昇は
失活閾値($\sim$5–10 °C)を大幅に下回る。

また,Parylene-HT/PEG-SAM 表面により,
生体分子吸着が 1/10 以下に低減され,
繰返し駆動(10$^6$ cycles)後も接触角変化 $\le$2° と安定。
これにより化学的・熱的・機械的な三重安定性を確保した。

\subsection{設計最密点の導出}
図\ref{fig:tradeoff} に,各設計制約の交点として導かれる
最適ピッチ $p$ の概念を示す。
電気的には,ピッチを縮小すると寄生容量が急増し,
ドライバ電流が増加する。
機械的には,膜間距離が減少し Pull-in 安全率が低下する。
一方,流体的にはノズル干渉が発生しやすくなる。

これらを同時に満たす領域が $p=31$--33\,µm($\sim$800\,dpi)であり,
ここが「最も密で,かつ破綻しない」実用最密構成点である。
本設計点は高密度化の上限を示すものではなく,
\emph{電気・機械・流体のバランスが自律的に決まる帰結値}である。

\begin{figure}[t]
\centering
\resizebox{0.88\columnwidth}{!}{%
\begin{tikzpicture}[>=latex,font=\scriptsize]
\draw[->] (0,0) -- (7,0) node[right]{ピッチ $p$ [µm]};
\draw[->] (0,0) -- (0,4.2) node[above]{制約強度(相対値)};
\draw[blue,thick] plot[smooth] coordinates{(1,4)(2,2.8)(3,2)(4,1.4)(5,1)(6,0.9)(7,0.8)};
\node[blue,anchor=west] at (5.2,1.2){電界・配線};
\draw[red,thick,dashed] plot[smooth] coordinates{(1,0.8)(2,1.1)(3,1.5)(4,2.3)(5,3.2)(6,3.8)(7,4)};
\node[red,anchor=west] at (4.5,3.1){Pull-in / 機械応力};
\draw[orange,thick,dotted] plot[smooth] coordinates{(1,1.2)(2,1.1)(3,1)(4,1.1)(5,1.5)(6,2.2)(7,3)};
\node[orange,anchor=west] at (4.8,2.0){流体干渉};
\draw[very thick,->,gray!70] (3.8,1.2)--(4.2,1.9);
\node[fill=white,draw,gray!60,rounded corners=1pt,anchor=west] at (4.25,1.7){$p^*\simeq31.75$\,µm (800\,dpi)};
\end{tikzpicture}}
\caption{電気・機械・流体制約のトレードオフによる最適ピッチの導出概念図。}
\label{fig:tradeoff}
\end{figure}

\subsection{総合的合理性}
以上の解析より,本設計点は以下の 4 条件を同時に満たす:
\begin{itemize}[leftmargin=6mm]
\item Pull-in 安全率 $\ge2$(電気的安定性)
\item 配線・ドライバの実装容易性(45\,V, 3.3\,V整合)
\item ノズル干渉・滴分離の安定(Re–We–Oh 条件内)
\item 温度上昇 $<2\,^\circ$C(生体安全)
\end{itemize}
これらを同時に成立させた結果として,
\textbf{800 dpi} が「限界値」ではなく「自然な到達点」として決まる。
すなわち本設計は,
精密・効率・倫理性を兼ね備えた
\emph{実用最密構成 (Practically Densest Configuration, PDC)} と位置づけられる。
