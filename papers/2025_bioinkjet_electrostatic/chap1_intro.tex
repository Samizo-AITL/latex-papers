% chap1_intro.tex — 序論
\section{序論:精密技術の再定義とその倫理的転換}
従来の精密技術は,「高出力・高応力・高温」を追求することで,
高速・高密度・高エネルギー効率を実現してきた。
その中心にあったのがPZT圧電アクチュエータであり,
強誘電特性による高エネルギー密度と量産信頼性によって,
インクジェット,マイクロポンプ,マイクロアクチュエータ分野を牽引してきた。
しかし,この構造的利点は同時に,
(i) Pb含有による環境負荷,  
(ii) 高温焼成($\geq$600\,\si{\celsius})によるプロセス整合性の制約,  
(iii) 大電流駆動に伴う熱的劣化,  
という3つの根本課題を抱えている。

一方で,近年のバイオマテリアル・創薬・再生医療分野では,
「精密」とは\emph{出力を大きくすることではなく,壊さないこと}を意味する。
すなわち,生体分子・細胞・タンパク質といった脆弱対象を扱う領域において,
\emph{力の最小化,温度上昇の抑制,電界ストレスの低減}こそが
新しい精密の要件である。
この文脈において,静電駆動型の薄膜MEMSアクチュエータは,
従来の圧電駆動とは異なる価値軸を提示する。

静電アクチュエータは容量性駆動に基づくため,
(1) \textbf{低電流・低発熱}であること,  
(2) \textbf{Pbフリー材料}(SiN$_x$, Al$_2$O$_3$)が適用可能であること,  
(3) \textbf{低温プロセス}($\leq$400\,\si{\celsius})でCMOS/ポリマー系との整合が得られること,  
という特長を併せ持つ。
さらに,構造設計の自由度が高く,
電界強度,膜応力,ギャップ寸法を適切に協調させることで,
低電圧($\leq$45\,V)でも実用的な変位を得ることができる。

本研究は,これらの特徴を総合的に活かし,
\textbf{Pbフリー・非熱・非接触・生体適合}という新たな設計哲学のもとで,
Bioインクジェットヘッドを再構築することを目的とする。
特に,電場・膜変位・寄生容量・熱・流体の\emph{五領域同時最適化}を通じて,
\textbf{約800\,dpi(31.75\,µmピッチ)}が
電気・機械・流体的整合から自然に導かれる
合理設計点であることを明らかにする。
この枠組みは,単なるデバイス開発にとどまらず,
\emph{精密=節度ある制御}としての
倫理的精密(Ethical Precision)への転換を実装するものである。
