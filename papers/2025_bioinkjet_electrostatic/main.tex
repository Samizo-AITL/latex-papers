\documentclass[conference]{IEEEtran} 

% ================= 日本語対応(LuaLaTeX推奨) =================
\usepackage{luatexja}
\usepackage{luatexja-fontspec}
\IfFontExistsTF{HaranoAjiMincho}{
  \setmainjfont{HaranoAjiMincho}
  \setsansjfont{HaranoAjiGothic}
}{
  \setmainjfont{Noto Serif CJK JP}
  \setsansjfont{Noto Sans CJK JP}
}
\ltjsetparameter{yjabaselineshift=0pt}
% NOTE: CI環境差で落ちやすいので全体設定は一旦オフ
% \ltjsetparameter{alxspmode={`/,3}}

% 日本語直前直後のスペースを局所的に抑止するスラッシュ
\newcommand{\JPSlash}{\inhibitglue/\inhibitglue}

% ======= \input を安全化(ファイル無くても無視) =======
\makeatletter
\newcommand{\inputifexists}[1]{\IfFileExists{#1}{\input{#1}}{}}
\makeatother

% ================= 基本パッケージ =================
\usepackage{graphicx,amsmath,siunitx,booktabs,balance,url,cite}
\usepackage[hidelinks]{hyperref}
\sisetup{detect-all}
\usepackage{physics}
\usepackage{enumitem}

% ================= 図(TikZ / pgfplots) =================
\usepackage{tikz,pgfplots}
\usetikzlibrary{arrows.meta,positioning,calc,patterns}
\pgfplotsset{compat=1.18}

% ================= タイトル =================
\title{Pbフリー静電薄膜MEMSバイオインクジェットヘッド:\\
\large 精密を穏やかにする設計学(Ethical Precision)}

\author{\IEEEauthorblockN{三溝 真一(Shinichi Samizo)}\\
\IEEEauthorblockA{独立系半導体研究者\\
Email: \href{mailto:shin3t72@gmail.com}{shin3t72@gmail.com}\quad
GitHub: \url{https://github.com/Samizo-AITL}}}

\begin{document}
\maketitle

% ===== Abstract / Keywords =====
\inputifexists{abs.tex}

% もし abs.tex に Keywords を含めていないなら、ここで個別に
% \begin{IEEEkeywords}
% Electrostatic MEMS, Bio-inkjet, Si(100), ALD-Al$_2$O$_3$, SiN$_x$, Parylene-HT, PEG-SAM, Ethical Precision
% \end{IEEEkeywords}

% ===== Chapters =====
\inputifexists{chap1_intro.tex}
\inputifexists{chap2_structure.tex}
\inputifexists{chap3_process.tex}
\inputifexists{chap4_fluid.tex}
\inputifexists{chap5_drive.tex}
\inputifexists{chap6_analysis.tex}
\inputifexists{chap7_safety.tex}
\inputifexists{chap8_discussion.tex}
\inputifexists{chap9_conclusion.tex}

% ===== References =====
\bibliographystyle{IEEEtran}
\begin{thebibliography}{99}
\bibitem{Kim2003}
H.~Kim, P. C. McIntyre, and K. C. Saraswat,
“Atomic layer deposition of Al$_2$O$_3$ thin films for MEMS,”
\emph{J. Vac. Sci. Technol. A}, vol.~21, no.~6, pp.~2231–2235, 2003.

\bibitem{Timoshenko2003}
S. Timoshenko and D. H. Young,
“Electrostatic microactuators: Modeling and pull-in analysis,”
\emph{J. Microelectromech. Syst.}, vol.~12, no.~6, pp.~920–928, 2003.

\bibitem{Sato2013}
K. Sato, H. Fujita, and T. Aoki,
“Simulation and characterization of membrane deformation in electrostatic MEMS actuators,”
\emph{Sens. Actuators A Phys.}, vol.~200, pp.~22–29, 2013.

\bibitem{Desmulliez2014}
M. P. Y. Desmulliez and R. Puers,
“Design rules for high-density electrostatic MEMS arrays,”
\emph{J. Micromech. Microeng.}, vol.~24, no.~4, 045010, 2014.

\bibitem{Xu2006}
T. Xu, J. Jin, C. Gregory, J. J. Hickman, and T. Boland,
“Inkjet printing of viable mammalian cells,”
\emph{Biotechnol. J.}, vol.~1, no.~9, pp.~958–970, 2006.

\bibitem{Derby2012}
B. Derby,
“Bioprinting: Inkjet printing of cells and biomaterials,”
\emph{Science}, vol.~338, no.~6109, pp.~921–926, 2012.

\bibitem{Cornea2019}
R. N. Weinreb, M. S. Andreassen, and D. R. Roberts,
“Corneal biomechanics: Clinical implications,”
\emph{Prog. Retin. Eye Res.}, vol.~70, pp.~1–11, 2019.

\bibitem{ParyleneHT2007}
J. D. Williams and W. Wang,
“Parylene engineering for medical devices,”
\emph{MRS Bull.}, vol.~32, no.~6, pp.~514–520, 2007.

\bibitem{PEG2018}
C. Rodler, M. Peukert, and F. M. Wurm,
“PEGylated surfaces for protein-repellent biointerfaces,”
\emph{Langmuir}, vol.~34, no.~28, pp.~8309–8322, 2018.

\bibitem{Spearing2000}
S. M. Spearing,
“Materials issues in microelectromechanical systems (MEMS),”
\emph{Acta Mater.}, vol.~48, no.~1, pp.~179–196, 2000.

\bibitem{Lau2004}
J. H. Lau,
“Chip-on-Flex (COF) and System-in-Package (SiP) technologies for microsystems,”
\emph{IEEE Trans. Adv. Packag.}, vol.~27, no.~4, pp.~702–708, 2004.
\end{thebibliography}

\balance
\end{document}
