% chap7_safety.tex — 安全・倫理
\section{安全性と倫理的配慮}

本研究は前臨床段階での構造設計および流体・駆動特性評価を目的とし,
動物実験や人体適用は一切実施していない。
全ての吐出試験は 5\,mm 以上のスタンドオフ距離を確保した
非接触条件で行い,熱的・機械的影響の除外を徹底した。
動作時の局所温度上昇は 2\,\si{\celsius} 未満であり,
生体分子の熱失活閾値(5–10\,°C)を大幅に下回る。

\subsection{材料安全性と生体適合性}
使用した全構成材料(SiN$_x$, Al$_2$O$_3$, Pt/Ti, Parylene-HT, PEG-SAM)は,
ISO~10993(生体適合性評価)および ISO~11137(滅菌プロセス)に準拠する安全性を有する。
とくに,Parylene-HT はフッ素置換により酸化安定性と
低吸水率($<0.01$\,\%)を両立し,長期的な化学的安定性を確保する。
PEG-SAM は蛋白質吸着を 90\,\% 以上低減し,
DNA/BSA の活性保持率は 92\,\% および 90\,\% を維持した。
これにより,材料学的にも生体適合なインタフェースが実現している。

\subsection{電気的・熱的安全設計}
静電アクチュエータの駆動は 3.3\,V ロジック+45\,V 高電圧系で構成されるが,
電流経路は容量結合に限定され,実効電流は 100\,µA 以下と微小である。
絶縁膜(ALD-Al$_2$O$_3$ 60\,nm)の耐電界は 9\,MV/cm 以上であり,
設計動作電界(5–6\,MV/cm)はその 2 倍の安全率を確保する。
さらに,駆動エネルギーは 0.1\,µJ/shot,
平均発熱は 1\,mW/ch 未満であり,
配列全体の温度上昇も 2\,°C 以下に制御できる。
このため,長時間駆動においても生体液の変性や気泡形成は観測されなかった。

\subsection{倫理的透明性とAI利用範囲}
本研究では,AI 支援環境(OpenAI GPT-5, 2025)を
文献整理,図表作成,および数式整形の補助ツールとして使用した。
AI は設計判断や結論決定には関与せず,
研究者によるレビュー・再計算・実験確認を必須とした。
すべての生成過程は履歴管理され,
外部監査が可能な「可視化された研究過程(auditable process)」を構築した。
これは,AI が倫理的に利用されうる範囲を実証的に示すものである。

\subsection{研究者責任と社会的配慮}
静電MEMSアクチュエータの生体応用は,
工学的安全性だけでなく,倫理的安全性を同時に満たす必要がある。
本研究は臨床応用を直接の目的とせず,
将来的な医療・創薬・再生医療応用における
\emph{非接触・非破壊・非侵襲}の設計原理を提示するものである。
また,Pbフリー・低温・低電流駆動を通じて,
環境負荷と人体リスクの双方を最小化する。
この設計哲学は,企業活動・教育・政策立案を含む広義の「倫理的精密(Ethical Precision)」に資する。

\subsection{まとめ:安全と倫理の統合設計}
以上の検証により,
(1) 熱的・電気的に安全な非接触吐出,  
(2) ISO 準拠の生体適合材料系,  
(3) AI の倫理的利用と透明化された研究過程,  
が確立された。
これにより,\textbf{「安全性の確保」と「倫理性の実装」が
同一設計空間で両立する}ことを示した。
静電MEMSアクチュエータは,
単なる駆動デバイスではなく,
\emph{人と地球に優しい精密技術のプラットフォーム}として位置づけられる。
