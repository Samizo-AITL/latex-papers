% chap9_conclusion.tex — 結論
\section{結論}

本研究では,Pbフリーかつ低温プロセス整合な
\textbf{静電薄膜MEMSアクチュエータ}を提案し,
従来の高出力指向から一歩離れた
「\emph{穏やかな力による精密制御}」の実証を行った。
Si(100)基板上に形成した SiN$_x$/ALD-Al$_2$O$_3$/Pt/Ti 積層構造により,
45\,V級の現実的電装で膜変位0.10–0.12\,µm,
圧力 $\sim$50\,kPa,吐出速度2–5\,m/s,
エネルギー0.1\,µJ/shotという穏やかな駆動を達成した。
Parylene-HT/PEG-SAM 表面改質により,
蛋白吸着を抑制し DNA/BSA の活性保持率 $\ge$90\% を確認した。

また,静電-機械-流体連成解析により,
Pull-in 安全率,寄生容量,ノズル干渉,発熱を同時に満たす
自然な設計点として \textbf{$\sim$800\,dpi(31.75\,µm ピッチ)} が導かれることを示した。
これは高密度化の限界値ではなく,
\emph{電気・機械・流体の調和によって自ずと定まる最適構成}である。

本研究の意義は,単なるデバイス性能の向上にとどまらず,
次の三点に要約される:
\begin{enumerate}[label=(\arabic*),leftmargin=6mm]
\item Pbフリー・低温・低電流という\textbf{環境・生体整合性の確立}  
\item FEM/実測/設計理論による\textbf{穏やかな精密(Gentle Precision)の実装}  
\item 精密技術における\textbf{倫理的価値軸(Ethical Precision)の提示}  
\end{enumerate}

この成果は,
PZT方式が築いた「力の精密(Force Precision)」に対し,
静電MEMSによる「節度の精密(Moderated Precision)」を対置するものであり,
両者の連携によって「\textbf{Eco-Precision for Life}」という
新しい精密工学の方向性を拓く。

\subsection*{今後の展望}
今後は,
(1) 多ノズル並列化と波形最適化による高スループット化,  
(2) MEMS-SoC統合による自律・省電力駆動,  
(3) 生体適合材料や微細流路とのハイブリッド化,  
(4) AI支援設計による感応的制御(adaptive waveform tuning),  
を通じて,\textbf{倫理的精密(Ethical Precision)の社会実装}を加速する。

静電薄膜MEMSアクチュエータは,
単なるマイクロデバイスではなく,
\emph{人と地球の双方に優しい精密の象徴}である。
本稿で示した設計哲学と方法論は,
将来の医療・環境・教育分野において,
「壊さない技術」から「護る精密」へと
価値基準を転換する一つの出発点となる。
