\begin{figure*}[t]
\centering
\begin{tikzpicture}[%
  font=\small,
  node distance=9mm and 11mm,
  box/.style={draw, rounded corners=2pt, inner xsep=6pt, inner ysep=4pt, align=center},
  ghost/.style={draw, dashed, rounded corners=2pt, inner xsep=6pt, inner ysep=4pt, align=center},
  >={Latex[length=2mm]}
]

% ----------- ノード配置 -----------
% LLM
\node[ghost] (llm) at (0,2.6) {LLM Analyzer (Next):\\retunes $K_p,K_i,K_d$; regenerates FSM rules};

% 上段: Config, PID, FSM
\node[box] (yaml) at (-4.6,0.6) {YAML Config\\FW / CSR};
\node[box] (pid)  at (-0.6,0.6) {PID Controller};
\node[box] (fsm)  [right=18mm of pid] {FSM Supervisor};

% 中段: Telemetry
\node[box] (tele) [below=12mm of pid] {Telemetry:\\Delay / Temp / Jitter};

% 下段: FEM, S-param
\node[box] (fem)  [below=10mm of tele] {FEM\\(Thermal / Stress)};
\node[box] (spar) [below=10mm of fem] {S-parameters (SI/EMI)};

% 左列: EDA
\node[box] (eda) [left=18mm of tele] {EDA Flow\\(Synth / P\&R / STA)};

% ----------- 矢印 -----------
% LLM → PID/FSM (破線)
\draw[dashed,->] (llm.south) |- ($(pid.north)+(0,0.2)$) -- (pid.north);
\draw[dashed,->] (llm.south) |- ($(fsm.north)+(0,0.2)$) -- (fsm.north);

% ラベル(矢印とは独立)
\node[font=\scriptsize] at ($(pid.north)!0.5!(fsm.north)+(0,0.55)$)
  {gain \& threshold adapt};

% YAML → PID
\draw[->] (yaml.east) -- (pid.west);

% FSM → EDA(制約線)
\draw[->] (fsm.west) -- ++(-0.8,0) |- (eda.north);
\node[font=\scriptsize,align=center] at ($(fsm.west)!0.5!(eda.north)+(0,0.25)$)
  {constraints};

% Telemetry → PID
\draw[->] (tele.north) -- (pid.south);

% FEM → Telemetry
\draw[->] (fem.north) -- (tele.south);

% S-param → FEM
\draw[->] (spar.north) -- (fem.south);

% EDA → Telemetry(実測値)
\draw[->] (eda.east) -- ++(0.8,0) |- (tele.west);

\end{tikzpicture}
\caption{System overview (two-column): runtime telemetry $\rightarrow$ compact physics models $\rightarrow$ PID/FSM runtime control $\rightarrow$ actuators, with EDA sign-off integration; an optional LLM (Next) provides adaptive gain retuning and FSM rule regeneration. All arrows are routed along edges/gaps and do not overlap node borders.}
\label{fig:system}
\end{figure*}
