\documentclass[conference]{IEEEtran}

% ====== Packages ======
\usepackage{amsmath,amssymb}
\usepackage{graphicx}
\usepackage{booktabs}
\usepackage{hyperref}
\usepackage{tikz}
\usetikzlibrary{arrows.meta,positioning,fit,calc}

% ====== Title & Author ======
\title{SystemDK with AITL: Physics-Aware Runtime\\
DTCO via PID, FSM, and LLM Integration}

\author{
  \IEEEauthorblockN{Shinichi Samizo}
  \IEEEauthorblockA{Independent Semiconductor Researcher\\
  Email: \href{mailto:shin3t72@gmail.com}{shin3t72@gmail.com}}
}

\begin{document}
\maketitle

% ====== Abstract ======
\begin{abstract}
This paper presents SystemDK with AITL, a framework that extends conventional Design--Technology Co-Optimization (DTCO) by embedding control-theoretic loops directly into EDA flows. Compact PID controllers and FSM supervisors stabilize runtime variations (RC delay, thermal coupling, EMI/EMC disturbances). Future extensions (AITL Next) integrate LLM analyzers for adaptive PID retuning and FSM rule regeneration. The framework incorporates FEM analysis and S-parameter measurements into synthesis, P\&R, and STA, ensuring physics-aware closure. Simulations demonstrate order-of-magnitude improvements in timing stability, thermal robustness, and jitter suppression.
\end{abstract}

\begin{IEEEkeywords}
DTCO, CFET, PID control, FSM, LLM, EMI/EMC, thermal management, timing jitter, EDA.
\end{IEEEkeywords}

% ====== Introduction ======
\section{Introduction}
Scaling to sub-2nm nodes and CFET integration introduces critical runtime effects:
\begin{itemize}
  \item RC delay variation due to interconnect scaling,
  \item Vertical thermal coupling in 3D-ICs,
  \item Stress-induced $V_{\text{th}}$ shifts,
  \item EMI/EMC noise degrading jitter and reliability.
\end{itemize}
Traditional DTCO applies static guardbands, leading to inefficiency. SystemDK with AITL proposes embedding runtime control (PID+FSM) and, in the future, LLM-based adaptation.

% ====== Proposed Framework ======
\section{Proposed Framework}

\subsection{AITL Base}
PID compensates delay/thermal/voltage variations; FSM supervises modes and safety thresholds.

\subsection{AITL Next}
A lightweight LLM analyzes EDA logs, retunes PID gains ($K_p,K_i,K_d$), and regenerates FSM rules.

% ====== Fig.1 (TikZ System Overview) ======
\begin{figure*}[t]
  \centering
  \tikzset{
    blk/.style={draw,rounded corners=2pt,minimum height=8mm,inner xsep=4mm,align=center},
    lab/.style={font=\footnotesize,inner sep=1pt},
    line/.style={-Latex,thick},
    dashedbox/.style={draw,dashed,rounded corners=2pt,inner sep=4pt}
  }

  \begin{tikzpicture}[node distance=7mm and 12mm]
    % Top control chain
    \node[blk]                             (yaml) {YAML Config\\FW / CSR};
    \node[blk, right=20mm of yaml]         (pid)  {PID Controller};
    \node[blk, right=20mm of pid]          (fsm)  {FSM Supervisor};

    % Telemetry bridge
    \node[blk, below=12mm of pid]          (tele) {Telemetry:\\Delay / Temp / Jitter};

    % Physics models
    \node[blk, below=12mm of tele]         (fem)  {FEM\\(Thermal / Stress)};
    \node[blk, below=10mm of fem]          (spar) {S-parameters (SI/EMI)};

    % EDA flow (left column)
    \node[blk, left=28mm of tele]          (eda)  {EDA Flow\\(Synth / P\&R / STA)};

    % Future LLM (dashed frame + note)
    \node[dashedbox, above=10mm of pid, minimum width=76mm, minimum height=10mm] (llm) {};
    \node[lab] at (llm) {LLM Analyzer (Next): retunes $K_p,K_i,K_d$; regenerates FSM rules};

    % -------- Connections (no block-crossing) --------
    % Top chain
    \draw[line] (yaml.east) -- (pid.west);
    \draw[line] (pid.east) -- node[lab,above,near end]{gain \& threshold adapt} (fsm.west);

    % EDA constraints to control blocks (via left corridor)
    \coordinate (corrL) at ([xshift=-8mm]tele.west);
    \draw[line] (eda.east) -- (corrL) |- (yaml.west);
    \draw[line] (corrL) |- (fsm.east);
    \node[lab,anchor=east] at (corrL) {constraints};

    % Telemetry upward path (from physics -> PID)
    \draw[line] (fem.north)  -- (tele.south);
    \draw[line] (tele.north) -- (pid.south);

    % Physics chain (S-params -> FEM)
    \draw[line] (spar.north) -- (fem.south);

    % EDA -> Physics (route horizontally then down)
    \draw[line] (eda.east) -- ++(8mm,0) |- (spar.west);
  \end{tikzpicture}

  \caption{System overview (two-column): runtime telemetry $\rightarrow$ compact physics models $\rightarrow$ PID/FSM runtime control $\rightarrow$ actuators, with EDA sign-off integration; an optional LLM (Next) provides adaptive gain retuning and FSM rule regeneration.}
  \label{fig:system_overview}
\end{figure*}

% ====== Analytical Models ======
\section{Analytical Models and EDA Mapping}
\subsection{RC Delay Model}
\begin{equation}
t_{pd}(T,\sigma,f) = R_0 \big(1+\alpha_T (T-T_0) + \alpha_\sigma \sigma \big) C(f) + \Delta E_{MI}(f).
\end{equation}
Mapped to STA path-delay constraints.

\subsection{Thermal Coupling}
\begin{equation}
C_{\text{th}} \frac{dT}{dt} + \frac{T - T_{\text{amb}}}{R_{\text{th}}} = P_{\text{chip}}(t).
\end{equation}
Mapped to P\&R thermal placement constraints.

% ====== Simulation Results ======
\section{Simulation Results with EDA Implications}
\subsection{RC Delay Compensation}
PID reduces normalized RC delay variation versus uncontrolled case.

\subsection{Thermal Response Control}
FSM-supervised PID stabilizes thermal $\Delta T$ under dynamic load.

\subsection{EMI Jitter Suppression}
Runtime control suppresses jitter induced by injected EMI.

\subsection{FEM Analysis}
FEM maps capture thermal hotspots and TSV-induced stress distributions.

\subsection{S-Parameter Analysis}
$S_{11}/S_{21}$ measurements validate SI/EMI resilience with PID+FSM.

% ====== Implementation ======
\section{Implementation PoC}
RTL excerpts (PID controller), FSM transitions, and YAML configuration were implemented in Verilog and integrated with APB/AXI-Lite CSRs.

% ====== Discussion ======
\section{Discussion}
\begin{itemize}
  \item Guardbands $\rightarrow$ adaptive loops,
  \item Static sign-off $\rightarrow$ dynamic runtime closure,
  \item Reliability $\rightarrow$ cross-domain resilience (delay, thermal, stress, EMI).
\end{itemize}

% ====== Conclusion ======
\section{Conclusion and Future Work}
AITL Base (PID+FSM) establishes runtime stabilization.  
AITL Next will integrate lightweight LLM modules for real-time EDA log analysis and control redesign.  
Industrial relevance: prototype chips, EDA tool collaboration, and AI-driven DTCO.

% ====== References ======
\begin{thebibliography}{1}
\bibitem{yakimets}
D. Yakimets \emph{et al.}, ``Challenges for cfet integration,'' in \emph{Proc. IEEE IEDM}, 2020, pp. 11.9.1--11.9.4.
\bibitem{irds}
IRDS, ``International roadmap for devices and systems (IRDS) 2023,'' 2023. [Online]. Available: \url{https://irds.ieee.org/roadmap-2023}
\bibitem{franklin}
G. Franklin, J. D. Powell, and A. Emami-Naeini, \emph{Feedback Control of Dynamic Systems}, 7th ed. Pearson, 2015.
\bibitem{khalil}
H. K. Khalil, \emph{Nonlinear Systems}. Prentice Hall, 2002.
\bibitem{anderson}
B. D. O. Anderson and J. B. Moore, \emph{Optimal Control: Linear Quadratic Methods}. Dover, 2007.
\bibitem{iec}
IEC, ``Electromagnetic Compatibility (EMC) -- Part 4: Testing and Measurement Techniques,'' IEC Std. 61000-4, 2019.
\end{thebibliography}

% ====== Author Biography ======
\begin{IEEEbiographynophoto}{Shinichi Samizo}
received the M.S. degree in Electrical and Electronic Engineering from Shinshu University, Japan.  
He worked at Seiko Epson Corporation as an engineer in semiconductor memory and mixed-signal device development, and also contributed to inkjet MEMS actuators and PrecisionCore printhead technology.  
He is currently an independent semiconductor researcher focusing on process/device education, memory architecture, and AI system integration.  
\textbf{Contact:} \href{mailto:shin3t72@gmail.com}{shin3t72@gmail.com}, \href{https://github.com/Samizo-AITL}{Samizo-AITL}
\end{IEEEbiographynophoto}

\end{document}
