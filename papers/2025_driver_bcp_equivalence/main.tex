% =====================================================
% μTFP Driver IC BCP論文(Process Equivalence)
% =====================================================
\documentclass[conference]{IEEEtran}
\usepackage{graphicx}
\usepackage{url}
\usepackage{hyperref}
\usepackage{amsmath}
\usepackage{multirow}

\begin{document}

% -----------------------------------------------------
% タイトル・著者情報
% -----------------------------------------------------
\title{%
μTFPヘッド用ドライバICのBCP対応:\\
プロセス同等化による二拠点生産体制の確立\\[2mm]
\textit{Business Continuity Implementation for μTFP Printhead Driver IC\\
via Process Equivalence at Sakata and Lapis Semiconductor}
}

\author{%
三溝 真一(Shinichi Samizo)\\
独立系半導体研究者(元セイコーエプソン株式会社)\\
Email: \href{mailto:shin3t72@gmail.com}{shin3t72@gmail.com} \\
GitHub: \url{https://github.com/Samizo-AITL}
}

\maketitle

% -----------------------------------------------------
% Abstract
% -----------------------------------------------------
\begin{abstract}
本研究は,東日本大震災を契機に,μTFPプリントヘッド用
ドライバICの生産リスク低減を目的として実施したBCP対応事例を報告する。
ファンドリとしてラピスセミコンダクタを追加し,
0.35 µm CMOS(3.3V/45V WSiゲート)プロセスを対象に,
従来の「パラメータ合わせ」ではなく「プロセス同等化」により
酒田工場と同一仕様を再現した。
その結果,電気特性・吐出特性・EMCいずれも酒田品と同等であり,
ヘッド側設計変更を要しない完全互換体制を確立した。
本事例は,設計資産を保護しつつ冗長化を実現する
「Process-Identical BCP」の有効性を示すものである。
\end{abstract}

\begin{IEEEkeywords}
BCP, μTFP, Driver IC, Process Equivalence, Lapis Semiconductor, Sakata Factory, Reliability, Inkjet Head
\end{IEEEkeywords}

% -----------------------------------------------------
\section{序論}
2011年の東日本大震災は,半導体製造拠点の地理的集中リスクを顕在化させた。
エプソン酒田工場ではμTFPヘッド用ドライバICを一拠点で生産していたが,
事業継続性の観点から,宮城県のラピスセミコンダクタに
代替パスを構築することが急務となった。

\section{BCP対応方針}
本BCP対応では,ファンドリ追加に際し次の2方式を検討した:
\begin{enumerate}
\item ラピス側のプロセスパラメータに合わせて再設計する(Parameter Matching)
\item 酒田工場のマスクとプロセス条件をそのままラピスで再現する(Process Matching)
\end{enumerate}
短期的には前者が容易であるが,ヘッド電特・COFコード・ファームウェアの再評価が必要となる。
本研究では,ヘッド仕様維持を最優先し,後者の「プロセス同等化」を採用した。

\section{プロセス同等化手法}
\subsection{対象デバイス}
0.35 µm CMOS, 3.3V/45V WSiゲート構造を有する高耐圧ドライバIC。
酒田マスクをラピス製造ラインに流用し,
熱酸化・RTA・金属膜堆積条件を逐次マッピングした。

\subsection{立上げフロー}
\begin{enumerate}
\item 酒田レシピの提供と装置パラメータ変換
\item 試作ロット評価:膜厚・CD・シート抵抗比較
\item Parametric Test:酒田ロットとの統計照合(3σ内合格)
\item 信頼性評価:HTOL・TDDB・EM試験
\end{enumerate}

\section{評価結果}
\subsection{電気特性}
出力ドライバのIV特性、ゲートリーク電流ともに
酒田品との偏差は±2\%以内であった。

\subsection{ヘッド特性およびEMC}
ラピス製ICを搭載したプリントヘッドで吐出試験を実施。
波形・速度・安定性いずれも同等であり、
システムレベルEMC試験でも差異を確認しなかった。

\begin{figure}[htbp]
\centering
\includegraphics[width=0.45\textwidth]{figs/fig1_bcp_options.pdf}
\caption{BCP対応方式の比較(Parameter Matching vs Process Matching)}
\end{figure}

\begin{table}[htbp]
\centering
\caption{ラピス・酒田プロセスの主要特性比較}
\begin{tabular}{lcc}
\hline
項目 & 酒田 & ラピス \\
\hline
ゲート酸化膜厚 (Å) & 800 & 805 \\
WSi抵抗値 (Ω/□) & 2.3 & 2.4 \\
リーク電流 (A/cm$^2$) & $<1\times10^{-9}$ & $<1\times10^{-9}$ \\
臨界電圧Vbd (V) & 48.2 & 47.9 \\
\hline
\end{tabular}
\end{table}

\section{品質保証とトレーサビリティ}
製造:ラピスセミコンダクタ,
ウエハテスト:酒田工場,
出荷保証:エプソン本体。
COFコードは共通運用としつつ,
ロット別管理によるトレーサビリティを確保した。

\section{考察}
プロセス同等化は、短期的には高工数・長期間を要するが、
製品側再設計を不要とし、結果的に全体工数を最小化する。
本方式は「設計資産保護型BCP」として有効であり、
今後のMEMS・混載IC製造にも応用可能である。

\section{結論}
本報では,μTFPドライバICのBCP対応として,
酒田工場とラピスセミコンダクタにおける
同一マスク・同一プロセス立上げを実現した。
電気特性・ヘッド性能とも完全同等であり,
高信頼な二拠点供給体制を確立した。
この手法は「Design Preservation through Process Reproduction」
の有効事例として位置づけられる。

% -----------------------------------------------------
\section*{謝辞}
本研究の実施にあたり,酒田工場,宮城ラピスセミコンダクタの各位に感謝する。

% -----------------------------------------------------
\section*{参考文献}
\begin{thebibliography}{99}
\bibitem{Samizo2024_uTEP}
S.~Samizo, ``uTEPヘッド用ドライバICにおけるウエハ酸素濃度変動起因の品質問題対応,'' 2024.

\bibitem{Samizo2023_TFP}
S.~Samizo, ``薄膜PZTアクチュエータにおける振動板クラックと端部焼損の原因解析,'' 2023.

\bibitem{Samizo2022_PZT}
S.~Samizo, ``薄膜PZT技術の系譜,'' 2022.
\end{thebibliography}

% -----------------------------------------------------
% 著者略歴
% -----------------------------------------------------
\section*{著者略歴}
\textbf{三溝 真一}(Shinichi Samizo)は、信州大学大学院 工学系研究科 電気電子工学専攻にて修士号を取得した。  
その後、セイコーエプソン株式会社に勤務し、半導体ロジック/メモリ/高耐圧インテグレーション、そして、インクジェット薄膜ピエゾアクチュエータ及びPrecisionCoreプリントヘッドの製品化に従事した。  
現在は独立系半導体研究者として、プロセス/デバイス教育、メモリアーキテクチャ、AIシステム統合などに取り組んでいる。  
連絡先: \href{mailto:shin3t72@gmail.com}{shin3t72@gmail.com}.

\end{document}
