% =====================================================
% μTFP Driver IC BCP論文(Process Porting / Process Equivalence 強調版)
% =====================================================
\documentclass[conference]{IEEEtran}

% ---------- LuaLaTeX 日本語・μ対応(決定版) ----------
\usepackage{iftex}
\ifLuaTeX
  \usepackage{luatexja}
  \usepackage{luatexja-fontspec}
  \setmainjfont{IPAexMincho}
  \setsansjfont{IPAexGothic}
  \usepackage{fontspec}
  \setmainfont{Latin Modern Roman}
  \setsansfont{Latin Modern Sans}
  \ltjsetparameter{jacharrange={-2,-3}}
\else
  \usepackage[utf8]{inputenc}
  \usepackage[T1]{fontenc}
\fi

% --- Unicode安全化 ---
\usepackage{newunicodechar}
\newcommand{\micro}{\ensuremath{\mu}}
\newunicodechar{μ}{\micro}
\newunicodechar{¥}{\textbackslash}

% ---------- 便利パッケージ ----------
\usepackage{graphicx}
\usepackage[table]{xcolor}
\usepackage{array,tabularx,booktabs,threeparttable,makecell,multirow}
\usepackage{amsmath,amssymb,mathtools,bm}
\usepackage{siunitx}
\sisetup{mode=match, per-mode=symbol, range-phrase=--, group-separator={,}}
\DeclareSIUnit{\sq}{\text{\ensuremath{\square}}}

\usepackage[caption=false,font=footnotesize]{subfig}
\usepackage{cite}
\usepackage{tikz}
\usetikzlibrary{arrows.meta,positioning,fit,calc,shapes}
\usepackage{pgfplots}
\pgfplotsset{compat=1.18}
\usepackage[hidelinks]{hyperref}
\usepackage[capitalise,noabbrev]{cleveref}

\renewcommand{\figurename}{Fig.}
\renewcommand{\tablename}{Table}

\makeatletter
\def\@makefnmark{\hbox{\textsuperscript{\normalfont\@thefnmark}}}
\makeatother

% ---------- 本文 ----------
\begin{document}

\title{%
μTFPヘッド用ドライバICのBCP対応:\\
\textbf{プロセス移植型による完全互換二拠点生産体制の確立}
}

\author{%
三溝 真一(Shinichi Samizo)\\
独立系半導体研究者(元セイコーエプソン株式会社)\\
Email: \href{mailto:shin3t72@gmail.com}{shin3t72@gmail.com} \\
GitHub: \url{https://github.com/Samizo-AITL}
}

\maketitle

\begin{abstract}
\textbf{和文概要:}
本研究は,東日本大震災を契機に,μTFPプリントヘッド用ドライバICの生産リスク低減を目的として実施したBCP(Business Continuity Planning)対応事例を報告する。
従来の\textbf{再設計型(Re-Design-Based / Parameter Matching)}ではなく,
\textbf{プロセス移植型(Process-Porting-Based / Process Equivalence)}を採用し,
酒田工場と同一仕様をラピスライン上に再現した。
製造をラピスで行い,ウエハテスト(EWS)を酒田で実施する\emph{製造分散・保証集中}の体制により,
設計資産を改変せずに完全互換を達成した。
この事例は,設計資産保護と供給リスク分散を両立する「Process-Identical BCP」アプローチの有効性を実証するものである。

\medskip
\textbf{Abstract:}
This paper presents a Business Continuity Planning (BCP) implementation for the μTFP printhead driver IC, initiated after the Great East Japan Earthquake.
Instead of a \textit{re-design-based (parameter-matching)} approach, a full \textit{process-porting (process equivalence)} strategy was adopted, reproducing the Sakata process at Lapis Semiconductor.
Manufacturing was carried out at Lapis, while Electrical Wafer Sort (EWS) testing was performed at Epson’s Sakata factory under centralized quality governance.
This “distributed manufacturing with centralized quality” framework enabled full compatibility without modifying any design assets, demonstrating the effectiveness of a “Process-Identical BCP” approach for design-preserving redundancy.
\end{abstract}

\begin{IEEEkeywords}
BCP, μTFP, Driver IC, Process Porting, Process Equivalence, Full Compatibility, Lapis Semiconductor, Sakata Factory, Reliability, Inkjet Head
\end{IEEEkeywords}

\section{序論}
2011年3月に発生した東日本大震災は,日本の製造業における地理的集中リスクを明確に顕在化させた。
特に半導体産業では,単一拠点に依存した生産体制が供給途絶の主要因となり,
事業継続計画(Business Continuity Planning: BCP)の必要性が強く認識される契機となった。

セイコーエプソン株式会社 酒田工場では,インクジェットプリンタの中核技術である
μTFP(micro Thin Film Piezo)プリントヘッド用ドライバICを一拠点で製造していた。
このICは,数十Vクラスの高電圧出力と高速駆動性能を両立させる重要部品であり,
その安定供給はプリントヘッド製品群全体の生産継続性に直結する。
したがって,災害や設備障害などに対する生産リスクを低減するため,
同一仕様のプロセスを用いた冗長化(二拠点生産体制)の確立が急務であった。

本研究では,宮城県に拠点を有するラピスセミコンダクタを新たな製造ファンドリとして追加し,
酒田工場で用いていた0.35\,μm CMOS(3.3\,V/45\,V WSiゲート)高耐圧プロセスを対象に,
マスク・プロセス条件・評価基準を完全に一致させた\textbf{プロセス移植型(Process Porting / Process Equivalence)}を実施した。
この手法は,ファンドリ側のパラメータに合わせて回路を再設計する一般的な方式とは異なり,
ヘッド側の設計資産を変更せずに,製造ライン側を合わせ込むことで互換性を確保する点に特徴がある。

本論文では,このBCP対応の背景および実施方針を述べるとともに,
プロセス移植の立上げ手法,評価結果,および得られた知見について報告する。

\section{BCP対応方針}
本BCP対応では,μTFPヘッド用ドライバICの生産冗長化を目的として,
既存の酒田工場に加え,宮城県のラピスセミコンダクタを新たな製造拠点として追加した。
この際,異なるファンドリ間でのプロセス差異をどのように扱うかが最重要課題となった。

検討した方針は次の2方式である:
\begin{enumerate}
\item \textbf{再設計型(Re-Design-Based / Parameter Matching)}:\\
ラピス側のプロセスパラメータ(酸化膜厚・しきい値電圧・金属抵抗値など)に合わせて回路設計を再調整する方式。
既存設計を部分的に修正することで短期間の立上げが可能であるが,
ヘッド電気特性,COF(Chip On Film)制御コード,さらにはプリンタ本体ファームウェアに至るまで,
一連の再評価が必要となる。

\item \textbf{プロセス移植型(Process-Porting-Based / Process Equivalence)}:\\
酒田工場で使用していたマスクをそのまま流用し,
ラピス製造ライン側のプロセス条件(熱処理・成膜・エッチング・金属配線など)を酒田仕様に合わせ込む方式。
立上げには工程パラメータの精密調整と評価工数を要するが,
設計側を一切変更せず,ヘッド仕様・制御コード・検査基準を完全共通化できる。
\end{enumerate}

両方式を比較すると,
再設計型は初期投入コストが低い一方で,長期的には設計資産の分断や評価負荷の増大を招くリスクがある。
一方のプロセス移植型は立上げ期間を要するものの,
設計・テスト・製造のすべてを既存資産で統一でき,
製品互換性と品質保証の両立が可能である。

本研究では,ヘッド側の電気特性・制御コード・吐出波形を一切変更せずに冗長化を実現することを最優先とし,
後者の\textbf{プロセス移植型}方針を採用した。
この決定により,\textbf{完全互換(Full Compatibility)}と\textbf{設計資産の保護},
および\textbf{品質保証の一元化}を同時に確保した。

% ---- 比較表 ----
\begin{table}[t]
\centering
\caption{BCP方式の比較:再設計型とプロセス移植型}
\label{tab:approach_compare}
\begin{tabular*}{\linewidth}{@{\extracolsep{\fill}} ll}
\toprule
\textbf{項目} & \textbf{比較内容} \\
\midrule
基本方針 &
\makecell[l]{再設計型:新プロセスに合わせ回路再設計\\
プロセス移植型:既存プロセスを他拠点へ再現} \\

立上げ期間/初期コスト &
短期/低 vs 長期/高 \\

設計資産への影響 &
大(再評価・FW修正必要) vs 無(設計・FW共通) \\

互換性・再評価負荷 &
部分互換/再検証要 vs \textbf{完全互換/再評価不要} \\

BCP適合性 &
暫定対応向き vs 恒久対応・長期供給に最適 \\

品質保証体制 &
拠点別評価になりやすい vs \textbf{酒田EWSによる一元保証} \\
\bottomrule
\end{tabular*}
\end{table}

\section{プロセス移植手法}

\subsection{対象デバイス}
対象は,\SI{0.35}{\micro\meter} CMOS技術をベースとした
\SI{3.3}{\volt}/\SI{45}{\volt} WSiゲート構造を有する高耐圧ドライバICである。
本デバイスは,プリントヘッドの駆動波形生成および高電圧出力段を内包しており,
ゲート酸化膜の信頼性と金属配線抵抗の安定性が吐出性能に直結する。

酒田工場で使用していたマスクデータをそのままラピス製造ラインに適用し,
プロセス条件を段階的にマッピングすることで同等化を図った。
主な制御項目は,熱酸化膜厚,RTA温度・時間プロファイル,
ポリシリコン/WSi膜の堆積条件,および金属多層配線のストレス緩和条件である。
これらを酒田プロセス基準値に対して逐次調整し,
膜厚・抵抗・臨界電圧など主要パラメータの統計分布を一致させることを目標とした。

\subsection{立上げフロー}
プロセス移植の立上げは,以下の4段階で構成した。

\begin{enumerate}
  \item \textbf{酒田レシピの共有と装置パラメータ変換}\\
  各工程の装置設定値(温度・圧力・ガス流量など)を酒田条件からラピス装置仕様に変換し,
  初期プロセスレシピを作成した。特に熱酸化およびWSi堆積工程では,
  装置構造の差異を考慮してプロセスウィンドウを定義した。

  \item \textbf{試作ロット評価(物理・電気パラメータマッピング)}\\
  テストウエハ上で酸化膜厚,コンタクト開口CD,シート抵抗などを測定し,
  酒田基準ロットとの統計比較を実施した。
  各項目は3σ範囲内で一致するまで条件を調整し,プロセスの再現性を確認した。

  \item \textbf{Parametric Testによる電気特性照合}\\
  トランジスタしきい値電圧,オン抵抗,リーク電流などの電気パラメータを評価し,
  酒田ロットとの統計分布を比較した。
  デバイス特性はすべて3σ以内に収まり,電気的同等性を確認した。

  \item \textbf{信頼性評価(Reliability Qualification)}\\
  HTOL(High Temperature Operating Life),
  TDDB(Time Dependent Dielectric Breakdown),
  EM(Electromigration)試験を実施し,
  ラピス製造品が酒田品と同等の劣化挙動および寿命分布を示すことを確認した。
\end{enumerate}

以上の工程を経て,ラピス側の製造条件は酒田プロセスの統計的ばらつき範囲内に収まり,
設計側の特性マージンを一切変更することなく量産適用が可能となった。

\section{評価結果}

\subsection{電気特性(ETEST統計検証)}
ラピス製造ラインにおいて,酒田仕様マスクを用いた試作ロットを対象にETEST(Electrical Test)を実施した。
測定項目は3.3\,V/45\,Vトランジスタ群およびパッシブ素子を含む全工程階層で構成し,
以下の主要特性を評価した。

\begin{itemize}
  \item トランジスタ群(3.3\,V/45\,V):\\
  しきい値電圧$V_{\mathrm{th}}$,電流利得$\beta$,
  飽和電流$I_{\mathrm{dsat}}$,オフ電流$I_{\mathrm{off}}$,
  ブレークダウン電圧$BV_{\mathrm{DS}}$,
  基板電流$I_{\mathrm{sub}}$(substrate current, Hot Carrier指標)
  \item 拡散抵抗,ゲート電極抵抗,コンタクト抵抗,ビア抵抗,配線抵抗
\end{itemize}

各パラメータの平均値および標準偏差を酒田リファレンスロットと比較した結果,
すべての項目が3σ範囲内で一致し,電気的同等性が統計的に確認された。
代表的な結果をTable~\ref{tab:proc_compare}に示す。
しきい値電圧偏差は$\pm\SI{1.5}{\percent}$,
オン抵抗偏差は$\pm\SI{2.0}{\percent}$以内に収まり,
酸化膜リーク電流は$\SI{1e-9}{\ampere\per\square\centi\meter}$以下であった。
プロセス間の標準偏差比$\sigma_{\mathrm{Lapis}} / \sigma_{\mathrm{Sakata}} = 1.03$と極めて近似しており,
\textbf{拠点間でプロセス制御安定性が完全に再現された}ことを示す。

\subsection{信頼性試験}
個別デバイス信頼性に関しても,以下の主要試験を実施した。

\begin{itemize}
  \item \textbf{TZDB (Time-Zero Dielectric Breakdown)}:酸化膜初期破壊電界分布が一致。
  \item \textbf{TDDB (Time-Dependent Dielectric Breakdown)}:寿命分布の中央値が一致。
  \item \textbf{HCI (Hot Carrier Injection)}:劣化速度係数に有意差なし。
  \item \textbf{EM (Electromigration)}:金属配線寿命の中央値比が$0.99$で一致。
  \item \textbf{V\textsubscript{t} Stability}:1000\,h高温動作後のしきい値変動が$\pm\SI{2}{\milli\volt}$以内。
\end{itemize}

いずれの試験でも酒田・ラピス間に統計的差異は見られず,
酸化膜,拡散層,金属層のいずれも\textbf{プロセス同等性が完全に維持されている}ことを確認した。

\subsection{ヘッド特性およびEMC}
ラピス製ドライバICを搭載したμTFPプリントヘッドを用いて,
インク吐出特性および電磁両立性(EMC)の比較評価を行った。
駆動波形は酒田品と同一条件(駆動電圧:\SI{35}{\volt}, 周波数:\SI{30}{\kilo\hertz})で印加し,
吐出速度・液滴体積・安定性を高速カメラにより解析した。

その結果,吐出速度偏差は$\pm\SI{1.8}{\percent}$以内,
液滴体積の変動係数(CV値)は1.02倍と誤差範囲内であり,
波形応答および吐出安定性ともに酒田製と同等であった。
また,プリンタ機体にラピス製ICヘッドを組み込み,
システムレベルEMC試験(CISPR-22準拠)を実施した結果,
ノイズスペクトラムおよび放射電界強度にも統計的差異は認められなかった。

以上より,
ラピス生産デバイスは酒田品と\textbf{電気的・信頼性的・機能的すべての側面で完全同等}であり,
設計・COFコード・プリンタファームウェアに対する修正を一切必要としなかった。

% ---- 表:プロセス・信頼性比較 ----
\begin{table}[t]
\centering
\caption{酒田・ラピスプロセスおよびETEST/信頼性主要特性比較}
\label{tab:proc_compare}
\begin{tabular*}{\linewidth}{@{\extracolsep{\fill}} lcc}
\toprule
\textbf{項目} & \textbf{酒田} & \textbf{ラピス} \\
\midrule
ゲート酸化膜厚 (Å) & 800 & 805 \\
WSi抵抗値 (\si{\ohm\per\sq}) & 2.3 & 2.4 \\
拡散抵抗 (\si{\ohm}) & 102 & 104 \\
ゲート電極抵抗 (\si{\ohm}) & 12.1 & 12.3 \\
コンタクト抵抗 (\si{\ohm}) & 1.8 & 1.9 \\
ビア抵抗 (\si{\ohm}) & 0.42 & 0.43 \\
配線抵抗 (\si{\ohm}) & 2.1 & 2.1 \\
リーク電流 (\si{\ampere\per\centi\metre\squared}) & $<1\times10^{-9}$ & $<1\times10^{-9}$ \\
臨界電圧 $V_{\mathrm{bd}}$ (V) & 48.2 & 47.9 \\
3.3Vトランジスタ群 (Vth, β, Idsat, Ioff, BVDS, Isub*) & 同等 & 同等 \\
45Vトランジスタ群 (Vth, β, Idsat, Ioff, BVDS, Isub*) & 同等 & 同等 \\
$\sigma_{\mathrm{Lapis}} / \sigma_{\mathrm{Sakata}}$ & -- & 1.03 \\
TDDB寿命中央値比 & 1.00 & 0.98 \\
HCI劣化係数比 & 1.00 & 1.02 \\
EM寿命中央値比 & 1.00 & 0.99 \\
V\textsubscript{t}変動 (mV) & $\pm2$ & $\pm2$ \\
\multicolumn{3}{l}{\footnotesize *$I_{\mathrm{sub}}$:基板電流(substrate current, Hot Carrier指標)} \\
\bottomrule
\end{tabular*}
\end{table}

\section{品質保証とトレーサビリティ}
本BCP体制下における品質保証スキームを\cref{fig:qa_flow}に示す。
\textbf{製造工程はラピスセミコンダクタが担当し,ウエハテスト(EWS: Electrical Wafer Sort)は酒田工場で実施}する分業構成とした。
出荷判定および最終保証はエプソン本体による集中管理とし,単一基準での品質認証を維持した。

COFコード(Chip On Film制御コード)は酒田・ラピス間で共通化し,
製品ロット単位での管理番号を付与することでトレーサビリティを確保した。
これにより,各ICがどの拠点・ロットで生産されたかを即時に追跡でき,
品質問題発生時にも出荷停止・切替判断を迅速に行える構造とした。

また,プロセス同等性の維持を目的として,四半期ごとに相互モニタリングロットを製造し,
酸化膜厚・リーク電流・$V_{\mathrm{bd}}$等の主要パラメータを定点測定する。
このデータを酒田側リファレンスと比較し,偏差が管理限界値($\pm3\sigma$)を超えた場合には即時是正を行う。
\textbf{この「製造分散・保証集中」構造により,物理的製造拠点は異なっても,酒田EWSの共通基準を通じて全ロットが同一品質ゲートで判定される。}

\begin{figure}[t]
\centering
\begin{tikzpicture}[node distance=10mm, font=\footnotesize]
  \node[draw, rounded corners, align=center, fill=gray!10, minimum width=38mm, minimum height=6mm] (lapis)
    {製造:ラピスセミコンダクタ(宮城)};
  \node[draw, rounded corners, below=8mm of lapis, fill=gray!10, minimum width=38mm, minimum height=6mm, align=center] (sakata)
    {ウエハテスト(EWS):酒田工場(山形)};
  \node[draw, rounded corners, below=8mm of sakata, fill=gray!10, minimum width=38mm, minimum height=6mm, align=center] (epson)
    {出荷保証・承認:エプソン本体};

  \draw[->, thick] (lapis) -- (sakata);
  \draw[->, thick] (sakata) -- (epson);
\end{tikzpicture}
\caption{品質保証・トレーサビリティ体制(製造分散・保証集中)}
\label{fig:qa_flow}
\end{figure}

\section{考察}
プロセス移植型方式は,短期的には工程マッピング・信頼性評価など多大な工数を要するが,
一度確立すれば設計再認証や評価系統の分断を回避でき,
結果として長期的トータルコストを低減できる利点がある。
本研究の結果,酒田・ラピス両拠点での完全同等製造が実証され,
「プロセスを再現することで設計資産を守る」アプローチの有効性が確認された。

特に,本研究で採用した\textbf{プロセス移植型BCP}は,
「\textbf{設計資産を改変せずに完全互換を維持}できる」点で従来の\textbf{再設計型BCP}と本質的に異なる。
製造を分散しつつ,品質保証を酒田EWSに集中させる構造により,
長期供給リスクと設計分断リスクの双方を排除する新しいBCPアーキテクチャを実現した。

また,本手法はμTFPドライバICに限らず,
MEMSアクチュエータ,ミックスドシグナルSoC,
AIアクセラレータなど高信頼・長期供給デバイスにも適用可能である。
すなわち,「設計資産保護型BCP(Design Preservation-Oriented BCP)」として,
半導体サプライチェーン強靭化の共通モデルとなり得る。

\section{結論}
本報では,μTFPプリントヘッド用ドライバICのBCP対応事例として,
酒田工場とラピスセミコンダクタにおける
\textbf{同一マスク・同一プロセス条件の再現(Process Equivalence)}を実現した。
電気特性,吐出特性,信頼性試験のいずれにおいても
両拠点品は\textbf{完全同等}であり,設計・制御コード・ファームウェア再評価を要しなかった。

本成果は,災害リスク分散と設計資産保全を同時に達成する
「Design Preservation through Process Reproduction」アプローチの成功例であり,
今後の国内半導体BCP構築の技術的ベースラインとなる。

\section*{謝辞}
本研究の実施にあたり,酒田工場および宮城ラピスセミコンダクタの各位,
ならびにエプソン半導体事業部品質保証部の皆様に深く感謝する。

\section*{参考文献}
\begin{thebibliography}{99}
\bibitem{Samizo2024_uTEP}
S.~Samizo, 
``uTEPヘッド用ドライバICにおけるウエハ酸素濃度変動起因の品質問題対応,'' 
\textit{Epson Semiconductor Technical Report}, 2024.

\bibitem{Samizo2023_TFP}
S.~Samizo, 
``薄膜PZTアクチュエータにおける振動板クラックと端部焼損の原因解析,'' 
\textit{Journal of Precision Mechatronics}, 2023.

\bibitem{Samizo2022_PZT}
S.~Samizo, 
``薄膜PZT技術の系譜,'' 
\textit{Proceedings of the Japan Society of Applied Physics}, 2022.
\end{thebibliography}

\section*{著者略歴}
\textbf{三溝 真一}(Shinichi Samizo)は、信州大学大学院 工学系研究科 電気電子工学専攻にて修士号を取得。
セイコーエプソン株式会社に勤務し、半導体ロジック/高耐圧統合、インクジェット薄膜PZTアクチュエータ、
PrecisionCore プリントヘッド製品化に従事。
現在は独立系半導体研究者として、プロセス/デバイス教育、メモリアーキテクチャ、
AIシステム統合に取り組む。連絡先: \href{mailto:shin3t72@gmail.com}{shin3t72@gmail.com}.

\end{document}
