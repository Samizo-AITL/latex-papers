% =====================================================
% μTFP Driver IC BCP論文(Process Equivalence)
% =====================================================
\documentclass[conference]{IEEEtran}

% ---------- LuaLaTeX 日本語・μ対応(決定版) ----------
\usepackage{iftex}
\ifLuaTeX
  \usepackage{luatexja}
  \usepackage{luatexja-fontspec}
  \setmainjfont{IPAexMincho}
  \setsansjfont{IPAexGothic}
  \usepackage{fontspec}
  \setmainfont{Latin Modern Roman}
  \setsansfont{Latin Modern Sans}
  \ltjsetparameter{jacharrange={-2,-3}}
\else
  \usepackage[utf8]{inputenc}
  \usepackage[T1]{fontenc}
\fi

% --- Unicode直打ちの安全化(μ/¥を確実に通す) ---
\usepackage{newunicodechar}
\newcommand{\micro}{\ensuremath{\mu}}
\newunicodechar{μ}{\micro}
\newunicodechar{¥}{\textbackslash}

% ---------- 便利パッケージ群 ----------
\usepackage{graphicx}
\usepackage[table]{xcolor}
\usepackage{array,tabularx,booktabs,threeparttable,makecell,multirow}
\usepackage{amsmath,amssymb,mathtools,bm}
\usepackage{siunitx}
\sisetup{
  mode = match,
  propagate-math-font = true,
  reset-math-version = false,
  reset-text-family = false,
  reset-text-series = false,
  reset-text-shape = false,
  text-family-to-math = true,
  text-series-to-math = true,
  per-mode = symbol,
  uncertainty-mode = separate,
  range-phrase = --,
  group-separator = {,}
}
% □(per square)を単位として登録
\DeclareSIUnit{\sq}{\text{\ensuremath{\square}}}

\usepackage[caption=false,font=footnotesize]{subfig}
\usepackage{cite}
\usepackage{tikz}
\usetikzlibrary{arrows.meta,positioning,fit,calc,shapes,decorations.pathmorphing}
\usepackage{pgfplots}
\pgfplotsset{compat=1.18}
\usepackage[hidelinks]{hyperref}
\usepackage[capitalise,noabbrev]{cleveref}

\renewcommand{\figurename}{Fig.}
\renewcommand{\tablename}{Table}

\makeatletter
\def\@makefnmark{\hbox{\textsuperscript{\normalfont\@thefnmark}}}
\makeatother

% ---------- 本文 ----------
\begin{document}

\title{%
μTFPヘッド用ドライバICのBCP対応:\\
プロセス同等化による二拠点生産体制の確立
}

\author{%
三溝 真一(Shinichi Samizo)\\
独立系半導体研究者(元セイコーエプソン株式会社)\\
Email: \href{mailto:shin3t72@gmail.com}{shin3t72@gmail.com} \\
GitHub: \url{https://github.com/Samizo-AITL}
}

\maketitle

\begin{abstract}
\textbf{和文概要:}
本研究は,東日本大震災を契機に,μTFPプリントヘッド用ドライバICの生産リスク低減を目的として実施した事業継続計画(BCP: Business Continuity Planning)対応事例を報告する。
エプソン酒田工場で製造されていた0.35\,μm CMOS(3.3\,V/45\,V, WSiゲート)ドライバICについて,ファンドリとして宮城県のラピスセミコンダクタを追加した。
従来の「パラメータ合わせ(Parameter Matching)」ではなく,「プロセス同等化(Process Equivalence)」により酒田工場と同一仕様をラピスライン上に再現する手法を採用した。
立上げにはプロセス条件の逐次マッピングと統計的検証を要したが,結果として,電気特性・吐出特性・EMCのいずれも酒田品と同等であり,ヘッド設計の再評価を不要とする完全互換を実現した。
本事例は,設計資産を改変せずに冗長化を達成する「Process-Identical BCP」の有効性を実証するものである。

\medskip
\textbf{Abstract:}
This paper reports a Business Continuity Planning (BCP) implementation for the μTFP printhead driver IC, initiated in response to the Great East Japan Earthquake.
To mitigate production risk, Lapis Semiconductor (Miyagi, Japan) was introduced as an additional foundry to complement the original Epson Sakata factory.
The target device was a 0.35\,μm CMOS driver IC (3.3 V / 45 V WSi-gate process). 
Instead of redesigning the circuit to match the Lapis process parameters, a full \textit{process equivalence} approach was adopted—reproducing the Sakata process using identical masks and matched fabrication conditions.
Although the process alignment required extensive parameter mapping and verification, the resulting electrical characteristics, ink ejection performance, and EMC behavior were indistinguishable from the Sakata production.
This work demonstrates the feasibility of a “Process-Identical BCP” approach that ensures design compatibility and product redundancy without sacrificing existing design assets.
\end{abstract}

\begin{IEEEkeywords}
BCP, μTFP, Driver IC, Process Equivalence, Lapis Semiconductor, Sakata Factory, Reliability, Inkjet Head
\end{IEEEkeywords}

\section{序論}
2011年3月に発生した東日本大震災は,日本の製造業における地理的集中リスクを明確に顕在化させた。
特に半導体産業では,単一拠点に依存した生産体制が供給途絶の主要因となり,
事業継続計画(Business Continuity Planning: BCP)の必要性が強く認識される契機となった。

セイコーエプソン株式会社 酒田工場では,インクジェットプリンタの中核技術である
μTFP(micro Thin Film Piezo)プリントヘッド用ドライバICを一拠点で製造していた。
このICは,数十Vクラスの高電圧出力と高速駆動性能を両立させる重要部品であり,
その安定供給はプリントヘッド製品群全体の生産継続性に直結する。
したがって,災害や設備障害などに対する生産リスクを低減するため,
同一仕様のプロセスを用いた冗長化(二拠点生産体制)の確立が急務であった。

本研究では,宮城県に拠点を有するラピスセミコンダクタを新たな製造ファンドリとして追加し,
酒田工場で用いていた0.35\,μm CMOS(3.3\,V/45\,V WSiゲート)高耐圧プロセスを対象に,
マスク・プロセス条件・評価基準を完全に一致させた「プロセス同等化(Process Equivalence)」を実施した。
この手法は,ファンドリ側のパラメータに合わせて回路を再設計する一般的な方式とは異なり,
ヘッド側の設計資産を変更せずに,製造ライン側を合わせ込むことで互換性を確保する点に特徴がある。

本論文では,このBCP対応の背景および実施方針を述べるとともに,
プロセス同等化の立上げ手法,評価結果,および得られた知見について報告する。

\section{BCP対応方針}
本BCP対応では,μTFPヘッド用ドライバICの生産冗長化を目的として,
既存の酒田工場に加え,宮城県のラピスセミコンダクタを新たな製造拠点として追加した。
この際,異なるファンドリ間でのプロセス差異をどのように扱うかが最重要課題となった。

検討した方針は次の2方式である:
\begin{enumerate}
\item \textbf{Parameter Matching方式}:\\
ラピス側のプロセスパラメータ(酸化膜厚・しきい値電圧・金属抵抗値など)に合わせて回路設計を再調整する方法。
既存設計を一部修正することで短期間での立上げが可能であるが,
ヘッド電気特性やCOF(Chip On Film)制御コード,さらにはプリンタ本体のファームウェアを含む一連のシステム再評価が必要となる。
\item \textbf{Process Matching方式}:\\
酒田工場で使用していたマスクをそのまま流用し,
ラピス製造ライン側のプロセス条件(熱処理・成膜・エッチング・金属配線など)を酒田仕様に合わせ込む方法。
立上げには多くの工程パラメータ調整と評価工数を要するが,
設計側の変更を一切伴わず,ヘッド仕様・制御コード・検査基準を完全に共通化できる。
\end{enumerate}

両方式を比較すると,
Parameter Matchingは初期投入コストが低い一方で,
長期的には設計資産の分断や評価負荷の増大を招くリスクがある。
一方のProcess Matchingは初期立上げに時間を要するものの,
設計・テスト・製造のすべてを既存資産で統一でき,
製品互換性と品質保証の両立が可能である。

本研究では,ヘッド側の電気特性・制御コード・吐出波形を一切変更せずに冗長化を実現することを最優先とし,
後者の\textbf{「プロセス同等化(Process Equivalence)」}方針を採用した。
この決定により,設計資産の保護とサプライチェーンの耐災害性を同時に確保することを狙った。

\section{プロセス同等化手法}

\subsection{対象デバイス}
対象は,\SI{0.35}{\micro\meter} CMOS技術をベースとした
\SI{3.3}{\volt}/\SI{45}{\volt} WSiゲート構造を有する高耐圧ドライバICである。
本デバイスは,プリントヘッドの駆動波形生成および高電圧出力段を内包しており,
ゲート酸化膜の信頼性と金属配線抵抗の安定性が吐出性能に直結する。

酒田工場で使用していたマスクデータをそのままラピス製造ラインに適用し,
プロセス条件を段階的にマッピングすることで同等化を図った。
主な制御項目は,熱酸化膜厚,RTA温度・時間プロファイル,
ポリシリコン/WSi膜の堆積条件,および金属多層配線のストレス緩和条件である。
これらを酒田プロセス基準値に対して逐次調整し,
膜厚・抵抗・臨界電圧など主要パラメータの統計分布を一致させることを目標とした。

\subsection{立上げフロー}
プロセス同等化の立上げは,以下の4段階で構成した。

\begin{enumerate}
  \item \textbf{酒田レシピの共有と装置パラメータ変換}\\
  各工程の装置設定値(温度・圧力・ガス流量など)を酒田条件からラピス装置仕様に変換し,
  初期プロセスレシピを作成した。特に熱酸化およびWSi堆積工程では,
  装置構造の差異を考慮してプロセスウィンドウを定義した。

  \item \textbf{試作ロット評価(物理・電気パラメータマッピング)}\\
  テストウエハ上で酸化膜厚,コンタクト開口CD,シート抵抗などを測定し,
  酒田基準ロットとの統計比較を実施した。
  各項目は3σ範囲内で一致するまで条件を調整し,プロセスの再現性を確認した。

  \item \textbf{Parametric Testによる電気特性照合}\\
  トランジスタしきい値電圧,オン抵抗,リーク電流などの電気パラメータを評価し,
  酒田ロットとの統計分布を比較した。
  デバイス特性はすべて3σ以内に収まり,電気的同等性を確認した。

  \item \textbf{信頼性評価(Reliability Qualification)}\\
  HTOL(High Temperature Operating Life),
  TDDB(Time Dependent Dielectric Breakdown),
  EM(Electromigration)試験を実施し,
  ラピス製造品が酒田品と同等の劣化挙動および寿命分布を示すことを確認した。
\end{enumerate}

以上の工程を経て,ラピス側の製造条件は酒田プロセスの統計的ばらつき範囲内に収まり,
設計側の特性マージンを一切変更することなく量産適用が可能となった。

\section{評価結果}

\subsection{電気特性}
プロセス同等化後のラピス製デバイスについて,出力ドライバ段および高耐圧MOSトランジスタの
IV特性を測定した。
その結果,酒田製デバイスとのしきい値電圧偏差は$\pm\SI{1.5}{\percent}$以内,
オン抵抗は$\pm\SI{2}{\percent}$以内に収まり,電気的にほぼ完全一致を示した。
ゲートリーク電流は$\SI{1e-9}{\ampere\per\square\centi\meter}$以下であり,
酒田プロセスと同等の酸化膜信頼性が確認された。

さらに,チップ全面でのデバイスパラメータのばらつきを統計解析した結果,
プロセス間の標準偏差比$\sigma_{\mathrm{Lapis}} / \sigma_{\mathrm{Sakata}} = 1.03$と極めて近似しており,
プロセス制御安定性の再現が確認された。

\subsection{ヘッド特性およびEMC}
ラピス製ドライバICを搭載したμTFPプリントヘッドを用いて,
インク吐出特性および電磁両立性(EMC)の比較評価を行った。
駆動波形は酒田品と同一条件(駆動電圧:\SI{35}{\volt}, 周波数:\SI{30}{\kilo\hertz})で印加し,
吐出速度・液滴体積・安定性を高速カメラにより解析した。

その結果,吐出速度偏差は$\pm\SI{1.8}{\percent}$以内,
液滴体積の変動係数(CV値)は1.02倍と誤差範囲内であり,
波形応答および吐出安定性ともに酒田製と同等であった。
また,プリンタ機体にラピス製ICヘッドを組み込み,
システムレベルEMC試験(CISPR-22準拠)を実施した結果,
ノイズスペクトラム・放射電界強度ともに統計的差異は認められなかった。

これらの結果より,
ラピス生産デバイスは電気的・機能的に酒田品と完全同等であり,
ヘッド設計・COFコード・プリンタファームウェアに対する修正を必要としないことが確認された。

\begin{figure}[t]
\centering
\begin{tikzpicture}[>=Stealth, node distance=10mm, font=\footnotesize]
  \node[draw, rounded corners, align=center, minimum width=35mm, minimum height=7mm, fill=gray!10] (pm)
    {Parameter\\Matching\\(再設計)};
  \node[draw, rounded corners, align=center, minimum width=35mm, minimum height=7mm, right=22mm of pm, fill=gray!10] (pe)
    {Process\\Matching\\(本研究)};
  \node[below=12mm of pm, align=left, text width=35mm] (pm_bullets) {
    \begin{itemize}\itemsep2pt
      \item 立上げ短期
      \item ヘッド側再評価必要
      \item 部分互換
    \end{itemize}
  };
  \node[below=12mm of pe, align=left, text width=35mm] (pe_bullets) {
    \begin{itemize}\itemsep2pt
      \item 工数・期間大
      \item ヘッド側影響ゼロ
      \item 完全互換
    \end{itemize}
  };
\end{tikzpicture}
\caption{BCP対応方式の比較(Parameter Matching vs Process Matching)}
\label{fig:bcp_compare}
\end{figure}

\begin{table}[t]
\centering
\caption{ラピス・酒田プロセスの主要特性比較}
\label{tab:proc_compare}
\begin{tabular}{lcc}
\toprule
項目 & 酒田 & ラピス \\
\midrule
ゲート酸化膜厚 (Å) & 800 & 805 \\
WSi抵抗値 (\si{\ohm\per\sq}) & 2.3 & 2.4 \\
リーク電流 (\si{\ampere\per\centi\metre\squared}) & $<1\times10^{-9}$ & $<1\times10^{-9}$ \\
臨界電圧 V\textsubscript{bd} (V) & 48.2 & 47.9 \\
\bottomrule
\end{tabular}
\end{table}

\section{品質保証とトレーサビリティ}
本BCP体制下における品質保証スキームを図\ref{fig:qa_flow}に示す。
製造工程はラピスセミコンダクタが担当し,
ウエハテスト(EWS: Electrical Wafer Sort)は酒田工場で実施する分業構成とした。
出荷判定および最終保証はエプソン本体による集中管理とし,
単一基準での品質認証を維持した。

COFコード(Chip On Film制御コード)は酒田・ラピス間で共通化し,
製品ロット単位での管理番号を付与することでトレーサビリティを確保した。
これにより,個々のICがどの拠点・ロットで生産されたかを即時に追跡でき,
品質問題発生時にも出荷停止・切替判断を迅速に行える構造とした。

また,プロセス同等化の維持を目的として,
四半期ごとに相互モニタリングロットを製造し,
酸化膜厚・リーク電流・Vbd等の主要パラメータを定点測定する。
このデータを酒田側のリファレンスと比較し,
偏差が管理限界値($\pm3\sigma$)を超えた場合には即時是正を行う。
これにより,二拠点体制下でも長期的なプロセス同等性を維持する体制を確立した。

\begin{figure}[t]
\centering
\begin{tikzpicture}[node distance=10mm, font=\footnotesize]
  \node[draw, rounded corners, align=center, fill=gray!10, minimum width=32mm, minimum height=6mm] (lapis)
    {製造:ラピスセミコンダクタ};
  \node[draw, rounded corners, below=8mm of lapis, fill=gray!10, minimum width=32mm, minimum height=6mm, align=center] (sakata)
    {ウエハテスト:酒田工場};
  \node[draw, rounded corners, below=8mm of sakata, fill=gray!10, minimum width=32mm, minimum height=6mm, align=center] (epson)
    {出荷保証:エプソン本体};

  \draw[->, thick] (lapis) -- (sakata);
  \draw[->, thick] (sakata) -- (epson);
\end{tikzpicture}
\caption{品質保証・トレーサビリティ体制の構成}
\label{fig:qa_flow}
\end{figure}

\section{考察}
プロセス同等化方式は,短期的には工程マッピング・信頼性評価など多大な工数を要するが,
一度確立すれば製品設計の再認証や評価系統の分断を避けられ,
結果として長期的なトータルコストを低減できる利点がある。
本研究の結果,酒田・ラピス両拠点における製造条件の完全同等化が実証され,
「プロセスを再現することで設計資産を守る」アプローチの有効性が確認された。

また,この手法はμTFPドライバICに限定されず,
将来的にはMEMSアクチュエータ,ミックスドシグナルSoC,
さらにはAIアクセラレータなど,
高信頼性と長期供給が求められるデバイス群にも適用可能である。
すなわち,「設計資産保護型BCP(Design Preservation-Oriented BCP)」として
半導体サプライチェーン強靭化の共通モデルとなり得る。

\section{結論}
本報では,μTFPプリントヘッド用ドライバICのBCP対応事例として,
酒田工場とラピスセミコンダクタにおける
\textbf{同一マスク・同一プロセス条件の再現(Process Equivalence)}を実現した。
電気特性,吐出特性,および信頼性試験のいずれにおいても
両拠点品は完全同等であり,製品設計・制御コード・ファームウェアの再評価を要しなかった。

本成果は,災害リスク分散と設計資産保全を同時に達成する
「Design Preservation through Process Reproduction」アプローチの実用的成功例であり,
今後の国内半導体BCP構築における技術的ベースラインとなる。

\section*{謝辞}
本研究の実施にあたり,酒田工場および宮城ラピスセミコンダクタの各位,
ならびにエプソン半導体事業部品質保証部の皆様に深く感謝する。

\section*{参考文献}
\begin{thebibliography}{99}

\bibitem{Samizo2024_uTEP}
S.~Samizo, 
``uTEPヘッド用ドライバICにおけるウエハ酸素濃度変動起因の品質問題対応
[Quality issue mitigation in uTEP driver IC caused by wafer oxygen concentration fluctuation],'' 
\textit{Epson Semiconductor Technical Report}, 2024.

\bibitem{Samizo2023_TFP}
S.~Samizo, 
``薄膜PZTアクチュエータにおける振動板クラックと端部焼損の原因解析
[Failure analysis of diaphragm cracking and edge burn in thin-film PZT actuators],'' 
\textit{Journal of Precision Mechatronics}, 2023.

\bibitem{Samizo2022_PZT}
S.~Samizo, 
``薄膜PZT技術の系譜
[Historical evolution of thin-film PZT technology],'' 
\textit{Proceedings of the Japan Society of Applied Physics}, 2022.

\end{thebibliography}

\section*{著者略歴}
\textbf{三溝 真一}(Shinichi Samizo)は、信州大学大学院 工学系研究科 電気電子工学専攻にて修士号を取得した。その後、セイコーエプソン株式会社に勤務し、半導体ロジック/メモリ/高耐圧インテグレーション、そして、インクジェット薄膜ピエゾアクチュエータ及びPrecisionCoreプリントヘッドの製品化に従事した。現在は独立系半導体研究者として、プロセス/デバイス教育、メモリアーキテクチャ、AIシステム統合などに取り組んでいる。連絡先: \href{mailto:shin3t72@gmail.com}{shin3t72@gmail.com}.

\end{document}
