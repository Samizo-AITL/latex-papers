% =====================================================
% μTFP Driver IC BCP論文(Process Equivalence)
% =====================================================
\documentclass[conference]{IEEEtran}

% ---------- LuaLaTeX 日本語・μ対応(決定版) ----------
\usepackage{iftex}
\ifLuaTeX
  \usepackage{luatexja}
  \usepackage{luatexja-fontspec}      % 日本語フォントをfontspecで指定
  \setmainjfont{IPAexMincho}          % 和文明朝(fonts-ipaexfont 提供名)
  \setsansjfont{IPAexGothic}          % 和文ゴシック
  % (必要なら)欧文も明示:省略可
  \usepackage{fontspec}
  \setmainfont{Latin Modern Roman}
  \setsansfont{Latin Modern Sans}
  \ltjsetparameter{jacharrange={-2,-3}}
\else
  \usepackage[utf8]{inputenc}
  \usepackage[T1]{fontenc}
\fi

% --- Unicode直打ちの安全化(μ/¥を確実に通す) ---
\usepackage{newunicodechar}
\newcommand{\micro}{\ensuremath{\mu}}
\newunicodechar{μ}{\micro}
\newunicodechar{¥}{\textbackslash}

% ---------- 以下は従来通り ----------
\usepackage{graphicx}
\usepackage[table]{xcolor}
\usepackage{array,tabularx,booktabs,threeparttable,makecell,multirow}
\usepackage{amsmath,amssymb,mathtools,bm}
\usepackage{siunitx}
\sisetup{detect-all,per-mode=symbol,separate-uncertainty=true,range-phrase=--,group-separator={,}}
\usepackage[caption=false,font=footnotesize]{subfig}
\usepackage{cite}
\usepackage{tikz}
\usetikzlibrary{arrows.meta,positioning,fit,calc,shapes,decorations.pathmorphing}
\usepackage{pgfplots}
\pgfplotsset{compat=1.18}
\usepackage[hidelinks]{hyperref}
\usepackage[capitalise,noabbrev]{cleveref}
\renewcommand{\figurename}{Fig.}
\renewcommand{\tablename}{Table}
\newcommand{\ohm}{\ensuremath{\Omega}}

\makeatletter
\def\@makefnmark{\hbox{\textsuperscript{\normalfont\@thefnmark}}}
\makeatother

% ---------- ここから本文 ----------
\begin{document}

% -----------------------------------------------------
% タイトル・著者情報
% -----------------------------------------------------
\title{%
μTFPヘッド用ドライバICのBCP対応:\\
プロセス同等化による二拠点生産体制の確立\\[2mm]
\textit{Business Continuity Implementation for μTFP Printhead Driver IC\\
via Process Equivalence at Sakata and Lapis Semiconductor}
}

\author{%
三溝 真一(Shinichi Samizo)\\
独立系半導体研究者(元セイコーエプソン株式会社)\\
Email: \href{mailto:shin3t72@gmail.com}{shin3t72@gmail.com} \\
GitHub: \url{https://github.com/Samizo-AITL}
}

\maketitle

% -----------------------------------------------------
% Abstract
% -----------------------------------------------------
\begin{abstract}
本研究は,東日本大震災を契機に,μTFPプリントヘッド用ドライバICの生産リスク低減を目的として実施したBCP対応事例を報告する。ファンドリとしてラピスセミコンダクタを追加し,\SI{0.35}{\micro\meter} CMOS(\SI{3.3}{\volt}/\SI{45}{\volt} WSiゲート)プロセスを対象に,従来の「パラメータ合わせ」ではなく「プロセス同等化」により酒田工場と同一仕様を再現した。その結果,電気特性・吐出特性・EMCいずれも酒田品と同等であり,ヘッド側設計変更を要しない完全互換体制を確立した。本事例は,設計資産を保護しつつ冗長化を実現する「Process-Identical BCP」の有効性を示すものである。
\end{abstract}

\begin{IEEEkeywords}
BCP, μTFP, Driver IC, Process Equivalence, Lapis Semiconductor, Sakata Factory, Reliability, Inkjet Head
\end{IEEEkeywords}

% -----------------------------------------------------
\section{序論}
2011年の東日本大震災は,半導体製造拠点の地理的集中リスクを顕在化させた。エプソン酒田工場ではμTFPヘッド用ドライバICを一拠点で生産していたが,事業継続性の観点から,宮城県のラピスセミコンダクタに代替パスを構築することが急務となった。

\section{BCP対応方針}
本BCP対応では,ファンドリ追加に際し次の2方式を検討した:
\begin{enumerate}
\item ラピス側のプロセスパラメータに合わせて再設計する(Parameter Matching)
\item 酒田工場のマスクとプロセス条件をそのままラピスで再現する(Process Matching)
\end{enumerate}
短期的には前者が容易であるが,ヘッド電特・COFコード・ファームウェアの再評価が必要となる。本研究では,ヘッド仕様維持を最優先し,後者の「プロセス同等化」を採用した。

\section{プロセス同等化手法}
\subsection{対象デバイス}
\SI{0.35}{\micro\meter} CMOS,\SI{3.3}{\volt}/\SI{45}{\volt} WSiゲート構造を有する高耐圧ドライバIC。酒田マスクをラピス製造ラインに流用し,熱酸化・RTA・金属膜堆積条件を逐次マッピングした。

\subsection{立上げフロー}
\begin{enumerate}
\item 酒田レシピの提供と装置パラメータ変換
\item 試作ロット評価:膜厚・CD・シート抵抗比較
\item Parametric Test:酒田ロットとの統計照合($3\sigma$内合格)
\item 信頼性評価:HTOL・TDDB・EM試験
\end{enumerate}

\section{評価結果}
\subsection{電気特性}
出力ドライバのIV特性,ゲートリーク電流ともに酒田品との偏差は$\pm\SI{2}{\percent}$以内であった。

\subsection{ヘッド特性およびEMC}
ラピス製ICを搭載したプリントヘッドで吐出試験を実施。波形・速度・安定性いずれも同等であり,システムレベルEMC試験でも差異を確認しなかった。

% --- 外部PDFを使わずにTikZで簡易図を作成(ビルド安定化) ---
\begin{figure}[t]
\centering
\begin{tikzpicture}[>=Stealth, node distance=10mm, font=\footnotesize]
  \node[draw, rounded corners, align=center, minimum width=35mm, minimum height=7mm, fill=gray!10] (pm)
    {Parameter\\Matching\\(再設計)};
  \node[draw, rounded corners, align=center, minimum width=35mm, minimum height=7mm, right=22mm of pm, fill=gray!10] (pe)
    {Process\\Matching\\(本研究)};

  \draw[->, thick] (pm) -- +(0,-8mm) node[midway, left] {} ;
  \draw[->, thick] (pe) -- +(0,-8mm) node[midway, right] {} ;

  \node[below=12mm of pm, align=left, text width=35mm] (pm_bullets) {
    \begin{itemize}\itemsep2pt
      \item 立上げ短期
      \item ヘッド側再評価必要
      \item 部分互換
    \end{itemize}
  };
  \node[below=12mm of pe, align=left, text width=35mm] (pe_bullets) {
    \begin{itemize}\itemsep2pt
      \item 工数・期間大
      \item ヘッド側影響ゼロ
      \item 完全互換
    \end{itemize}
  };
\end{tikzpicture}
\caption{BCP対応方式の比較(Parameter Matching vs Process Matching)}
\label{fig:bcp_compare}
\end{figure}

\begin{table}[t]
\centering
\caption{ラピス・酒田プロセスの主要特性比較}
\label{tab:proc_compare}
\begin{tabular}{lcc}
\toprule
項目 & 酒田 & ラピス \\
\midrule
ゲート酸化膜厚 (Å) & 800 & 805 \\
WSi抵抗値 (\si{\ohm\per\square}) & 2.3 & 2.4 \\
リーク電流 (A/cm$^2$) & $<1\times10^{-9}$ & $<1\times10^{-9}$ \\
臨界電圧 V\textsubscript{bd} (V) & 48.2 & 47.9 \\
\bottomrule
\end{tabular}
\end{table}

\section{品質保証とトレーサビリティ}
製造:ラピスセミコンダクタ,ウエハテスト:酒田工場,出荷保証:エプソン本体。COFコードは共通運用としつつ,ロット別管理によるトレーサビリティを確保した。

\section{考察}
プロセス同等化は短期的には高工数・長期間を要するが,製品側再設計を不要とし,結果的に全体工数を最小化する。本方式は「設計資産保護型BCP」として有効であり,今後のMEMS・混載IC製造にも応用可能である。

\section{結論}
本報では,μTFPドライバICのBCP対応として,酒田工場とラピスセミコンダクタにおける同一マスク・同一プロセス立上げを実現した。電気特性・ヘッド性能とも完全同等であり,高信頼な二拠点供給体制を確立した。この手法は「Design Preservation through Process Reproduction」の有効事例として位置づけられる。

% -----------------------------------------------------
\section*{謝辞}
本研究の実施にあたり,酒田工場,宮城ラピスセミコンダクタの各位に感謝する。

% -----------------------------------------------------
\section*{参考文献}
\begin{thebibliography}{99}
\bibitem{Samizo2024_uTEP}
S.~Samizo, ``uTEPヘッド用ドライバICにおけるウエハ酸素濃度変動起因の品質問題対応,'' 2024.

\bibitem{Samizo2023_TFP}
S.~Samizo, ``薄膜PZTアクチュエータにおける振動板クラックと端部焼損の原因解析,'' 2023.

\bibitem{Samizo2022_PZT}
S.~Samizo, ``薄膜PZT技術の系譜,'' 2022.
\end{thebibliography}

% -----------------------------------------------------
% 著者略歴
% -----------------------------------------------------
\section*{著者略歴}
\textbf{三溝 真一}(Shinichi Samizo)は、信州大学大学院 工学系研究科 電気電子工学専攻にて修士号を取得した。  
その後、セイコーエプソン株式会社に勤務し、半導体ロジック/メモリ/高耐圧インテグレーション、そして、インクジェット薄膜ピエゾアクチュエータ及びPrecisionCoreプリントヘッドの製品化に従事した。  
現在は独立系半導体研究者として、プロセス/デバイス教育、メモリアーキテクチャ、AIシステム統合などに取り組んでいる。  
連絡先: \href{mailto:shin3t72@gmail.com}{shin3t72@gmail.com}.

\end{document}
