\documentclass[conference]{IEEEtran}
\IEEEoverridecommandlockouts

\title{Educational Perspectives on Complementary FETs (CFET): \\
Evolution Beyond GAA and Open Challenges}

\author{
\IEEEauthorblockN{Shinichi Samizo}
\IEEEauthorblockA{
Project Design Hub, Samizo-AITL \\
Email: samizo-aitl@example.com
}
}

\begin{document}
\maketitle

\begin{abstract}
This tutorial paper provides an educational overview of the emerging
\textit{Complementary FET (CFET)} technology, which vertically stacks
nFET and pFET devices to overcome scaling limits beyond Gate-All-Around
(GAA) nanosheets. CFET represents a paradigm shift where the
cross-section itself forms a CMOS inverter, enabling significant area
reduction and delay minimization. While promising, CFET fabrication and
design remain in research stages, and industrial process design kits
(PDKs) or compact models are not yet available. This paper reviews the
evolution from planar MOSFETs to FinFETs, GAA, and CFET, summarizes
structural concepts, electrical impacts, and fabrication challenges, and
emphasizes the educational value of integrating CFET into semiconductor
curricula as an example of an open, unresolved technology challenge.
\end{abstract}

\begin{IEEEkeywords}
CFET, GAA, FinFET, nanosheet FET, scaling, semiconductor education,
tutorial, vertical stacking, PDK.
\end{IEEEkeywords}

\section{Introduction}
Scaling of CMOS devices has followed Moore's law for decades, enabled
by planar CMOS, FinFET, and most recently Gate-All-Around (GAA)
nanosheet FETs. However, beyond the 2nm node, interconnect delay and
cell area become critical bottlenecks. CFET (Complementary FET) has
emerged as a candidate solution, stacking nFET and pFET vertically so
that the cross-section itself forms a CMOS inverter. This paper
positions CFET as both a technological roadmap element and an
educational opportunity to expose students and engineers to open
research challenges.

\section{Device Evolution}
The historical trajectory of CMOS devices is illustrated in
Fig.~\ref{fig:evolution}.
\begin{itemize}
\item Planar CMOS: dominant until 90nm node
\item FinFET: 3D fins, improved control (22nm and beyond)
\item GAA Nanosheet: full gate wrap, scalable to 2nm
\item CFET: vertical n/p stacking, projected beyond 2030
\end{itemize}

\section{CFET Structural Concepts}
Two main structural variants exist:
\begin{enumerate}
\item Sequential CFET: bottom nFET + top pFET, formed sequentially
with strict thermal budget management.
\item Forksheet CFET: orthogonal layout of n/p nanosheets, reducing
routing congestion.
\end{enumerate}
Both approaches share the concept of integrating n/p devices in the
vertical dimension to double effective cell density.

\section{Electrical Features and Design Impacts}
Key educational aspects include:
\begin{itemize}
\item \textbf{Area Efficiency:} Standard cell area nearly halved.
\item \textbf{Delay Reduction:} n/p direct vertical connection reduces
RC delay.
\item \textbf{Symmetry:} Improved inverter balance due to stacked
structure.
\item \textbf{New Parasitics:} Vertical vias introduce variability and
resistance challenges.
\end{itemize}

\section{Manufacturing Challenges}
CFET fabrication requires:
\begin{itemize}
\item Independent doping control for stacked n/p channels.
\item Thermal budget separation: top devices must be fabricated under
low temperature.
\item Precise selective epitaxy and etching.
\item BEOL integration with vertical vias for VDD/GND and output.
\end{itemize}

\section{Modeling and EDA Limitations}
Currently, no compact models fully support CFET:
\begin{itemize}
\item BSIM-CMG: up to GAA only.
\item BSIM-BULK: planar MOSFETs only.
\item Verilog-A level research models exist, but no standard.
\item No CFET-ready PDK libraries available for EDA tools.
\end{itemize}
From an educational perspective, this limitation is valuable: students
learn that modeling and design environments often lag behind device
innovation.

\section{Educational Value}
CFET offers unique teaching opportunities:
\begin{itemize}
\item Illustrates the direct link between device scaling and circuit
design.
\item Provides students with exposure to \textit{unsolved} challenges.
\item Enables cross-disciplinary learning (device, process, CAD).
\item Positions learners to think about semiconductor technology in the
2030s.
\end{itemize}

\section{Conclusion and Outlook}
CFET is a concept where the CMOS inverter is realized in the
cross-section. While still in research, it provides an invaluable
educational case study for illustrating unresolved semiconductor
challenges. Forksheet and 3D-CFET extensions, as well as integration
with system-on-stack (SoS) architectures, will likely define the next
decade of transistor innovation. Incorporating CFET into educational
curricula can help bridge device physics, circuit design, and system
integration for future engineers.

\section*{Acknowledgment}
The author thanks the Project Design Hub community for discussions and
support in developing educational resources on advanced semiconductor
nodes.

\begin{thebibliography}{1}
\bibitem{imec2022}
IMEC, ``Complementary FETs for sub-1nm nodes,'' in \textit{IEDM Tech.
Digest}, 2022.

\bibitem{intel2023}
Intel Labs, ``Research progress on CFET integration,'' in \textit{VLSI
Symposium}, 2023.

\bibitem{ibm2021}
IBM Research, ``Gate-all-around nanosheets and beyond,'' \textit{Nature
Electronics}, vol. 4, pp. 456--463, 2021.

\end{thebibliography}

\end{document}
