\documentclass[conference]{IEEEtran}

% ---- Fonts & math (LuaLaTeX対応) ----
\usepackage{newtxtext,newtxmath}

% ---- Graphics & color ----
\usepackage{graphicx}
\usepackage[dvipsnames]{xcolor}

% ---- Links / \url (← 追加) ----
\usepackage[hidelinks]{hyperref}

% ---- Spacing around floats (tight for 2 pages) ----
\setlength{\textfloatsep}{5pt plus 1pt minus 1pt}
\setlength{\floatsep}{5pt plus 1pt minus 1pt}
\setlength{\intextsep}{5pt plus 1pt minus 1pt}

% ---- Tables ----
\usepackage{booktabs}

% ---- Float control ----
\usepackage{placeins}

% ---- TikZ ----
\usepackage{tikz}
\usetikzlibrary{
  arrows.meta,
  positioning,
  fit,
  calc,
  shapes.geometric,
  shapes.misc
}

% ---- TikZ styles (NOTE: step -> flowstep to avoid key clash) ----
\tikzset{
  line/.style={-Latex, line width=0.45pt},
  box/.style={draw, rounded corners=2pt, align=center, inner sep=3pt, minimum height=7.2mm},
  smallbox/.style={box, minimum width=26mm},
  midbox/.style={box, minimum width=32mm},
  bigbox/.style={box, minimum width=64mm},
  flowstep/.style={box, minimum width=30mm} % <— was 32mm; a bit tighter
}

% ---- Convenience ----
\newcommand{\etal}{\textit{et~al.}}

% =========================================================
\begin{document}

\title{AITL on Space: A Robust Three-Layer Architecture\\
with a Tri-NVM Hierarchy (SRAM / MRAM / FRAM)\\
for Long-Duration Spacecraft Autonomy}

\author{\IEEEauthorblockN{Shinichi Samizo}\\
\IEEEauthorblockA{Independent Semiconductor Researcher\\
Former Engineer at Seiko Epson Corporation\\
Email: shin3t72@gmail.com\quad GitHub: \url{https://github.com/Samizo-AITL}}
}

\maketitle

\begin{abstract}
\noindent
We propose \textit{AITL on Space}, a robust three-layer control architecture (Robust Core, FSM Supervisor, AI Adaptor) integrated with a tri-NVM hierarchy (SRAM/MRAM/FRAM) and mapped to a 22\,nm FDSOI SoC. The contribution is a complete end-to-end design flow from mission-level specification to ASIC: requirements are formalized as JSON via EduController, synthesized by the AITL-H module, validated in FPGA HIL with fault injection, stress-tested through SystemDK FEM (thermal/radiation/packaging), and finally implemented as ASIC. This methodology enables resilient autonomy for long-duration spacecraft missions.
\end{abstract}

\section{Introduction}
Deep-space missions require ultra-robust control under total ionizing dose (TID), single event effects (SEE), and thermal cycling. Conventional PID\,+\,Flash architectures face lifetime limits due to charge-trap drift and endurance. We present \emph{AITL on Space}: a resilient three-layer architecture with a tri-NVM hierarchy and a reproducible design flow from specification to ASIC.

\section{Specification and Design Flow}
The process begins with \textbf{Mission Specification}. Requirements such as pointing accuracy, power stability, and thermal tolerance are captured in \textbf{EduController}, a model-based tool that exports plant matrices and weighting functions as JSON (portable across simulators). The JSON is consumed by \textbf{AITL-H}, which synthesizes an $H_\infty$ output-feedback controller $K$ with mixed-sensitivity weighting and generates a fixed-point implementation for RTL/FPGA/ASIC. The design then undergoes:
\begin{itemize}[itemsep=0pt,topsep=1pt,leftmargin=*]
  \item \textbf{FPGA HIL}: hardware-in-the-loop validation with SEU\,+\,outage injection; metrics include safe-mode entry $<\!1$\,s, recovery rate $\ge 99\%$, and ECC scrubbing efficiency.
  \item \textbf{SystemDK FEM}: co-simulation of thermal cycles, radiation effects, and packaging stress, closing the verification loop before silicon.
  \item \textbf{ASIC Mapping}: implementation on GlobalFoundries 22FDX FDSOI hardened for long-duration missions.
\end{itemize}
\textit{Toolchain at a Glance}. \emph{EduController} $\rightarrow$ spec\,$\rightarrow$\,JSON exporter (plant $A,B,C,D$, noise/disturbance models, weights $W_1,W_2,W_3$). \emph{AITL-H}: $H_\infty$ synthesizer (Riccati/LMI)\,$\rightarrow$\,fixed-point $K$. \emph{SystemDK FEM} = thermal/radiation/packaging derating \& memory scrubbing validation.

\section{System Architecture}
AITL consists of three layers:
\begin{itemize}[itemsep=0pt,topsep=1pt,leftmargin=*]
  \item \textbf{Robust Core}: $H_\infty$/MPC/SMC controllers for stability.
  \item \textbf{FSM Supervisor}: mode switching (Safe/Nominal/Recovery) with FDI/FDI\!I for fault management.
  \item \textbf{AI Adaptor}: long-term re-identification and drift compensation.
\end{itemize}
A tri-NVM hierarchy ensures persistence: SRAM for execution, MRAM for logs/code with ECC scrubbing and dual slots, and FRAM for safe boot and FSM states. Target SoC is 22\,nm FDSOI hardened for radiation and temperature stress.

\section{Mathematical Model and $H_\infty$ Design}
We consider an 11D discrete-time state-space plant coupling attitude (6), power bus (2), and thermal nodes (3):
\begin{align}
x_{k+1} &= A x_k + B u_k + E w_k,\\
y_k     &= C x_k + D u_k + v_k,
\end{align}
where $w_k$ and $v_k$ are disturbance and noise. The model extends to 20D by adding translational axes and bias states. Weights $(W_1,W_2,W_3)$ shape sensitivity, control effort, and complementary sensitivity. \textit{EduController} outputs them as JSON; \textit{AITL-H} synthesizes $K$ with robustness margins and exports a fixed-point realization for RTL/FPGA/ASIC.

\section{Verification Pipeline}
FPGA HIL injects SEUs and sensor outages. Metrics include safe-mode entry time ($<\!1$\,s), recovery rate ($\ge 99\%$), and ECC scrubbing efficiency. \textit{SystemDK FEM} validates thermal and radiation stress, ensuring packaging reliability before ASIC tape-out.

\section{Conclusion}
\textit{AITL on Space} combines robust control, supervisory safety, AI re-identification, and hardened memory. The proposed end-to-end flow—from mission specification to ASIC—provides a reproducible methodology for resilient autonomy in long-duration space missions.

% Balance last page columns around ref. 3 (tune as needed)
\IEEEtriggeratref{3}

\begin{thebibliography}{4}
\bibitem{doyle} J.~C. Doyle, B.~A. Francis, and A.~R. Tannenbaum, \emph{Feedback Control Theory}. Macmillan, 1992.
\bibitem{colinge} J.-P. Colinge, \emph{Silicon-on-Insulator Technology: Materials to VLSI}, 3rd ed. Springer, 2004.
\bibitem{wolf} W. Wolf, \emph{FPGA-Based System Design}. Prentice Hall, 2004.
\bibitem{rabey} J.~M. Rabaey, A. Chandrakasan, and B. Nikolić, \emph{Digital Integrated Circuits: A Design Perspective}, 2nd ed. Prentice Hall, 2003.
\end{thebibliography}

\section*{Author Biography}
\begingroup\small
Shinichi Samizo received the M.S. degree in Electrical and Electronic Engineering from Shinshu University, Japan. He worked at Seiko Epson Corporation as an engineer in semiconductor memory and mixed-signal device development, and contributed to inkjet MEMS actuators and PrecisionCore printhead technology. He is currently an independent semiconductor researcher focusing on process/device education, memory architecture, and AI system integration. Contact: \texttt{shin3t72@gmail.com}.
\endgroup

% --- Biography のすぐ後は「本文確定」だけにして、図はここから上に来るように ---
\FloatBarrier % ← ここまでを本文として確定(図はこの行“より前”に出る)

% ========== Fig.1: End-to-end design flow(単欄) ==========
\begin{figure}[!t]
  \centering
  \resizebox{0.96\columnwidth}{!}{%
\begin{tikzpicture}[node distance=10mm,
  every node/.style={font=\footnotesize},
  line/.style={-Latex, line width=0.5pt},
  box/.style={draw, rounded corners=2pt, align=center, inner sep=3pt, minimum height=7.2mm},
  smallbox/.style={box, minimum width=26mm}
]
  \tikzset{title/.style={font=\bfseries, inner sep=0pt}}

  \node[smallbox] (edu) {EduController\\\small モデル化 \& JSON出力};
  \node[smallbox, right=12mm of edu] (json) {JSON\\\small $A,B,C,D$, $W_1,W_2,W_3$};
  \node[smallbox, right=12mm of json] (aitl) {AITL-H\\\small $H_\infty$合成 $\to$ 固定小数点$K$};
  \node[smallbox, right=12mm of aitl] (hil) {FPGA HIL\\\small SEU/欠測注入, 安全率評価};
  \node[smallbox, right=12mm of hil] (fem) {SystemDK FEM\\\small 熱/放射/実装 ストレス};
  \node[smallbox, right=12mm of fem] (asic) {ASIC\\\small 22FDX FDSOI};

  \draw[line] (edu) -- (json);
  \draw[line] (json) -- (aitl);
  \draw[line] (aitl) -- (hil);
  \draw[line] (hil) -- (fem);
  \draw[line] (fem) -- (asic);

  \node[align=center, below=6mm of hil] (metrics)
    {\small 指標:セーフモード $<\!1$\,s,回復率 $\ge 99\%$,ECC スクラビング効率};

  \node[title, above=6mm of aitl] {End-to-end Design Flow};
\end{tikzpicture}
  }
  \caption{End-to-end design flow from mission spec to ASIC.}
  \label{fig:flow}
\end{figure}

% ========== Fig.2: AITL architecture(横幅) ==========
\begin{figure*}[!t]
  \centering
  \resizebox{0.92\textwidth}{!}{%
\begin{tikzpicture}[node distance=6mm,
  every node/.style={font=\footnotesize},
  line/.style={-Latex, line width=0.5pt},
  box/.style={draw, rounded corners=2pt, align=center, inner sep=3pt, minimum height=7.2mm},
  smallbox/.style={box, minimum width=26mm},
  bigbox/.style={box, minimum width=64mm}
]
  \tikzset{
    colA/.style={fill=gray!6},
    colB/.style={fill=gray!6},
    colC/.style={fill=gray!6},
    title/.style={font=\bfseries, inner sep=1pt}
  }

  % 全体外枠
  \node[box, minimum width=160mm, minimum height=56mm, align=left] (soc) {};

  % 3レイヤ
  \node[box, colA, minimum width=48mm, minimum height=46mm, anchor=west, xshift=4mm] (robust) at (soc.west) {};
  \node[box, colB, minimum width=48mm, minimum height=46mm, right=4mm of robust] (fsm) {};
  \node[box, colC, minimum width=48mm, minimum height=46mm, right=4mm of fsm] (ai) {};

  \node[title] at ([yshift=+22mm]robust.center) {Robust Core};
  \node[title] at ([yshift=+22mm]fsm.center)   {FSM Supervisor};
  \node[title] at ([yshift=+22mm]ai.center)    {AI Adaptor};

  % Robust Core
  \node[smallbox] (rc1) at ([yshift=+6mm]robust.center) {$H_\infty$/MPC/SMC\\\small 安定化制御};
  \node[smallbox, below=4mm of rc1] (rc2) {観測器/整形重み\\\small $W_1,W_2,W_3$};
  \node[smallbox, below=4mm of rc2] (rc3) {固定小数点 $K$\\\small RTL/FPGA/ASIC};

  % FSM
  \node[smallbox] (fs1) at ([yshift=+12mm]fsm.center) {モード管理\\\small Safe/Nominal/Recovery};
  \node[smallbox, below=4mm of fs1] (fs2) {FDI/FDII\\\small 故障検出・隔離};
  \node[smallbox, below=4mm of fs2] (fs3) {セーフブート\\\small ウォッチドッグ};

  % AI
  \node[smallbox] (ai1) at ([yshift=+12mm]ai.center) {長期リID/ドリフト補償};
  \node[smallbox, below=4mm of ai1] (ai2) {パラメタ更新\\\small 低頻度演算};
  \node[smallbox, below=4mm of ai2] (ai3) {検証ゲート\\\small 安全境界内のみ反映};

  % 信号流れ
  \draw[line] (rc1.east) -- (fs1.west);
  \draw[line] (fs1.east) -- (ai1.west);
  \draw[line] (ai3.west) -- ++(-6mm,0) |- (rc3.east);

  % Tri-NVM(右端)
  \node[box, minimum width=28mm, minimum height=46mm, anchor=east, xshift=-4mm, align=center] (nvm) at (soc.east) {};
  \node[title, above=0mm of nvm] {Tri-NVM};

  \node[smallbox, minimum width=24mm] (sram) at ([yshift=+12mm]nvm.center) {SRAM\\\small 実行};
  \node[smallbox, minimum width=24mm, below=3mm of sram] (mram) {MRAM\\\small ログ/コード\\ECC};
  \node[smallbox, minimum width=24mm, below=3mm of mram] (fram) {FRAM\\\small セーフブート/FSM};

  \draw[line] (rc3.east) -- ++(4mm,0) |- (sram.west);
  \draw[line] (fs3.east) -- ++(6mm,0) |- (fram.west);
  \draw[line] (ai2.east) -- ++(10mm,0) |- (mram.west);

  \node[title] at ([yshift=+30mm]soc.north) {AITL Architecture on 22\,nm FDSOI SoC};
\end{tikzpicture}
  }
  \caption{AITL architecture: three layers with a tri-NVM hierarchy.}
  \label{fig:arch}
\end{figure*}

% ========== Fig.3: Closed-loop(単欄) ==========
\begin{figure}[!t]
  \centering
  \resizebox{0.90\columnwidth}{!}{%
\begin{tikzpicture}[node distance=10mm,
  every node/.style={font=\footnotesize},
  line/.style={-Latex, line width=0.5pt},
  box/.style={draw, rounded corners=2pt, align=center, inner sep=3pt, minimum height=7.2mm},
  smallbox/.style={box, minimum width=22mm}
]
  \tikzset{
    sum/.style={draw, circle, inner sep=1.2pt, minimum size=3.6mm},
    port/.style={font=\small}
  }
  \node[smallbox] (K) {$K$};
  \node[smallbox, right=22mm of K, minimum width=26mm] (P) {$P$};
  \node[sum, left=10mm of K] (sumin) {$\Sigma$};
  \node[sum, right=12mm of P] (sumy) {$\Sigma$};

  \draw[line] (sumin) -- (K);
  \draw[line] (K) -- node[above, port] {$u$} (P);
  \draw[line] (P) -- node[above, port] {$y$} (sumy);
  \draw[line] (sumy) |- ++(0,-12mm) -| node[pos=0.25, below, port] {$-$} (sumin);

  \node[left=12mm of sumin] (r) {};
  \draw[line] (r.center) -- node[above, port] {$r$} (sumin);

  \node[above=10mm of P] (w) {};
  \draw[line] (w.center) -- node[right, port] {$w$} (P.north);

  \node[below=10mm of sumy] (v) {};
  \draw[line] (v.center) -- node[right, port] {$v$} (sumy.south);

  \node[smallbox, below=8mm of P] (W) {$W$};
  \draw[line] (P.south) -- (W.north);
  \node[right=14mm of W] (z) {};
  \draw[line] (W) -- node[above, port] {$z$} (z.center);

  \node[align=center, below=14mm of K] {設計:$H_\infty$ で $\|T_{w\to z}\|_\infty$ を最小化};
\end{tikzpicture}
  }
  \caption{Closed-loop structure used for robust design.}
  \label{fig:closed}
\end{figure}

\end{document}
