\section{BEOL(配線技術)の進化}
デバイススケーリングの進展と並行して、BEOL(Back End of Line)技術も著しく発展してきた。  
微細化に伴い、配線抵抗と配線間容量が増大し、RC遅延がシステム性能を律速する要因となったため、  
低誘電率材料(Low-$k$ dielectric)と高導電率金属の導入が進められた。

初期のアルミニウム配線から銅(Cu)配線への転換により、導電率が約40\%改善され、  
Dual Damascene プロセスによって高アスペクト比配線の形成が可能となった。  
このプロセスでは、ビアとラインを一体的にエッチングし、Cuを化学機械研磨(CMP)で平坦化することで、  
多層配線の高密度実装と優れた表面平坦性を両立している。

誘電体材料についても、SiO$_2$からSiOC系Low-$k$、さらには極低誘電率(Ultra Low-$k$)材料へと進化した。  
ただし、低密度化による機械的強度低下や水分吸収による信頼性低下が課題であり、  
近年ではエアギャップ構造やカーボン含有絶縁膜など、誘電率と機械強度の最適化を両立する研究が進められている。

7\,nm世代以降では、電源供給経路をシリコン裏面に再配置する  
Backside Power Rail(BPR)アーキテクチャが導入されつつある。  
BPRは、ロジック層上面の配線混雑を大幅に緩和し、信号配線の自由度とセル高さ縮小を同時に実現する。  
また、CFETのような垂直積層構造と組み合わせることで、  
前工程(FEOL)と後工程(BEOL)の電気的境界が曖昧化し、  
「デバイス—配線一体設計(Device-Interconnect Co-Optimization; DICO)」の重要性が高まっている。

さらに、従来のCu配線では微細化に伴うバリア層比率増大が抵抗上昇を招くため、  
ルテニウム(Ru)やコバルト(Co)といった次世代導電材料の適用が検討されている。  
これらはバリア不要配線(barrier-less interconnect)を可能にし、  
原子スケール領域での配線信頼性(Electromigration耐性)を改善する有力候補である。

このように、BEOL技術は単なる“配線”の領域を超え、  
電源分離、熱拡散、材料信頼性を含むシステム統合技術へと進化している。  
次章では、この配線スケーリングを支える電気的パラメータおよび構造スケーリング法則について検討する。
