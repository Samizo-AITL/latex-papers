\section{BEOL(配線技術)の進化}
デバイススケーリングの進展と並行して、BEOL(Back End of Line)技術も著しく発展してきた。  
トランジスタ性能の向上が配線遅延によって相殺される「RCボトルネック」が顕在化し、  
配線抵抗および配線間容量の削減がシステム性能を支配する時代へと移行した。  
これに対応するため、低誘電率材料(Low-$k$ dielectric)および高導電率金属の導入が進められてきた。

初期のアルミニウム配線から銅(Cu)配線への転換により、導電率が約40\%向上し、  
Dual Damasceneプロセスによって高アスペクト比配線の形成が可能となった。  
このプロセスでは、ビアおよびラインを同時にエッチングし、Cu充填後に化学機械研磨(CMP)で平坦化することで、  
多層配線の高密度実装と優れた表面平坦性を両立している。  
また、バリアメタル(Ta/TaN)の最適化により、エレクトロマイグレーション(EM)およびストレスマイグレーション耐性が大幅に向上した。

誘電体材料についても、SiO$_2$からSiOC系Low-$k$、さらにUltra Low-$k$(ULK)材料へと進化している。  
ただし、低密度化に伴う機械的強度低下や水分吸収による信頼性劣化が課題であり、  
近年ではエアギャップ構造やカーボン含有絶縁膜(SiOC:H, SiCN系)を利用した  
「誘電率と機械強度の同時最適化」が研究の主流となっている。  

7\,nm世代以降では、電源供給経路をシリコン裏面側に再配置する  
Backside Power Rail(BPR)アーキテクチャが導入されつつある。  
BPRは、ロジック層上面の配線混雑を緩和し、信号配線の自由度を拡大することで  
セル高さのさらなる縮小とIRドロップの低減を同時に実現する。  
また、CFETのような垂直積層デバイスと組み合わせることで、  
前工程(FEOL)と後工程(BEOL)の境界が曖昧化し、  
「デバイス–配線一体最適化(Device–Interconnect Co-Optimization; DICO)」が不可欠な設計指針となっている。

一方、Cu配線では微細化に伴うバリア層比率の増大が抵抗上昇を引き起こすため、  
ルテニウム(Ru)やコバルト(Co)などの次世代導電材料が注目されている。  
これらはバリアレス配線(barrier-less interconnect)を可能とし、  
界面拡散の抑制とEM耐性の両立を実現する有力候補である。  
さらに、カーボンナノチューブ(CNT)やグラフェン配線など、  
原子スケールでの新規配線材料も研究段階にある。

このように、BEOL技術は単なる「配線形成プロセス」から、  
電源分離、熱拡散、機械信頼性、そしてシステム全体最適化を担う統合技術へと進化している。  
次章では、この配線スケーリングを支える主要パラメータおよび構造スケーリング法則について詳述する。

\begin{table}[t]
  \centering
  \caption{BEOL(配線層)技術のスケーリング推移}
  \label{tab:beol_scaling}
  \begin{tabular}{lcccc}
    \toprule
    ノード & 配線材 & 絶縁材 & 最小ピッチ (nm) & 電源構造 \\
    \midrule
    65\,nm  & Cu & SiO$_2$ & 200 & Frontside Power \\
    28\,nm  & Cu & Low-k & 100 & Dual Damascene \\
    7\,nm   & Cu & Low-k+Co Liner & 40 & Power Grid \\
    3\,nm   & Ru/Cu & Porous Low-k & 28 & Backside Power Rail \\
    2\,nm   & Ru & Airgap Low-k & 24 & BPR + TSV Feed \\
    \bottomrule
  \end{tabular}
\end{table}

% =====================================================
% Fig. 8 : BEOL構造比較(Dual Damascene vs BPR)💯版
% =====================================================
\begin{figure}[t]
  \centering
  \tikzset{
    diel/.style   ={fill=black!5, draw=black, line width=0.3pt},
    metal/.style  ={fill=black!35, draw=black, line width=0.3pt},
    barrier/.style={fill=black!55, draw=black, line width=0.3pt},
    note/.style   ={font=\footnotesize, align=center},
    label/.style  ={font=\footnotesize}
  }

  %==============================
  % (a) Dual Damascene - Frontside Power
  %==============================
  \begin{tikzpicture}[scale=0.9]
    \node[label, anchor=west] at (-0.1,3.35) {\textbf{(a) Dual Damascene(前面電源配線)}};
    
    % Dielectric stack
    \draw[diel] (0,0) rectangle (4.0,3.0);
    
    % Cu lines with barrier
    \foreach \x in {0.4,1.8,3.2}{
      \draw[barrier] (\x,1.8) rectangle (\x+0.4,2.6);
      \draw[metal]   (\x+0.05,1.85) rectangle (\x+0.35,2.55);
    }

    % Vias connecting layers
    \draw[barrier] (0.55,0.6) rectangle (0.85,1.8);
    \draw[metal]   (0.6,0.65) rectangle (0.8,1.75);
    \draw[barrier] (3.35,0.6) rectangle (3.65,1.8);
    \draw[metal]   (3.4,0.65) rectangle (3.6,1.75);

    % Substrate (context hint)
    \fill[black!15] (0,0) rectangle (4.0,0.4);
    \draw[black] (0,0) rectangle (4.0,0.4);
    \node[label] at (2,0.2) {Si Substrate};

    % Note text
    \node[note] at (2,0.0) {信号と電源が同層で競合 → 配線混雑};
  \end{tikzpicture}

  \vspace{1.5ex}

  %==============================
  % (b) Backside Power Rail (BPR)
  %==============================
  \begin{tikzpicture}[scale=0.9]
    \node[label, anchor=west] at (-0.1,3.6) {\textbf{(b) Backside Power Rail(裏面電源分離)}};

    % Signal dielectric (frontside)
    \draw[diel] (0,0.8) rectangle (4.0,3.2);

    % Frontside signal lines
    \foreach \x in {0.7,2.0,3.3}{
      \draw[metal] (\x,2.2) rectangle (\x+0.35,3.0);
    }
    \node[label, anchor=west] at (3.5,2.7) {信号配線};

    % Substrate
    \draw[fill=black!10, draw=black, line width=0.3pt] (0,0.2) rectangle (4.0,0.8);
    \node[label] at (2,0.05) {Si Substrate};

    % Backside Power Rails
    \draw[metal] (0.3,-0.2) rectangle (1.9,0.2);
    \draw[metal] (2.1,-0.2) rectangle (3.7,0.2);
    \node[label] at (1.1,-0.35) {BPR-VDD};
    \node[label] at (2.9,-0.35) {BPR-VSS};

    % TSV-like power vias
    \foreach \x in {0.6,3.4}{
      \draw[metal] (\x,0.2) rectangle (\x+0.25,0.8);
    }

    % Note text
    \node[note] at (2.0,0.45) {裏面電源でセル上面が空き\\配線自由度・性能が向上};
  \end{tikzpicture}

  \caption{BEOL配線技術の概念比較図。\\
  \footnotesize (a) Dual Damasceneでは電源・信号が同一面に配置され配線混雑が発生。\\
  (b) BPR構造では電源を裏面に再配置し、セル上部の配線密度と設計自由度を向上させる。}
  \label{fig:beol_bpr}
\end{figure}

