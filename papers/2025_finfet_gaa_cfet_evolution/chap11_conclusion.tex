\section{結論}
本稿では、130\,nm以降のCMOSスケーリングを、FinFET、GAA、そしてCFETに至る構造的進化の観点から体系的に整理した。  
プレーナーMOSの電界制御限界を突破したFinFET、完全包囲ゲート構造による静電的最適化を実現したGAA、  
そして垂直積層により論理対称性と配線効率を両立したCFETへと至る流れは、  
スケーリングの中心軸が「寸法の縮小」から「構造の最適化」へと移行したことを明確に示している。

この構造的転換は、電界・熱・信頼性といった多次元設計パラメータを同時に最適化する  
“Structure-Driven Scaling” の概念を確立したものである。  
特に、High-$k$/Metal Gate(HKMG)、Backside Power Rail(BPR)、およびBSIM-CMGモデルの統合は、  
プロセス・デバイス・回路設計を貫く一貫した設計基盤を提供し、  
信頼性と性能を両立する「構造的CMOSアーキテクチャ」を実現した。

今後のポスト-CFET時代においては、スケーリングの焦点は  
材料・構造・AIによる設計統合(AI-driven Co-Optimization)へと拡張される。  
熱対称性や電源分離構造の自動最適化、信頼性の予測設計、そして  
マルチフィジックスを考慮した構造シミュレーションが標準化されることで、  
デバイス開発は「試作依存型」から「構造駆動型知能設計」へと進化するだろう。

すなわち、**構造そのものが信頼性と性能を設計する時代**が到来しており、  
FinFET–GAA–CFETの進化はその第一歩である。  
本稿で提示した体系的整理は、次世代CMOSスケーリングにおける  
物理・設計・AI統合の指針としての基盤を提供するものである。
