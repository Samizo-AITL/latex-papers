\section{GAA(Gate-All-Around)構造}
GAA(Gate-All-Around)構造は、チャネルを全周囲からゲート電極で包み込むことにより、静電制御性を極限まで高めたデバイス構造である。  
従来のFinFETではゲートが三方向からチャネルを制御していたのに対し、GAAはチャネルの上下方向にもゲート電界が作用するため、  
チャネルポテンシャルの均一性が飛躍的に向上し、ドレイン電界の侵入がほぼ完全に抑制される。  
この結果、短チャネル効果(SCE)およびドレイン誘起バリア低下(DIBL)がさらに低減され、  
しきい値電圧$V_{\mathrm{th}}$の安定性とサブスレッショルドスイング(SS)の改善が達成される。

ナノシート型GAA構造では、複数のチャネル層(Nanosheet)を垂直方向に積層し、それぞれを独立にゲートで包囲する。  
この積層構造により、面積効率を維持しつつ実効チャネル幅を拡大できる。  
有効チャネル幅$W_{\mathrm{eff}}$はFinFET構造に対して次式で表される:
\begin{equation}
W_{\mathrm{eff}} = 2n(H + W),
\end{equation}
ここで$n$はチャネル層数、$H$はシート厚、$W$はチャネル幅である。  
この式が示すように、GAAではFinFETと異なり上下両面の電流経路が追加されるため、同一フットプリントに対してより高い駆動能力を実現できる。

さらに、GAA構造は「構造的なスケーラビリティ」を備えており、チャネル幅$W$を減少させつつ層数$n$を増やすことで、  
電気的特性を維持したまま幾何学的スケーリングを継続できる。  
この特性により、IMECやSamsungをはじめとする研究機関・メーカーでは、5\,nm世代以降でのGAA採用を加速させ、  
1.4\,nmクラスのノードにおいてもチャネル制御性と電流駆動性能の両立が実現されつつある。

一方で、ナノシートの積層・分離工程には高い製造精度が要求される。  
層間酸化膜の厚さ均一性、ゲート堆積時の段差被覆性、およびチャネル層のエッチング選択性が信頼性に直結する。  
これらの課題に対処するため、Selective Epitaxial Growth(SEG)によるチャネル形成や、Atomic Layer Deposition(ALD)によるゲート絶縁膜制御技術が導入されている。

GAAは、FinFETの静電制御を限界まで拡張した「完全電界包囲デバイス」として、次世代スケーリングの中核技術となっている。  
次章では、n型およびp型デバイスを垂直方向に積層したCFET(Complementary FET)構造について、その統合設計思想と課題を論じる。

\begin{table}[t]
  \centering
  \caption{GAA(Gate-All-Around)ナノシート構造の代表パラメータ}
  \label{tab:gaa_params}
  \begin{tabular}{lccc}
    \toprule
    パラメータ & 記号 & 代表値 & 備考 \\
    \midrule
    ナノシート層数 & $n$ & 3–5 & 積層により$W_\mathrm{eff}$増大 \\
    シート厚 & $T_s$ & 5–8\,nm & チャネル厚制御が性能支配 \\
    シート間距離 & $S$ & 10–12\,nm & 熱干渉と寄生容量に影響 \\
    有効チャネル幅 & $W_\mathrm{eff}$ & $2n(H+W)$ & 全包囲ゲート寄与 \\
    しきい値電圧 & $V_\mathrm{th}$ & 0.35–0.4\,V & 対称制御が容易 \\
    サブスレッショルドスイング & $SS$ & 60–65\,mV/dec & 理想スイングに近似 \\
    電流駆動比(Fin比) & $I_\mathrm{ON}/I_\mathrm{OFF}$ & 2–3× & FinFET比 \\
    \bottomrule
  \end{tabular}
\end{table}

\begin{table*}[t]
  \centering
  \caption{CMOSスケーリングにおけるプロセス構造の進化(Process Evolution of CMOS Scaling)}
  \label{tab:process_evolution}
  \scriptsize
  \setlength{\tabcolsep}{4pt}
  \renewcommand{\arraystretch}{1.1}
  \begin{tabular}{ccccccc}
    \toprule
    ノード & 構造 & 電源電圧[V] & $T_\mathrm{ox}$[nm] & Min L[nm] & 主な特徴 & 技術課題 \\
    Node & Structure & $V_\mathrm{DD}$[V] & $T_\mathrm{ox}$[nm] & Min L[nm] & Key Features & Challenges \\
    \midrule
    90\,nm & プレーナMOS & 1.2 & $\sim$2.0 & $\sim$65 & NiSi導入、Strained-Si、LDD最適化 & リーク電流、寄生容量、リソグラフィ限界 \\
           & Planar MOS &     &              &           & NiSi, strained-Si, optimized LDD & Leakage, parasitics, lithography \\[2pt]
    45\,nm & プレーナMOS & 1.0 & $\sim$1.3 & $\sim$35 & HKMG導入準備、ULK試験導入 & ゲート制御限界、ばらつき拡大 \\
           & Planar MOS &     &             &          & HKMG prep, ULK intro & Gate control limit, variability \\[2pt]
    22\,nm & FinFET初代 & 0.85 & $\sim$0.9 & $\sim$20 & Tri-Gate構造採用、3Dチャネル化 & Finばらつき、設計難度増加 \\
           & 1st Gen FinFET & & & & Tri-Gate, 3D channel & Fin variation, design complexity \\[2pt]
    5\,nm & GAA導入 & 0.6 & $\sim$0.6 & $\sim$8 & Nanosheet構造試験導入 & Routing困難、シート幅制御 \\
           & GAA Pilot & & & & Nanosheet trials & Sheet width control, poor routability \\[2pt]
    2\,nm & CFET試作 & $\lesssim$0.5 & $\sim$0.4 & $\sim$4 & NMOS/PMOS縦積層化 & 熱干渉、配線分離難 \\
           & CFET (R\&D) & & & & Complementary FET stacking & Thermal interference, power routing split \\
    \bottomrule
  \end{tabular}
\end{table*}

