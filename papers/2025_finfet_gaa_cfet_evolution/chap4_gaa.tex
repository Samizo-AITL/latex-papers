\section{GAA(Gate-All-Around)構造}
GAA(Gate-All-Around)構造は、チャネルを全周囲からゲート電極で包み込むことにより、静電制御性を極限まで高めたデバイス構造である。  
従来のFinFETではゲートが三方向からチャネルを制御していたのに対し、GAAでは上下方向にもゲート電界が作用するため、  
チャネルポテンシャルの均一性が飛躍的に向上し、ドレイン電界の侵入がほぼ完全に抑制される。  
これにより、短チャネル効果(SCE)およびドレイン誘起バリア低下(DIBL)がさらに低減され、  
しきい値電圧$V_{\mathrm{th}}$の安定性とサブスレッショルドスイング(SS)の改善が同時に達成される。

ナノシート型GAA構造では、複数のチャネル層(Nanosheet)を垂直方向に積層し、それぞれを独立にゲートで包囲する。  
この積層構造により、平面面積を増加させずに実効チャネル幅を拡大できる。  
有効チャネル幅$W_{\mathrm{eff}}$はFinFET構造に対して次式で表される:
\begin{equation}
  W_{\mathrm{eff}} = 2n(H + W),
\end{equation}
ここで$n$はチャネル層数、$H$はシート厚、$W$はチャネル幅である。  
この式が示すように、GAAではFinFETと異なり上下両面に電流経路が追加されるため、  
同一フットプリントでより高い駆動能力を実現できる。

さらにGAA構造は「構造的スケーラビリティ」を備えており、チャネル幅$W$を減少させながら層数$n$を増やすことで、  
電気的特性を維持したまま幾何学的スケーリングを継続できる。  
この特性により、IMECやSamsungをはじめとする研究機関・メーカーでは、5\,nm世代以降でのGAA採用を加速させ、  
1.4\,nmクラスのノードにおいてもチャネル制御性と電流駆動性能の両立が実現されつつある。

一方で、ナノシートの積層・分離工程には極めて高い製造精度が要求される。  
層間酸化膜の厚さ均一性、ゲート堆積時の段差被覆性、およびチャネル層のエッチング選択性は、  
いずれもデバイス信頼性に直結する重要因子である。  
これらの課題に対処するため、Selective Epitaxial Growth(SEG)によるチャネル形成や、  
Atomic Layer Deposition(ALD)による高均一絶縁膜堆積技術が適用されている。

GAAは、FinFETの静電制御性を極限まで拡張した「完全電界包囲デバイス」として、  
次世代スケーリングの中核技術を形成している。  
次章では、n型およびp型デバイスを垂直方向に積層したCFET(Complementary FET)構造について、  
その統合アーキテクチャと設計上の課題を論じる。

\begin{table}[t]
  \centering
  \caption{GAA(Gate-All-Around)ナノシート構造の代表パラメータ}
  \label{tab:gaa_params}
  \begin{tabular}{lccc}
    \toprule
    パラメータ & 記号 & 代表値 & 備考 \\
    \midrule
    ナノシート層数 & $n$ & 3–5 & 積層により$W_\mathrm{eff}$増大 \\
    シート厚 & $T_s$ & 5–8\,nm & チャネル厚制御が性能支配 \\
    シート間距離 & $S$ & 10–12\,nm & 熱干渉と寄生容量に影響 \\
    有効チャネル幅 & $W_\mathrm{eff}$ & $2n(H+W)$ & 全包囲ゲート寄与 \\
    しきい値電圧 & $V_\mathrm{th}$ & 0.35–0.4\,V & 対称制御が容易 \\
    サブスレッショルドスイング & $SS$ & 60–65\,mV/dec & 理想スイングに近似 \\
    電流駆動比(Fin比) & $I_\mathrm{ON}/I_\mathrm{OFF}$ & 2–3× & FinFET比 \\
    \bottomrule
  \end{tabular}
\end{table}

% =====================================================
% Fig. 6 : GAA(Gate-All-Around)ナノシート構造 断面図(改訂版・💯)
% =====================================================
\begin{figure}[t]
  \centering
  \begin{tikzpicture}[scale=1.0, every node/.style={font=\footnotesize}]
    % ----- Substrate -----
    \fill[gray!25] (0,0) rectangle (4.0,0.4);
    \draw[black] (0,0) rectangle (4.0,0.4);
    \node[below] at (2.0,0) {Si Substrate};

    % ----- Gate stack (outer boundary) -----
    \draw[thick,fill=cyan!15] (0.5,0.5) rectangle (3.5,1.9);
    \node[right] at (3.55,1.2) {Gate-All-Around};

    % ----- Nanosheet channels (3層) -----
    \foreach \y in {0.65,1.05,1.45} {
      \fill[orange!35] (0.9,\y) rectangle (3.1,\y+0.18);
      \draw[black!60] (0.9,\y) rectangle (3.1,\y+0.18);
    }

    % ----- Gate oxide (thin layer hint) -----
    \draw[gray!60,densely dashed,line width=0.3pt] (0.85,0.6) rectangle (3.15,1.68);
    \node at (2.0,1.75) {Gate dielectric (SiO$_2$/High-$k$)};

    % ----- Source / Drain regions -----
    \fill[gray!60] (0.0,0.6) rectangle (0.5,1.8);
    \fill[gray!60] (3.5,0.6) rectangle (4.0,1.8);
    \draw[black] (0.0,0.6) rectangle (0.5,1.8);
    \draw[black] (3.5,0.6) rectangle (4.0,1.8);
    \node[left]  at (0.0,1.2) {Source};
    \node[right] at (4.0,1.2) {Drain};

    % ----- Dimension arrows -----
    \draw[-{Latex[length=2mm]}, line width=0.3pt] (0.9,0.55) -- (0.9,1.65)
      node[midway,left=2pt] {$T_s$};
    \draw[-{Latex[length=2mm]}, line width=0.3pt] (0.9,1.85) -- (3.1,1.85)
      node[midway,above=2pt] {$W$};

    % ----- Annotation for nanosheet stack -----
    \node at (2.0,0.9) {Nanosheet Channels ($n=3$)};
  \end{tikzpicture}

  \caption{GAAナノシート構造の模式断面図(S/D領域とゲート包囲構造を示す)\\
  \footnotesize Schematic cross-section of a GAA nanosheet FET showing gate wrapping around vertically stacked channels.}
  \label{fig:gaa_cross}
\end{figure}

