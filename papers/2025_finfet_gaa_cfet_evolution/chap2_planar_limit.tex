\section{プレーナーMOS構造の限界}
プレーナーMOSFETでは、チャネル長の短縮に伴い、ドレイン電界がチャネル領域へ深く侵入する短チャネル効果(SCE: Short Channel Effect)が顕著となる。  
その結果、しきい値電圧($V_{\mathrm{th}}$)の低下、ドレイン誘起バリア低下(DIBL: Drain-Induced Barrier Lowering)、  
およびサブスレッショルド領域における漏れ電流特性の劣化が発生する。  
これらの現象は、チャネルポテンシャルの空間分布がドレイン電位に強く依存することに起因しており、  
従来の二次元電界近似による設計指針が成立しなくなったことを示している。

130\,nm世代以降では、ゲート酸化膜厚が数\si{\nano\meter}領域に達し、トンネル電流が支配的なリーク要因となった。  
酸化膜をさらに薄膜化してしきい値電圧を維持しようとすると、ゲートリーク電流が指数関数的に増大し、  
絶縁破壊耐性および信頼性が急激に低下する。  
この時期を境に、従来のDennardスケーリング則は破綻し、単なる寸法縮小では性能・信頼性を両立できなくなった。

この臨界点において登場したのが、立体的なチャネル構造を備えたFinFETである。  
FinFETは、シリコン基板上に垂直フィン(Fin)を形成し、ゲートを三方向から包み込む構造により、  
チャネル電界制御性を飛躍的に高め、SCEおよびDIBLを効果的に抑制した。  
この構造的転換は、MOSFET設計の概念を「平面上の電界制御」から「体積全体の電位制御」へと拡張するものであり、  
以後のGAA(Gate-All-Around)およびCFET(Complementary FET)に代表される  
三次元構造進化の出発点となった。
