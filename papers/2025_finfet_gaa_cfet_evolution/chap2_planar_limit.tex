\section{プレーナーMOS構造の限界}
プレーナーMOSFETでは、チャネル長短縮に伴いドレイン電界がチャネル内部へ侵入する短チャネル効果(SCE)が顕著化する。  
これにより、しきい値電圧の低下、ドレイン誘起バリア低下(DIBL)、およびサブスレッショルド特性の劣化が発生する。  
130\,nm以降では、ゲート酸化膜の薄膜化と供にゲートリークが問題となり、絶縁破壊耐性が急速に低下した。
これらの要因が、三次元構造(Fin型チャネル)への移行を促す直接的契機となった。
