% chap2_planar_limit.tex  (IEEEtran対応・💯版)
% -------------------------------------------------
% 第2章:CMOS構造進化と三次元化の展開(IEEEtranは\chapter非対応)
% -------------------------------------------------

\section{CMOS構造進化と三次元化の展開}
本章では、プレーナーMOSの電界制御限界を出発点に、FinFET、GAA、CFETへ至る立体化の系譜を整理する。構造スケーリングは、単なる寸法縮小ではなく「電界制御を立体化する」設計転換である。

\subsection{プレーナーMOS構造の限界}
プレーナーMOSFETでは、短チャネル化に伴いドレイン電界がチャネルへ侵入し、SCEやDIBLが顕著となる。130\,nm世代以降、ゲート酸化膜厚が数\si{\nano\meter}に達するとトンネルリークが支配的となり、Dennard則は破綻した。この限界を超えるため、ゲートを立体的に包み込む構造が模索され、FinFETが登場した。

\begin{figure}[t]
  \centering
  % figs/fig_planar_sce.tex
% プレーナーMOSのSCE模式図(モノクロ・1カラム適合)

\tikzset{
  gate/.style   ={pattern=north east lines, pattern color=black, draw=black},
  oxide/.style  ={fill=black!7, draw=black},
  si/.style     ={fill=white, draw=black},
  sd/.style     ={fill=black!25, draw=black},
  field/.style  ={->, line width=0.3pt},
  label/.style  ={font=\footnotesize}
}
\begin{tikzpicture}[scale=0.95]
  % Substrate
  \draw[fill=black!10,draw=black] (-0.4,0) rectangle (4.4,-0.5);
  \node[label] at (2.0,-0.7) {Si Substrate};

  % Oxide + channel slab
  \draw[oxide] (0,0) rectangle (4,1.2);   % STI/BOX-ish region
  \draw[si]    (0.4,0.2) rectangle (3.6,0.6); % channel slab

  % Gate oxide + gate
  \draw[oxide, line width=0.4pt] (0.4,0.6) rectangle (3.6,0.7);
  \draw[gate]  (0.8,0.7) rectangle (3.2,1.1);
  \node[label] at (2.0,1.25) {Gate};

  % Source / Drain
  \draw[sd] (0.0,0.2) rectangle (0.4,0.9);
  \draw[sd] (3.6,0.2) rectangle (4.0,0.9);
  \node[label] at (0.2,1.05) {S};
  \node[label] at (3.8,1.05) {D};

  % Short-channel effect: field lines from D into channel
  \foreach \y in {0.25,0.35,0.45,0.55} {
    \draw[field] (3.8,\y) .. controls (3.1,\y+0.15) and (2.6,\y+0.05) .. (2.2,\y+0.02);
  }

  % Labels
  \node[label,align=left] at (3.2,0.2) {Drain電界の侵入\\(DIBL, SCE)};
  \node[label] at (2.0,0.0) {};
\end{tikzpicture}
%
  \caption{プレーナーMOSFETの短チャネル効果(SCE)模式図。ドレイン電界の侵入により$V_\mathrm{th}$低下とDIBLが発生。}
  \label{fig:planar_sce}
\end{figure}

\subsection{FinFET:三面ゲートによる制御強化}
FinFETは、シリコンFinをゲートが三面から包囲するTri-Gate構造により、体積的な電位制御を実現する。Finの高さ$H$、幅$W$、本数$n$に対して有効チャネル幅は
\[
W_{\mathrm{eff}} = n(2H + W)
\]
で与えられ、側面電流が支配的となる。高Fin化で$I_\mathrm{ON}$を高められる一方、LERや段差被覆性が信頼性を制約する。

\subsection{GAA:全包囲ゲートへの発展}
GAAはチャネルを上下左右から完全包囲し、静電制御を極限化する。ナノシート多層化により上下両面の電流経路が追加され、同一フットプリントでの駆動力が増す:
\[
W_{\mathrm{eff}} = 2n(H + W).
\]
SEGとALDによる選択成長・膜厚制御が鍵であり、5\,nm以降の主流技術となっている。

\subsection{CFET:垂直積層と電源分離}
CFETはn/pトランジスタの垂直積層によりセル面積を大幅に削減し、BPRと統合して信号・電源の物理分離を実現する。上下FET間の熱干渉と電気的アイソレーション、$\!<\!400^\circ$C級の低温プロセスが設計・量産の要件となる。

\subsection{構造比較と設計軸}
FinFET(3面制御)→GAA(全周制御)→CFET(上下分離+BPR)の流れは、
「電界制御の立体化」→「空間効率化」→「電源分離」の深化を示す。構造そのものが性能と信頼性を同時に規定する段階に到達した。

% 必要に応じて比較図
\begin{figure}[t]
  \centering
  \begin{figure}[t]
  \centering
  \tikzset{
    gate/.style   ={pattern=north east lines, pattern color=black, draw=black},
    oxide/.style  ={fill=black!5, draw=black},
    si/.style     ={fill=white, draw=black},
    sd/.style     ={fill=black!25, draw=black},
    metal/.style  ={fill=black!40, draw=black},
    label/.style  ={font=\footnotesize},
    dim/.style    ={-{Latex[length=2mm]}, line width=0.3pt}
  }

  %=== 1) FinFET ===
  \begin{tikzpicture}[scale=0.9]
    \node[label, anchor=west] at (-0.2,3.4) {\textbf{(a) FinFET}};
    % substrate & STI
    \draw[fill=black!10, draw=black] (-0.5,0) rectangle (4.5,-0.6);
    \node[label] at (2,-0.8) {Substrate};
    \draw[oxide] (0,0) rectangle (4,1.6); % BOX/STI領域
    % fin
    \draw[si] (1.7,0) rectangle (2.3,1.4);
    % gate oxide wrapper (薄層表示)
    \draw[oxide, line width=0.4pt] (1.6,-0.02) rectangle (2.4,1.42);
    % gate metal (3面ラップ)
    \draw[gate] (1.35,0.2) rectangle (2.65,1.25);
    % source/drain
    \draw[sd] (0.25,0.2) rectangle (1.35,1.25);
    \draw[sd] (2.65,0.2) rectangle (3.75,1.25);
    \node[label] at (0.8,1.45) {S};
    \node[label] at (3.2,1.45) {D};
    \node[label] at (2.0,1.55) {Fin (Si)};
    \node[label] at (2.0,0.95) {Gate};
    % dimensions (H, W)
    \draw[dim] (2.3,0.1) -- ++(0.6,0) node[right,label] {$H$};
    \draw[dim] (1.7,0.05) -- ++(-0.6,0) node[left,label] {$H$};
    \draw[dim] (1.7,1.4) -- ++(0.6,0) node[right,label] {$W$};
  \end{tikzpicture}

  \vspace{1.2ex}

  %=== 2) GAA (Nanosheet) ===
  \begin{tikzpicture}[scale=0.9]
    \node[label, anchor=west] at (-0.2,3.4) {\textbf{(b) GAA (Nanosheet)}};
    \draw[fill=black!10, draw=black] (-0.5,0) rectangle (4.5,-0.6);
    \node[label] at (2,-0.8) {Substrate};
    \draw[oxide] (0,0) rectangle (4,1.9);
    % three sheets with gate-all-around
    \foreach \y in {0.4,0.95,1.5}{
      \draw[oxide] (1.1,\y-0.22) rectangle (2.9,\y+0.22); % spacer
      \draw[si] (1.2,\y-0.12) rectangle (2.8,\y+0.12);    % sheet
      \draw[gate] (1.0,\y-0.34) rectangle (3.0,\y+0.34);  % 4面ゲート
    }
    % S/D
    \draw[sd] (0.25,0.25) rectangle (1.0,1.65);
    \draw[sd] (3.0,0.25) rectangle (3.75,1.65);
    \node[label] at (0.7,1.85) {S};
    \node[label] at (3.3,1.85) {D};
    \node[label] at (2.0,1.95) {Gate-All-Around};
    % dims for one sheet
    \draw[-{Latex[length=2mm]}] (2.8,0.95) -- ++(0.5,0) node[right,label] {$W$};
    \draw[-{Latex[length=2mm]}] (2.6,0.83) -- ++(0, -0.35) node[below,label] {$H$};
  \end{tikzpicture}

  \vspace{1.2ex}

  %=== 3) CFET (n/p stacked) ===
  \begin{tikzpicture}[scale=0.9]
    \node[label, anchor=west] at (-0.2,3.6) {\textbf{(c) CFET (n/p 垂直積層)}};
    % BPR (backside rails)
    \draw[metal] (-0.5,-0.9) rectangle (1.6,-1.3);
    \draw[metal] (2.4,-0.9) rectangle (4.5,-1.3);
    \node[label] at (0.55,-1.45) {BPR-VDD};
    \node[label] at (3.45,-1.45) {BPR-VSS};

    % substrate and window
    \draw[fill=black!10, draw=black] (-0.5,0) rectangle (4.5,-0.6);
    \node[label] at (2,-0.8) {Substrate};
    \draw[oxide] (0,0) rectangle (4,2.6);

    % lower nFET (GAA)
    \foreach \y in {0.55}{
      \draw[gate] (1.0,\y-0.3) rectangle (3.0,\y+0.3);
      \draw[oxide] (1.1,\y-0.2) rectangle (2.9,\y+0.2);
      \draw[si] (1.2,\y-0.1) rectangle (2.8,\y+0.1);
    }
    \node[label] at (3.25,0.55) {\footnotesize nFET (GAA)};

    % ILD spacer
    \draw[oxide, fill=black!12] (0,1.05) rectangle (4,1.25);

    % upper pFET (GAA)
    \foreach \y in {1.8}{
      \draw[gate] (1.0,\y-0.3) rectangle (3.0,\y+0.3);
      \draw[oxide] (1.1,\y-0.2) rectangle (2.9,\y+0.2);
      \draw[si] (1.2,\y-0.1) rectangle (2.8,\y+0.1);
    }
    \node[label] at (3.25,1.8) {\footnotesize pFET (GAA)};

    % shared S/D towers
    \draw[sd] (0.3,0.25) rectangle (1.0,2.15);
    \draw[sd] (3.0,0.25) rectangle (3.7,2.15);
    \node[label] at (0.65,2.35) {S/D};
    \node[label] at (3.35,2.35) {S/D};
  \end{tikzpicture}

  \caption{構造比較(モノクロ模式図):(a) FinFET(三面ゲート),(b) GAA(ナノシート多層),(c) CFET(n/p垂直積層+BPR整合)。灰色は誘電体,斜線はゲート金属,濃灰はソース/ドレイン。}
  \label{fig:structure_compare}
\end{figure}

  \caption{FinFET / GAA / CFET の比較模式図(1カラム収録)。}
  \label{fig:structure_compare_in_ch2}
\end{figure}
