\section{プレーナーMOS構造の限界}
プレーナーMOSFETにおいては、チャネル長の短縮に伴い、ドレイン電界がチャネル領域へ深く侵入する短チャネル効果(SCE: Short Channel Effect)が顕著化する。  
その結果、しきい値電圧($V_{\mathrm{th}}$)の低下、ドレイン誘起バリア低下(DIBL: Drain-Induced Barrier Lowering)、およびサブスレッショルド領域における漏れ電流特性の劣化が発生する。  
これらの現象は、チャネルポテンシャルの空間的制御がドレイン電位に依存することに起因しており、従来の二次元電界近似では十分に説明できない領域に突入したことを示している。

130\,nm世代以降では、ゲート酸化膜厚が数\si{\nano\meter}領域に達し、酸化膜のトンネル電流が支配的となった。  
酸化膜を薄膜化してしきい値電圧を維持しようとすると、ゲートリーク電流が指数関数的に増大し、絶縁破壊耐性が急激に低下する。  
これにより、単なるスケーリング則(Dennard則)に基づく寸法縮小はもはや成立せず、電界分布そのものを三次元的に制御する新構造の必要性が生じた。

この臨界点において登場したのが、立体的なチャネル構造を持つFinFETである。  
FinFETは、シリコン基板上に垂直フィンを形成し、ゲートを三方向から包み込むことでチャネル電界制御を強化し、SCEおよびDIBLを抑制した。  
この構造的転換は、MOSFET動作を「面制御」から「体積制御」へと拡張するものであり、  
以後のGAA(Gate-All-Around)およびCFET(Complementary FET)へと続く三次元構造進化の出発点となった。
