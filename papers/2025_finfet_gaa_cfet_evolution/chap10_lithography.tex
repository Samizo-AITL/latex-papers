\section{リソグラフィとマスク技術}
デバイス微細化の限界を押し広げるうえで、リソグラフィ技術の進歩は決定的な役割を果たしてきた。  
ArF液浸露光(ArF Immersion Lithography)からEUV(Extreme Ultraviolet)露光への移行は、  
露光波長の大幅短縮によって解像限界を縮小し、  
多重パターニング工程(Double/Quadruple Patterning)を劇的に削減した。  
これにより、配線ピッチの微細化に伴うプロセス変動(Overlay Error、CD Variation)が大幅に低減し、  
スループットおよび歩留まりの両立が可能となった。

\subsection{EUV露光と多重パターニング削減}
従来のArF液浸露光では、NA(Numerical Aperture)の限界により、  
40\,nm以下のパターン形成に複数回の露光・エッチングを要した。  
これに対し、13.5\,nm波長のEUV光を用いることで、  
1回の露光で30\,nmクラスのライン&スペースパターンが形成可能となった。  
この結果、工程数削減による線幅変動の抑制とともに、  
レジストパターンの位置ずれや累積誤差が大幅に緩和された。  
一方で、EUV光源の出力安定性やミラー反射損失、レジスト感度とLWR(Line Width Roughness)のトレードオフが  
次世代微細化の課題として残されている。

\subsection{マスク3D効果と光学補正}
微細パターン形成の高精度化には、マスク上の三次元効果(Mask 3D Effect)を考慮した補正設計が必須である。  
EUVマスクは多層ミラー構造を持ち、斜入射により反射位相と透過率がパターン位置によって変化する。  
このため、マスク転写像の歪みを補償するために、  
光学近接補正(OPC: Optical Proximity Correction)およびソースマスク最適化(SMO: Source Mask Optimization)が導入されている。  
これらの手法は、マスクパターン形状を露光シミュレーションと連動させ、  
最終的なシリコン上パターン精度をナノメートルスケールで保証する。

さらに、GAAおよびCFETのような三次元構造デバイスでは、  
パターン形状が深堀エッチングやゲート包囲構造と密接に関連するため、  
EUVマスク設計段階での形状歪み補正とアライメント精度がデバイス特性の再現性を左右する。  
特にCFETでは、上下層FETのゲートおよびソース・ドレインを個別に定義する必要があるため、  
多層マスク整合技術(Multi-Layer Alignment Technology)が歩留まり向上の鍵となる。

\subsection{次世代リソグラフィへの展望}
2\,nm世代以降では、High-NA EUV(開口数1.0以上)の適用が進み、  
より高い解像力とパターン忠実度を両立する見込みである。  
また、EUVペリクル(保護膜)の透過損失や、レジスト材料の化学増幅機構に起因するノイズ制限が  
量産性のボトルネックとなる可能性が指摘されている。  
これに対し、**EUVと電子線マスク補正(E-Beam Mask Repair)を組み合わせたハイブリッド露光戦略** や、  
**Computational Lithography(計算リソグラフィ)によるパターン最適化** の研究が進展している。

このように、リソグラフィとマスク技術は単なる描画プロセスではなく、  
**デバイス構造設計と一体化した「構造整合工学(Structural Lithography Engineering)」** へと進化している。  
EUV世代以降の微細化では、露光・エッチング・デバイス形状設計を統合した最適化が、  
CFET時代の高精度かつ高信頼な製造基盤を支える中心的要素となる。
