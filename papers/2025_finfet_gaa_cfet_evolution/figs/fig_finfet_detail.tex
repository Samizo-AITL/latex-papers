\begin{figure}[t]
  \centering

  % ---------- TikZ Styles ----------
  \tikzset{
    gate/.style   ={pattern=north east lines, pattern color=black, draw=black, line width=0.3pt},
    oxide/.style  ={fill=black!8, draw=black, line width=0.3pt},
    si/.style     ={fill=white, draw=black, line width=0.4pt},
    sd/.style     ={fill=black!25, draw=black, line width=0.3pt},
    substrate/.style={fill=black!15, draw=black, line width=0.3pt},
    label/.style  ={font=\footnotesize},
    dim/.style    ={-{Latex[length=2mm]}, line width=0.3pt}
  }

  % ---------- Drawing ----------
  \begin{tikzpicture}[scale=1.05]

    % Substrate
    \draw[substrate] (-0.5,0) rectangle (4.5,-0.5);
    \node[label] at (2.0,-0.8) {Substrate};

    % STI / isolation background
    \draw[oxide] (0,0) rectangle (4,1.5);

    % Fin (Si)
    \draw[si] (1.75,0) rectangle (2.25,1.2);

    % Gate oxide (thin)
    \draw[oxide] (1.65,0) rectangle (2.35,1.22);

    % Gate electrode (3 sides)
    \draw[gate] (1.35,0.2) rectangle (2.65,1.15);

    % Source / Drain
    \draw[sd] (0.25,0.2) rectangle (1.35,1.15);
    \draw[sd] (2.65,0.2) rectangle (3.75,1.15);

    % Labels (non-overlapping)
    \node[label] at (0.8,1.35) {Source (S)};
    \node[label] at (3.2,1.35) {Drain (D)};
    \node[label,align=center] at (2.0,1.4) {Gate};
    \node[label] at (2.0,0.6) {Fin (Si)};

    % Dimension arrows
    \draw[dim] (1.75,0.0) -- ++(0,1.2) node[midway,left=2pt,label] {$H$};
    \draw[dim] (1.75,1.25) -- ++(0.5,0) node[midway,above=2pt,label] {$W$};

  \end{tikzpicture}

  % ---------- Caption ----------
  \caption{FinFETの断面模式図(高精細版)。ゲート電極は三面を包囲し、Finの幾何パラメータ $H,\,W$ は有効チャネル幅 $W_{\mathrm{eff}} = n(2H + W)$ に寄与する。}
  \label{fig:finfet_detail}
\end{figure}
