% figs/fig_planar_sce.tex
% プレーナーMOSのSCE模式図(モノクロ・1カラム適合)

\tikzset{
  gate/.style   ={pattern=north east lines, pattern color=black, draw=black},
  oxide/.style  ={fill=black!7, draw=black},
  si/.style     ={fill=white, draw=black},
  sd/.style     ={fill=black!25, draw=black},
  field/.style  ={->, line width=0.3pt},
  label/.style  ={font=\footnotesize}
}
\begin{tikzpicture}[scale=0.95]
  % Substrate
  \draw[fill=black!10,draw=black] (-0.4,0) rectangle (4.4,-0.5);
  \node[label] at (2.0,-0.7) {Si Substrate};

  % Oxide + channel slab
  \draw[oxide] (0,0) rectangle (4,1.2);   % STI/BOX-ish region
  \draw[si]    (0.4,0.2) rectangle (3.6,0.6); % channel slab

  % Gate oxide + gate
  \draw[oxide, line width=0.4pt] (0.4,0.6) rectangle (3.6,0.7);
  \draw[gate]  (0.8,0.7) rectangle (3.2,1.1);
  \node[label] at (2.0,1.25) {Gate};

  % Source / Drain
  \draw[sd] (0.0,0.2) rectangle (0.4,0.9);
  \draw[sd] (3.6,0.2) rectangle (4.0,0.9);
  \node[label] at (0.2,1.05) {S};
  \node[label] at (3.8,1.05) {D};

  % Short-channel effect: field lines from D into channel
  \foreach \y in {0.25,0.35,0.45,0.55} {
    \draw[field] (3.8,\y) .. controls (3.1,\y+0.15) and (2.6,\y+0.05) .. (2.2,\y+0.02);
  }

  % Labels
  \node[label,align=left] at (3.2,0.2) {Drain電界の侵入\\(DIBL, SCE)};
  \node[label] at (2.0,0.0) {};
\end{tikzpicture}
