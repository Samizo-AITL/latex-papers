% =====================================================
% Fig. 8 : BEOL構造比較(Dual Damascene vs BPR)💯版
% =====================================================
\begin{figure}[t]
  \centering
  \tikzset{
    diel/.style   ={fill=black!5, draw=black, line width=0.3pt},
    metal/.style  ={fill=black!35, draw=black, line width=0.3pt},
    barrier/.style={fill=black!55, draw=black, line width=0.3pt},
    note/.style   ={font=\footnotesize, align=center},
    label/.style  ={font=\footnotesize}
  }

  %==============================
  % (a) Dual Damascene - Frontside Power
  %==============================
  \begin{tikzpicture}[scale=0.9]
    \node[label, anchor=west] at (-0.1,3.35) {\textbf{(a) Dual Damascene(前面電源配線)}};
    
    % Dielectric stack
    \draw[diel] (0,0) rectangle (4.0,3.0);
    
    % Cu lines with barrier
    \foreach \x in {0.4,1.8,3.2}{
      \draw[barrier] (\x,1.8) rectangle (\x+0.4,2.6);
      \draw[metal]   (\x+0.05,1.85) rectangle (\x+0.35,2.55);
    }

    % Vias connecting layers
    \draw[barrier] (0.55,0.6) rectangle (0.85,1.8);
    \draw[metal]   (0.6,0.65) rectangle (0.8,1.75);
    \draw[barrier] (3.35,0.6) rectangle (3.65,1.8);
    \draw[metal]   (3.4,0.65) rectangle (3.6,1.75);

    % Substrate (context hint)
    \fill[black!15] (0,0) rectangle (4.0,0.4);
    \draw[black] (0,0) rectangle (4.0,0.4);
    \node[label] at (2,0.2) {Si Substrate};

    % Note text
    \node[note] at (2,0.0) {信号と電源が同層で競合 → 配線混雑};
  \end{tikzpicture}

  \vspace{1.5ex}

  %==============================
  % (b) Backside Power Rail (BPR)
  %==============================
  \begin{tikzpicture}[scale=0.9]
    \node[label, anchor=west] at (-0.1,3.6) {\textbf{(b) Backside Power Rail(裏面電源分離)}};

    % Signal dielectric (frontside)
    \draw[diel] (0,0.8) rectangle (4.0,3.2);

    % Frontside signal lines
    \foreach \x in {0.7,2.0,3.3}{
      \draw[metal] (\x,2.2) rectangle (\x+0.35,3.0);
    }
    \node[label, anchor=west] at (3.5,2.7) {信号配線};

    % Substrate
    \draw[fill=black!10, draw=black, line width=0.3pt] (0,0.2) rectangle (4.0,0.8);
    \node[label] at (2,0.05) {Si Substrate};

    % Backside Power Rails
    \draw[metal] (0.3,-0.2) rectangle (1.9,0.2);
    \draw[metal] (2.1,-0.2) rectangle (3.7,0.2);
    \node[label] at (1.1,-0.35) {BPR-VDD};
    \node[label] at (2.9,-0.35) {BPR-VSS};

    % TSV-like power vias
    \foreach \x in {0.6,3.4}{
      \draw[metal] (\x,0.2) rectangle (\x+0.25,0.8);
    }

    % Note text
    \node[note] at (2.0,0.45) {裏面電源でセル上面が空き\\配線自由度・性能が向上};
  \end{tikzpicture}

  \caption{BEOL配線技術の概念比較図。\\
  \footnotesize (a) Dual Damasceneでは電源・信号が同一面に配置され配線混雑が発生。\\
  (b) BPR構造では電源を裏面に再配置し、セル上部の配線密度と設計自由度を向上させる。}
  \label{fig:beol_bpr}
\end{figure}
