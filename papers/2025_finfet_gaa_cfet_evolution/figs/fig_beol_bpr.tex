\begin{figure}[t]
  \centering
  \tikzset{
    diel/.style   ={fill=black!5, draw=black},
    metal/.style  ={fill=black!35, draw=black},
    barrier/.style={fill=black!60, draw=black},
    label/.style  ={font=\footnotesize}
  }

  %=== (a) Dual Damascene (frontside power) ===
  \begin{tikzpicture}[scale=0.9]
    \node[label, anchor=west] at (-0.2,3.3) {\textbf{(a) Dual Damascene(従来前面配線)}};
    % dielectric stack
    \draw[diel] (0,0) rectangle (4,3.0);
    % lines (Cu) with barrier
    \foreach \x in {0.4,1.8,3.2}{
      \draw[barrier] (\x,1.8) rectangle (\x+0.4,2.6);
      \draw[metal]   (\x+0.05,1.85) rectangle (\x+0.35,2.55);
    }
    % vias
    \draw[barrier] (0.55,0.6) rectangle (0.85,1.8);
    \draw[metal]   (0.6,0.65) rectangle (0.8,1.75);
    \draw[barrier] (3.35,0.6) rectangle (3.65,1.8);
    \draw[metal]   (3.4,0.65) rectangle (3.6,1.75);
    % cell area note
    \node[label] at (2,0.3) {信号・電源が同一面で競合 → 配線混雑};
  \end{tikzpicture}

  \vspace{1.2ex}

  %=== (b) Backside Power Rail (BPR) ===
  \begin{tikzpicture}[scale=0.9]
    \node[label, anchor=west] at (-0.2,3.6) {\textbf{(b) Backside Power Rail(裏面電源分離)}};
    % frontside dielectric (signal)
    \draw[diel] (0,0.8) rectangle (4,3.2);
    % frontside signal lines
    \foreach \x in {0.7,2.0,3.3}{
      \draw[metal] (\x,2.2) rectangle (\x+0.35,3.0);
    }
    \node[label] at (3.3,3.25) {信号配線};
    % substrate slab
    \draw[fill=black!10, draw=black] (0,0.2) rectangle (4,0.8);
    \node[label] at (2,0.05) {Si Substrate};
    % backside rails
    \draw[metal] (0.2,-0.2) rectangle (1.9,0.2);
    \draw[metal] (2.1,-0.2) rectangle (3.8,0.2);
    \node[label] at (1.05,-0.35) {BPR-VDD};
    \node[label] at (2.95,-0.35) {BPR-VSS};
    % backside vias (power TSV-like)
    \foreach \x in {0.5,3.5}{
      \draw[metal] (\x,0.2) rectangle (\x+0.25,0.8);
    }
    \node[label] at (2,0.45) {裏面電源でセル上面が空き → 配線自由度↑};
  \end{tikzpicture}

  \caption{BEOL配線の概念図:Dual Damascene(前面)では信号と電源が競合、BPRでは電源を裏面に再配置し配線混雑を緩和。}
  \label{fig:beol_bpr}
\end{figure}
