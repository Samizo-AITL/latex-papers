\begin{figure}[t]
  \centering
  \tikzset{
    box/.style   ={draw=black, fill=black!5},
    hot/.style   ={draw=black, fill=black!60},
    warm/.style  ={draw=black, fill=black!35},
    cool/.style  ={draw=black, fill=black!15},
    gate/.style  ={pattern=north east lines, pattern color=black, draw=black},
    sd/.style    ={fill=black!25, draw=black},
    label/.style ={font=\footnotesize}
  }

  %=== (a) FinFET thermal map ===
  \begin{tikzpicture}[scale=0.9]
    \node[label, anchor=west] at (-0.2,2.7) {\textbf{(a) FinFET: Self-Heating}};
    \draw[box] (0,0) rectangle (4,2.4);    % frame
    \draw[sd] (0.2,0.4) rectangle (0.9,2.0);
    \draw[sd] (3.1,0.4) rectangle (3.8,2.0);
    % fin thermal core
    \draw[cool] (1.8,0.4) rectangle (2.2,2.0);
    \draw[warm] (1.75,0.6) rectangle (2.25,1.8);
    \draw[hot]  (1.7,0.9) rectangle (2.3,1.5);
    % gate wrap
    \draw[gate] (1.3,0.5) rectangle (2.7,1.9);
    \node[label] at (2,2.2) {高温域がチャネル中央に集中};
  \end{tikzpicture}

  \vspace{1.2ex}

  %=== (b) CFET thermal coupling ===
  \begin{tikzpicture}[scale=0.9]
    \node[label, anchor=west] at (-0.2,3.2) {\textbf{(b) CFET: Vertical Thermal Coupling}};
    \draw[box] (0,0) rectangle (4,2.9);    % frame
    \draw[sd] (0.2,0.4) rectangle (0.9,2.5);
    \draw[sd] (3.1,0.4) rectangle (3.8,2.5);
    % lower device hotspot
    \draw[cool] (1.2,0.4) rectangle (2.8,1.2);
    \draw[warm] (1.4,0.55) rectangle (2.6,1.05);
    \draw[hot]  (1.6,0.7) rectangle (2.4,0.9);
    % ILD spacer
    \draw[box, fill=black!12] (0,1.25) rectangle (4,1.45);
    % upper device hotspot
    \draw[cool] (1.2,1.5) rectangle (2.8,2.3);
    \draw[warm] (1.4,1.65) rectangle (2.6,2.15);
    \draw[hot]  (1.6,1.8) rectangle (2.4,2.0);
    % gates (輪郭のみ)
    \draw[gate] (1.1,0.5) rectangle (2.9,1.1);
    \draw[gate] (1.1,1.6) rectangle (2.9,2.2);
    \node[label] at (2,2.7) {上下FET間で熱干渉(対称設計が鍵)};
  \end{tikzpicture}

  \caption{自己発熱と熱対称性の概念図:FinFETではチャネル中央に高温域、CFETでは上下FET間の熱干渉が顕在化。熱対称性の最適化が信頼性の鍵となる。}
  \label{fig:thermal_symmetry}
\end{figure}
