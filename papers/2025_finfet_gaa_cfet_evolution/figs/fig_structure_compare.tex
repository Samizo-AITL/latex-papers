% =====================================================
% Fig. 9 : FinFET, GAA, CFET 構造比較(改訂版・💯)
% =====================================================
\begin{figure}[t]
  \centering
  \tikzset{
    gate/.style   ={pattern=north east lines, pattern color=black, draw=black},
    oxide/.style  ={fill=black!7, draw=black},
    si/.style     ={fill=white, draw=black},
    sd/.style     ={fill=black!25, draw=black},
    metal/.style  ={fill=black!45, draw=black},
    label/.style  ={font=\footnotesize},
    dim/.style    ={-{Latex[length=2mm]}, line width=0.3pt}
  }

  %==================== (a) FinFET ====================
  \begin{tikzpicture}[scale=0.9]
    \node[label, anchor=west] at (-0.2,3.3) {\textbf{(a) FinFET}};
    % substrate
    \draw[fill=black!10, draw=black] (-0.5,0) rectangle (4.5,-0.6);
    \node[label] at (2.0,-0.8) {Si Substrate};
    % STI region
    \draw[oxide] (0,0) rectangle (4,1.5);
    % fin
    \draw[si] (1.7,0) rectangle (2.3,1.3);
    % gate oxide
    \draw[oxide,line width=0.4pt] (1.6,-0.02) rectangle (2.4,1.32);
    % gate metal (tri-gate)
    \draw[gate] (1.35,0.15) rectangle (2.65,1.20);
    % source/drain
    \draw[sd] (0.25,0.15) rectangle (1.35,1.20);
    \draw[sd] (2.65,0.15) rectangle (3.75,1.20);
    % labels
    \node[label] at (0.8,1.4) {S};
    \node[label] at (3.2,1.4) {D};
    \node[label] at (2.0,1.45) {Fin (Si)};
    \node[label] at (2.0,0.9) {Gate};
    % dimensions
    \draw[dim] (1.7,0.0) -- ++(0,1.3) node[midway,left=2pt] {$H$};
    \draw[dim] (1.7,1.35) -- ++(0.6,0) node[midway,above=1pt] {$W$};
  \end{tikzpicture}

  \vspace{1.5ex}

  %==================== (b) GAA ====================
  \begin{tikzpicture}[scale=0.9]
    \node[label, anchor=west] at (-0.2,3.4) {\textbf{(b) GAA (Nanosheet)}};
    % substrate
    \draw[fill=black!10, draw=black] (-0.5,0) rectangle (4.5,-0.6);
    \node[label] at (2,-0.8) {Si Substrate};
    % oxide box
    \draw[oxide] (0,0) rectangle (4,1.9);
    % nanosheets (3層)
    \foreach \y in {0.5,1.0,1.5}{
      \draw[gate] (1.0,\y-0.25) rectangle (3.0,\y+0.25);
      \draw[oxide] (1.1,\y-0.15) rectangle (2.9,\y+0.15);
      \draw[si] (1.2,\y-0.07) rectangle (2.8,\y+0.07);
    }
    % S/D
    \draw[sd] (0.25,0.3) rectangle (1.0,1.7);
    \draw[sd] (3.0,0.3) rectangle (3.75,1.7);
    \node[label] at (0.7,1.9) {S};
    \node[label] at (3.3,1.9) {D};
    % label
    \node[label] at (2.0,2.0) {Gate-All-Around};
    % dims
    \draw[dim] (2.8,0.9) -- ++(0.6,0) node[right] {$W$};
    \draw[dim] (2.6,0.7) -- ++(0,-0.35) node[below] {$H$};
  \end{tikzpicture}

  \vspace{1.5ex}

  %==================== (c) CFET ====================
  \begin{tikzpicture}[scale=0.9]
    \node[label, anchor=west] at (-0.2,3.6) {\textbf{(c) CFET (n/p 垂直積層)}};
    % BPR rails
    \draw[metal] (-0.5,-0.9) rectangle (1.6,-1.3);
    \draw[metal] (2.4,-0.9) rectangle (4.5,-1.3);
    \node[label] at (0.55,-1.45) {BPR-VDD};
    \node[label] at (3.45,-1.45) {BPR-VSS};
    % substrate
    \draw[fill=black!10, draw=black] (-0.5,0) rectangle (4.5,-0.6);
    \node[label] at (2.0,-0.8) {Substrate};
    % oxide bulk
    \draw[oxide] (0,0) rectangle (4,2.5);
    % lower nFET
    \draw[gate] (1.0,0.4) rectangle (3.0,1.0);
    \draw[oxide] (1.1,0.5) rectangle (2.9,0.9);
    \draw[si] (1.2,0.6) rectangle (2.8,0.8);
    \node[label] at (3.25,0.7) {n-FET};
    % ILD spacer
    \draw[oxide, fill=black!12] (0,1.05) rectangle (4,1.25);
    % upper pFET
    \draw[gate] (1.0,1.4) rectangle (3.0,2.0);
    \draw[oxide] (1.1,1.5) rectangle (2.9,1.9);
    \draw[si] (1.2,1.6) rectangle (2.8,1.8);
    \node[label] at (3.25,1.7) {p-FET};
    % shared S/D pillars
    \draw[sd] (0.3,0.25) rectangle (1.0,2.15);
    \draw[sd] (3.0,0.25) rectangle (3.7,2.15);
    \node[label] at (0.65,2.35) {S/D};
    \node[label] at (3.35,2.35) {S/D};
  \end{tikzpicture}

  \caption{CMOS構造の比較模式図(モノクロ対応)。\\
  \footnotesize (a) FinFET: 三面ゲート構造。 (b) GAA: 多層ナノシートの全包囲ゲート構造。 (c) CFET: n/p積層構造とBPR電源分離の統合。灰色=誘電体、斜線=ゲート金属、濃灰=ソース/ドレイン。}
  \label{fig:structure_compare}
\end{figure}
