\section{BSIM-CMGモデリング}
BSIM-CMG(Berkeley Short-channel IGFET Model – Common Multi-Gate)は、  
FinFETやGAA(Gate-All-Around)などの非平面デバイスに対応する統一SPICEモデルである。  
従来のBSIM4モデルがプレーナーMOSFETの電界分布を前提としていたのに対し、  
BSIM-CMGはチャネルの三次元的電界分布と多ゲート構造による電位結合効果を厳密に取り扱う。

本モデルでは、チャネル電位$\psi(x,y,z)$を多変数ポアソン方程式として近似的に解き、  
有効電荷密度$Q_\text{inv}$およびドレイン電流$I_\text{D}$をゲート電圧$V_\text{G}$、ドレイン電圧$V_\text{D}$の関数として定式化する:  
\begin{equation}
I_\text{D} = \mu_\text{eff} C_\text{inv} \frac{W_\text{eff}}{L_\text{g}} (V_\text{G} - V_\text{th}) V_\text{D} \, f_\text{sat}(E_\text{ch},T)
\end{equation}
ここで、$\mu_\text{eff}$は実効移動度、$C_\text{inv}$はチャネル反転層容量、  
$f_\text{sat}$はチャネル電界および温度$T$に依存する飽和関数である。  
FinFET構造では$W_\text{eff}=n(2H+W)$、GAA構造では$W_\text{eff}=2n(H+W)$と定義され、  
この形態パラメータが電流駆動力を直接決定する。

BSIM-CMGは、これらの幾何学パラメータを抽象化してモデル化するため、  
異なる構造(FinFET、Nanowire、Nanosheet)間でパラメータ再利用が可能である。  
主要な制御パラメータとして、  
\begin{itemize}
    \item ゲート絶縁膜の実効酸化膜厚(EOT)、  
    \item チャネル高さ$H_\text{fin}$および幅$W_\text{fin}$、  
    \item ソース/ドレイン接合抵抗、  
    \item サブスレッショルド・スロープ因子$n$,  
    \item キャリア散乱係数および温度依存項、  
\end{itemize}
などが定義されており、物理的パラメータと回路設計変数を統一的に結びつけることができる。

また、BSIM-CMGは、チャネル内部の電位カップリングを解くために「中心軸法(Core Potential Method)」を導入しており、  
チャネル電位を単一代表点で近似することで、非平面構造においても数値安定性と計算効率を確保している。  
これにより、GAAやCFETのような多層チャネルデバイスでもSPICEレベルでの解析が可能となった。

さらに、CFETにおいては、上下のトランジスタ間での電熱結合(self-heating coupling)や  
相互寄生容量を含めた拡張モデルが提案されつつある。  
この拡張では、各トランジスタを独立のBSIM-CMGサブモデルとして扱い、  
電流・温度・電位の相互干渉を双方向結合することで、  
垂直積層構造に特有の熱非対称性を高精度に再現できる。

BSIM-CMGの導入により、デバイス物理と回路設計の連携が一層強化され、  
トランジスタ寸法・材料・温度・信頼性パラメータを同一フレームで評価可能となった。  
このモデルは、ポストFinFET時代の設計基盤として、  
構造最適化から回路レベルの信頼性評価に至るまで広範に適用されている。
