\section{BSIM-CMGモデリング}
BSIM-CMG(Berkeley Short-channel IGFET Model – Common Multi-Gate)は、  
FinFETおよびGAA(Gate-All-Around)などの非平面構造デバイスを統一的に表現するために開発された  
物理ベースSPICEコンパクトモデルである。  
従来のBSIM4モデルがプレーナーMOSFETの二次元電界分布を前提としていたのに対し、  
BSIM-CMGは三次元チャネル電位分布および多ゲート構造における電位結合(electrostatic coupling)を厳密に扱うことが可能である。

本モデルでは、チャネル電位$\psi(x,y,z)$を多変数ポアソン方程式に基づき近似的に解き、  
有効電荷密度$Q_\mathrm{inv}$およびドレイン電流$I_\mathrm{D}$をゲート電圧$V_\mathrm{G}$およびドレイン電圧$V_\mathrm{D}$の関数として定式化する:
\begin{equation}
I_\mathrm{D} =
\mu_\mathrm{eff} \, C_\mathrm{inv} \,
\frac{W_\mathrm{eff}}{L_\mathrm{g}}
(V_\mathrm{G} - V_\mathrm{th}) V_\mathrm{D}
\, f_\mathrm{sat}(E_\mathrm{ch}, T),
\end{equation}
ここで、$\mu_\mathrm{eff}$は実効キャリア移動度、$C_\mathrm{inv}$は反転層容量、  
$f_\mathrm{sat}(E_\mathrm{ch},T)$はチャネル電界および温度依存性を表す飽和関数である。  
FinFET構造では$W_\mathrm{eff}=n(2H+W)$、GAA構造では$W_\mathrm{eff}=2n(H+W)$と定義され、  
形態パラメータが直接的に電流駆動力を支配する。

BSIM-CMGはこれらの幾何学パラメータを抽象化してモデル化するため、  
FinFET、Nanosheet、Nanowireといった異なる構造間でのパラメータ再利用が可能である。  
主な制御パラメータは以下の通りである:
\begin{itemize}
    \item 実効酸化膜厚(EOT: Equivalent Oxide Thickness)  
    \item チャネル高さ$H_\mathrm{fin}$および幅$W_\mathrm{fin}$  
    \item ソース/ドレイン接合抵抗  
    \item サブスレッショルド係数$n$およびキャリア散乱係数  
    \item 温度依存項および熱ノイズ係数  
\end{itemize}
これにより、物理パラメータと回路設計変数を単一フレームで統合し、デバイス–回路協調設計(co-design)が実現される。

さらに、BSIM-CMGはチャネル内部電位を代表値で近似する「中心軸法(Core Potential Method)」を採用しており、  
非平面構造においても数値安定性と高速収束性を両立している。  
これにより、GAAやCFETのような多層チャネル構造においてもSPICEレベルの解析が安定して実行可能である。

CFETに対しては、上下トランジスタ間の電熱結合(self-heating coupling)および  
相互寄生容量を考慮した拡張モデルが提案されている。  
このモデルでは各トランジスタを独立したBSIM-CMGサブブロックとして構築し、  
電流・温度・電位の相互干渉を双方向結合させることで、  
垂直積層構造に特有の熱非対称性を高精度に再現できる。

BSIM-CMGの導入により、デバイス物理と回路設計の整合が飛躍的に向上した。  
寸法スケーリング、材料特性、熱劣化、信頼性パラメータを統一的に評価できる本モデルは、  
ポストFinFET世代における設計基盤として、構造最適化から回路レベル信頼性解析に至るまで  
広範に応用されている。  

% figs/fig_bsim_cmg_model.tex
% BSIM-CMG モデル構成図(モノクロ/外側のfigure環境は書かない)
% 2カラム対応:transform shape で縮尺、ノード幅は固定値でオーバー防止
\begin{tikzpicture}[
  font=\scriptsize,
  node distance=6mm,
  scale=0.95,                      % ← 過大なら 0.9, 0.85 に下げる
  every node/.style={transform shape} % ← scaleで文字も一緒に縮む
]
  % styles
  \tikzset{
    blk/.style={draw=black, fill=black!5, rounded corners,
                minimum width=36mm, minimum height=7mm, align=center},
    io/.style ={draw=black, fill=black!15, rounded corners,
                minimum width=28mm, minimum height=7mm, align=center},
    grp/.style={draw=black, dashed, inner sep=3mm, rounded corners},
    arrow/.style={-Latex, line width=0.3pt}
  }

  % Inputs
  \node[io] (geom) {Geometry\\(Fin/GAA params)};
  \node[io, right=of geom] (mat) {Materials\\(EOT, $\mu$, HKMG)};
  \node[io, right=of mat] (bias) {Bias/Temp\\($V_G, V_D, T$)};

  % Solvers / kernels
  \node[blk, below=12mm of mat] (elec) {Electrostatics\\(core potential / Poisson approx.)};
  \node[blk, below=of elec] (mob) {Mobility \& Scattering\\(phonon, surface, impurity)};
  \node[blk, below left=6mm and 7mm of mob] (sdr) {Series $R$ (S/D, contact)};
  \node[blk, below right=6mm and 7mm of mob] (sh) {Self-heating\\(thermal coupling)};

  % Outputs
  \node[blk, below=18mm of mob, minimum width=42mm] (charge)
        {Charge/Capacitance\\($Q$, $C_{gg}$, $C_{gd}$, \dots)};
  \node[blk, below=8mm of charge, minimum width=42mm] (ids)
        {$I_D$ model\\(sat. func. $f_\mathrm{sat}(E_\mathrm{ch},T)$)};

  % Group box
  \node[grp, fit=(elec) (mob) (sdr) (sh) (charge) (ids),
        label={[font=\footnotesize]above:BSIM-CMG Core}] (core) {};

  % Arrows: inputs -> core
  \draw[arrow] (geom.south) -- (elec.north -| geom.south);
  \draw[arrow] (mat.south)  -- (elec.north);
  \draw[arrow] (bias.south) -- (elec.north -| bias.south);

  % Flow inside core
  \draw[arrow] (elec) -- (mob);
  \draw[arrow] (mob.west) |- (sdr.north);
  \draw[arrow] (mob.east) |- (sh.north);
  \draw[arrow] (mob) -- (charge);
  \draw[arrow] (charge) -- (ids);

  % Feedbacks (dashed)
  \draw[arrow, dashed] (ids.west) |- ($(elec.south west)!0.6!(elec.north west)$);
  \draw[arrow, dashed] (sh.east)  |- ($(mob.east)+(8pt,0)$) -| (mob.east);

  % Note
  \node[font=\scriptsize, align=left, anchor=west]
        at ($(core.south west)+(0,-5mm)$)
        {出力: $I_D(V_G,V_D,T)$, $Q$, 各種$C$(SPICE互換)};
\end{tikzpicture}

