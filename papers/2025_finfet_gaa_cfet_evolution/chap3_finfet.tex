\section{FinFET構造とその特徴}
FinFETは、シリコン基板上に形成された立体的なフィン(Fin)チャネルを、ゲートが三方向から包み込む構造を有する。  
この三面ゲート構造により、チャネル電位分布を高精度に制御でき、プレーナーMOSFETに比べてドレイン電界の侵入を大幅に抑制する。  
その結果、短チャネル効果(SCE)の緩和、ドレイン誘起バリア低下(DIBL)の低減、サブスレッショルドスイング(SS)の改善が同時に達成される。

Fin構造の幾何学的パラメータは電気特性を直接支配する。  
Finの高さを$H$、幅を$W$、Fin数を$n$とすると、有効チャネル幅$W_{\mathrm{eff}}$は次式で表される:
\begin{equation}
  W_{\mathrm{eff}} = n(2H + W).
\end{equation}
この関係式は、電流伝導が側面チャネルに支配的であることを示し、Fin高さ$H$を増すことでオン電流($I_{\mathrm{ON}}$)を向上できる一方、  
過度な高さは機械的強度およびエッチング制御精度の面で限界をもたらす。  
したがって、FinFET設計では「高さによる駆動能力」と「製造再現性」の最適点を探索することが重要である。

FinFETの主な利点は、(1) 優れたゲート制御性、(2) 低オフリーク電流、(3) 動作電圧の低減による低消費電力化である。  
一方で、製造上の課題として、Fin寸法の微小ばらつき(Line Edge Roughness: LER)やゲート包囲部の非対称性がしきい値電圧$V_{\mathrm{th}}$の揺らぎを引き起こし、  
デバイス間の性能均一性を制限する。  
さらに、高アスペクト比Fin構造では、ゲート酸化膜堆積やメタルゲート充填における段差被覆性(Step Coverage)が信頼性を支配する要因となる。

このようにFinFETは、平面構造の限界を克服するだけでなく、  
デバイス設計における「電界制御性とプロセス均一性の最適折衷」を体現する構造パラダイムである。  
本構造は、後続のGAA(Gate-All-Around)およびCFET(Complementary FET)への発展を導く基盤技術として位置付けられる。

% (FinFET節の本文のすぐ後ろ)
\begin{table}[t]
  \centering
  \caption{FinFET代表パラメータ例(概念値)}
  \label{tab:finfet_params}
  \begin{tabular}{lccc}
    \toprule
    パラメータ & 記号 & 代表値 & 備考 \\
    \midrule
    Fin高さ & $H$ & 40–60\,nm & 電流駆動に比例 \\
    Fin幅   & $W$ & 5–10\,nm & 短チャネル制御に影響 \\
    Fin数   & $n$ & 2–4 & セル幅内で最適化 \\
    有効チャネル幅 & $W_\mathrm{eff}$ & $n(2H+W)$ & 実効伝導面積 \\
    しきい値電圧 & $V_\mathrm{th}$ & 0.35–0.45\,V & LERに敏感 \\
    オン電流 & $I_\mathrm{ON}$ & $\sim$1\,mA/$\mu$m & 高$H$/小$W$で向上 \\
    オフ電流 & $I_\mathrm{OFF}$ & $<100$\,pA/$\mu$m & 三面ゲート効果 \\
    サブスレッショルドスイング & $SS$ & 65–75\,mV/dec & SCE抑制指標 \\
    \bottomrule
  \end{tabular}
\end{table}

\begin{figure}[t]
  \centering

  \tikzset{
    gate/.style   ={pattern=north east lines, pattern color=black, draw=black},
    oxide/.style  ={fill=black!8, draw=black},
    si/.style     ={fill=white, draw=black},
    sd/.style     ={fill=black!25, draw=black},
    label/.style  ={font=\footnotesize},
    dim/.style    ={-{Latex[length=2mm]}, line width=0.3pt}
  }

  \begin{tikzpicture}[scale=1.0]
    % Substrate
    \draw[fill=black!12, draw=black] (-0.5,0) rectangle (4.5,-0.5);
    \node[label] at (2.0,-0.75) {Substrate};

    % STI / isolation background
    \draw[oxide] (0,0) rectangle (4,1.6);

    % Fin (Si)
    \draw[si] (1.7,0) rectangle (2.3,1.4);

    % Thin gate dielectric hint
    \draw[oxide, line width=0.4pt] (1.6,-0.02) rectangle (2.4,1.42);

    % Gate wrapping three faces
    \draw[gate] (1.35,0.2) rectangle (2.65,1.25);

    % Source / Drain
    \draw[sd] (0.25,0.2) rectangle (1.35,1.25);
    \draw[sd] (2.65,0.2) rectangle (3.75,1.25);

    % Labels
    \node[label] at (0.8,1.4) {S};
    \node[label] at (3.2,1.4) {D};
    \node[label] at (2.0,1.5) {Fin (Si)};
    \node[label] at (2.0,0.95) {Gate};

    % Dimensions (H, W)
    \draw[dim] (1.7,0.0) -- ++(0,1.4) node[midway,left=2pt,label] {$H$};
    \draw[dim] (1.7,1.45) -- ++(0.6,0) node[midway,above=2pt,label] {$W$};
  \end{tikzpicture}

  \caption{FinFETの断面模式図(モノクロ)。ゲートは三面をラップし、幾何パラメータ $H,\,W$ は $W_{\mathrm{eff}}=n(2H+W)$ に寄与する。}
  \label{fig:finfet_detail}
\end{figure}

