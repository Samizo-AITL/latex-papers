\section{FinFET構造とその特徴}
FinFETは、立体的に形成されたシリコンフィン(Fin)をゲートが三方向から包み込む立体チャネル構造を有する。  
この三面ゲート構造により、チャネル電位の空間分布を高精度に制御でき、プレーナーMOSFETに比べてドレイン電界の侵入を大幅に抑制する。  
結果として、短チャネル効果(SCE)の緩和、ドレイン誘起バリア低下(DIBL)の低減、サブスレッショルドスイング(SS)の改善が同時に達成される。

Fin構造の幾何学的パラメータは、電気特性を直接支配する。  
Finの高さを$H$、幅を$W$、Fin数を$n$とすると、有効チャネル幅$W_{\mathrm{eff}}$は次式で表される:
\begin{equation}
W_{\mathrm{eff}} = n(2H + W).
\end{equation}
この式は、側面チャネルが電流伝導に支配的であることを示し、Finの高さを増すことでドライブ能力($I_{\mathrm{ON}}$)を高められる一方、過剰な高さは機械的強度やエッチング制御の面で限界をもたらす。

FinFETの主な利点は、(1)ゲート制御性の向上、(2)オフリーク電流の低減、(3)動作電圧の低下による低消費電力化である。  
一方で、製造上の課題として、Fin寸法の微小ばらつき(Line Edge Roughness, LER)やゲート包囲部の非対称性がしきい値電圧$V_{\mathrm{th}}$の揺らぎを引き起こし、デバイス間の性能均一性を制限する。  
さらに、高アスペクト比Fin構造では、ゲート酸化膜堆積やメタルゲート充填における段差被覆性(Step Coverage)が信頼性を支配する要因となる。

このようにFinFETは、平面構造の限界を克服するだけでなく、デバイス設計における「電界制御性とプロセス均一性の最適折衷」を追求する新たな設計パラダイムを提示した。  
次章では、このFin構造をさらに発展させた全包囲ゲート構造GAA(Gate-All-Around)について述べる。
