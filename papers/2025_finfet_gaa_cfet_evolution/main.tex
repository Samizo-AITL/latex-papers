% =====================================================
%  FinFET〜CFET構造進化チュートリアル論文(日本語版 / IEEEスタイル)
% =====================================================

\documentclass[conference]{IEEEtran}

% -----------------------------------------------------
% パッケージ設定
% -----------------------------------------------------
\usepackage[utf8]{inputenc}
\usepackage[T1]{fontenc}
\usepackage{xeCJK}            % 日本語対応
\setCJKmainfont{IPAexMincho}  % 明朝体
\usepackage{graphicx}
\usepackage{amsmath,amssymb}
\usepackage{siunitx}
\usepackage{booktabs}
\usepackage{hyperref}
\usepackage{url}
\usepackage{cite}

% -----------------------------------------------------
% タイトルと著者情報
% -----------------------------------------------------
\title{FinFET〜CFET構造進化チュートリアル:\\
スケーリング臨界と構造信頼性設計}

\author{%
  \IEEEauthorblockN{三溝 真一 (Shinichi Samizo)}%
  \IEEEauthorblockA{独立系半導体研究者(元セイコーエプソン) / Independent Semiconductor Researcher (ex-Seiko Epson)\\%
  Email: \href{mailto:shin3t72@gmail.com}{shin3t72@gmail.com}\quad
  GitHub: \url{https://github.com/Samizo-AITL}}%
}

% -----------------------------------------------------
% ドキュメント開始
% -----------------------------------------------------
\begin{document}
\maketitle

% -----------------------------------------------------
% 要旨(Abstract)
% -----------------------------------------------------
\begin{abstract}
\textbf{(日本語要旨)}  
本稿は、130\,nm以降におけるCMOSスケーリング技術の構造的進化を体系的に整理したチュートリアル論文である。  
プレーナーCMOSの限界からFinFET、GAA(Nanosheet)、そしてCFET(Complementary FET)へと至る構造変遷を、  
電界制御、熱対称性、電源分離、および再現性設計の観点から俯瞰する。  
さらに、High-k/Metal Gate(HKMG)技術、BEOL配線スケーリング(Low-$k$絶縁膜、Dual Damascene、Backside Power Rail)および  
BSIM-CMGモデリングを統合し、微細化の最終段階における「構造によって信頼性を設計する時代」の到来を示す。  
本稿は、プロセス・デバイス・設計の各領域を横断した構造最適化の指針を提供することを目的とする。  

\medskip
\textbf{(English Abstract)}  
This tutorial paper provides a systematic overview of the structural evolution of CMOS scaling technology beyond the 130\,nm node.  
The transition from planar CMOS to FinFET, Gate-All-Around (GAA) nanosheet, and ultimately Complementary FET (CFET) architectures  
is analyzed from the perspectives of electrostatic control, thermal symmetry, power rail separation, and reproducibility in design.  
Furthermore, this study integrates High-$k$/Metal Gate (HKMG) technology, BEOL scaling (Low-$k$ dielectrics, Dual Damascene,  
and Backside Power Rail), and BSIM-CMG device modeling to demonstrate the paradigm shift toward “designing reliability through structure.”  
The paper aims to provide unified design insights across process, device, and circuit domains for next-generation nanoscale CMOS integration.
\end{abstract}

% -----------------------------------------------------
% キーワード(Keywords)
% -----------------------------------------------------
\begin{IEEEkeywords}
FinFET, GAA, CFET, HKMG, BEOL, Backside Power Rail, BSIM-CMG, Thermal Symmetry, Structural Reliability, CMOS Scaling
\end{IEEEkeywords}

% -----------------------------------------------------
% 本文章構成
% -----------------------------------------------------
% ============================================================
% 1. はじめに(Revised Final)
% ============================================================
\section{はじめに}

現代の複合システム設計では,
構造・材料・熱・応力・電磁・信頼性といった物理現象の解析と,
PID・状態遷移・AI制御などの制御系設計が,
独立した領域として個別に進められることが多い。
この分離は,設計階層間の情報不整合を招き,
パラメータ再設定やモデル再解析の繰り返しによる
開発遅延や信頼性劣化を引き起こす主要因となっている。

従来のEDAツールや設計支援システムは,
各領域(回路,熱,応力,信号,制御)における
局所的な最適化や解析精度向上には寄与しているが,
設計全体を俯瞰的に連結する統合的情報基盤を欠いている。
そのため,一部の設計変更が他の解析・制御領域へ自動伝搬せず,
設計全体としての一貫性と制御安定性を同時に保証することが困難である。

本研究では,この課題を解決するために,
設計・解析・制御を単一のデータスキーマ上で連結する
新しい工学アーキテクチャ
\textbf{SystemDK with AITL Core} を提案する。
SystemDK(System Design Kernel)は,
仕様策定,制御系設計,FPGA/ASIC回路設計,構造設計,
および FEM/ノイズ解析を統合的に接続し,
設計情報の流れを閉ループ化する知識基盤である。
これにより,構造的変更が制御モデルや解析条件へ即時反映され,
設計全体が自律的に安定化・最適化される環境を実現する。

さらに,AITL(Adaptive Intelligent Tri-Layer)は,
SystemDKの中核制御構造として機能する知的制御フレームワークであり,
PID・FSM・LLMの三層から構成される。
PID層は物理量の実時間安定化を,
FSM層は動作モードと状態遷移の一貫性を,
LLM層は設計データ間の論理整合性を監督する。
この三層制御構造により,
SystemDK全体が設計変更や解析結果に応じて
自動的に再整合・再最適化を行う自律的設計基盤が形成される。

本論文では,
SystemDK with AITL Core の
統合設計フロー,制御理論,および信頼性統合手法を体系的に示す。
提案する設計体系は,
仕様策定から実装検証までの全工程を閉ループで接続し,
物理的安定性と情報的整合性を同時に保証する
次世代の自律設計アーキテクチャの基礎をなすものである。

\section{プレーナーMOS構造の限界}
プレーナーMOSFETでは、チャネル長短縮に伴いドレイン電界がチャネル内部へ侵入する短チャネル効果(SCE)が顕著化する。  
これにより、しきい値電圧の低下、ドレイン誘起バリア低下(DIBL)、およびサブスレッショルド特性の劣化が発生する。  
130\,nm以降では、ゲート酸化膜の薄膜化と供にゲートリークが問題となり、絶縁破壊耐性が急速に低下した。
これらの要因が、三次元構造(Fin型チャネル)への移行を促す直接的契機となった。

\section{FinFET構造とその特徴}
FinFETは、立体的に形成されたシリコンフィン(Fin)をゲートが三方向から包み込む立体チャネル構造を有する。  
この三面ゲート構造により、チャネル電位の空間分布を高精度に制御でき、プレーナーMOSFETに比べてドレイン電界の侵入を大幅に抑制する。  
結果として、短チャネル効果(SCE)の緩和、ドレイン誘起バリア低下(DIBL)の低減、サブスレッショルドスイング(SS)の改善が同時に達成される。

Fin構造の幾何学的パラメータは、電気特性を直接支配する。  
Finの高さを$H$、幅を$W$、Fin数を$n$とすると、有効チャネル幅$W_{\mathrm{eff}}$は次式で表される:
\begin{equation}
W_{\mathrm{eff}} = n(2H + W).
\end{equation}
この式は、側面チャネルが電流伝導に支配的であることを示し、Finの高さを増すことでドライブ能力($I_{\mathrm{ON}}$)を高められる一方、過剰な高さは機械的強度やエッチング制御の面で限界をもたらす。

FinFETの主な利点は、(1)ゲート制御性の向上、(2)オフリーク電流の低減、(3)動作電圧の低下による低消費電力化である。  
一方で、製造上の課題として、Fin寸法の微小ばらつき(Line Edge Roughness, LER)やゲート包囲部の非対称性がしきい値電圧$V_{\mathrm{th}}$の揺らぎを引き起こし、デバイス間の性能均一性を制限する。  
さらに、高アスペクト比Fin構造では、ゲート酸化膜堆積やメタルゲート充填における段差被覆性(Step Coverage)が信頼性を支配する要因となる。

このようにFinFETは、平面構造の限界を克服するだけでなく、デバイス設計における「電界制御性とプロセス均一性の最適折衷」を追求する新たな設計パラダイムを提示した。  
次章では、このFin構造をさらに発展させた全包囲ゲート構造GAA(Gate-All-Around)について述べる。

% (FinFET節の本文のすぐ後ろ)
\begin{table}[t]
  \centering
  \caption{FinFET代表パラメータ例(概念値)}
  \label{tab:finfet_params}
  \begin{tabular}{lccc}
    \toprule
    パラメータ & 記号 & 代表値 & 備考 \\
    \midrule
    Fin高さ & $H$ & 40–60\,nm & 電流駆動に比例 \\
    Fin幅   & $W$ & 5–10\,nm & 短チャネル制御に影響 \\
    Fin数   & $n$ & 2–4 & セル幅内で最適化 \\
    有効チャネル幅 & $W_\mathrm{eff}$ & $n(2H+W)$ & 実効伝導面積 \\
    しきい値電圧 & $V_\mathrm{th}$ & 0.35–0.45\,V & LERに敏感 \\
    オン電流 & $I_\mathrm{ON}$ & $\sim$1\,mA/$\mu$m & 高$H$/小$W$で向上 \\
    オフ電流 & $I_\mathrm{OFF}$ & $<100$\,pA/$\mu$m & 三面ゲート効果 \\
    サブスレッショルドスイング & $SS$ & 65–75\,mV/dec & SCE抑制指標 \\
    \bottomrule
  \end{tabular}
\end{table}

\begin{figure}[t]
  \centering

  \tikzset{
    gate/.style   ={pattern=north east lines, pattern color=black, draw=black},
    oxide/.style  ={fill=black!8, draw=black},
    si/.style     ={fill=white, draw=black},
    sd/.style     ={fill=black!25, draw=black},
    label/.style  ={font=\footnotesize},
    dim/.style    ={-{Latex[length=2mm]}, line width=0.3pt}
  }

  \begin{tikzpicture}[scale=1.0]
    % Substrate
    \draw[fill=black!12, draw=black] (-0.5,0) rectangle (4.5,-0.5);
    \node[label] at (2.0,-0.75) {Substrate};

    % STI / isolation background
    \draw[oxide] (0,0) rectangle (4,1.6);

    % Fin (Si)
    \draw[si] (1.7,0) rectangle (2.3,1.4);

    % Thin gate dielectric hint
    \draw[oxide, line width=0.4pt] (1.6,-0.02) rectangle (2.4,1.42);

    % Gate wrapping three faces
    \draw[gate] (1.35,0.2) rectangle (2.65,1.25);

    % Source / Drain
    \draw[sd] (0.25,0.2) rectangle (1.35,1.25);
    \draw[sd] (2.65,0.2) rectangle (3.75,1.25);

    % Labels
    \node[label] at (0.8,1.4) {S};
    \node[label] at (3.2,1.4) {D};
    \node[label] at (2.0,1.5) {Fin (Si)};
    \node[label] at (2.0,0.95) {Gate};

    % Dimensions (H, W)
    \draw[dim] (1.7,0.0) -- ++(0,1.4) node[midway,left=2pt,label] {$H$};
    \draw[dim] (1.7,1.45) -- ++(0.6,0) node[midway,above=2pt,label] {$W$};
  \end{tikzpicture}

  \caption{FinFETの断面模式図(モノクロ)。ゲートは三面をラップし、幾何パラメータ $H,\,W$ は $W_{\mathrm{eff}}=n(2H+W)$ に寄与する。}
  \label{fig:finfet_detail}
\end{figure}


\section{GAA(Gate-All-Around)構造}
GAA(Gate-All-Around)構造は、チャネルを全周囲からゲート電極で包み込むことにより、静電制御性を極限まで高めたデバイス構造である。  
従来のFinFETではゲートが三方向からチャネルを制御していたのに対し、GAAはチャネルの上下方向にもゲート電界が作用するため、  
チャネルポテンシャルの均一性が飛躍的に向上し、ドレイン電界の侵入がほぼ完全に抑制される。  
この結果、短チャネル効果(SCE)およびドレイン誘起バリア低下(DIBL)がさらに低減され、  
しきい値電圧$V_{\mathrm{th}}$の安定性とサブスレッショルドスイング(SS)の改善が達成される。

ナノシート型GAA構造では、複数のチャネル層(Nanosheet)を垂直方向に積層し、それぞれを独立にゲートで包囲する。  
この積層構造により、面積効率を維持しつつ実効チャネル幅を拡大できる。  
有効チャネル幅$W_{\mathrm{eff}}$はFinFET構造に対して次式で表される:
\begin{equation}
W_{\mathrm{eff}} = 2n(H + W),
\end{equation}
ここで$n$はチャネル層数、$H$はシート厚、$W$はチャネル幅である。  
この式が示すように、GAAではFinFETと異なり上下両面の電流経路が追加されるため、同一フットプリントに対してより高い駆動能力を実現できる。

さらに、GAA構造は「構造的なスケーラビリティ」を備えており、チャネル幅$W$を減少させつつ層数$n$を増やすことで、  
電気的特性を維持したまま幾何学的スケーリングを継続できる。  
この特性により、IMECやSamsungをはじめとする研究機関・メーカーでは、5\,nm世代以降でのGAA採用を加速させ、  
1.4\,nmクラスのノードにおいてもチャネル制御性と電流駆動性能の両立が実現されつつある。

一方で、ナノシートの積層・分離工程には高い製造精度が要求される。  
層間酸化膜の厚さ均一性、ゲート堆積時の段差被覆性、およびチャネル層のエッチング選択性が信頼性に直結する。  
これらの課題に対処するため、Selective Epitaxial Growth(SEG)によるチャネル形成や、Atomic Layer Deposition(ALD)によるゲート絶縁膜制御技術が導入されている。

GAAは、FinFETの静電制御を限界まで拡張した「完全電界包囲デバイス」として、次世代スケーリングの中核技術となっている。  
次章では、n型およびp型デバイスを垂直方向に積層したCFET(Complementary FET)構造について、その統合設計思想と課題を論じる。

\begin{table}[t]
  \centering
  \caption{GAA(Gate-All-Around)ナノシート構造の代表パラメータ}
  \label{tab:gaa_params}
  \begin{tabular}{lccc}
    \toprule
    パラメータ & 記号 & 代表値 & 備考 \\
    \midrule
    ナノシート層数 & $n$ & 3–5 & 積層により$W_\mathrm{eff}$増大 \\
    シート厚 & $T_s$ & 5–8\,nm & チャネル厚制御が性能支配 \\
    シート間距離 & $S$ & 10–12\,nm & 熱干渉と寄生容量に影響 \\
    有効チャネル幅 & $W_\mathrm{eff}$ & $2n(H+W)$ & 全包囲ゲート寄与 \\
    しきい値電圧 & $V_\mathrm{th}$ & 0.35–0.4\,V & 対称制御が容易 \\
    サブスレッショルドスイング & $SS$ & 60–65\,mV/dec & 理想スイングに近似 \\
    電流駆動比(Fin比) & $I_\mathrm{ON}/I_\mathrm{OFF}$ & 2–3× & FinFET比 \\
    \bottomrule
  \end{tabular}
\end{table}

\begin{table*}[t]
  \centering
  \caption{CMOSスケーリングにおけるプロセス構造の進化(Process Evolution of CMOS Scaling)}
  \label{tab:process_evolution}
  \scriptsize
  \setlength{\tabcolsep}{4pt}
  \renewcommand{\arraystretch}{1.1}
  \begin{tabular}{ccccccc}
    \toprule
    ノード & 構造 & 電源電圧[V] & $T_\mathrm{ox}$[nm] & Min L[nm] & 主な特徴 & 技術課題 \\
    Node & Structure & $V_\mathrm{DD}$[V] & $T_\mathrm{ox}$[nm] & Min L[nm] & Key Features & Challenges \\
    \midrule
    90\,nm & プレーナMOS & 1.2 & $\sim$2.0 & $\sim$65 & NiSi導入、Strained-Si、LDD最適化 & リーク電流、寄生容量、リソグラフィ限界 \\
           & Planar MOS &     &              &           & NiSi, strained-Si, optimized LDD & Leakage, parasitics, lithography \\[2pt]
    45\,nm & プレーナMOS & 1.0 & $\sim$1.3 & $\sim$35 & HKMG導入準備、ULK試験導入 & ゲート制御限界、ばらつき拡大 \\
           & Planar MOS &     &             &          & HKMG prep, ULK intro & Gate control limit, variability \\[2pt]
    22\,nm & FinFET初代 & 0.85 & $\sim$0.9 & $\sim$20 & Tri-Gate構造採用、3Dチャネル化 & Finばらつき、設計難度増加 \\
           & 1st Gen FinFET & & & & Tri-Gate, 3D channel & Fin variation, design complexity \\[2pt]
    5\,nm & GAA導入 & 0.6 & $\sim$0.6 & $\sim$8 & Nanosheet構造試験導入 & Routing困難、シート幅制御 \\
           & GAA Pilot & & & & Nanosheet trials & Sheet width control, poor routability \\[2pt]
    2\,nm & CFET試作 & $\lesssim$0.5 & $\sim$0.4 & $\sim$4 & NMOS/PMOS縦積層化 & 熱干渉、配線分離難 \\
           & CFET (R\&D) & & & & Complementary FET stacking & Thermal interference, power routing split \\
    \bottomrule
  \end{tabular}
\end{table*}


\section{CFET(Complementary FET)構造}
CFET(Complementary Field-Effect Transistor)は、n型およびp型のトランジスタを垂直方向に積層した三次元デバイス構造であり、  
従来の平面配置(lateral arrangement)から垂直配置(vertical stacking)への構造転換を特徴とする。  
このアプローチにより、トランジスタ・セルが占有する平面面積を大幅に削減し、  
同一チップ面積あたりの集積度をFinFETやGAA構造を超えて向上させることが可能となる。

CFETでは、nFETとpFETが垂直方向に積層され、それぞれが独立したゲートおよびソース/ドレイン構造を持つ。  
この積層構造により、同一セル内で上下デバイスが相補的に機能するため、標準セル高さを縮小しながら論理駆動能力を維持できる。  
さらに、バックサイドから電源を供給するBackside Power Rail(BPR)技術との統合が容易であり、  
信号線と電源線の物理分離によって配線抵抗・寄生結合を低減し、電源ノイズ耐性を向上させる。

また、GAA技術を前提とするCFETでは、上下のトランジスタを独立に電気的制御できるようにするため、  
チャネル層間のアイソレーション精度と熱干渉の抑制が重要となる。  
n型とp型が発熱特性および移動度特性の異なる材料で構成される場合、  
垂直方向の熱対称性(thermal symmetry)が設計上の支配要因となり、  
電流駆動能力および信頼性(NBTI/HCI耐性)に影響を及ぼす。

製造上の課題としては、垂直積層におけるソース/ドレインの選択的エピタキシャル成長、  
層間絶縁膜(Inter-Layer Dielectric; ILD)の平坦化精度、  
およびn/pトランジスタ間の電気的アイソレーション確保が挙げられる。  
特に、CFETの上部デバイス形成後に下部デバイスの特性が変化しないよう、  
低温プロセス化($<400\,^{\circ}\mathrm{C}$)が必須となる。

CFETは、GAAを超えて「論理対称性と物理空間効率の両立」を目指す究極の三次元CMOSアーキテクチャである。  
現在、IMECやIntel、SamsungなどがCFETプロトタイプを試作段階に進めており、  
1\,nmクラス以降のロジックデバイスにおける主流候補として注目されている。  
次章では、こうした構造進化を支える配線および電源インテグレーション技術(BEOLおよびBPR)について述べる。

\section{BEOL(配線技術)の進化}
デバイススケーリングの進展と並行して、BEOL(Back End of Line)技術も著しく発展してきた。  
トランジスタ性能の向上が配線遅延によって相殺される「RCボトルネック」が顕在化し、  
配線抵抗および配線間容量の削減がシステム性能を支配する時代へと移行した。  
これに対応するため、低誘電率材料(Low-$k$ dielectric)および高導電率金属の導入が進められてきた。

初期のアルミニウム配線から銅(Cu)配線への転換により、導電率が約40\%向上し、  
Dual Damasceneプロセスによって高アスペクト比配線の形成が可能となった。  
このプロセスでは、ビアおよびラインを同時にエッチングし、Cu充填後に化学機械研磨(CMP)で平坦化することで、  
多層配線の高密度実装と優れた表面平坦性を両立している。  
また、バリアメタル(Ta/TaN)の最適化により、エレクトロマイグレーション(EM)およびストレスマイグレーション耐性が大幅に向上した。

誘電体材料についても、SiO$_2$からSiOC系Low-$k$、さらにUltra Low-$k$(ULK)材料へと進化している。  
ただし、低密度化に伴う機械的強度低下や水分吸収による信頼性劣化が課題であり、  
近年ではエアギャップ構造やカーボン含有絶縁膜(SiOC:H, SiCN系)を利用した  
「誘電率と機械強度の同時最適化」が研究の主流となっている。  

7\,nm世代以降では、電源供給経路をシリコン裏面側に再配置する  
Backside Power Rail(BPR)アーキテクチャが導入されつつある。  
BPRは、ロジック層上面の配線混雑を緩和し、信号配線の自由度を拡大することで  
セル高さのさらなる縮小とIRドロップの低減を同時に実現する。  
また、CFETのような垂直積層デバイスと組み合わせることで、  
前工程(FEOL)と後工程(BEOL)の境界が曖昧化し、  
「デバイス–配線一体最適化(Device–Interconnect Co-Optimization; DICO)」が不可欠な設計指針となっている。

一方、Cu配線では微細化に伴うバリア層比率の増大が抵抗上昇を引き起こすため、  
ルテニウム(Ru)やコバルト(Co)などの次世代導電材料が注目されている。  
これらはバリアレス配線(barrier-less interconnect)を可能とし、  
界面拡散の抑制とEM耐性の両立を実現する有力候補である。  
さらに、カーボンナノチューブ(CNT)やグラフェン配線など、  
原子スケールでの新規配線材料も研究段階にある。

このように、BEOL技術は単なる「配線形成プロセス」から、  
電源分離、熱拡散、機械信頼性、そしてシステム全体最適化を担う統合技術へと進化している。  
次章では、この配線スケーリングを支える主要パラメータおよび構造スケーリング法則について詳述する。

\begin{table}[t]
  \centering
  \caption{BEOL(配線層)技術のスケーリング推移}
  \label{tab:beol_scaling}
  \begin{tabular}{lcccc}
    \toprule
    ノード & 配線材 & 絶縁材 & 最小ピッチ (nm) & 電源構造 \\
    \midrule
    65\,nm  & Cu & SiO$_2$ & 200 & Frontside Power \\
    28\,nm  & Cu & Low-k & 100 & Dual Damascene \\
    7\,nm   & Cu & Low-k+Co Liner & 40 & Power Grid \\
    3\,nm   & Ru/Cu & Porous Low-k & 28 & Backside Power Rail \\
    2\,nm   & Ru & Airgap Low-k & 24 & BPR + TSV Feed \\
    \bottomrule
  \end{tabular}
\end{table}

\begin{figure}[t]
  \centering
  \tikzset{
    diel/.style   ={fill=black!5, draw=black},
    metal/.style  ={fill=black!35, draw=black},
    barrier/.style={fill=black!60, draw=black},
    label/.style  ={font=\footnotesize}
  }

  %=== (a) Dual Damascene (frontside power) ===
  \begin{tikzpicture}[scale=0.9]
    \node[label, anchor=west] at (-0.2,3.3) {\textbf{(a) Dual Damascene(従来前面配線)}};
    % dielectric stack
    \draw[diel] (0,0) rectangle (4,3.0);
    % lines (Cu) with barrier
    \foreach \x in {0.4,1.8,3.2}{
      \draw[barrier] (\x,1.8) rectangle (\x+0.4,2.6);
      \draw[metal]   (\x+0.05,1.85) rectangle (\x+0.35,2.55);
    }
    % vias
    \draw[barrier] (0.55,0.6) rectangle (0.85,1.8);
    \draw[metal]   (0.6,0.65) rectangle (0.8,1.75);
    \draw[barrier] (3.35,0.6) rectangle (3.65,1.8);
    \draw[metal]   (3.4,0.65) rectangle (3.6,1.75);
    % cell area note
    \node[label] at (2,0.3) {信号・電源が同一面で競合 → 配線混雑};
  \end{tikzpicture}

  \vspace{1.2ex}

  %=== (b) Backside Power Rail (BPR) ===
  \begin{tikzpicture}[scale=0.9]
    \node[label, anchor=west] at (-0.2,3.6) {\textbf{(b) Backside Power Rail(裏面電源分離)}};
    % frontside dielectric (signal)
    \draw[diel] (0,0.8) rectangle (4,3.2);
    % frontside signal lines
    \foreach \x in {0.7,2.0,3.3}{
      \draw[metal] (\x,2.2) rectangle (\x+0.35,3.0);
    }
    \node[label] at (3.3,3.25) {信号配線};
    % substrate slab
    \draw[fill=black!10, draw=black] (0,0.2) rectangle (4,0.8);
    \node[label] at (2,0.05) {Si Substrate};
    % backside rails
    \draw[metal] (0.2,-0.2) rectangle (1.9,0.2);
    \draw[metal] (2.1,-0.2) rectangle (3.8,0.2);
    \node[label] at (1.05,-0.35) {BPR-VDD};
    \node[label] at (2.95,-0.35) {BPR-VSS};
    % backside vias (power TSV-like)
    \foreach \x in {0.5,3.5}{
      \draw[metal] (\x,0.2) rectangle (\x+0.25,0.8);
    }
    \node[label] at (2,0.45) {裏面電源でセル上面が空き → 配線自由度↑};
  \end{tikzpicture}

  \caption{BEOL配線の概念図:Dual Damascene(前面)では信号と電源が競合、BPRでは電源を裏面に再配置し配線混雑を緩和。}
  \label{fig:beol_bpr}
\end{figure}


\section{スケーリングパラメータの推移}
トランジスタの微細化は、従来「Dennardスケーリング」に基づき、  
チャネル長、酸化膜厚、電源電圧を一定の比率で縮小することで性能と消費電力の両立を図ってきた。  
しかし、90\,nm世代以降ではリーク電流と電界強度の増大により、この単純な比例則が崩壊し、  
電気的・構造的な新しいスケーリング指標が求められるようになった。

表\ref{tab:scaling}に主要プロセスノードにおけるスケーリングパラメータの推移を示す。  
電源電圧$V_{DD}$は130\,nm世代の1.2\,Vから、2\,nm世代では0.6\,V付近まで低下した。  
一方で、ゲート酸化膜厚$T_\text{ox}$は物理的に1.5\,nmを下回る領域に突入し、  
トンネル電流によるゲートリークが急増するため、High-$k$/Metal Gate(HKMG)技術が導入された。  
これにより、実効酸化膜厚(EOT: Equivalent Oxide Thickness)を維持しつつ、  
リーク電流$I_\text{G}$を数桁低減できるようになった。

さらに、FinFETおよびGAA構造の採用により、静電的なゲート制御性が強化され、  
短チャネル効果(SCE)に対する耐性が飛躍的に向上した。  
電源電圧の低下による駆動電流$I_\text{ON}$の減少は、  
チャネル形状の三次元化とキャリア移動度改善によって補償されている。  
これらの構造的最適化が、物理限界に接近するスケーリングを可能にしている。

\begin{table}[htbp]
\centering
\caption{スケーリングパラメータの推移}
\label{tab:scaling}
\begin{tabular}{lccc}
\toprule
ノード & 電源電圧 (V) & $T_\text{ox}$ (nm) & 構造 \\
\midrule
130\,nm & 1.2 & 2.5 & Planar \\
65\,nm  & 1.0 & 1.8 & Planar / HKMG \\
28\,nm  & 0.9 & 1.2 & FinFET \\
5\,nm   & 0.7 & 0.9 & GAA \\
2\,nm   & 0.6 & 0.7 & CFET \\
\bottomrule
\end{tabular}
\end{table}

図示されるように、$V_{DD}$の低下と$T_\text{ox}$の薄膜化はもはや線形関係を持たず、  
熱雑音限界およびトンネルリーク限界がスケーリングの支配要因となっている。  
この結果、今後の微細化では、単なる寸法縮小ではなく、  
電気・材料・構造パラメータを同時最適化する「More-than-Moore型スケーリング」への転換が進行している。

CFET世代以降では、  
\begin{itemize}
    \item 高移動度チャネル材料(SiGe, Ge, III–V)による$I_\text{ON}/I_\text{OFF}$改善、  
    \item BPRと低抵抗BEOLによる電圧降下抑制、  
    \item 電界集中を抑制する多層ゲート絶縁膜構造、  
\end{itemize}
などの多次元設計が必要となる。  
これにより、スケーリングパラメータの概念は幾何学的な寸法値から、  
「電気的・熱的・構造的設計空間の最適化指標」へと拡張されつつある。

\section{BSIM-CMGモデリング}
BSIM-CMG(Common Multi-Gate)モデルは、FinFETおよびGAA構造に適用可能な統一的SPICEモデルである。  
非平面構造特有の電界分布やチャネル結合を考慮し、ゲート面積、ドレイン電界、温度依存性を正確に再現できる。  
このモデルにより、回路設計段階での性能・信頼性予測が高精度化した。

\section{信頼性と構造設計}
微細化の進行に伴い、デバイス信頼性は動作限界を規定する主要因となっている。  
特に、BTI(Bias Temperature Instability)、HCI(Hot Carrier Injection)、および自己発熱(Self-Heating)など、  
時間依存劣化(Time-Dependent Degradation)がデバイス寿命を支配する要素として顕在化している。  
スケーリングの最終段階では、電気・熱・機械応力の複合的相互作用を考慮した「構造的信頼性設計(Structural Reliability Design)」が不可欠となる。

\subsection{電界劣化と界面反応}
BTIはゲート酸化膜界面での電荷捕獲・放出反応に起因し、  
長時間のゲートバイアス印加によりしきい値電圧$V_\mathrm{th}$が時間経過とともに変化する。  
特に$p$MOSでは負バイアス温度不安定性(NBTI)が支配的であり、  
酸化膜中の水素脱離反応および界面欠陥生成が主な劣化メカニズムとされる。  
一方、HCIは高電界ドレイン領域でのキャリア加速と衝突イオン化により発生し、  
酸化膜損傷およびホットキャリア捕獲を引き起こす。  
これらはいずれも局所電界強度および温度上昇の積分効果に比例して進行するため、  
デバイス形状と電界分布設計の両面からの対策が求められる。

\subsection{熱対称性と構造的緩和}
FinFETやGAAのような三次元構造では、チャネル周囲の熱伝導経路が複雑化し、  
局所的な温度勾配が形成されやすくなる。  
Fin側壁の酸化膜は熱伝導率が低く、チャネル内部で自己発熱が蓄積し、  
移動度劣化やBTI加速を引き起こす。  
したがって、構造設計段階において「熱対称性(Thermal Symmetry)」を確保することが重要である。  
具体的には、チャネル上下の温度勾配を最小化するようにゲート金属やSTI(Shallow Trench Isolation)を配置し、  
熱拡散経路をシリコン基板や金属層側へ誘導する設計が有効である。

さらに、GAAやCFET構造では複数のチャネル層が積層されるため、  
上下トランジスタ間の熱干渉が新たな信頼性課題となる。  
下層に高熱伝導材料(例:Geチャネル/Wゲート)を用いる、  
または熱拡散経路を金属配線層に接続する「熱ブリッジ構造(Thermal Bridge)」を導入することで、  
積層構造全体の熱シンメトリを最適化できる。

\subsection{構造信頼性設計のパラダイム}
FinFETからCFETへの進化は、単なる寸法縮小ではなく、  
「構造を通じて信頼性を設計する」という新たな設計哲学への転換を意味する。  
従来は事後的に評価されていた劣化現象を、  
設計初期段階における構造パラメータ最適化で未然に抑制するアプローチである。  
たとえば、Finピッチ、ゲート包囲角度、チャネル積層間距離などを、  
熱・電界分布解析と統合的に最適化することにより、  
長期信頼性(Lifetime Reliability)を設計ルールの一部として保証することが可能となる。

今後のCFET世代では、熱、電界、機械応力の三者を同時に考慮した  
「マルチフィジックス信頼性設計(Multiphysics Reliability Design)」が必須となる。  
すなわち、デバイスの形状そのものが信頼性を左右する設計変数であり、  
**構造そのものが信頼性パラメータである時代** に突入している。

\begin{figure}[t]
  \centering
  \tikzset{
    box/.style   ={draw=black, fill=black!5},
    hot/.style   ={draw=black, fill=black!60},
    warm/.style  ={draw=black, fill=black!35},
    cool/.style  ={draw=black, fill=black!15},
    gate/.style  ={pattern=north east lines, pattern color=black, draw=black},
    sd/.style    ={fill=black!25, draw=black},
    label/.style ={font=\footnotesize}
  }

  %=== (a) FinFET thermal map ===
  \begin{tikzpicture}[scale=0.9]
    \node[label, anchor=west] at (-0.2,2.7) {\textbf{(a) FinFET: Self-Heating}};
    \draw[box] (0,0) rectangle (4,2.4);    % frame
    \draw[sd] (0.2,0.4) rectangle (0.9,2.0);
    \draw[sd] (3.1,0.4) rectangle (3.8,2.0);
    % fin thermal core
    \draw[cool] (1.8,0.4) rectangle (2.2,2.0);
    \draw[warm] (1.75,0.6) rectangle (2.25,1.8);
    \draw[hot]  (1.7,0.9) rectangle (2.3,1.5);
    % gate wrap
    \draw[gate] (1.3,0.5) rectangle (2.7,1.9);
    \node[label] at (2,2.2) {高温域がチャネル中央に集中};
  \end{tikzpicture}

  \vspace{1.2ex}

  %=== (b) CFET thermal coupling ===
  \begin{tikzpicture}[scale=0.9]
    \node[label, anchor=west] at (-0.2,3.2) {\textbf{(b) CFET: Vertical Thermal Coupling}};
    \draw[box] (0,0) rectangle (4,2.9);    % frame
    \draw[sd] (0.2,0.4) rectangle (0.9,2.5);
    \draw[sd] (3.1,0.4) rectangle (3.8,2.5);
    % lower device hotspot
    \draw[cool] (1.2,0.4) rectangle (2.8,1.2);
    \draw[warm] (1.4,0.55) rectangle (2.6,1.05);
    \draw[hot]  (1.6,0.7) rectangle (2.4,0.9);
    % ILD spacer
    \draw[box, fill=black!12] (0,1.25) rectangle (4,1.45);
    % upper device hotspot
    \draw[cool] (1.2,1.5) rectangle (2.8,2.3);
    \draw[warm] (1.4,1.65) rectangle (2.6,2.15);
    \draw[hot]  (1.6,1.8) rectangle (2.4,2.0);
    % gates (輪郭のみ)
    \draw[gate] (1.1,0.5) rectangle (2.9,1.1);
    \draw[gate] (1.1,1.6) rectangle (2.9,2.2);
    \node[label] at (2,2.7) {上下FET間で熱干渉(対称設計が鍵)};
  \end{tikzpicture}

  \caption{自己発熱と熱対称性の概念図:FinFETではチャネル中央に高温域、CFETでは上下FET間の熱干渉が顕在化。熱対称性の最適化が信頼性の鍵となる。}
  \label{fig:thermal_symmetry}
\end{figure}


\section{リソグラフィとマスク技術}
デバイス微細化の限界を押し広げるうえで、リソグラフィ技術の進歩は決定的な役割を果たしてきた。  
ArF液浸露光(ArF Immersion Lithography)からEUV(Extreme Ultraviolet)露光への移行は、  
露光波長の大幅短縮によって解像限界を縮小し、  
多重パターニング工程(Double/Quadruple Patterning)を劇的に削減した。  
これにより、配線ピッチの微細化に伴うプロセス変動(Overlay Error、CD Variation)が大幅に低減し、  
スループットおよび歩留まりの両立が可能となった。

\subsection{EUV露光と多重パターニング削減}
従来のArF液浸露光では、NA(Numerical Aperture)の限界により、  
40\,nm以下のパターン形成に複数回の露光・エッチングを要した。  
これに対し、13.5\,nm波長のEUV光を用いることで、  
1回の露光で30\,nmクラスのライン&スペースパターンが形成可能となった。  
この結果、工程数削減による線幅変動の抑制とともに、  
レジストパターンの位置ずれや累積誤差が大幅に緩和された。  
一方で、EUV光源の出力安定性やミラー反射損失、レジスト感度とLWR(Line Width Roughness)のトレードオフが  
次世代微細化の課題として残されている。

\subsection{マスク3D効果と光学補正}
微細パターン形成の高精度化には、マスク上の三次元効果(Mask 3D Effect)を考慮した補正設計が必須である。  
EUVマスクは多層ミラー構造を持ち、斜入射により反射位相と透過率がパターン位置によって変化する。  
このため、マスク転写像の歪みを補償するために、  
光学近接補正(OPC: Optical Proximity Correction)およびソースマスク最適化(SMO: Source Mask Optimization)が導入されている。  
これらの手法は、マスクパターン形状を露光シミュレーションと連動させ、  
最終的なシリコン上パターン精度をナノメートルスケールで保証する。

さらに、GAAおよびCFETのような三次元構造デバイスでは、  
パターン形状が深堀エッチングやゲート包囲構造と密接に関連するため、  
EUVマスク設計段階での形状歪み補正とアライメント精度がデバイス特性の再現性を左右する。  
特にCFETでは、上下層FETのゲートおよびソース・ドレインを個別に定義する必要があるため、  
多層マスク整合技術(Multi-Layer Alignment Technology)が歩留まり向上の鍵となる。

\subsection{次世代リソグラフィへの展望}
2\,nm世代以降では、High-NA EUV(開口数1.0以上)の適用が進み、  
より高い解像力とパターン忠実度を両立する見込みである。  
また、EUVペリクル(保護膜)の透過損失や、レジスト材料の化学増幅機構に起因するノイズ制限が  
量産性のボトルネックとなる可能性が指摘されている。  
これに対し、**EUVと電子線マスク補正(E-Beam Mask Repair)を組み合わせたハイブリッド露光戦略** や、  
**Computational Lithography(計算リソグラフィ)によるパターン最適化** の研究が進展している。

このように、リソグラフィとマスク技術は単なる描画プロセスではなく、  
**デバイス構造設計と一体化した「構造整合工学(Structural Lithography Engineering)」** へと進化している。  
EUV世代以降の微細化では、露光・エッチング・デバイス形状設計を統合した最適化が、  
CFET時代の高精度かつ高信頼な製造基盤を支える中心的要素となる。

\section{結論}
本稿では、130\,nm以降のCMOSスケーリングをFinFET、GAA、CFETの構造的観点から体系的に整理した。  
構造進化は、微細化限界を超えるための「設計上の自由度」を拡大するものであり、  
今後は、熱・電源・信頼性を同時最適化する統合設計が重要となる。  
CFET以降の時代には、材料・構造・AI設計が連携する新しいスケーリングパラダイムが求められる。


% -----------------------------------------------------
% 謝辞
% -----------------------------------------------------
\section*{謝辞}
本稿の作成にあたり、半導体デバイススケーリング・信頼性・プロセス統合に関して示唆に富む議論を行ってくださった
産業界および学術界の関係各位に深く感謝する。

% -----------------------------------------------------
% 参考文献
% -----------------------------------------------------
% --- force at least one item for IEEEtran ---
\nocite{*}
\bibliographystyle{IEEEtran}
\bibliography{refs}

% -----------------------------------------------------
% 著者略歴
% -----------------------------------------------------
\section*{著者略歴}
\textbf{三溝 真一}(Shinichi Samizo)は、信州大学大学院 工学系研究科 電気電子工学専攻にて修士号を取得した。  
その後、セイコーエプソン株式会社に勤務し、半導体ロジック/メモリ/高耐圧インテグレーション、  
およびインクジェット薄膜ピエゾアクチュエータならびにPrecisionCoreプリントヘッドの製品化に従事した。  
現在は独立系半導体研究者として、プロセス/デバイス教育、メモリアーキテクチャ、AIシステム統合などに取り組んでいる。  
連絡先: \href{mailto:shin3t72@gmail.com}{shin3t72@gmail.com}.

\end{document}
