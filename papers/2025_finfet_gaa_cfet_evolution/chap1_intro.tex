\section{序論}
本章では、CMOSスケーリングの歴史的背景と、プレーナーMOS構造の限界について概説する。  
130\,nm世代以降、チャネル長の短縮に伴って短チャネル効果(SCE: Short Channel Effect)が顕在化し、  
ゲート電界によるチャネル制御力が低下した。これにより、サブスレッショルドリーク電流やゲートリーク電流が増大し、  
単純な寸法スケーリングによる性能向上は限界に達した。  
この課題を解決するため、電界制御効率を根本的に高める新たな三次元構造が求められるようになった。

FinFETは、その最初の転換点として、立体的なフィン(Fin)チャネルを三方向からゲートで包み込む構造を採用し、  
優れた電界制御性とオン/オフ特性の両立を実現した。  
続くGAA(Gate-All-Around)構造では、チャネルを全周囲からゲートで完全に包み込むことにより、  
静電制御性・デバイス均一性・しきい値電圧安定性のさらなる向上を達成した。  
さらにCFET(Complementary FET)では、n型およびp型デバイスを垂直方向に積層することで、  
面積効率と配線密度を飛躍的に高め、電源・信号経路の空間分離を可能にした。  
このように、トランジスタは“平面の微細化”から“構造の最適化”へと進化している。

本論文では、FinFETからGAA、さらにCFETへと至る構造進化を、  
(1) 電界制御、(2) 熱対称性、(3) 電源分離、(4) 再現性設計の四つの観点から体系的に整理する。  
あわせて、High-$k$/Metal Gate(HKMG)技術、BEOLスケーリング(Low-$k$誘電体、Dual Damascene、Backside Power Rail)、  
およびBSIM-CMGモデリングを統合的に扱い、  
次世代CMOS技術における「\textbf{構造そのものが信頼性を設計するパラメータとなる時代}」の到来を明確に示すことを目的とする。
