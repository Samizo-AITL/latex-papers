\section{序論}
本章では、CMOSスケーリングの歴史的背景とプレーナーMOS構造の限界について概説する。  
130\,nm世代以降、チャネル長の短縮に伴い短チャネル効果(SCE: Short Channel Effect)が顕著化し、  
ゲート電界によるチャネル制御力が低下するとともに、サブスレッショルドリーク電流やゲートリーク電流の増大が問題となった。  
このため、単純な寸法スケーリングによる性能向上は限界に達し、電界制御効率を高めるための新たな三次元構造が求められるようになった。

FinFETは、その最初の転換点として、立体的なフィン構造を用いることでゲート制御性とオン/オフ特性の両立を実現した。  
続くGAA(Gate-All-Around)構造では、チャネルを全周囲からゲートで包み込むことにより、さらなる電界制御性とデバイス均一性を獲得した。  
さらにCFET(Complementary FET)では、n型およびp型デバイスを垂直積層することにより、面積効率および配線密度を飛躍的に向上させ、  
トランジスタ構造をデバイス・インテグレーションの観点から再定義する方向へ進化している。

本論文では、これらFinFETからGAA、さらにCFETに至る構造進化を、  
(1) 電界制御、(2) 熱対称性、(3) 電源分離、(4) 再現性設計の四つの観点から体系的に整理する。  
また、High-$k$/Metal Gate(HKMG)技術、BEOLスケーリング(Low-$k$誘電体、Dual Damascene、Backside Power Rail)およびBSIM-CMGモデリングを統合的に扱い、  
「構造によって信頼性を設計する時代」の到来を示すことを目的とする。
