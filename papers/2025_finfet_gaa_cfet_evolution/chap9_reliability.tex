\section{信頼性と構造設計}
微細化の進行に伴い、デバイス信頼性は動作限界を規定する主要因となっている。  
特に、BTI(Bias Temperature Instability)、HCI(Hot Carrier Injection)、および自己発熱(Self-Heating)など、  
時間依存劣化(Time-Dependent Degradation)がデバイス寿命を支配する要素として顕在化している。  
スケーリングの最終段階では、電気・熱・機械応力の複合的相互作用を考慮した「構造的信頼性設計(Structural Reliability Design)」が不可欠となる。

\subsection{電界劣化と界面反応}
BTIはゲート酸化膜界面での電荷捕獲・放出反応に起因し、  
長時間のゲートバイアス印加によりしきい値電圧$V_\mathrm{th}$が時間経過とともに変化する。  
特に$p$MOSでは負バイアス温度不安定性(NBTI)が支配的であり、  
酸化膜中の水素脱離反応および界面欠陥生成が主な劣化メカニズムとされる。  
一方、HCIは高電界ドレイン領域でのキャリア加速と衝突イオン化により発生し、  
酸化膜損傷およびホットキャリア捕獲を引き起こす。  
これらはいずれも局所電界強度および温度上昇の積分効果に比例して進行するため、  
デバイス形状と電界分布設計の両面からの対策が求められる。

\subsection{熱対称性と構造的緩和}
FinFETやGAAのような三次元構造では、チャネル周囲の熱伝導経路が複雑化し、  
局所的な温度勾配が形成されやすくなる。  
Fin側壁の酸化膜は熱伝導率が低く、チャネル内部で自己発熱が蓄積し、  
移動度劣化やBTI加速を引き起こす。  
したがって、構造設計段階において「熱対称性(Thermal Symmetry)」を確保することが重要である。  
具体的には、チャネル上下の温度勾配を最小化するようにゲート金属やSTI(Shallow Trench Isolation)を配置し、  
熱拡散経路をシリコン基板や金属層側へ誘導する設計が有効である。

さらに、GAAやCFET構造では複数のチャネル層が積層されるため、  
上下トランジスタ間の熱干渉が新たな信頼性課題となる。  
下層に高熱伝導材料(例:Geチャネル/Wゲート)を用いる、  
または熱拡散経路を金属配線層に接続する「熱ブリッジ構造(Thermal Bridge)」を導入することで、  
積層構造全体の熱シンメトリを最適化できる。

\subsection{構造信頼性設計のパラダイム}
FinFETからCFETへの進化は、単なる寸法縮小ではなく、  
「構造を通じて信頼性を設計する」という新たな設計哲学への転換を意味する。  
従来は事後的に評価されていた劣化現象を、  
設計初期段階における構造パラメータ最適化で未然に抑制するアプローチである。  
たとえば、Finピッチ、ゲート包囲角度、チャネル積層間距離などを、  
熱・電界分布解析と統合的に最適化することにより、  
長期信頼性(Lifetime Reliability)を設計ルールの一部として保証することが可能となる。

今後のCFET世代では、熱、電界、機械応力の三者を同時に考慮した  
「マルチフィジックス信頼性設計(Multiphysics Reliability Design)」が必須となる。  
すなわち、デバイスの形状そのものが信頼性を左右する設計変数であり、  
**構造そのものが信頼性パラメータである時代** に突入している。

\begin{figure}[t]
  \centering
  \tikzset{
    box/.style   ={draw=black, fill=black!5},
    hot/.style   ={draw=black, fill=black!60},
    warm/.style  ={draw=black, fill=black!35},
    cool/.style  ={draw=black, fill=black!15},
    gate/.style  ={pattern=north east lines, pattern color=black, draw=black},
    sd/.style    ={fill=black!25, draw=black},
    label/.style ={font=\footnotesize}
  }

  %=== (a) FinFET thermal map ===
  \begin{tikzpicture}[scale=0.9]
    \node[label, anchor=west] at (-0.2,2.7) {\textbf{(a) FinFET: Self-Heating}};
    \draw[box] (0,0) rectangle (4,2.4);    % frame
    \draw[sd] (0.2,0.4) rectangle (0.9,2.0);
    \draw[sd] (3.1,0.4) rectangle (3.8,2.0);
    % fin thermal core
    \draw[cool] (1.8,0.4) rectangle (2.2,2.0);
    \draw[warm] (1.75,0.6) rectangle (2.25,1.8);
    \draw[hot]  (1.7,0.9) rectangle (2.3,1.5);
    % gate wrap
    \draw[gate] (1.3,0.5) rectangle (2.7,1.9);
    \node[label] at (2,2.2) {高温域がチャネル中央に集中};
  \end{tikzpicture}

  \vspace{1.2ex}

  %=== (b) CFET thermal coupling ===
  \begin{tikzpicture}[scale=0.9]
    \node[label, anchor=west] at (-0.2,3.2) {\textbf{(b) CFET: Vertical Thermal Coupling}};
    \draw[box] (0,0) rectangle (4,2.9);    % frame
    \draw[sd] (0.2,0.4) rectangle (0.9,2.5);
    \draw[sd] (3.1,0.4) rectangle (3.8,2.5);
    % lower device hotspot
    \draw[cool] (1.2,0.4) rectangle (2.8,1.2);
    \draw[warm] (1.4,0.55) rectangle (2.6,1.05);
    \draw[hot]  (1.6,0.7) rectangle (2.4,0.9);
    % ILD spacer
    \draw[box, fill=black!12] (0,1.25) rectangle (4,1.45);
    % upper device hotspot
    \draw[cool] (1.2,1.5) rectangle (2.8,2.3);
    \draw[warm] (1.4,1.65) rectangle (2.6,2.15);
    \draw[hot]  (1.6,1.8) rectangle (2.4,2.0);
    % gates (輪郭のみ)
    \draw[gate] (1.1,0.5) rectangle (2.9,1.1);
    \draw[gate] (1.1,1.6) rectangle (2.9,2.2);
    \node[label] at (2,2.7) {上下FET間で熱干渉(対称設計が鍵)};
  \end{tikzpicture}

  \caption{自己発熱と熱対称性の概念図:FinFETではチャネル中央に高温域、CFETでは上下FET間の熱干渉が顕在化。熱対称性の最適化が信頼性の鍵となる。}
  \label{fig:thermal_symmetry}
\end{figure}

