\section{信頼性と構造設計}
微細化の進行に伴い、デバイス信頼性は動作限界を規定する主要因となっている。  
特に、BTI(Bias Temperature Instability)、HCI(Hot Carrier Injection)、および自己発熱(Self-Heating)といった
時間依存劣化(Time-Dependent Degradation)が支配的となる。

\subsection{電界劣化と界面反応}
BTIは、ゲート酸化膜界面における電荷捕獲・放出反応に起因し、  
ゲート電圧ストレス下でしきい値電圧$V_\text{th}$が時間とともにシフトする現象である。  
特に負バイアス温度不安定性(NBTI)は$p$MOSに顕著であり、  
酸化膜中の水素脱離反応や界面欠陥生成が支配的メカニズムとされる。  
一方、HCIは高電界ドレイン領域でのキャリア加速と衝突イオン化により発生し、  
酸化膜損傷とチャネルホットキャリア蓄積を引き起こす。  
これらの現象はいずれも、電界集中と局所温度上昇の積分効果に比例して進行する。

\subsection{熱対称性と構造的緩和}
FinFETおよびGAAデバイスでは、三次元構造により放熱経路が複雑化する。  
Fin形状の側壁酸化膜は熱伝導率が低く、チャネル内で局所的な温度勾配が形成されるため、  
Self-Heating効果が移動度劣化およびBTI加速を引き起こす。  
このため、構造設計段階における「熱シンメトリ(Thermal Symmetry)」の確保が不可欠である。  
チャネル上下での温度勾配を最小化し、ゲート金属やSTI(Shallow Trench Isolation)の熱伝導経路を最適化することにより、  
局所過熱領域を抑制できる。

GAAおよびCFET構造では、チャネル層の積層により上下FET間の熱干渉が生じる。  
これに対して、下層トランジスタに高熱伝導材料(例:Geチャネル/Wゲート)を用いる、  
または熱拡散パスを金属層側に逃がす「熱ブリッジ構造(Thermal Bridge)」を設計的に導入するなど、  
構造的放熱設計が信頼性確保の中心的要素となる。

\subsection{構造信頼性設計のパラダイム}
FinFETからCFETへの進化は、単なる寸法縮小ではなく、  
「構造を通じて信頼性を設計する」という新たな設計思想への転換を意味する。  
従来は事後的に評価されていた劣化現象を、  
設計初期段階での構造パラメータ最適化により抑制するアプローチである。  
たとえば、Finピッチやゲート包囲角度、チャネル積層間距離を熱・電界分布解析と統合的に最適化することで、  
長期信頼性(Lifetime Reliability)をデザインルールの一部として保証することが可能となる。

今後は、CFETを含む垂直統合構造に対して、  
熱・電界・機械応力を同時に考慮した「マルチフィジックス信頼性設計」が必須となる。  
このように、スケーリングの最終段階では、  
**構造そのものが信頼性の設計パラメータとなる**という新たな設計パラダイムが形成されつつある。

% 例:信頼性節の中盤に
\begin{figure}[t]
  \centering
  \tikzset{
    box/.style   ={draw=black, fill=black!5},
    hot/.style   ={draw=black, fill=black!60},
    warm/.style  ={draw=black, fill=black!35},
    cool/.style  ={draw=black, fill=black!15},
    gate/.style  ={pattern=north east lines, pattern color=black, draw=black},
    sd/.style    ={fill=black!25, draw=black},
    label/.style ={font=\footnotesize}
  }

  %=== (a) FinFET thermal map ===
  \begin{tikzpicture}[scale=0.9]
    \node[label, anchor=west] at (-0.2,2.7) {\textbf{(a) FinFET: Self-Heating}};
    \draw[box] (0,0) rectangle (4,2.4);    % frame
    \draw[sd] (0.2,0.4) rectangle (0.9,2.0);
    \draw[sd] (3.1,0.4) rectangle (3.8,2.0);
    % fin thermal core
    \draw[cool] (1.8,0.4) rectangle (2.2,2.0);
    \draw[warm] (1.75,0.6) rectangle (2.25,1.8);
    \draw[hot]  (1.7,0.9) rectangle (2.3,1.5);
    % gate wrap
    \draw[gate] (1.3,0.5) rectangle (2.7,1.9);
    \node[label] at (2,2.2) {高温域がチャネル中央に集中};
  \end{tikzpicture}

  \vspace{1.2ex}

  %=== (b) CFET thermal coupling ===
  \begin{tikzpicture}[scale=0.9]
    \node[label, anchor=west] at (-0.2,3.2) {\textbf{(b) CFET: Vertical Thermal Coupling}};
    \draw[box] (0,0) rectangle (4,2.9);    % frame
    \draw[sd] (0.2,0.4) rectangle (0.9,2.5);
    \draw[sd] (3.1,0.4) rectangle (3.8,2.5);
    % lower device hotspot
    \draw[cool] (1.2,0.4) rectangle (2.8,1.2);
    \draw[warm] (1.4,0.55) rectangle (2.6,1.05);
    \draw[hot]  (1.6,0.7) rectangle (2.4,0.9);
    % ILD spacer
    \draw[box, fill=black!12] (0,1.25) rectangle (4,1.45);
    % upper device hotspot
    \draw[cool] (1.2,1.5) rectangle (2.8,2.3);
    \draw[warm] (1.4,1.65) rectangle (2.6,2.15);
    \draw[hot]  (1.6,1.8) rectangle (2.4,2.0);
    % gates (輪郭のみ)
    \draw[gate] (1.1,0.5) rectangle (2.9,1.1);
    \draw[gate] (1.1,1.6) rectangle (2.9,2.2);
    \node[label] at (2,2.7) {上下FET間で熱干渉(対称設計が鍵)};
  \end{tikzpicture}

  \caption{自己発熱と熱対称性の概念図:FinFETではチャネル中央に高温域、CFETでは上下FET間の熱干渉が顕在化。熱対称性の最適化が信頼性の鍵となる。}
  \label{fig:thermal_symmetry}
\end{figure}

