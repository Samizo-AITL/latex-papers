\begin{table*}[t]
  \centering
  \caption{CMOSスケーリングにおけるプロセス構造の進化(Process Evolution of CMOS Scaling)}
  \label{tab:process_evolution}
  % フォントと列間調整
  \scriptsize
  \setlength{\tabcolsep}{2pt}
  \renewcommand{\arraystretch}{1.05}
  % カラム幅にリサイズ
  \resizebox{\textwidth}{!}{%
  \begin{tabular}{ccccccc}
    \toprule
    ノード / \textbf{Node} & 構造 / \textbf{Structure} & \textbf{$V_\mathrm{DD}$ [V]} &
    \textbf{$T_\mathrm{ox}$ [nm]} & \textbf{Min L [nm]} &
    主な特徴 / \textbf{Key Features} & 技術課題 / \textbf{Challenges} \\
    \midrule
    90\,nm & プレーナMOS / Planar MOS & 1.2 & $\sim$2.0 & $\sim$65 &
    NiSi導入、Strained-Si、LDD最適化 / NiSi, strained-Si, optimized LDD &
    リーク電流、寄生容量、リソグラフィ限界 / Leakage, parasitics, lithography \\
    65\,nm & プレーナMOS / Planar MOS & 1.1 & $\sim$1.7 & $\sim$50 &
    高濃度チャネル、Low-k導入 / Heavily doped channel, Low-k ILD &
    短チャネル効果、配線遅延 / SCE, interconnect delay \\
    45\,nm & プレーナMOS / Planar MOS & 1.0 & $\sim$1.3 & $\sim$35 &
    HKMG導入準備、ULK試験導入 / HKMG prep, ULK intro &
    ゲート制御限界、ばらつき拡大 / Gate control limit, variability \\
    32\,nm & HKMGプレーナ / HKMG Planar & 0.9 & $\sim$1.0 & $\sim$28 &
    High-k / Metal Gate正式採用 / HK/MG full adoption &
    $V_\mathrm{t}$ばらつき、$T_\mathrm{inv}$制御困難 / $V_\mathrm{t}$ variation, $T_\mathrm{inv}$ control \\
    22\,nm & FinFET(初代) / 1st Gen FinFET & 0.85 & $\sim$0.9 & $\sim$20 &
    Tri-Gate構造、3Dチャネル化 / Tri-Gate, 3D channel &
    Finばらつき、設計難度増加 / Fin variation, design complexity \\
    14/10\,nm & 主流FinFET / Mainstream FinFET & 0.75--0.80 & $\sim$0.8 & $\sim$16 &
    マルチパターニング化、BEOL低誘電率化 / Multi-patterning, low-k BEOL &
    SRAM縮小限界、配線混雑 / SRAM scaling limit, routing congestion \\
    7\,nm & FinFET+EUV / FinFET + EUV & 0.65--0.70 & $\sim$0.7 & $\sim$12 &
    EUV導入開始、LELELE形成 / EUV intro, LELELE patterning &
    遮光膜設計、熱分布管理 / Mask design, thermal issues \\
    5\,nm & GAA試験導入 / GAA Pilot & 0.60--0.65 & $\sim$0.6 & $\sim$8 &
    Nanosheet構造試験導入 / Nanosheet trials &
    シート幅制御、Routing困難 / Sheet width control, poor routability \\
    3\,nm & GAA主流化 / GAA Mainstream & 0.55--0.60 & $\sim$0.5 & $\sim$5 &
    TSMC/Samsungで本格導入 / Adopted by TSMC \& Samsung &
    高密度寄生、ばらつき管理 / Parasitics, process variability \\
    $<$2\,nm & CFET構造開発中 / CFET in R\&D & $\lesssim$0.5 & $\sim$0.4 & $\sim$4 &
    NMOS・PMOS縦積層化 / Complementary FET stacking &
    熱干渉、電源・配線分離難 / Thermal interference, power-routing split \\
    \bottomrule
  \end{tabular}
  }% end resizebox
\end{table*}
