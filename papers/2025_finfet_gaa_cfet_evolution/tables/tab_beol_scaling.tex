\begin{table}[t]
  \centering
  \caption{BEOL(配線層)技術のスケーリング推移(Scaling Trend of BEOL Interconnect Technology)}
  \label{tab:beol_scaling}
  \setlength{\tabcolsep}{1.5pt}      % 列間を圧縮
  \renewcommand{\arraystretch}{1.05} % 行間を調整
  \footnotesize                      % 小さめフォント(IEEE標準)
  % 列幅調整を自動で行い1カラムに収める
  \resizebox{\columnwidth}{!}{%
  \begin{tabular}{lcccc}
    \toprule
    ノード / \textbf{Node} & 配線材 / \textbf{Metal} &
    絶縁材 / \textbf{Dielectric} & 最小ピッチ (nm) / \textbf{Pitch} &
    電源構造 / \textbf{Power Structure} \\
    \midrule
    65\,nm  & Cu & SiO$_2$ & 200 & Frontside Power \\
    28\,nm  & Cu & Low-k & 100 & Dual Damascene \\
    7\,nm   & Cu & Low-k+Co Liner & 40 & Power Grid \\
    3\,nm   & Ru/Cu & Porous Low-k & 28 & Backside Power Rail \\
    2\,nm   & Ru & Airgap Low-k & 24 & BPR + TSV Feed \\
    \bottomrule
  \end{tabular}
  } % end resizebox
\end{table}
