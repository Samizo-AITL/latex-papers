\section{スケーリングパラメータの推移}
トランジスタの微細化は長年にわたり、いわゆる「Dennardスケーリング則」に基づいて進展してきた。  
この比例則は、チャネル長、酸化膜厚、電源電圧を一定比率で同時に縮小することで、  
トランジスタ性能を維持しながら消費電力を低減できるという理論的基盤を提供してきた。  
しかし、90\,nm世代以降、リーク電流および電界強度の増大によりこの単純なスケーリング則は破綻し、  
電気的・構造的な最適化を同時に考慮する新しいパラダイムが必要となった。

この技術的変遷を俯瞰するために、表\ref{tab:process_evolution}に主要プロセスノードにおける  
**構造的および材料的進化(Process Revolution)** を示す。  
同表に見られるように、電源電圧$V_{DD}$は130\,nm世代の1.2\,Vから2\,nm世代ではおよそ0.6\,Vへと低下し、  
一方でゲート酸化膜厚$T_\text{ox}$および実効酸化膜厚(EOT)は1\,nm前後にまで到達している。  
この臨界領域では、トンネルリークが支配的となるため、High-$k$/Metal Gate(HKMG)技術が導入された。  
HKMGは実効EOTを維持しつつリーク電流$I_\text{G}$を数桁低減し、  
FinFETおよびGAA構造への橋渡し技術として機能している。

さらに、FinFETおよびGAA構造の採用により、ゲート電界による静電制御性が飛躍的に向上した。  
電源電圧の低下に伴う駆動電流$I_\text{ON}$の減少は、チャネル形状の立体化と高移動度材料の導入によって補償されている。  
このように、「スケーリングの限界」は単なる寸法縮小の限界ではなく、  
**構造変革(Structural Revolution)** としてのスケーリング再定義を意味している。

今後のCFET世代以降では、以下のような多次元設計最適化が求められる:
\begin{itemize}
    \item 高移動度チャネル材料(SiGe, Ge, III–V族)による$I_\text{ON}/I_\text{OFF}$改善  
    \item Backside Power Rail(BPR)および低抵抗BEOLによる電圧降下抑制  
    \item 多層絶縁膜スタックによる電界集中の緩和  
\end{itemize}
これらにより、スケーリングは幾何学的寸法値ではなく、  
**「電気的・熱的・構造的最適化空間」** の設計パラメータとして再定義されつつある。

\begin{table*}[t]
  \centering
  \caption{CMOSスケーリングにおけるプロセス構造の進化(Process Evolution of CMOS Scaling)}
  \label{tab:process_evolution}
  \scriptsize
  \setlength{\tabcolsep}{4pt}
  \renewcommand{\arraystretch}{1.1}
  \begin{tabular}{ccccccc}
    \toprule
    ノード & 構造 & 電源電圧[V] & $T_\mathrm{ox}$[nm] & Min L[nm] & 主な特徴 & 技術課題 \\
    Node & Structure & $V_\mathrm{DD}$[V] & $T_\mathrm{ox}$[nm] & Min L[nm] & Key Features & Challenges \\
    \midrule
    90\,nm & プレーナMOS & 1.2 & $\sim$2.0 & $\sim$65 & NiSi導入、Strained-Si、LDD最適化 & リーク電流、寄生容量、リソグラフィ限界 \\
           & Planar MOS &     &              &           & NiSi, strained-Si, optimized LDD & Leakage, parasitics, lithography \\[2pt]
    45\,nm & プレーナMOS & 1.0 & $\sim$1.3 & $\sim$35 & HKMG導入準備、ULK試験導入 & ゲート制御限界、ばらつき拡大 \\
           & Planar MOS &     &             &          & HKMG prep, ULK intro & Gate control limit, variability \\[2pt]
    22\,nm & FinFET初代 & 0.85 & $\sim$0.9 & $\sim$20 & Tri-Gate構造採用、3Dチャネル化 & Finばらつき、設計難度増加 \\
           & 1st Gen FinFET & & & & Tri-Gate, 3D channel & Fin variation, design complexity \\[2pt]
    5\,nm & GAA導入 & 0.6 & $\sim$0.6 & $\sim$8 & Nanosheet構造試験導入 & Routing困難、シート幅制御 \\
           & GAA Pilot & & & & Nanosheet trials & Sheet width control, poor routability \\[2pt]
    2\,nm & CFET試作 & $\lesssim$0.5 & $\sim$0.4 & $\sim$4 & NMOS/PMOS縦積層化 & 熱干渉、配線分離難 \\
           & CFET (R\&D) & & & & Complementary FET stacking & Thermal interference, power routing split \\
    \bottomrule
  \end{tabular}
\end{table*}

