\section{スケーリングパラメータの推移}
トランジスタの微細化は、従来「Dennardスケーリング」に基づき、  
チャネル長、酸化膜厚、電源電圧を一定の比率で縮小することで性能と消費電力の両立を図ってきた。  
しかし、90\,nm世代以降ではリーク電流と電界強度の増大により、この単純な比例則が崩壊し、  
電気的・構造的な新しいスケーリング指標が求められるようになった。

表\ref{tab:scaling}に主要プロセスノードにおけるスケーリングパラメータの推移を示す。  
電源電圧$V_{DD}$は130\,nm世代の1.2\,Vから、2\,nm世代では0.6\,V付近まで低下した。  
一方で、ゲート酸化膜厚$T_\text{ox}$は物理的に1.5\,nmを下回る領域に突入し、  
トンネル電流によるゲートリークが急増するため、High-$k$/Metal Gate(HKMG)技術が導入された。  
これにより、実効酸化膜厚(EOT: Equivalent Oxide Thickness)を維持しつつ、  
リーク電流$I_\text{G}$を数桁低減できるようになった。

さらに、FinFETおよびGAA構造の採用により、静電的なゲート制御性が強化され、  
短チャネル効果(SCE)に対する耐性が飛躍的に向上した。  
電源電圧の低下による駆動電流$I_\text{ON}$の減少は、  
チャネル形状の三次元化とキャリア移動度改善によって補償されている。  
これらの構造的最適化が、物理限界に接近するスケーリングを可能にしている。

\begin{table*}[t]
  \centering
  \caption{CMOSスケーリングにおけるプロセス構造の進化(Process Evolution of CMOS Scaling)}
  \label{tab:process_evolution}
  \scriptsize
  \setlength{\tabcolsep}{4pt}
  \renewcommand{\arraystretch}{1.1}
  \begin{tabular}{ccccccc}
    \toprule
    ノード & 構造 & 電源電圧[V] & $T_\mathrm{ox}$[nm] & Min L[nm] & 主な特徴 & 技術課題 \\
    Node & Structure & $V_\mathrm{DD}$[V] & $T_\mathrm{ox}$[nm] & Min L[nm] & Key Features & Challenges \\
    \midrule
    90\,nm & プレーナMOS & 1.2 & $\sim$2.0 & $\sim$65 & NiSi導入、Strained-Si、LDD最適化 & リーク電流、寄生容量、リソグラフィ限界 \\
           & Planar MOS &     &              &           & NiSi, strained-Si, optimized LDD & Leakage, parasitics, lithography \\[2pt]
    45\,nm & プレーナMOS & 1.0 & $\sim$1.3 & $\sim$35 & HKMG導入準備、ULK試験導入 & ゲート制御限界、ばらつき拡大 \\
           & Planar MOS &     &             &          & HKMG prep, ULK intro & Gate control limit, variability \\[2pt]
    22\,nm & FinFET初代 & 0.85 & $\sim$0.9 & $\sim$20 & Tri-Gate構造採用、3Dチャネル化 & Finばらつき、設計難度増加 \\
           & 1st Gen FinFET & & & & Tri-Gate, 3D channel & Fin variation, design complexity \\[2pt]
    5\,nm & GAA導入 & 0.6 & $\sim$0.6 & $\sim$8 & Nanosheet構造試験導入 & Routing困難、シート幅制御 \\
           & GAA Pilot & & & & Nanosheet trials & Sheet width control, poor routability \\[2pt]
    2\,nm & CFET試作 & $\lesssim$0.5 & $\sim$0.4 & $\sim$4 & NMOS/PMOS縦積層化 & 熱干渉、配線分離難 \\
           & CFET (R\&D) & & & & Complementary FET stacking & Thermal interference, power routing split \\
    \bottomrule
  \end{tabular}
\end{table*}


図示されるように、$V_{DD}$の低下と$T_\text{ox}$の薄膜化はもはや線形関係を持たず、  
熱雑音限界およびトンネルリーク限界がスケーリングの支配要因となっている。  
この結果、今後の微細化では、単なる寸法縮小ではなく、  
電気・材料・構造パラメータを同時最適化する「More-than-Moore型スケーリング」への転換が進行している。

CFET世代以降では、  
\begin{itemize}
    \item 高移動度チャネル材料(SiGe, Ge, III–V)による$I_\text{ON}/I_\text{OFF}$改善、  
    \item BPRと低抵抗BEOLによる電圧降下抑制、  
    \item 電界集中を抑制する多層ゲート絶縁膜構造、  
\end{itemize}
などの多次元設計が必要となる。  
これにより、スケーリングパラメータの概念は幾何学的な寸法値から、  
「電気的・熱的・構造的設計空間の最適化指標」へと拡張されつつある。
