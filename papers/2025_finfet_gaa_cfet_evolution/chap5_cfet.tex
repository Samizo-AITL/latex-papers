\section{CFET(Complementary FET)構造}
CFETは、n型およびp型のトランジスタを垂直方向に積層した構造であり、配線層上での面積占有を最小化できる。  
これにより、セル高さをさらに短縮し、電源・信号経路を分離した効率的な電源配線構造(BPR:Backside Power Rail)と整合する。  
GAA技術を前提とした積層形成により、スケーリング限界を超える新しい集積アーキテクチャが実現される。
