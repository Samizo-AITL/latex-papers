\section{CFET(Complementary FET)構造}
CFET(Complementary Field-Effect Transistor)は、n型およびp型のトランジスタを垂直方向に積層した三次元デバイス構造であり、  
従来の平面配置(lateral arrangement)から垂直配置(vertical stacking)への構造転換を特徴とする。  
このアプローチにより、トランジスタ・セルが占有する平面面積を大幅に削減し、  
同一チップ面積あたりの集積度をFinFETやGAA構造を超えて向上させることが可能となる。

CFETでは、nFETとpFETが垂直方向に積層され、それぞれが独立したゲートおよびソース/ドレイン構造を持つ。  
この積層構造により、同一セル内で上下デバイスが相補的に機能するため、標準セル高さを縮小しながら論理駆動能力を維持できる。  
さらに、バックサイドから電源を供給するBackside Power Rail(BPR)技術との統合が容易であり、  
信号線と電源線の物理分離によって配線抵抗・寄生結合を低減し、電源ノイズ耐性を向上させる。

また、GAA技術を前提とするCFETでは、上下のトランジスタを独立に電気的制御できるようにするため、  
チャネル層間のアイソレーション精度と熱干渉の抑制が重要となる。  
n型とp型が発熱特性および移動度特性の異なる材料で構成される場合、  
垂直方向の熱対称性(thermal symmetry)が設計上の支配要因となり、  
電流駆動能力および信頼性(NBTI/HCI耐性)に影響を及ぼす。

製造上の課題としては、垂直積層におけるソース/ドレインの選択的エピタキシャル成長、  
層間絶縁膜(Inter-Layer Dielectric; ILD)の平坦化精度、  
およびn/pトランジスタ間の電気的アイソレーション確保が挙げられる。  
特に、CFETの上部デバイス形成後に下部デバイスの特性が変化しないよう、  
低温プロセス化($<400\,^{\circ}\mathrm{C}$)が必須となる。

CFETは、GAAを超えて「論理対称性と物理空間効率の両立」を目指す究極の三次元CMOSアーキテクチャである。  
現在、IMECやIntel、SamsungなどがCFETプロトタイプを試作段階に進めており、  
1\,nmクラス以降のロジックデバイスにおける主流候補として注目されている。  
次章では、こうした構造進化を支える配線および電源インテグレーション技術(BEOLおよびBPR)について述べる。
