\documentclass[tikz,border=2pt]{standalone}
\usepackage{amsmath}
\usepackage{newtxtext,newtxmath} % Times-like to match IEEEtran
\usetikzlibrary{arrows.meta,positioning,calc,fit,shapes.multipart,decorations.pathmorphing}
\tikzset{>=Latex, line/.style={line width=0.8pt}, box/.style={draw, rounded corners=2pt, minimum width=28mm, minimum height=12mm, align=center},
note/.style={font=\footnotesize}}

\begin{document}
\begin{tikzpicture}[x=2.1cm,y=0.9cm]
  \draw[thick] (0,0) rectangle (5,4);
  \foreach \y in {1,2,3} \draw (0,\y) -- (5,\y);
  \foreach \x in {1,2,3,4} \draw (\x,0) -- (\x,4);
  \node[small] at (0.5,3.7) {Domain};
  \node[small] at (1.5,3.7) {Throughput};
  \node[small] at (2.5,3.7) {Uniformity};
  \node[small] at (3.5,3.7) {Alignment};
  \node[small] at (4.5,3.7) {Viability};
  \node[small] at (0.5,2.5) {Printing};
  \node[small] at (0.5,1.5) {Electronics};
  \node[small] at (0.5,0.5) {Bio};
  \node[small] at (0.5,3.5-2) {Semiconductor};
  \foreach \x/\y in {2/2.5,3/2.5,2/1.5,4/1.5,3/0.5,4/0.5,3/1.5} \fill (\x,\y) circle (2pt);
\end{tikzpicture}
\end{document}
