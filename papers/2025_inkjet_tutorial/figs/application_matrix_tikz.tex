\documentclass[tikz,border=2pt]{standalone}
\usepackage{amsmath}
\usepackage{newtxtext,newtxmath} % IEEEtranのTimes系に合わせる
\usetikzlibrary{arrows.meta,positioning,calc,fit,shapes.multipart,decorations.pathmorphing}
\tikzset{
  >=Latex,
  line/.style={line width=0.5pt},
}

\begin{document}
% Fig.6: Application matrix(1カラム向け/重なり無し)
\begin{tikzpicture}[font=\footnotesize, x=1cm, y=1cm]
  % ---- 全体サイズ(1カラム幅想定) ----
  \def\W{8.2}   % 横幅
  \def\H{4.8}   % 高さ
  % 縦横の罫線位置
  \def\xA{0} \def\xB{2.3} \def\xC{4.2} \def\xD{6.1} \def\xE{8.2}
  \def\yA{0} \def\yB{1.2} \def\yC{2.4} \def\yD{3.6} \def\yE{4.8}

  % ---- 外枠とグリッド ----
  \draw[line] (\xA,\yA) rectangle (\xE,\yE);
  \foreach \x in {\xB,\xC,\xD} \draw[line] (\x,\yA)--(\x,\yE);
  \foreach \y in {\yB,\yC,\yD} \draw[line] (\xA,\y)--(\xE,\y);

  % ---- ヘッダ(枠の外、上側に配置=重なり回避) ----
  \def\yHead{\yE + 0.35}
  \node[align=left, text width=2.0cm, anchor=mid] at ($(\xA,0)!0.5!(\xB,0)$ |- 0,\yHead) {Domain};
  \node[anchor=mid] at ($(\xB,0)!0.5!(\xC,0)$ |- 0,\yHead) {Throughput};
  \node[anchor=mid] at ($(\xC,0)!0.5!(\xD,0)$ |- 0,\yHead) {Uniformity};
  \node[anchor=mid] at ($(\xD,0)!0.5!(\xE,0)$ |- 0,\yHead) {Alignment};
  \node[anchor=mid] at ($(\xD,0)!0.5!(\xE,0)$!1!+(\xC-\xB,0) |- 0,\yHead) {Viability};

  % ---- 行ラベル(セル内・自動折返し)----
  \node[align=left, text width=2.0cm, anchor=center]
        at ($(\xA,\yC)!0.5!(\xB,\yD)$) {Printing};
  \node[align=left, text width=2.0cm, anchor=center]
        at ($(\xA,\yB)!0.5!(\xB,\yC)$) {Electronics};
  \node[align=left, text width=2.0cm, anchor=center]
        at ($(\xA,\yA)!0.5!(\xB,\yB)$) {Semi\-conductor};
  \node[align=left, text width=2.0cm, anchor=center]
        at ($(\xA,\yD)!0.5!(\xB,\yE)$) {Bio};

  % ---- セル中心座標(列B〜E/各行)----
  % 横方向
  \coordinate (cThru)  at ($(\xB,0)!0.5!(\xC,0)$);
  \coordinate (cUnif)  at ($(\xC,0)!0.5!(\xD,0)$);
  \coordinate (cAlign) at ($(\xD,0)!0.5!(\xE,0)$);
  \coordinate (cViab)  at ($(\xD,0)!0.5!(\xE,0)$); % Viability列は右端列と同じ中心
  % 縦方向
  \coordinate (rPrint) at ($(0,\yC)!0.5!(0,\yD)$);
  \coordinate (rElec)  at ($(0,\yB)!0.5!(0,\yC)$);
  \coordinate (rSemi)  at ($(0,\yA)!0.5!(0,\yB)$);
  \coordinate (rBio)   at ($(0,\yD)!0.5!(0,\yE)$);

  % ---- ●用マクロ ----
  \newcommand{\putdot}[2]{\fill ($(#1)!(#2)$) circle[radius=0.06cm];}

  % ---- ドット配置(例は元データと同じ意味)----
  % Printing
  \fill ($(cThru)+(rPrint)$)  circle[radius=0.06cm];
  \fill ($(cUnif)+(rPrint)$)  circle[radius=0.06cm];
  % Electronics
  \fill ($(cAlign)+(rElec)$)  circle[radius=0.06cm];
  % Semiconductor
  \fill ($(cUnif)+(rSemi)$)   circle[radius=0.06cm];
  \fill ($(cAlign)+(rSemi)$)  circle[radius=0.06cm];
  % Bio
  \fill  ($(cViab)+(rBio)$)   circle[radius=0.06cm];

\end{tikzpicture}
\end{document}
