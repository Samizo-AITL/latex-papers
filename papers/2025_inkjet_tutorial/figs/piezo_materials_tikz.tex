\documentclass[tikz,border=2pt]{standalone}
\usepackage{amsmath}
\usepackage{newtxtext,newtxmath} % Times-like to match IEEEtran
\usetikzlibrary{arrows.meta,positioning,calc,fit,shapes.multipart,decorations.pathmorphing}
\tikzset{>=Latex, line/.style={line width=0.8pt}, box/.style={draw, rounded corners=2pt, minimum width=28mm, minimum height=12mm, align=center},
note/.style={font=\footnotesize}}

\begin{document}
\begin{tikzpicture}
\draw[line] (0,0) rectangle (10,7);
\node[note, anchor=west] at (10,0) {CMOS Compatibility $\rightarrow$};
\node[note, rotate=90] at (-0.8,3.5) {Displacement Efficiency $\uparrow$};

\fill (3.5,6.0) circle (1.2pt) node[above left=2pt] {PZT};
\fill (4.6,4.8) circle (1.2pt) node[above left=2pt] {KNN};
\fill (8.0,5.0) circle (1.2pt) node[above left=2pt] {ScAlN};

\draw[line, dashed] (1,5.5) -- (9,5.5);
\draw[line, dashed] (6.5,1) -- (6.5,6);
\node[note] at (2.0,5.8) {High efficiency};
\node[note, rotate=90] at (6.8,2.0) {High CMOS};
\end{tikzpicture}
\end{document}
