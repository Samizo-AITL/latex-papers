% Common TikZ styles (robust & clash-safe)
\usetikzlibrary{arrows.meta,positioning,fit,calc,shapes,decorations.markings}

% 分割宣言で Runaway 回避
\tikzset{every picture/.style={line cap=round, line join=round}}
\tikzset{box/.style={draw, rectangle, rounded corners,
  minimum width=2.2cm, minimum height=0.9cm, align=center, very thick}}
\tikzset{arrow/.style={-{Stealth}, very thick}}
\tikzset{line/.style={draw, very thick}}
\tikzset{dashedline/.style={draw, dashed, thick}}

% "node" は予約キーなので再定義しない
\tikzset{point/.style={circle, draw, minimum size=5.5mm, inner sep=0pt, thick}}
\tikzset{dot/.style={circle, fill, inner sep=0pt, minimum size=2.2pt}}

\tikzset{group/.style={draw, rounded corners, inner sep=5pt, thick}}
\tikzset{small/.style={font=\footnotesize}}

% ===== ラベル用(今回の未定義キー対策) =====
\tikzset{
  lbl/.style={font=\footnotesize, inner sep=1.5pt, fill=white, text=black, draw=none},
  lblW/.style={lbl, anchor=west},   % 右に余白が欲しいとき
  lblE/.style={lbl, anchor=east},   % 左に余白が欲しいとき
  lblN/.style={lbl, anchor=south},  % 上側配置("above" 相当)
  lblS/.style={lbl, anchor=north},  % 下側配置("below" 相当)
}

\tikzset{note/.style={rectangle, draw, fill=yellow!18, rounded corners,
  inner sep=2.5pt, font=\scriptsize\sffamily, thick}}
