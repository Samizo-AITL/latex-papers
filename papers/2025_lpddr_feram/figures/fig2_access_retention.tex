% ==== Fig. 2: Access time vs. retention (LPDDR vs FeRAM) ====
\begin{figure}[t]
  \centering
  \begin{tikzpicture}
    \begin{loglogaxis}[
      width=\columnwidth,
      height=0.62\columnwidth,
      xlabel={Access time (ns)},
      ylabel={Retention (s)},
      xmin=8, xmax=300,
      ymin=1e-3, ymax=1e9,
      xtick={10,20,50,100,200},
      ytick={1e-3,1e-2,1e-1,1,1e1,1e3,1e5,1e7,1e9},
      legend style={at={(0.97,0.03)},anchor=south east,fill=none,draw=none},
      tick align=outside,
      grid=both,
      grid style={line width=.1pt, draw=gray!40}
    ]

    % --- LPDDR (volatile / needs refresh): red squares ---
    % Access time ~15-60 ns, effective "retention" ~32-64 ms before refresh
    \addplot+[
      only marks,
      mark=square*,
      mark size=2.2pt
    ] table[row sep=\\]{
      x   y
      15  3.2e-2  % 32 ms
      20  6.4e-2  % 64 ms
      30  3.2e-2
      60  6.4e-2
    };

    % --- FeRAM: blue circles ---
    % Access time ~80-150 ns, retention years (1e7-1e8 s)
    \addplot+[
      only marks,
      mark=*,
      mark size=2.2pt
    ] table[row sep=\\]{
      x    y
      80   1.0e7   % ~0.3 year
      100  3.0e7   % ~1 year
      120  1.0e8   % ~3.2 years
      150  3.0e8   % ~9.5 years
    };

    \legend{LPDDR (typ.), FeRAM (typ.)}
    \end{loglogaxis}
  \end{tikzpicture}
  \vspace{-1ex}
  \caption{Access time vs. retention comparison between LPDDR (red squares) and FeRAM (blue circles). LPDDR offers fast access but requires frequent refresh (tens of ms), whereas FeRAM provides orders-of-magnitude longer retention with modest access latency.}
  \label{fig:access_retention_lpddr_feram}
\end{figure}
