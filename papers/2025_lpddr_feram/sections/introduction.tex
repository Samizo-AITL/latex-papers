% =========================
% sections/introduction.tex
% =========================
\section{Introduction}

Mobile edge AI systems such as smartphones, wearables, and embedded accelerators require memory subsystems that balance
\emph{bandwidth}, \emph{energy efficiency}, and \emph{responsiveness}.
Low-power DRAM (LPDDR) has become the de-facto main memory for these platforms, offering tens to hundreds of~GB/s bandwidth
with relatively low I/O energy compared to server-class HBM~\cite{ChoiIEDM2022}.
However, LPDDR remains \emph{volatile} and relies on periodic refresh, which introduces standby power overhead
and limits energy efficiency in always-connected modes.

Non-volatile memories (NVMs) such as ReRAM, MRAM, and FeRAM have been proposed as alternatives or complements to DRAM~\cite{NohedaNature2023,WeebitIEDM2022}.
Among them, FeRAM based on HfO$_2$ offers low-voltage switching, fast rewriting, and long retention,
making it attractive as a lightweight assistive memory~\cite{KimIEDM2021}.
Yet, direct monolithic integration of LPDDR and FeRAM is infeasible due to process-temperature mismatch:
DRAM capacitors require high-$T$ anneals ($>$700$^\circ$C), whereas FeRAM crystallization must remain near 400$^\circ$C.
This motivates heterogeneous integration at the \emph{package level} rather than within a single process flow.

In this work, we propose \textbf{LPDDR+FeRAM integration via chiplet or system-in-package (SiP/PoP) assembly}.
LPDDR remains the primary working memory, while a small FeRAM die provides checkpointing and refresh offloading.
The organization is supervised by the \emph{SystemDK} co-design framework, which coordinates policies across hardware, packaging, and runtime software.
Figure~\ref{fig:concept_lpddr_feram} illustrates the high-level concept:
LPDDR supplies high-bandwidth working memory, FeRAM chiplets retain state with negligible standby power, 
and SystemDK supervision ensures seamless operation within mobile edge AI devices.
