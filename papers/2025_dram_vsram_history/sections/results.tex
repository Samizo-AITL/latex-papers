\section{Results}

The initial yield was about 65\%, with pause-refresh errors dominating (Bin-5).
SEM and process tracing suggested cumulative plasma damage due to resist ashing after WSA-ET and multiple LDD steps.
Switching to a wet-oriented resist strip minimized plasma exposure; together with strengthened body back-bias (e.g., from $-1$~V to $-3$~V), yield improved to about 80\% and passed long-term reliability.
Figure~\ref{fig:yield} shows the lot-by-lot yield recovery.

\begin{figure}[!t]
  \centering
  \pgfplotstableread[col sep=comma]{data/yield_lot.csv}\yieldtable
  \begin{tikzpicture}
    \begin{axis}[
      width=0.45\textwidth,
      height=0.32\textwidth,
      xlabel={Lot \#},
      ylabel={Yield (\%)},
      ymin=60, ymax=85,
      xtick=data,
      xmajorgrids, ymajorgrids,
      grid style=dashed,
      tick label style={/pgf/number format/fixed},
      legend style={at={(0.02,0.98)},anchor=north west,draw=none,fill=none}
    ]
      \addplot+[thick,mark=*] table[x=lot,y=yield]{\yieldtable};
      \addplot+[domain=1:11, samples=2, very thick] {80};
      \addlegendentry{Measured}
      \addlegendentry{Target 80\%}
    \end{axis}
  \end{tikzpicture}
  \caption{Lot-by-lot yield recovery after process modification.}
  \label{fig:yield}
\end{figure}
