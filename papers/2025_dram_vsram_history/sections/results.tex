\section{Results}

The initial yield was approximately 65\%, with pause-refresh errors being the dominant failure mode (Bin-5). 
Scanning electron microscopy (SEM) and process tracing indicated that cumulative plasma damage during resist ashing after WSA-ET and multiple LDD steps was the primary cause. 
By adopting a wet-oriented resist strip to minimize plasma exposure and reinforcing the body back-bias (e.g., from $-1$~V to $-3$~V), the yield improved to around 80\%, successfully meeting long-term reliability requirements. 
Figure~\ref{fig:yield} illustrates the lot-by-lot yield recovery.

\begin{figure}[t]
    \centering
    % ダミー図(最終版で差し替え)
    \fbox{\rule[0pt]{0pt}{2in} \rule[0pt]{3in}{0pt}} 
    \caption{Lot-by-lot yield recovery after process modification.}
    \label{fig:yield}
\end{figure}
