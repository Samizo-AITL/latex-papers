\section{Introduction}

In the late 1990s, Japan's semiconductor industry was in transition. At Epson's Sakata 8-inch fab, DRAM was \emph{not} pursued as an end business; rather, DRAM technology transfer was used as a \textbf{strategic vehicle} to absorb submicron process technologies at and beyond 0.35~\si{\micro\meter} and redeploy them into Epson's core devices (ASICs, logic ICs, display drivers, and inkjet driver ICs).

The technology transfer from Mitsubishi covered three nodes, each with a clear role:
(1) 0.5~\si{\micro\meter} 16~Mbit DRAM --- to establish mass-production capability and stabilize fab operation; 
(2) 0.35~\si{\micro\meter} 64~Mbit DRAM (2nd gen) --- to introduce a scaled process while tackling yield window narrowing; 
(3) 0.25~\si{\micro\meter} 64~Mbit DRAM (3rd gen) --- as the next-stage validation bed and the basis for in-house deployment.

This paper focuses on the 0.25~\si{\micro\meter} (3rd gen) ramp-up in 1998: a process overview, the ramp-up method, and a failure-analysis–driven yield-improvement cycle. We also trace how these results enabled the 0.25~\si{\micro\meter} mobile pseudo-SRAM (VSRAM) in 2001 and why trench-based 0.18~\si{\micro\meter} VSRAM was abandoned.
