\section{Discussion}
Table~\ref{tab:humanoid_comparison} positions the proposed Samizo-AITL PoC
relative to two leading humanoid platforms: Boston Dynamics Atlas and Tesla Optimus.
Atlas demonstrates world-class dynamic acrobatics, while Optimus targets scalable
industrial deployment. In contrast, the proposed system prioritizes autonomy,
fault tolerance, and sustainability.

The integration of LLMs into the hierarchical control loop is a distinctive feature.
Rather than replacing classical controllers, the LLM layer provides high-level
goal generation, anomaly interpretation, and conversational interfaces.
This complements the FSM and PID/state-space layers, ensuring stability and safety
while expanding the robot’s cognitive capabilities.
Such a design aligns with emerging trends in control systems where hybrid
architectures bridge model-based methods and data-driven intelligence.

Another differentiating factor is energy autonomy.
Through piezoelectric, photovoltaic, and regenerative harvesting,
the PoC sustains up to 20\% of its power budget without external charging.
This contrasts with Atlas and Optimus, which rely entirely on battery packs.
Combined with fast checkpoint-and-resume capability, the system supports resilient
operation in remote or resource-constrained environments.

Finally, educational reproducibility strengthens the broader impact of this work.
All specifications, models, and PoC results are openly published in bilingual
(Japanese–English) format on GitHub Pages, enabling replication and serving
as an instructional resource for control engineering education.
This open-science approach differentiates the PoC from closed industrial projects
and reinforces its role as both a research prototype and an educational benchmark.
