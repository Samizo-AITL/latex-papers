\begin{abstract}
This paper presents a proof-of-concept humanoid robot control framework that integrates
finite-state machines (FSM), proportional–integral–derivative (PID) control,
state-space methods (LQR/LQG), and large language models (LLMs) into a unified
three-layer architecture. In contrast to prior platforms (e.g., Atlas or Optimus),
the focus is on autonomy, fault tolerance, and energy sustainability.

The architecture is realized as a heterogeneous cross-node chipset:
a 22\,nm system-on-chip executes LLM inference, FSM management, and state-space control;
a 0.18\,$\mu$m AMS hub processes multimodal sensing (vision, IMU, force, audio);
and a 0.35\,$\mu$m LDMOS power drive with GaN/MOSFET stages delivers high-torque actuation.
Energy harvesting through piezoelectric, photovoltaic, and regenerative pathways
extends mission endurance in off-grid scenarios.

System-level verification using a SystemDK-style co-simulation demonstrates
posture recovery within 200\,ms after push disturbances,
a 30\% reduction in center-of-mass deviation compared with PID-only control,
and a 15\% improvement in walking energy efficiency with hybrid harvesting.
Nonvolatile checkpointing (FRAM/EEPROM) further enables resume within 10\,ms,
supporting robust mission continuity.

These results demonstrate the feasibility of combining classical control
and AI-based supervision in a sustainable, fault-tolerant humanoid
robot control system.
\end{abstract}
