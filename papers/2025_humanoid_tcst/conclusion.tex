\section{Conclusion}
This paper presented a flagship proof-of-concept humanoid control system
that integrates finite state machines (FSM), PID/state-space methods,
and large language models (LLMs) within a cross-node chipset architecture.
The system spans a 22~nm SoC for inference and control,
a 0.18~µm AMS sensor hub, and a 0.35~µm LDMOS drive with external GaN/MOSFET integration.
SystemDK-based validation confirmed posture recovery within 200~ms,
gait stability improved by 30\%, and energy efficiency gains of 15\%.
Energy harvesting contributed up to 20\% of the power budget,
while checkpoint-and-resume enabled robust mission continuity.

The main contributions are:
\begin{itemize}
  \item A hierarchical control framework combining FSM, PID/state-space, and LLM layers
        for autonomy and fault tolerance;
  \item Cross-node semiconductor co-design integrating digital, AMS, and power technologies;
  \item Experimental validation of resilience and sustainability via posture recovery,
        gait stability, and energy harvesting KPIs;
  \item Open publication of models and PoC results, supporting reproducibility and education.
\end{itemize}

Future work will extend this PoC to larger-scale prototypes with enhanced GaN-based
actuation, optimized harvesting, and field deployment in resource-constrained environments.
Beyond humanoid robotics, the proposed hybrid control paradigm points toward a
general framework that bridges classical model-based control and AI-driven reasoning.
