\section{Introduction}
Humanoid robots represent one of the most demanding applications in modern control engineering,
requiring dynamic stabilization, real-time disturbance rejection, and high-level
decision-making. Recent platforms such as Boston Dynamics Atlas and Tesla Optimus
have demonstrated remarkable mobility and manipulation. Nevertheless, existing systems
tend to emphasize either dynamic performance or industrial deployment, while
autonomy, fault tolerance, and energy sustainability remain relatively underexplored.

This paper addresses these gaps by presenting a proof-of-concept humanoid robot control
framework that integrates finite-state machines (FSM), proportional–integral–derivative (PID)
controllers, state-space methods (LQR/LQG), and large language models (LLMs)
into a unified hierarchical architecture. In this design,
(1) low-level PID and state-space control ensure stable actuation,
(2) mid-level FSM handles task sequencing and mode switching,
and (3) the LLM layer provides high-level goal reasoning and anomaly interpretation.

The proposed architecture is realized as a heterogeneous cross-node design,
consisting of a 22\,nm system-on-chip for inference and control, a 0.18\,$\mu$m
AMS sensor hub for multimodal data acquisition, and a 0.35\,$\mu$m LDMOS-based
drive stage with external GaN/MOSFET modules for high-torque actuation.
System-level validation using SystemDK co-simulation demonstrates posture
recovery within 200\,ms after disturbances, a 30\% improvement in gait stability,
and a 15\% increase in energy efficiency compared with PID-only baselines.
These results highlight the feasibility of combining classical control theory
and AI-based supervision to realize sustainable and fault-tolerant humanoid robots.
