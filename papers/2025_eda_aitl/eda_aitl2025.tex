\documentclass[conference]{IEEEtran}

% --- Robust preamble ---
\usepackage[utf8]{inputenc}
\usepackage[T1]{fontenc}
\usepackage{amsmath,amssymb}
\usepackage{graphicx}
\usepackage{cite}
\usepackage{url}
\usepackage{hyperref}

\title{EDA on AITL: A Dynamic Control Paradigm for Semiconductor Design Automation}

\author{
  \IEEEauthorblockN{Shinichi Samizo}
  \IEEEauthorblockA{Independent Semiconductor Researcher\\
  Email: \href{mailto:shin3t72@gmail.com}{shin3t72@gmail.com}}
}

\begin{document}
\maketitle

\begin{abstract}
Electronic Design Automation (EDA) has traditionally relied on static optimization flows, 
where design iterations are executed in batch mode with limited adaptability during runtime. 
This paper introduces a new paradigm, \textbf{EDA on AITL} 
(PID $\times$ FSM $\times$ LLM architecture), 
which incorporates real-time feedback, supervisory logic, and AI-guided adaptation 
directly into the design automation process. 
By embedding control loops within simulation and optimization stages, 
EDA tools transition from static execution to dynamic closed-loop operation. 
Proof-of-concept experiments on CFET delay/thermal modeling demonstrate 
faster convergence, improved robustness under parameter drifts, 
and the potential to reshape design-technology co-optimization (DTCO) 
for sub-2\,nm nodes and beyond.
\end{abstract}

\section{Introduction}
EDA has been the backbone of semiconductor scaling, 
yet its fundamental workflow remains a static pipeline:
design $\to$ simulation $\to$ analysis $\to$ redesign. 
While efficient for deterministic models, this paradigm lacks adaptability 
to emerging device architectures such as CFETs, forksheets, and 3D sequential integration, 
where strong coupling (thermal, RC, variability) 
causes unpredictable behavior under real workloads.

Inspired by control theory, 
we propose \textbf{EDA on AITL}, a cross-layer approach that embeds:
\begin{itemize}
  \item \textbf{PID}: continuous feedback adjustment of simulation parameters,
  \item \textbf{FSM}: supervisory guards to enforce safe transitions,
  \item \textbf{LLM}: high-level reasoning to retune control and optimization strategies.
\end{itemize}
This layered architecture transforms EDA from an \emph{open-loop design tool} 
to a \emph{closed-loop adaptive system}.

\section{Background and Motivation}
\subsection{Limitations of Static EDA}
Traditional flows (SPICE, STA, P\&R) rely on fixed compact models and corner analysis. 
However, nanosheet and CFET devices exhibit:
\begin{itemize}
  \item Strong temperature–delay coupling,
  \item Workload-dependent variation,
  \item Need for rapid DTCO across process corners.
\end{itemize}
Static optimization cannot adapt in-flight, leading to excessive guardbands.

\subsection{AITL Concept}
AITL (Adaptive Intelligence with Three Layers) integrates:
\begin{enumerate}
  \item PID (inner loop) for real-time stability,
  \item FSM (middle layer) for state/event management,
  \item LLM (outer layer) for adaptive learning and reconfiguration.
\end{enumerate}
This concept, initially proposed for system control, 
is here applied to semiconductor EDA as a new paradigm.

\section{Proposed Framework: EDA on AITL}
Figure~\ref{fig:framework} illustrates the architecture.

\begin{figure}[h]
\centering
\includegraphics[width=0.9\columnwidth]{figs/eda_aitl_framework.pdf}
\caption{Proposed ``EDA on AITL'' paradigm: 
simulation kernels are wrapped in PID controllers, 
monitored by FSM, and supervised by LLM.}
\label{fig:framework}
\end{figure}

\subsection{PID in Simulation Kernels}
During device/circuit simulation, PID controllers adjust:
\begin{itemize}
  \item Voltage/frequency scaling during transient analysis,
  \item Mesh step size in thermal solvers,
  \item Iterative solver tolerances.
\end{itemize}
This ensures stable convergence and minimizes overshoot.

\subsection{FSM Supervisory Layer}
FSM monitors simulation states:
\begin{itemize}
  \item HOT mode when $T>85^\circ$C,
  \item UNSTABLE mode when iteration divergence detected,
  \item NORMAL mode otherwise.
\end{itemize}
Transitions trigger safe throttling or rescheduling.

\subsection{LLM Supervisory Layer}
The LLM evaluates metrics across runs:
\begin{itemize}
  \item Retunes $(K_p,K_i,K_d)$ gains,
  \item Suggests parameter sweeps for unexplored regions,
  \item Rewrites FSM thresholds if instability persists.
\end{itemize}
This outer layer injects adaptability absent from traditional EDA.

\section{Proof-of-Concept: CFET Delay/Thermal Control}
To validate the paradigm, we implemented a compact CFET interconnect model 
with vertical coupling ($k_c$). 
The delay model follows:
\begin{equation}
T_{FO1} = (R_{wire}+R_{via})(C_{load}+C_{inter}),
\end{equation}
with temperature-dependent resistance:
\begin{equation}
R(T) = R_0\big(1+\alpha(T-25^\circ C)\big).
\end{equation}
Thermal dynamics are modeled by a first-order RC:
\begin{equation}
C_{th}\frac{dT}{dt} = P\cdot R_{th} - (T-T_{amb}).
\end{equation}

\subsection{Results}
\begin{itemize}
  \item \textbf{Static EDA baseline:} 0.9\% peak deviation, large overshoot.
  \item \textbf{EDA+PID+FSM:} deviation $<10^{-3}\%$, overshoot suppressed.
  \item \textbf{EDA on AITL (auto-tuned):} deviation $2.6\times10^{-5}\%$, 
  steady-state error $<10^{-6}\%$.
\end{itemize}
These results highlight dynamic adaptability as the key advantage.

\section{Discussion}
The AITL paradigm shifts EDA from \emph{batch static optimization} 
to \emph{real-time dynamic adaptation}. 
This enables:
\begin{itemize}
  \item Faster DTCO convergence under emerging nodes,
  \item Robustness against unmodeled coupling effects,
  \item A pathway to integrate AI-driven supervisory logic into design flows.
\end{itemize}

\section{Conclusion and Outlook}
We have proposed \textbf{EDA on AITL}, 
a dynamic control paradigm that embeds PID, FSM, and LLM layers 
within semiconductor EDA workflows. 
Initial validation on CFET models shows significant robustness gains. 
Future work includes:
\begin{itemize}
  \item Integrating with commercial SPICE and STA engines,
  \item Extending to NoC-aware thermal and delay control,
  \item Exploring LLM-in-the-loop for automated compact model refinement.
\end{itemize}

\section*{Acknowledgment}
The author thanks the Project Design Hub community for insights and discussions.

\bibliographystyle{IEEEtran}
\bibliography{refs}

\section*{Author Biography}
\noindent\textbf{Shinichi Samizo}
received the M.S. degree in Electrical and Electronic Engineering from Shinshu University, Japan.
He worked at Seiko Epson Corporation as an engineer in semiconductor memory and mixed-signal device development,
and contributed to inkjet MEMS actuators and PrecisionCore printhead technology.
He is currently an independent semiconductor researcher focusing on process/device education,
memory architecture, and AI system integration.\\[2pt]
\textbf{Contact:} \href{mailto:shin3t72@gmail.com}{shin3t72@gmail.com},
\href{https://github.com/Samizo-AITL}{Samizo-AITL}

\end{document}
