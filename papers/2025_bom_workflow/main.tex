% !TeX program = lualatex
\documentclass[10pt,conference]{IEEEtran}

% --- hyperref のオプションを先渡し ---
\PassOptionsToPackage{hidelinks}{hyperref}

% --- 日本語とフォント(LuaLaTeX) ---
\usepackage{luatexja}
\usepackage{luatexja-fontspec}

\setmainfont{TeX Gyre Termes}[Ligatures=TeX,NFSSFamily=ptm]
\setsansfont{TeX Gyre Heros}[Ligatures=TeX,NFSSFamily=phv]
\setmonofont{TeX Gyre Cursor}[NFSSFamily=pcr]
\setmainjfont{HaranoAjiMincho} % IPAexMincho でも可

% --- 数式とパッケージ ---
\usepackage{newtxmath}          % 数式 Times 系
\usepackage{amsmath}
\usepackage{graphicx}
\usepackage{xcolor}
\usepackage{cite}
\usepackage{hyperref}
\usepackage{bookmark}           % しおり管理(必要なら)

% --- Title & Author ---
\title{%
設計から量産部品発注に至る一般的実務フローとBOM運用ルールの体系化\\
\large Systematization of Workflow from Design to Mass Production Ordering and Rules for BOM Operation
}

\author{%
  \IEEEauthorblockN{三溝 真一 (Shinichi Samizo)}%
  \IEEEauthorblockA{独立系半導体研究者(元セイコーエプソン) / Independent Semiconductor Researcher (ex-Seiko Epson)\\%
  Email: \href{mailto:shin3t72@gmail.com}{shin3t72@gmail.com}\quad
  GitHub: \url{https://github.com/Samizo-AITL}}%
}

\begin{document}
\maketitle

\begin{abstract}
本研究は、製造業における「設計図面検討~量産部品発注」に至る一般的な実務プロセスを教育目的で抽象化し、部品表(BOM)を中核に据えた \textbf{部品コード体系(6桁+枝番)・属性統合・積み上げ管理} の一体フレームを提示する。環境(RoHS/REACH)、コスト、輸出管理(ECCN/HS/用途説明)をBOMで同時に扱う点に新規性がある。
\end{abstract}

\begin{IEEEkeywords}
BOM, Part Numbering, Roll-up Management, Export Control, RoHS, REACH, ECCN, HS Code, PLM, ERP
\end{IEEEkeywords}

% -------------------------------
\section{Introduction}
設計・調達・品質保証・輸出管理が分断されがちな現場において、暗黙知となりやすいコード運用と属性連携を形式知化する。本稿の目的は以下である。
\begin{enumerate}
  \item 一般的な設計~調達~発注フローの体系化
  \item BOM中心のデータ連携ルールの提示
  \item 教育・実務の両立性の検証
\end{enumerate}

% -------------------------------
\section{General Workflow}
設計図面検討 $\rightarrow$ 技術図面展開 $\rightarrow$ 部署配布 $\rightarrow$ BOM接続 $\rightarrow$ 調達・発注。  
環境・コスト・輸出属性を並列に付与・更新することで、監査対応力と教育効果を高める。

% -------------------------------
\section{BOM Generation and Structured Data}
BOMは単なる部品リストではなく、設計~調達~輸出をつなぐ共通言語である。  
親子関係・数量・参照図面ID・属性(環境/コスト/輸出)を最小単位で保持し、部品更新時に積み上げ判定を再実行する。

% -------------------------------
\section{Part Numbering System}
6桁コードを機能識別子、枝番を条件差管理とする。例:\texttt{ABCDEF-XX}。  
桁別カテゴリ:1=機械部品、2=電子部品、6=材料(SDS必須)、9=治具。  
機能が変わる場合は新しい6桁コードを発行する。

% -------------------------------
\section{Attributes and Export Control}
図面・環境・コスト・輸出・消防法を属性として統合。  
特に材料コード(先頭6)は危険物・環境・輸出の要配慮領域。  
SDS更新時は必ずBOMを再評価する。

% -------------------------------
\section{Roll-up Management}
部品属性 $\rightarrow$ サブAssy $\rightarrow$ 製品BOMへ集約。  
製品レベルで環境適合(RoHS/REACH)、コスト合計、輸出可否(ECCN/HS/用途)を同時判定する。

% -------------------------------
\section{Rules for BOM Operation}
設計変更時のルール:
\begin{itemize}
  \item 機能変化なし:枝番更新(軽微変更)
  \item 機能変化あり:新6桁コード+新図面
  \item 材料コード:SDS・消防法判定必須、更新時はロールアップ再評価
\end{itemize}

% -------------------------------
\section{Discussion}
教育的効果:新人・学生の俯瞰理解を支援。  
実務的効果:監査対応力・リスク低減。  
課題:ERP/PLM実装差、規制改訂への追随コスト。

% -------------------------------
\section{Conclusion}
BOMを統合情報基盤と再定義し、設計から輸出管理に至るプロセスを統合的に扱う教育・実務両立フレームを提示した。  
今後はAIによる属性自動付与や国際標準との接続を進める。

% -------------------------------
\section*{Acknowledgment}
本稿は公開可能な一般論に限定し、特定企業の機密情報は一切含まない。

% -------------------------------
% --- ここを削除(またはコメントアウト) ---
%\bibliographystyle{IEEEtran}
%\bibliography{refs}

% --- ここから本文の末尾あたり(結論や謝辞の直後)に追加 ---
\begin{thebibliography}{00}

\bibitem{lecun2015deep}
Y.~LeCun, Y.~Bengio, and G.~Hinton, ``Deep learning,'' \emph{Nature}, vol.~521,
no.~7553, pp.~436--444, 2015. doi:10.1038/nature14539.

\bibitem{vaswani2017attention}
A.~Vaswani \emph{et al.}, ``Attention is all you need,'' in \emph{Proc. NeurIPS},
2017, pp.~5998--6008. [Online]. Available: \url{https://arxiv.org/abs/1706.03762}

\bibitem{ieeehowto}
M.~Shell, ``How to use the {IEEEtran} \LaTeX\ class,'' 2002. [Online]. Available:
\url{http://www.michaelshell.org/tex/ieeetran/}

\bibitem{jpn-example}
A.~Yamada and K.~Suzuki, ``大規模言語モデルの産業応用,'' \emph{情報処理学会論文誌}, vol.~65,
no.~3, pp.~123--134, 2024. (日本語論文)

% 必要なだけ \bibitem を追加
\end{thebibliography}

% -------------------------------
\section*{著者略歴 / Author Biography}
\noindent\textbf{三溝 真一}(Shinichi Samizo)は、信州大学大学院 工学系研究科 電気電子工学専攻にて修士号を取得した。その後、セイコーエプソン株式会社に勤務し、半導体ロジック/メモリ/高耐圧インテグレーション、そして、インクジェット薄膜ピエゾアクチュエータ及び PrecisionCore プリントヘッドの製品化に従事した。現在は独立系半導体研究者として、プロセス/デバイス教育、メモリアーキテクチャ、AIシステム統合などに取り組んでいる。連絡先: \href{mailto:shin3t72@gmail.com}{shin3t72@gmail.com}.

\end{document}
