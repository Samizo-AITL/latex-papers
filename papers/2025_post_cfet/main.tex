% main.tex : Post-CFET paper (IEEEtran, two-column, detailed fixed version)

\documentclass[conference]{IEEEtran}

% ---------- Packages ----------
\usepackage{newtxtext,newtxmath}
\usepackage{xcolor}
\usepackage{graphicx}
\usepackage{booktabs}
\usepackage{multirow}
\usepackage{standalone}          % for TikZ input
\usepackage{tikz}
\usetikzlibrary{positioning,fit,shapes,arrows.meta,mindmap,trees,calc}
\usepackage{siunitx}
\usepackage{url}
\usepackage[hidelinks]{hyperref}
\usepackage{placeins}

% ---------- TikZ styles ----------
\tikzset{
  milestone/.style={circle,draw=red!70,fill=red!60,minimum size=4pt,inner sep=0pt},
  bubble/.style={rectangle,rounded corners=2pt,draw=black!25,fill=white,
                 font=\footnotesize,align=center,inner sep=2pt},
  arrow/.style={-Latex,thick}
}

% ---------- Macros ----------
\newcommand{\figpath}{figures}
\newcommand{\tikzcol}[2][\linewidth]{\resizebox{#1}{!}{\input{#2}}}
\newcommand{\etal}{\textit{et al.}}

% ---------- siunitx setup ----------
\sisetup{
  mode = match,
  propagate-math-font = true,
  detect-weight = true,
  detect-family = true,
  per-mode = symbol,
  range-phrase = --,
  range-units = single
}
\DeclareSIUnit{\decade}{dec}

% ---------- Title & Author ----------
\title{Post-CFET Device Architectures: Materials, Integration, and Design Perspectives}

\author{
\IEEEauthorblockN{Shinichi Samizo}
\IEEEauthorblockA{Independent Semiconductor Researcher\\
Project Design Hub, Samizo-AITL\\
\textit{Email:} \href{mailto:shin3t72@gmail.com}{shin3t72@gmail.com}\quad
\textit{GitHub:} \href{https://github.com/Samizo-AITL}{Samizo-AITL}}
}

\begin{document}
\maketitle

% ---------- Abstract ----------
\begin{abstract}
CMOS scaling has evolved from planar MOSFETs to FinFETs, Gate-All-Around (GAA) nanosheets and Complementary FETs (CFETs).
While CFET improves electrostatic control and mitigates local wiring bottlenecks, further gains are constrained by material, thermal and reliability limits of silicon.
This paper surveys post-CFET device options---two-dimensional (2D) material FETs, monolithic 3D (M3D) integration, spintronics/quantum devices, and heterogeneous atomic-scale integration.
We discuss their physical principles, process and integration challenges, demonstrated performance, reliability concerns, design/EDA implications, and educational aspects.
A comparison matrix and a 2030--2045 roadmap are provided.
\end{abstract}

\FloatBarrier % prevent floats before abstract

% ---------- 1. Introduction ----------
\section{Introduction}
Over five decades, the industry advanced by shrinking devices and re-architecting transistors: Planar $\rightarrow$ FinFET $\rightarrow$ GAA $\rightarrow$ CFET.
Dennard scaling has long broken; beyond \SI{\sim 3}{\nano\meter} nodes, \emph{interconnect delay and power density} dominate, and device self-heating and variability constrain further gains.
Wiring RC delay already exceeds intrinsic device delay for many paths; thermal design power densities approach \SI{1}{\watt\per\square\milli\meter} in high-performance logic, stressing reliability (BTI, EM) and packaging.
Consequently, progress increasingly relies on \textbf{materials innovation}, \textbf{vertical integration}, and \textbf{alternative state variables} (spin, photon, bio) rather than lateral device scaling alone.

% ---------- 2. Evolution ----------
\section{Evolution from CMOS to Post-CFET}
\begin{figure*}[!t]
  \centering
  \tikzcol[0.9\textwidth]{\figpath/evolution_tree.tex}
  \caption{Evolution tree: CMOS $\rightarrow$ CFET $\rightarrow$ post-CFET candidates.}
  \label{fig:evolution}
\end{figure*}

\subsection{Planar $\rightarrow$ FinFET}
Below \SI{45}{\nano\meter}, short-channel effects (SCE) and junction leakage degraded $SS$ and $I_{off}$.
Tri-gate FinFETs provided stronger gate coupling and reduced SCE.

\subsection{FinFET $\rightarrow$ GAA}
Nanosheets fully surrounded by the gate improved electrostatics and $V_T$ tunability, at the cost of process complexity and variability.

\subsection{GAA $\rightarrow$ CFET}
CFET stacks nFET and pFET vertically, cutting local wiring delay but raising thermal and alignment challenges.

\subsection{Beyond CFET}
Post-CFET requires new materials (2D), vertical monolithic integration, and non-charge state variables (spin, photon, bio).

% ---------- 3. Technologies ----------
\section{Post-CFET Candidate Technologies}

\subsection{2D Material FETs}
\textbf{State-of-the-art:}
\SIrange{10}{20}{\nano\meter} MoS$_2$ FETs show $I_{on}/I_{off}\sim10^7$ and $SS\sim\SIrange{60}{70}{\milli\volt\per\decade}$.
Contact resistivity remains \SIrange{0.5}{1}{\kilo\ohm\cdot\micro\meter}, with \SIrange{5}{10}{\percent} wafer non-uniformity.

\textbf{Integration:}
Low-$T$ BEOL-compatible growth/transfer, defect control, and metal/TMD contacts are central.

\textbf{Design/EDA:}
Compact models need to include Schottky contacts, confinement, and variability.

\textbf{Applications:}
Flexible electronics, IoT, bio-sensing, steep-slope concepts.

\subsection{Monolithic 3D Integration (M3D)}
\textbf{State-of-the-art:}
3D SRAM/logic prototypes cut delay by 30\% and area by 40\%; AI SoCs achieved 1.7$\times$ energy efficiency.

\textbf{Challenges:}
Thermal budget \SI{<450}{\celsius}, ILV pitch $<\SI{200}{\nano\meter}$, alignment, yield.

\textbf{Design/EDA:}
3D P\&R, ILV budgeting, thermal/stress co-simulation.

\textbf{Applications:}
AI accelerators, memory-centric computing.

\subsection{Spintronics / Quantum}
\textbf{State-of-the-art:}
STT-MRAM endurance $10^{12}$ cycles; SOT-MRAM reduces write current.

\textbf{Challenges:}
Lower $I_\text{write}$ from mA to $\mu$A, compatibility, read disturb.

\textbf{Design/EDA:}
Stochastic models, CIM-aware mapping.

\textbf{Applications:}
Radiation-hard, neuromorphic, last-level caches.

\subsection{Heterogeneous Atomic-Scale}
\textbf{State-of-the-art:}
Si+MoS$_2$ PD with \SI{200}{\milli\ampere\per\watt} at \SI{1.55}{\micro\meter}; CMOS+MEMS sensors; photonic I/O.

\textbf{Challenges:}
Bonding yield, CTE mismatch.

\textbf{Design/EDA:}
Cross-domain PDKs (optical/mechanical/electrical).

\textbf{Applications:}
Optical interconnect, aerospace, medical sensing.

% ---------- 4. Comparison ----------
\begin{table}[!t]
\centering
\caption{Comparison of post-CFET candidate technologies}
\label{tab:comparison}
\begin{tabular}{@{}lcccc@{}}
\toprule
Tech & Demo & Bottlenecks & Reliability & Design impact \\
\midrule
2D-FET & $I_{on}/I_{off}=10^7$ & $R_c$, uniformity & Traps, heat & New models \\
M3D & Delay −30\%, Area −40\% & $<450^\circ$C, ILV $R$ & Hotspots, stress & 3D P\&R \\
Spin & MRAM $10^{12}$ cyc & $I_\text{write}$, TMR var. & Retention & Stochastic models \\
Hetero & Si+2D PD 200 mA/W & Bonding, mismatch & Interface aging & Cross-domain EDA \\
\bottomrule
\end{tabular}
\end{table}

\section{Comparison and Positioning}
Table~\ref{tab:comparison} shows contrasts.  
Practical roadmap is hybrid: CFET density + M3D distance reduction + spin non-volatility + 2D/photonics I/O.

% ---------- 5. Design / Education ----------
\section{Design and Educational Perspectives}
\subsection{EDA/PDK Requirements}
\begin{itemize}
  \item Multi-physics compact models (quantum, stochastic, thermo-mechanical).
  \item 3D P\&R with ILV planning and thermal-aware signoff.
  \item Reliability kits (BTI, EM, retention, bonding voids).
\end{itemize}

\subsection{Education}
Curricula: scaling history, candidate reviews, modeling labs, 3D thermal projects, cross-domain design capstones.

% ---------- 6. Roadmap ----------
\section{Future Scenarios (2030--2045)}
2030s: Lab demos of 2D+CFET, M3D+SRAM, MRAM caches.  
2035–2040: Sequential 3D logic-memory, SOT-MRAM adoption.  
2040–2045: HPC/aerospace using non-volatile 3D stacks, radiation-hard 2D, co-packaged optics.

% ---------- 7. Conclusion ----------
\section{Conclusion}
Post-CFET shifts progress from lateral scaling to co-optimized materials, vertical integration, and new states.
Hybridization is key; success requires multi-physics EDA/PDKs and workforce trained across device–system levels.

% ---------- Figures at end ----------
\FloatBarrier
\begin{figure*}[!t]
  \centering
  \tikzcol[0.95\textwidth]{\figpath/block_diagram.tex}
  \caption{Conceptual block diagram of candidate options.}
\end{figure*}

\begin{figure*}[!t]
  \centering
  \tikzcol[0.95\textwidth]{\figpath/mindmap.tex}
  \caption{Post-CFET technology mind map.}
\end{figure*}

\begin{figure*}[!t]
  \centering
  \tikzcol[0.95\textwidth]{\figpath/roadmap.tex}
  \caption{2030--2045 roadmap (materials, integration, applications, EDA).}
\end{figure*}

% ---------- References ----------
\begin{thebibliography}{1}
\bibitem{irds2024} IRDS, \emph{International Roadmap for Devices and Systems}, 2024.
\bibitem{takagi2023} S. Takagi \etal, IEDM Tech Digest, 2023.
\bibitem{liu2022} Z. Liu \etal, \emph{Nature Electronics}, 2022.
\bibitem{fert2019} A. Fert \etal, \emph{Rev. Mod. Phys.}, 2019.
\bibitem{wong2020} H.-S. P. Wong, \emph{Nat. Rev. Mater.}, 2020.
\bibitem{batude2019} P. Batude \etal, IEDM, 2019.
\end{thebibliography}

% ---------- Biography ----------
\section*{Author Biography}
\textbf{Shinichi Samizo} received the M.S. degree in Electrical and Electronic Engineering from Shinshu University, Japan.  
He worked at Seiko Epson Corporation on semiconductor memory and mixed-signal device development, contributing to inkjet MEMS actuators and PrecisionCore printhead technology.  
He is currently an independent semiconductor researcher focusing on process/device education, memory architecture, and AI system integration.  
\par\smallskip Contact: \href{mailto:shin3t72@gmail.com}{shin3t72@gmail.com}, GitHub: \href{https://github.com/Samizo-AITL}{Samizo-AITL}.
\end{document}
