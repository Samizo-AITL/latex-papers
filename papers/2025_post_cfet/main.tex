% --------------------------------------------------
% Post-CFET Device Architecture Paper (English Version)
% --------------------------------------------------
\documentclass[conference]{IEEEtran}
\IEEEoverridecommandlockouts

% Packages
\usepackage{graphicx}
\usepackage{amsmath,amssymb}
\usepackage{hyperref}
\usepackage{booktabs}
\usepackage{multirow}
\usepackage{url}
\usepackage{standalone}   % for \includestandalone
\usepackage{tikz}         % required by standalone TikZ figures

% --------------------------------------------------
% Title & Author
% --------------------------------------------------
\title{Post-CFET Device Architectures: Materials, Integration, and Design Perspectives}

\author{
\IEEEauthorblockN{Shinichi Samizo}
\IEEEauthorblockA{Independent Semiconductor Researcher\\
Project Design Hub, Samizo-AITL\\
\textit{Email:} \href{mailto:shin3t72@gmail.com}{shin3t72@gmail.com}\\
\textit{GitHub:} \href{https://github.com/Samizo-AITL}{Samizo-AITL}}
}

\begin{document}
\maketitle

% --------------------------------------------------
% Abstract
% --------------------------------------------------
\begin{abstract}
CMOS scaling has evolved from Planar MOSFETs to FinFETs, Gate-All-Around (GAA) nanosheets, and Complementary FETs (CFETs). CFET improves electrostatic control and mitigates wiring bottlenecks, but silicon is approaching its material and thermal limits. This paper reviews \textbf{post-CFET device candidates} including \textit{two-dimensional (2D) material FETs, monolithic 3D integration, spintronics/quantum devices, and heterogeneous atomic-scale integration}. Their physical principles, fabrication challenges, experimental demonstrations, reliability concerns, application domains, and implications for design and education are compared.
\end{abstract}

% --------------------------------------------------
% Introduction
% --------------------------------------------------
\section{Introduction}
The semiconductor industry has advanced for more than five decades through device scaling and structural innovation. 
The trajectory from Planar MOSFET $\to$ FinFET $\to$ GAA $\to$ CFET represents the pursuit of enhanced electrostatic control and integration efficiency.  
However, mobility degradation, leakage, wiring delay, and thermal density have become limiting factors, demanding exploration of \textbf{post-CFET technologies}.

% --------------------------------------------------
% Evolution Path
% --------------------------------------------------
\section{Evolution from CMOS to Post-CFET}
Fig.~\ref{fig:evolution} shows the historical pathway of CMOS device evolution.

\begin{figure}[t]
  \centering
  % Compile figures/evolution_tree.tex (standalone TikZ) on the fly
  \includestandalone[width=0.46\textwidth]{figures/evolution_tree}
  \caption{Evolution tree: CMOS $\to$ CFET $\to$ Post-CFET candidates.}
  \label{fig:evolution}
\end{figure}

\subsection{Planar MOSFET to FinFET}
Short-channel effects and leakage became critical below 45 nm. Tri-gate FinFETs improved gate control and became the industry standard.

\subsection{FinFET to GAA}
Cell height constraints required nanosheets fully surrounded by gates. GAA provides stronger electrostatics but increases process variability.

\subsection{GAA to CFET}
CFET vertically stacks nFET and pFET, improving cell density and reducing interconnect delay. Challenges include heat removal and process complexity.

\subsection{Beyond CFET}
Silicon material limits and wiring dominance demand novel materials, new integration schemes, and alternative state variables (spin, photon, bio).

% --------------------------------------------------
% Candidate Technologies
% --------------------------------------------------
\section{Post-CFET Candidate Technologies}

\subsection{2D Material FETs}
\begin{itemize}
  \item \textbf{Demonstrations:} MoS$_2$ FET ($L_g$=12 nm, IEDM 2023) with Ion/Ioff=$10^7$, SS=65 mV/dec. 
  \item \textbf{Challenges:} High contact resistance ($R_c \approx 1$ k$\Omega\cdot\mu$m), film non-uniformity (5--10\%), interface traps.  
  \item \textbf{Applications:} Ultra-low power IoT, flexible electronics, bio-sensing.  
  \item \textbf{Design impact:} Immature SPICE models, poor statistical variability data.  
\end{itemize}

\subsection{Monolithic 3D Integration (M3D)}
\begin{itemize}
  \item \textbf{Demonstrations:} SRAM stacking (IEDM 2019): delay -30\%, area -40\%. AI SoC (Nat. Electronics 2022): energy efficiency +1.7$\times$.  
  \item \textbf{Challenges:} Low-temperature processing ($<450^\circ$C), inter-layer $V_{th}$ variation, thermal hotspots $>1$ W/mm$^2$.  
  \item \textbf{Applications:} AI accelerators, memory-centric computing.  
  \item \textbf{Design impact:} Requires 3D P\&R EDA, thermal/mechanical co-simulation.  
\end{itemize}

\subsection{Spintronics / Quantum Devices}
\begin{itemize}
  \item \textbf{Demonstrations:} STT-MRAM endurance $10^{12}$ cycles (IBM), SOT-MRAM write current -40\%, Topological FET on/off=$10^3$ at RT.  
  \item \textbf{Challenges:} CMOS compatibility, reducing write current from mA to $\mu$A.  
  \item \textbf{Applications:} Neuromorphic computing, radiation-hardened space systems, in-memory logic.  
  \item \textbf{Design impact:} Logic-memory fusion beyond von Neumann architecture.  
\end{itemize}

\subsection{Heterogeneous Atomic-Scale Integration}
\begin{itemize}
  \item \textbf{Demonstrations:} Si + MoS$_2$ photodetector, responsivity 200 mA/W @ 1.55 $\mu$m (Nat. Photonics 2020). CMOS+MEMS sensor chips.  
  \item \textbf{Challenges:} Interface stability, lattice/thermal mismatch, yield.  
  \item \textbf{Applications:} Optical interconnect, medical sensing, aerospace.  
  \item \textbf{Design impact:} Cross-domain EDA required (electrical + optical + mechanical).  
\end{itemize}

% --------------------------------------------------
% Block diagram
% --------------------------------------------------
\begin{figure}[t]
  \centering
  \includestandalone[width=0.47\textwidth]{figures/block_diagram}
  \caption{Conceptual block diagrams of Post-CFET candidate devices.}
  \label{fig:block}
\end{figure}

% --------------------------------------------------
% Mindmap
% --------------------------------------------------
\begin{figure}[t]
  \centering
  \includestandalone[width=0.47\textwidth]{figures/mindmap}
  \caption{Mindmap of Post-CFET technologies (demonstrations, challenges, applications).}
  \label{fig:mindmap}
\end{figure}

% --------------------------------------------------
% Comparison Table
% --------------------------------------------------
\section{Comparison Matrix}
Table~\ref{tab:matrix} compares the four candidate technologies.

\begin{table}[ht]
\centering
\caption{Comparison of Post-CFET candidate technologies}
\label{tab:matrix}
\begin{tabular}{lccccc}
\toprule
Tech. & Demonstrations & Rc/Thermal & Reliability & Applications & TRL \\
\midrule
2D-FET & Ion/Ioff=1e7 & 1k$\Omega\mu$m & Film variation & IoT/Flex & 3--5 \\
M3D & Delay-30\% & $<450^\circ$C & $V_{th}$ shift & AI/Memory & 4--6 \\
Spin & MRAM 1e12 cyc & RT stability & Write stress & Neuro/Space & 3--5 \\
Hetero & Si+MoS$_2$ PD & Interface limits & Yield & Optics/Med & 2--4 \\
\bottomrule
\end{tabular}
\end{table}

% --------------------------------------------------
% Roadmap
% --------------------------------------------------
\begin{figure}[t]
  \centering
  \includestandalone[width=0.49\textwidth]{figures/roadmap}
  \caption{Roadmap toward 2030--2045 for Post-CFET technologies (materials, integration, applications, and EDA/PDK).}
  \label{fig:roadmap}
\end{figure}

% --------------------------------------------------
% Design and Education
% --------------------------------------------------
\section{Design and Educational Perspectives}
Future EDA must integrate multi-physics: heat, stress, quantum, and cross-domain effects.  
Educational curricula should include: (1) scaling history, (2) candidate technology reviews, (3) multi-physics simulations, (4) case studies, (5) system-level design integration.

% --------------------------------------------------
% Future Scenarios
% --------------------------------------------------
\section{Future Scenarios}
\begin{itemize}
  \item \textbf{2030s (early):} Lab-scale demonstrations of 2D+CFET and M3D+2D hybrids.  
  \item \textbf{2030s (late):} Partial adoption in IoT/AI edge devices.  
  \item \textbf{2040s:} Mainstream in HPC and aerospace. Fusion of spintronics and M3D enabling non-volatile 3D logic-memory.  
\end{itemize}

% --------------------------------------------------
% Conclusion
% --------------------------------------------------
\section{Conclusion}
Post-CFET represents a paradigm shift from structural scaling to material innovation, integration schemes, alternative physical states, and heterogeneous fusion.  
It holds educational, design, and industrial significance, bridging the gap beyond the silicon era.

% --------------------------------------------------
% References
% --------------------------------------------------
\begin{thebibliography}{00}
\bibitem{irds2024} IRDS, International Roadmap for Devices and Systems, 2024.
\bibitem{takagi2023} S. Takagi et al., IEDM Tech Digest, 2023.
\bibitem{liu2022} Liu et al., Nature Electronics, 2022.
\bibitem{fert2019} A. Fert et al., Rev. Mod. Phys., 2019.
\bibitem{wong2020} H.-S. P. Wong, Nat. Rev. Mater., 2020.
\bibitem{batude2019} P. Batude et al., IEDM, 2019.
\end{thebibliography}

% --------------------------------------------------
% Author Biography
% --------------------------------------------------
\section*{Author Biography}
\noindent\textbf{Shinichi Samizo}
received the M.S. degree in Electrical and Electronic Engineering from Shinshu University, Japan.
He worked at Seiko Epson Corporation on semiconductor memory and mixed-signal device development, and contributed to inkjet MEMS actuators and PrecisionCore printhead technology.
He is currently an independent semiconductor researcher focusing on process/device education, memory architecture, and AI system integration.\\
\textbf{Contact:} \href{mailto:shin3t72@gmail.com}{shin3t72@gmail.com}, GitHub: \href{https://github.com/Samizo-AITL}{Samizo-AITL}.

\end{document}