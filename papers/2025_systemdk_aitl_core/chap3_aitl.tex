% ============================================================
% 3. AITL:三層制御構造
% ============================================================
\section{AITL:三層制御構造}
AITL (Adaptive Intelligent Tri-Layer) は,
SystemDKを動的に安定化させるための知的制御構造である。

\subsection{PID層}
物理量(温度・応力・電流など)のフィードバック制御を行う:
\begin{equation}
u(t) = K_P e(t) + K_I \int e(t)\,dt + K_D \frac{de(t)}{dt}
\end{equation}

\subsection{FSM層}
状態遷移を管理し,PIDの安定領域を維持する:
\begin{center}
\texttt{NORMAL → SATURATE → COOLDOWN → NORMAL}
\end{center}

\subsection{LLM層}
LLM層は,構造・挙動・制御モデルの形式的整合を検証する。
自然言語処理ではなく,論理整合チェック機能に限定する。
