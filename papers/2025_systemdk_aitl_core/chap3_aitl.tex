% ============================================================
% 3. AITL:三層制御構造(PIDを中核とする監督型構成)
% ============================================================
\section{AITL:三層制御構造}

AITL(Adaptive Intelligent Tri-Layer)は,
SystemDK全体を動的に安定化させるための知的制御構造である。
AITLは,内側に位置するPIDの閉ループ制御を中核とし,
その外側にFSM(状態監督層)とLLM(整合監督層)を配置する
多層監督型アーキテクチャとして構成される。

この構造により,
PIDが物理的安定性を実時間で維持し,
FSMがその動作モードを監督し,
LLMが設計情報の整合性を保証するという
三重の監督ループが形成される。

% ------------------------------------------------------------
\subsection{PID層:内側の実時間閉ループ制御}
PID層はAITLの最内層に位置し,
温度・電流・応力・振動などの物理量に対して
閉ループ制御を行う実時間制御中核である。
制御式は次式で定義される。

\begin{equation}
u(t) = K_P e(t) + K_I \int_{0}^{t} e(\tau)\,d\tau + K_D \frac{de(t)}{dt}
\end{equation}

ここで,$e(t)$ は偏差(目標値−観測値)であり,
$K_P$,$K_I$,$K_D$ は比例・積分・微分ゲインである。
PID層はSystemDKに登録された物理モデル(FEM解析など)から
環境応答パラメータを取得し,
負荷変動や外乱に対して安定動作を維持するようゲインを動的に調整する。
PIDはAITLにおける唯一の**閉ループ構造**であり,
外界との直接的なインタフェースを担う。

% ------------------------------------------------------------
\subsection{FSM層:中間のモード監督層}
FSM(Finite State Machine)層は,
PID層の外側に位置し,
その動作モードと制御状態を監督する。
FSMはPIDが安定動作範囲を逸脱しないように制御を切り替え,
安全動作や過渡回復を実現する。

代表的な遷移例を以下に示す:

\begin{center}
\texttt{NORMAL → SATURATE → COOLDOWN → NORMAL}
\end{center}

- \texttt{NORMAL}:PIDが安定動作中  
- \texttt{SATURATE}:PID出力が物理上限近傍に到達  
- \texttt{COOLDOWN}:PID動作を一時抑制し,熱・応力を緩和  

FSMはこれらの状態遷移をSystemDKスキーマで定義し,
PID制御器の許容範囲を管理する。
FSMは閉ループではなく**監督型(supervisory loop)**として機能し,
PIDからの情報をもとに動作モードを調整する。

% ------------------------------------------------------------
\subsection{LLM層:最外の整合監督層}
LLM(Logical Layer for Modeling)層は,
AITLの最外層に位置し,
設計全体の論理整合性を監督する。
ここでのLLMは,大規模言語モデルではなく,
設計情報・解析結果・制御設定の
整合性を形式的に検証するための整合監査層である。

LLMは,SystemDKスキーマを監視し,
構造設計・解析・制御設計のデータが整合しているかを自動的に検証する。
FEM解析で更新された物理データがPID設定条件に反映されているか,
FSMの状態定義が仕様条件と一致しているかをチェックし,
不整合があればスキーマ修正や再設計を指令する。

LLMはFSMやPIDの外側で動作し,
設計情報の**形式的健全性(formal integrity)**を保証する。

% ------------------------------------------------------------
\subsection{AITL全体の階層構造と動作原理}
AITLの階層的構造を以下に示す:

\begin{center}
\texttt{[LLM:整合監督層]}\\
\texttt{     ↑}\\
\texttt{[FSM:モード監督層]}\\
\texttt{     ↑}\\
\texttt{[PID:閉ループ制御層(物理系)]}
\end{center}

PID層は物理量を安定化する実時間制御ループであり,
FSM層はその動作状態をモード単位で監督し,
LLM層はSystemDKスキーマを介して設計情報の整合を監督する。
これにより,
AITLは「内側で物理を安定化し,外側で知的監督を行う」
多層閉ループシステムとして機能する。

さらにAITLはSystemDKの階層構造(BRDK/IPDK/PKGDK/SystemDK)と接続し,
各階層における制御・監督・整合機能を統合する。
これにより,ボードからシステムレベルまで
一貫した自律安定化制御を実現する。
