% ============================================================
% 3. AITL:三層制御構造(PIDを中核とする監督型構成:Final 💯採録水準)
% ============================================================
\section{AITL:三層制御構造}

AITL(Adaptive Intelligent Tri-Layer)は,
SystemDK全体を動的に安定化させるための知的制御構造である。
AITLは,内側の実時間制御から外側の知的監督までを
三層で構成し,各層が異なる時間スケールと抽象度で
システムの安定性および設計整合性を維持する。

すなわち,
PID層が物理的安定性を即応的に制御し,
FSM層がその動作モードを監督し,
LLM層が全体設計の整合性と再構成を実施するという,
\textbf{多層監督型の閉ループ制御体系}を形成する。
これにより,SystemDKは「設計そのものを制御対象とする」
自律的アーキテクチャとして機能する。

% ------------------------------------------------------------
\subsection{PID層:内側の実時間閉ループ制御}
PID層はAITLの最内層に位置し,
温度,電流,応力,位置,速度などの物理量に対して
閉ループ制御を行う中核制御層である。
制御式は次式で定義される:

\begin{equation}
u(t) = K_P e(t) + K_I \int_{0}^{t} e(\tau)\,d\tau + K_D \frac{de(t)}{dt}
\label{eq:PID}
\end{equation}

ここで,$e(t)$ は偏差(目標値と観測値の差),
$K_P$,$K_I$,$K_D$ は比例・積分・微分ゲインである。
PID層はSystemDKに格納された物理応答モデル(例:FEM解析結果)を参照し,
負荷変動や温度変化,外乱に応じてゲインを動的に再設定する。
特に,信頼性統合層で得られる劣化関数
$\Delta V_{th}(t)$ に基づき,
次式のようにゲイン補正が行われる:

\begin{equation}
K_P(t) = K_{P0} \left(1 - \alpha \cdot \Delta V_{th}(t)\right)
\label{eq:gainupdate}
\end{equation}

PID層はAITLにおける唯一の\textbf{実時間閉ループ構造}であり,
アクチュエータやロボット機構などの物理系に直接作用する。
これにより,SystemDK全体の即時安定性を担保する。

% ------------------------------------------------------------
\subsection{FSM層:中間のモード監督層}
FSM(Finite State Machine)層は,
PID層の外側に位置し,その制御モードと状態遷移を監督する。
FSMはPID出力の挙動を監視し,
異常や飽和を検出した際には安全モードや回復モードへの遷移を実行する。

代表的な状態遷移は次のように表される:

\begin{center}
\texttt{NORMAL → SATURATE → COOLDOWN → NORMAL}
\end{center}

- \texttt{NORMAL}:PIDが安定動作している通常制御状態  
- \texttt{SATURATE}:PID出力が物理上限に近づいた状態  
- \texttt{COOLDOWN}:PID制御を一時的に抑制し,発熱や応力を緩和して再安定化を待つ  

FSM層はSystemDKスキーマ内において遷移条件を形式的に定義し,
$\theta_{\mathrm{warn}}$, $\theta_{\mathrm{crit}}$ などの閾値を動的に更新する。
この層は閉ループ制御ではなく,
\textbf{PIDを監督する中間制御層(supervisory layer)}として機能する。
FSM層により,局所的な制御不安定性が上位層に伝搬する前に抑制され,
全体の動作モード一貫性が保証される。

% ------------------------------------------------------------
\subsection{LLM層:最外の知的整合・再設計層}
LLM(Large Language Model)層はAITLの最外層に位置し,
SystemDK全体の設計整合性と制御最適化を知的に監督する。
ここでのLLMは自然言語処理AIにとどまらず,
SystemDKスキーマに格納された構造・解析・制御データを統合的に解析し,
因果関係を理解して再構成を行う上位知能層である。

LLM層の主な機能は以下の通りである:

\begin{itemize}
  \item \textbf{異常要因の推論と説明:}  
  PIDおよびFSMの動作ログを解析し,
  発熱・振動・遅延などの異常発生源を特定・説明する。  
  例:\texttt{「アクチュエータM2のトルク飽和が熱暴走の要因」}。

  \item \textbf{再設計指令の生成:}  
  構造・制御・解析モデルを統合的に評価し,修正案を生成する。  
  例:\texttt{「Kpを12\%減少,FEM解析条件を再実行,FSM閾値を更新」}。

  \item \textbf{設計スキーマの再構築:}  
  修正案をSystemDKスキーマへ反映し,
  制御・解析・構造データを自律的に再同期させる。
\end{itemize}

このようにLLM層は,
PIDおよびFSMが維持する制御安定性の外側で,
システム全体の\textbf{知的再設計(intelligent redesign)}を実行する。
すなわちLLM層は,
SystemDK全体を理解し異常時に設計そのものを再構成する
「設計監督AI」として機能する。

% ------------------------------------------------------------
\subsection{AITL全体の階層構造と動作原理}
AITLの階層構造は次のように整理される:

\begin{center}
\texttt{[LLM:知的整合・再設計層]}\\
\texttt{     ↑}\\
\texttt{[FSM:モード監督層]}\\
\texttt{     ↑}\\
\texttt{[PID:閉ループ制御層(物理系)]}
\end{center}

PID層が物理安定性を実時間で維持し,
FSM層がモード単位で安全性を監督し,
LLM層がSystemDK全体の設計整合性と再構成を担う。
三層間の時定数は,
\[
\tau_{\mathrm{PID}} \ll \tau_{\mathrm{FSM}} \ll \tau_{\mathrm{LLM}}
\]
の関係を満たし,
各層が異なる時間スケールで相互補完的に動作する。

この階層協調構造により,
AITLは設計の安定性(PID),
動作の一貫性(FSM),
情報整合と再構成(LLM)を同時に保証する。
さらにAITLは,
BRDK/IPDK/PKGDK/SystemDKの各設計階層と連携し,
ボードレベルからシステムレベルまで,
統合的かつ自律的な制御・再設計を実現する。

\textbf{したがってAITLは,単なる制御機構ではなく,}
\textbf{「自ら設計を理解し,異常時に再構成する知的アーキテクチャ」}
としてSystemDKの中核を形成する。
