% ============================================================
% 3. AITL:三層制御構造(PIDを中核とする監督型構成)
% ============================================================
\section{AITL:三層制御構造}

AITL(Adaptive Intelligent Tri-Layer)は,
SystemDK全体を動的に安定化させるための知的制御構造である。
AITLは,内側の実時間制御から外側の知的監督までを
三層で構成し,それぞれが異なる時間スケールと抽象度で
システムの安定性と整合性を維持する。

すなわち,
PIDが物理的安定性を即応的に制御し,
FSMがその動作モードを監督し,
LLMが全体設計の整合性と再構成を行うという,
**三重の監督ループ構造**を形成する。

% ------------------------------------------------------------
\subsection{PID層:内側の実時間閉ループ制御}
PID層はAITLの最内層に位置し,
温度,電流,応力,位置,速度などの物理量に対して
閉ループ制御を行う。
制御式は以下で表される:

\begin{equation}
u(t) = K_P e(t) + K_I \int_{0}^{t} e(\tau)\,d\tau + K_D \frac{de(t)}{dt}
\end{equation}

ここで,$e(t)$ は偏差(目標値−観測値)であり,
$K_P$,$K_I$,$K_D$ は比例・積分・微分ゲインを示す。
PID層は,SystemDKに登録された物理応答モデル(例:FEM解析結果)を参照し,
負荷変動・温度変化・外乱に対して安定動作を維持するようゲインを動的に再設定する。

PIDはAITLの中核を成す唯一の**実時間閉ループ構造**であり,
アクチュエータやロボット機構などの物理系に直接作用する。

% ------------------------------------------------------------
\subsection{FSM層:中間のモード監督層}
FSM(Finite State Machine)層は,
PID層の外側に位置し,その制御モードを監督する。
FSMはPID出力の状態を監視し,
異常や飽和を検出した際には安全モードや回復モードへの切替を実行する。

代表的な遷移例を以下に示す:

\begin{center}
\texttt{NORMAL → SATURATE → COOLDOWN → NORMAL}
\end{center}

- \texttt{NORMAL}:通常制御状態(PIDが安定動作)  
- \texttt{SATURATE}:PID出力が上限に達し,制御限界に近い状態  
- \texttt{COOLDOWN}:PIDを一時抑制し,発熱・応力を緩和して再安定化を待つ  

FSM層はSystemDKスキーマ内で遷移条件を定義し,
制御ループの動作状態を形式的に管理する。
この層は閉ループ制御を行わず,
**PIDを監督する中間制御層(supervisory layer)**として動作する。

% ------------------------------------------------------------
\subsection{LLM層:最外の知的整合・再設計層}
LLM(Large Language Model)層はAITLの最外層に位置し,
SystemDK全体の設計整合性と制御最適化を知的に監督する。
ここでのLLMは単なる自然言語AIではなく,
SystemDKに格納された設計スキーマ(構造,解析,制御)を読み取り,
因果的関係を理解した上で再構成を行う上位知能層である。

LLM層は以下のような高度な監督を担う:

\begin{itemize}
  \item \textbf{異常要因の推論と説明:}  
  PIDやFSMからのログを解析し,発熱・振動・遅延などの異常発生源を特定する。  
  例:\texttt{「アクチュエータM2のトルク飽和が熱暴走の原因」}。

  \item \textbf{再設計指令の生成:}  
  構造・制御・FEMモデルを解析し,修正案を生成する。  
  例:\texttt{「Kpを12\%減少,FEM解析条件を再実行,FSM閾値を更新」}。

  \item \textbf{設計スキーマの再構築:}  
  修正案をSystemDKのスキーマ構造(JSON/YAML層)へ反映し,
  制御系・解析条件・構造モデルを自律的に再同期させる。
\end{itemize}

このようにLLM層は,
PIDとFSMが守る制御安定性の外側で,
システム全体の**知的再設計(intelligent redesign)**を行う。
つまり,AITLの最外層は,
「システムの状態を理解し,必要に応じて設計そのものを変更する」
という知能的機能を持つ。

% ------------------------------------------------------------
\subsection{AITL全体の階層構造と動作原理}

AITLの階層構造は以下のように整理される:

\begin{center}
\texttt{[LLM:知的整合・再設計層]}\\
\texttt{     ↑}\\
\texttt{[FSM:モード監督層]}\\
\texttt{     ↑}\\
\texttt{[PID:閉ループ制御層(物理系)]}
\end{center}

PID層が実時間制御を担い,
FSM層がその安全・安定動作を監督し,
LLM層がSystemDK全体の設計整合性を維持・再構成する。
この三層は時間スケールを異にしつつも
共通データスキーマを介して常時同期し,
SystemDK全体を**物理的・論理的・知的に安定化**させる。

さらにAITLは,
BRDK/IPDK/PKGDK/SystemDKの各階層と連携し,
ボードレベルからシステムレベルに至るまで
一貫した自律制御・再設計を可能にする。

\textbf{AITLはしたがって,単なる制御機構ではなく,}
\textbf{「自ら設計を理解し,異常時に再構成する知的アーキテクチャ」}
としてSystemDKの中核を形成する。
