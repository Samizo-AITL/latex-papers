% ============================================================
% 3. AITL:三層制御構造
% ============================================================
\section{AITL:三層制御構造}

AITL(Adaptive Intelligent Tri-Layer)は,
SystemDK全体を動的に安定化させるための知的制御構造である。
設計・解析・制御の各要素を統一的に扱うSystemDKにおいて,
AITLは「動作の安定性」「状態の一貫性」「情報の整合性」を同時に保証する
三層構造の制御モデルとして機能する。

AITLは以下の3つの層で構成される。

\begin{itemize}
  \item PID層(Physical Control Layer)
  \item FSM層(State Supervision Layer)
  \item LLM層(Logical Consistency Layer)
\end{itemize}

これら三層は階層的に連携し,
PIDがリアルタイム安定性を,
FSMが動作モードの論理的整合を,
LLMが設計データの形式的整合性を維持することで,
SystemDK全体の安定性を閉ループ的に保証する。

% ------------------------------------------------------------
\subsection{PID層:物理的安定化制御}
PID層は,SystemDKにおける最下層のリアルタイム制御層であり,
物理量(温度・応力・電流・振動など)を対象に,
安定化・追従・補償を行う。
制御式は以下の一般形で定義される。

\begin{equation}
u(t) = K_P e(t) + K_I \int_{0}^{t} e(\tau)\,d\tau + K_D \frac{de(t)}{dt}
\end{equation}

ここで,$e(t)$ は目標値と観測値の偏差であり,
$K_P$,$K_I$,$K_D$ は比例・積分・微分ゲインを示す。
PID層は,SystemDKの挙動層(FEM解析など)から得られる物理応答を
入力とし,環境変化や負荷変動に対して安定動作を維持するようゲインを自律的に調整する。
また,FSM層との連携により,
過渡応答時や限界領域における安全動作範囲を保証する。

% ------------------------------------------------------------
\subsection{FSM層:状態監督と動作モード遷移}
FSM(Finite State Machine)層は,
PID制御による物理安定化を上位から監督し,
設計・解析・制御の各モードを安全に切り替えるための状態管理層である。
各状態は,設計条件・解析結果・制御応答などの複数指標に基づいて定義される。

例として,以下のような遷移を考える:

\begin{center}
\texttt{NORMAL → SATURATE → COOLDOWN → NORMAL}
\end{center}

- \texttt{NORMAL}:設計および制御が安定動作中  
- \texttt{SATURATE}:PID制御量が物理上限に到達しつつある状態  
- \texttt{COOLDOWN}:PID制御の負荷を一時的に軽減し,安全動作に復帰するフェーズ  

FSM層はこれらの遷移ルールをSystemDKスキーマ上で形式的に定義し,
状態異常や飽和状態を検出した場合には,
PIDゲインや解析条件を再設定する命令を発行する。
この層により,制御系の局所不安定性が設計全体へ波及することを防ぐ。

% ------------------------------------------------------------
\subsection{LLM層:設計情報の論理的整合性検証}
LLM(Logical Layer for Modeling)層は,
SystemDK全体におけるデータ整合性と形式的一貫性を保証する上位層である。
ここでのLLMは,いわゆる大規模言語モデルを意味するものではなく,
論理構造の検証(Logical Consistency Checking)を目的とする。

LLM層では,構造層・挙動層・制御層間で交換されるデータが,
整合したスキーマ構造に従っているかを自動的に検証する。
例えば,FEM解析で更新された熱応力データが
制御層のPID制御パラメータと形式的に一致しているかを確認する。
不整合が検出された場合には,
その原因(欠落・型不一致・時系列差異など)を特定し,
SystemDKスキーマに修正フィードバックを行う。

このように,LLM層はSystemDK全体の
「情報的健全性(Informational Integrity)」を担保する知的監査層として機能する。
PIDとFSMが物理的・動作的安定性を保証するのに対し,
LLM層は情報的安定性を保証することで,
三層全体が協調的に閉ループ制御を形成する。

% ------------------------------------------------------------
\subsection{AITLの統合的動作原理}
AITLの三層は,SystemDKの構造・挙動・制御モデルと密に連携し,
リアルタイムで双方向に作用する。
PID層は物理的応答を安定化し,
FSM層はその動作をモード単位で監督し,
LLM層は全データ構造の一貫性を保証する。
これにより,
「物理安定性」「動作一貫性」「情報整合性」を同時に達成し,
SystemDK全体を自律的に最適化する閉ループアーキテクチャが成立する。
