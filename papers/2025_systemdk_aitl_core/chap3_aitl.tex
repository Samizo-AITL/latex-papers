% ============================================================
% 3. AITL:三層制御構造(PIDを中核とする監督型構成:Final+)
% ============================================================
\section{AITL:三層制御構造}

AITL(Adaptive Intelligent Tri-Layer)は,
SystemDK全体を動的に安定化させるための知的制御構造である。
AITLは,内側の実時間制御から外側の知的監督までを
三層で構成し,それぞれが異なる時間スケールと抽象度で
システムの安定性と設計整合性を維持する。

すなわち,
PID層が物理的安定性を即応的に制御し,
FSM層がその動作モードを監督し,
LLM層が全体設計の整合性と再構成を行うという,
\textbf{多層監督型の閉ループ制御体系}を形成する。

% ------------------------------------------------------------
\subsection{PID層:内側の実時間閉ループ制御}
PID層はAITLの最内層に位置し,
温度,電流,応力,位置,速度などの物理量に対して
閉ループ制御を行う中核制御層である。
制御式は次式で定義される:

\begin{equation}
u(t) = K_P e(t) + K_I \int_{0}^{t} e(\tau)\,d\tau + K_D \frac{de(t)}{dt}
\end{equation}

ここで,$e(t)$ は偏差(目標値と観測値の差),
$K_P$,$K_I$,$K_D$ は比例・積分・微分ゲインである。
PID層はSystemDKに格納された物理応答モデル(例:FEM解析結果)を参照し,
負荷変動・温度変化・外乱に応じてゲインを動的に再設定する。
PIDはAITLにおける唯一の**実時間閉ループ構造**であり,
アクチュエータやロボット機構といった物理系に直接作用する。

% ------------------------------------------------------------
\subsection{FSM層:中間のモード監督層}
FSM(Finite State Machine)層は,
PID層の外側に位置し,その制御モードや状態遷移を監督する。
FSMはPID出力の状態を監視し,
異常や飽和を検出した際には安全モードや回復モードへの切替を実行する。

代表的な状態遷移は次のように表される:

\begin{center}
\texttt{NORMAL → SATURATE → COOLDOWN → NORMAL}
\end{center}

- \texttt{NORMAL}:PIDが安定動作している通常制御状態  
- \texttt{SATURATE}:PID出力が物理上限に近づいた状態  
- \texttt{COOLDOWN}:PID制御を一時抑制し,発熱や応力を緩和して再安定化を待つ  

FSM層はSystemDKスキーマ内で遷移条件を定義し,
制御モードと安全性の一貫性を形式的に保証する。
この層は閉ループ制御ではなく,
\textbf{PIDを監督する中間制御層(supervisory layer)}として機能する。

% ------------------------------------------------------------
\subsection{LLM層:最外の知的整合・再設計層}
LLM(Large Language Model)層はAITLの最外層に位置し,
SystemDK全体の設計整合性と制御最適化を知的に監督する。
ここでのLLMは,単なる自然言語AIではなく,
SystemDKスキーマに格納された構造・解析・制御データを統合的に解析し,
因果関係を理解した上で再構成を行う上位知能層である。

LLM層の主な機能は以下の通りである:

\begin{itemize}
  \item \textbf{異常要因の推論と説明:}  
  PIDやFSMの動作ログを解析し,発熱・振動・遅延などの異常発生源を特定・説明する。  
  例:\texttt{「アクチュエータM2のトルク飽和が熱暴走の要因」}。

  \item \textbf{再設計指令の生成:}  
  構造・制御・解析モデルを統合的に評価し,修正案を生成する。  
  例:\texttt{「Kpを12\%減少,FEM解析条件を再実行,FSM閾値を更新」}。

  \item \textbf{設計スキーマの再構築:}  
  修正案をSystemDKのスキーマ(データ構造)へ反映し,
  制御・解析・構造データを自律的に再同期させる。
\end{itemize}

このようにLLM層は,
PIDおよびFSMが維持する制御安定性の外側で,
システム全体の**知的再設計(intelligent redesign)**を実行する。
すなわち,LLMはSystemDK全体を理解し,
異常時に設計そのものを再構成する「設計監督AI」として機能する。

% ------------------------------------------------------------
\subsection{AITL全体の階層構造と動作原理}
AITLの階層構造は次のように整理される:

\begin{center}
\texttt{[LLM:知的整合・再設計層]}\\
\texttt{     ↑}\\
\texttt{[FSM:モード監督層]}\\
\texttt{     ↑}\\
\texttt{[PID:閉ループ制御層(物理系)]}
\end{center}

PID層が物理安定性を実時間で維持し,
FSM層がモード単位で安全性を監督し,
LLM層がSystemDK全体の設計整合性と再構成を担う。
これら三層は時間スケールと抽象度を異にしながら,
共通のSystemDKスキーマを介して常時同期している。

AITLはこの三層協調構造により,
設計の安定性(PID),
動作の一貫性(FSM),
情報整合と再構成(LLM)を同時に保証する。
さらにAITLは,
BRDK/IPDK/PKGDK/SystemDKの各階層と連携し,
ボードレベルからシステムレベルまで
統合的かつ自律的な制御・再設計を実現する。

\textbf{AITLはしたがって,単なる制御機構ではなく,}
\textbf{「自ら設計を理解し,異常時に再構成する知的アーキテクチャ」}
としてSystemDKの中核を形成する。
