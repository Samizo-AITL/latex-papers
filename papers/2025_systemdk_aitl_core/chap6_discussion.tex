% ============================================================
% 6. 考察
% ============================================================
\section{考察}

SystemDK with AITL Core は,構造・挙動・制御の三要素を
統一的に扱うことにより,従来の階層分断型設計が抱えていた
情報不整合と制御不安定性の問題を根本的に解消する枠組みを提供する。
その特徴は,単なる設計支援ツールや解析環境ではなく,
設計そのものを「制御可能な動的システム」として再定義した点にある。

\subsection{(1) 階層統合による設計一貫性の確立}
SystemDKは,構造層・挙動層・制御層を共通データスキーマで接続し,
構造変更が解析および制御設計へ即時反映される動的更新機構を備える。
このため,設計段階における局所最適化や再解析の手戻りを最小化でき,
設計情報全体が整合的に保たれる。
従来,EDAやCAEツールが個別の最適化を担っていた領域を,
SystemDKは統合的な「設計知識基盤(Design Knowledge Kernel)」として再構築する。

\subsection{(2) AITLによる動的安定化機構}
AITL(Adaptive Intelligent Tri-Layer)は,
PID,FSM,LLMの三層制御構造により,
物理的安定性・状態的一貫性・情報的整合性を同時に保証する。
PID層はFEM解析などから得られる動的応答に対してリアルタイム補償を行い,
FSM層は制御状態の安全遷移を監督し,
LLM層はデータ構造全体の論理一貫性を検証する。
これにより,SystemDKは静的な解析体系ではなく,
「自律安定化を内包した設計制御系」として機能する。

\subsection{(3) 信頼性統合による長期安定性の保証}
本体系では,NBTIやHCIなどの物理的劣化モデルをPID層に組み込み,
FSM層が閾値を監視し,LLM層がデータ整合性を保証する。
この三層協調により,従来分離されていた
「信頼性解析」と「制御設計」を一体化し,
時間依存的な劣化進行を制御ループ内で補償可能とした。
すなわち,SystemDK with AITL Core は,
設計寿命・安定性・安全性の全要素を動的ループで最適化する
新しい信頼性設計体系である。

\subsection{(4) 純工学的理論体系としての意義}
SystemDK with AITL Core は,
教育・運用・経営的要素を排除した純粋工学的アーキテクチャとして位置づけられる。
その目的は,人為的意思決定や外部条件に依存しない
「設計の自律安定化理論」を確立することである。
これは,PID・FSM・LLMの三層構造を通じて,
物理モデル・制御モデル・情報モデルを統一的に扱う
形式的な工学理論体系の基盤を形成する。

\subsection{(5) 今後の展開}
今後は,本体系を半導体・メカトロ・制御システムなど
各分野の設計プロセスに適用し,
SystemDKスキーマを拡張して汎用設計基盤として発展させることが課題である。
特に,AITL層のパラメータ再同定とFSM遷移則の自動最適化を行うことで,
自律的に設計を再構成する「自己進化型設計システム」への発展が期待される。

本考察を通じて,
SystemDK with AITL Core は,設計・信頼性・制御を統合する
次世代自律設計理論の中核となることを示した。
