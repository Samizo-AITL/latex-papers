% ============================================================
% 6. 考察
% ============================================================
\section{考察}

SystemDK with AITL Core は,構造・挙動・制御の三要素を
統一的に扱うことにより,従来の階層分断型設計が抱えていた
情報不整合と制御不安定性の問題を根本的に解消する枠組みを提供する。
その特徴は,単なる設計支援ツールや解析環境ではなく,
設計そのものを「制御可能な動的システム」として再定義した点にある。

% ------------------------------------------------------------
\subsection{(1) 階層統合による設計一貫性の確立}
SystemDKは,構造層・挙動層・制御層を共通スキーマで接続し,
構造変更が解析および制御設計に即時反映される動的更新機構を備える。
これにより,構造設計(BRDK/IPDK/PKGDK)と
制御設計(AITL層)が常に整合した状態で維持される。
また,構造—解析—制御の各要素が同一データ空間で管理されるため,
設計段階での局所最適や手戻りを最小化し,
設計全体を「整合性を保つ閉ループ系」として扱うことが可能になる。
すなわちSystemDKは,EDAやCAEを越えた
「設計知識基盤(Design Knowledge Kernel)」としての機能を持つ。

% ------------------------------------------------------------
\subsection{(2) AITLによる動的安定化機構}
AITL(Adaptive Intelligent Tri-Layer)は,
PID,FSM,LLMの三層制御構造によって,
SystemDK上の設計情報と物理挙動を動的に安定化させる中核である。
PID層は,FEM解析やノイズシミュレーションから得られた
時定数・応答データを基にリアルタイム制御を行う。
FSM層は,PID動作の状態遷移を監督し,
制御モードの安定性と安全性を確保する。
LLM層は,設計構造・解析モデル・制御条件の整合を常時検証し,
不整合検出時にはパラメータ再同定を指示する。
この三層協調により,SystemDKは単なる静的データベースではなく,
「制御理論を内包した自律設計系」として機能する。

% ------------------------------------------------------------
\subsection{(3) 信頼性統合による長期安定性の保証}
本体系では,物理的劣化(NBTI, HCI, TDDB, 熱疲労)を
制御ループ内の可観測変数として扱い,
PID層が補償,FSM層が閾値監視,LLM層が長期整合検証を行う。
従来の「設計後に行う信頼性解析」を
制御設計段階に統合することにより,
時間依存的な劣化進行を自律的に補償可能とした。
この「信頼性—制御統合」は,
SystemDK内で寿命・安定性・安全性を動的最適化する
新しい設計理論体系を形成している。

% ------------------------------------------------------------
\subsection{(4) 制御主導設計体系としての意義}
SystemDK with AITL Core は,
教育・運用・経営的要素を排除した純粋工学的アーキテクチャとして構築される。
その基本思想は,設計を人為的意思決定から切り離し,
制御理論(PID/FSM/LLM)によって
設計空間そのものを安定化させることである。
PID層が物理制御を担い,
FSM層が設計モードの一貫性を保証し,
LLM層が論理整合を保つという三階層制御構造により,
設計自体が制御対象となる「自律的設計制御系(Autonomous Design Control System)」が成立する。

% ------------------------------------------------------------
\subsection{(5) 今後の展開}
今後は,本体系を半導体・メカトロ・制御・材料工学など
異分野の設計プロセスへ適用し,
SystemDKスキーマを共通基盤化することが課題となる。
特に,AITL層におけるFSM遷移則の自動最適化と
LLM層によるゲイン再同定のリアルタイム化を進めることで,
自己修復型・自己進化型の設計システムへの発展が期待される。

本考察を通じて,
SystemDK with AITL Core は,
設計・信頼性・制御を統合する純粋工学的理論体系の中核を成し,
自律設計理論の基礎を構築するものである。
