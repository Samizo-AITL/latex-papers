% ============================================================
% 6. 考察(Revised++ / 実装主導・非抽象化)
% ============================================================
\section{考察}

SystemDK with AITL Core は,
設計構造・制御設計・物理解析・信頼性評価を
単一のスキーマ上で動的に結合することで,
従来のツール分断型設計が抱えていた
情報不整合と制御不安定性の問題を根本的に解消する。
その本質は,「回路・構造・制御・信頼性を同一データ基盤で同期させる」
という実装的・制御理論的統合にある。

% ------------------------------------------------------------
\subsection{(1) 設計階層間の同期と一貫性の確立}
SystemDKは,FPGA/ASIC設計,構造設計(BRDK/IPDK/PKGDK),
および FEM・ノイズ解析を共通スキーマで連携させる。
これにより,構造修正(材料・配線・レイアウト変更)が
解析条件や制御パラメータへ即時伝搬し,
設計情報全体が常に整合した状態を維持できる。
すなわちSystemDKは,設計全体を「一貫的に閉じた制御可能系」として扱う
知識統合基盤である。

% ------------------------------------------------------------
\subsection{(2) AITLによる動的安定化機構}
AITL(Adaptive Intelligent Tri-Layer)は,
PID・FSM・LLMの三層制御構造を用いて,
SystemDK全体の動作安定性と設計整合性を維持する中核である。
PIDはFPGA/ASIC上の物理制御ループをリアルタイム補償し,
FSMはそのモード遷移を監督して安全範囲を保証する。
さらにLLM層は,制御応答・構造解析・設計仕様の整合を常時監視し,
不整合発生時には自動的に再設計指令を生成する。
この構造により,SystemDKは単なる設計データベースではなく,
「制御理論を内包する自律的設計システム」として動作する。

% ------------------------------------------------------------
\subsection{(3) 信頼性統合による長期安定性の保証}
SystemDKでは,NBTI・HCI・TDDBなどの劣化現象を
制御ループ内で扱うことにより,
時間経過に伴う物理性能低下をリアルタイムで補償する。
PID層は劣化進行を反映してゲイン補正を行い,
FSM層は閾値監視により安全遷移を指令,
LLM層は長期整合性を維持し設計パラメータを再同定する。
これにより,従来の「設計後解析」ではなく,
「運転中に自己補償を行う信頼性制御体系」が成立する。

% ------------------------------------------------------------
\subsection{(4) 制御主導設計体系への転換}
SystemDK with AITL Core は,
設計を制御の一形態として扱う「制御主導設計(Control-driven Design)」を実現する。
PIDは物理応答を安定化し,
FSMは動作モードの一貫性を管理し,
LLMは設計情報全体を再構成する。
この三層協調によって,
人手による試行錯誤や逐次調整を排除し,
設計自体を制御対象とする
自律設計制御系(Autonomous Design Control System)が確立する。

% ------------------------------------------------------------
\subsection{(5) 今後の展開}
今後は,本体系を半導体,メカトロニクス,制御装置設計,
および材料工学領域へ拡張し,
SystemDKスキーマを共通プラットフォーム化することが課題となる。
特に,AITLのFSM遷移則自動最適化と,
LLMによるPIDゲイン再同定のリアルタイム化を進めることで,
自己修復型・自己進化型の設計システムへの発展が期待される。

以上より,
SystemDK with AITL Core は,
設計・解析・制御・信頼性を一体的に扱う
実装主導の工学アーキテクチャであり,
将来の自律設計理論の中核を形成するものである。
