% ============================================================
% 4. 統合設計フロー(Final++: Robot/Actuator × Chiplet Integration / 💯)
% ============================================================
\section{統合設計フロー}

SystemDK with AITL Core の設計フローは,
ロボット/アクチュエータの制御設計を起点に,
FPGA/ASIC,チップレット構造,および実時間検証までを
単一スキーマで結合する\textbf{閉ループ型最適化プロセス}である。
PID・FSM・LLMから成るAITLを中核に,
FEMで得た剛性・慣性・熱・応力・ノイズ情報を制御へ反映し,
位置・速度・力の安定性と長期信頼性を同時に保証する。

% --- 設計フロー図(1カラム幅に自動フィット) ---
\begin{figure}[t]
  \centering
  \resizebox{\columnwidth}{!}{%
  \begin{tikzpicture}[
    font=\small,
    flowstep/.style={
      rounded corners, draw, thick, align=center,
      minimum width=38mm, minimum height=7mm
    },
    flowarr/.style={-{Latex[length=3mm]}, thick}
  ]

  % Nodes (top to bottom)
  \node[flowstep] (spec) {① 仕様策定・全体アーキテクチャ\\モジュール選定};
  \node[flowstep, below=6mm of spec] (ctrl) {② 制御系設計\\MATLAB/Simulink(PID, FSM)\\LLM整合 $\rightarrow$ C $\rightarrow$ Verilog};
  \node[flowstep, below=6mm of ctrl] (fpga) {③ FPGA回路設計\\RTL/SDC・固定小数点・I/O定義};
  \node[flowstep, below=6mm of fpga] (asic) {④ ASIC回路設計\\PDK適用/合成・P\&R・CTS};
  \node[flowstep, below=6mm of asic] (struct) {⑤ 構造設計(BRDK/IPDK/PKGDK)\\電源・信号・サーマル経路定義};
  \node[flowstep, below=6mm of struct] (fem) {⑥ FEM/ノイズ解析\\熱・応力・PI/SI/EMI};
  \node[flowstep, below=6mm of fem] (fpga_test) {⑦ FPGA検証(HIL)\\実時間応答・安定性評価};

  % Main arrows
  \draw[flowarr] (spec) -- (ctrl);
  \draw[flowarr] (ctrl) -- (fpga);
  \draw[flowarr] (fpga) -- (asic);
  \draw[flowarr] (asic) -- (struct);
  \draw[flowarr] (struct) -- (fem);
  \draw[flowarr] (fem) -- (fpga_test);

  % Feedbacks (dashed, short)
  \draw[-{Latex[length=2mm]}, dashed]
    (fem.west) .. controls +(-18mm,0mm) and +(-18mm,0mm) .. (ctrl.west)
    node[midway, left, align=right, xshift=-1mm] {\footnotesize 物理定数\\反映};

  \draw[-{Latex[length=2mm]}, dashed]
    (fpga_test.east) .. controls +(18mm,0mm) and +(18mm,0mm) .. (fpga.east)
    node[midway, right, align=left, xshift=1mm] {\footnotesize 応答補正};

  \end{tikzpicture}}
  \caption{制御主導の統合フロー(1カラム幅フィット)}
  \label{fig:design_flow}
\end{figure}

% ------------------------------------------------------------
\subsection{1) 仕様策定}
対象のロボット/アクチュエータ仕様(位置分解能,応答時間,力/トルク,剛性,
熱上限,ノイズ制約)を定義し,SystemDKスキーマ最上位ノードに登録する。
以降の制御・構造・解析と双方向リンクされ,変更は即時伝播する。

% ------------------------------------------------------------
\subsection{2) 制御系設計(MATLAB/Simulink,FSM,LLM統合)}
AITLに基づき制御モデルを設計する。
PID層は位置・速度・力の閉ループ制御,FSM層はモード(NORMAL/HOLD/RETURN/FAULT)
監督,LLM層は仕様と制御モデルの論理整合検証と修正指令を担当する。
設計後,Simulink CoderでCを生成し,ツールチェーンでVerilogへ変換して
ハードウェア実装互換を確保する。

% ------------------------------------------------------------
\subsection{3) FPGA回路設計・実時間制御検証}
生成VerilogをFPGA実装し,センサ入出力とアクチュエータを閉ループ接続する。
HIL検証で振動応答・追従誤差・過渡安定性を計測し,
PIDゲイン・FSM閾値を更新,データはSystemDKへ格納する。

% ------------------------------------------------------------
\subsection{4) 仮構造定義(Pre-BRDK/IPDK/PKGDK)}
FPGA検証結果に基づき,実装可能な仮構造(BRDK/IPDK/PKGDK)を定義する。
初期の熱経路・剛性・配線制約をモデル化し,チップレット解析の前提を整える。

% ------------------------------------------------------------
\subsection{5) FEM/ノイズ解析(Chiplet-level Physical Analysis)}
ASICダイ,インターポーザ(2.5D/3D: TSV, $\mu$-bump),PKG配線を対象に,
熱伝導・応力集中・機械共振(FEM)と,SI/PI/EMI(ノイズ)を解析する。
結果はSystemDKへ統合され,PIDの時定数やFSM閾値へ反映される。

% ------------------------------------------------------------
\subsection{6) 制御・構造・解析の再設計ループ}
制御応答と物理解析の整合が得られるまで,
PIDゲイン/FSM閾値/構造パラメータを反復最適化する。
LLM層が差分を検知し,SystemDKスキーマへ自動反映する。

% ------------------------------------------------------------
\subsection{7) 構造確定設計(BRDK/IPDK/PKGDK Fix)}
収束結果から最終のボード/インターポーザ/パッケージ設計を確定し,
再解析・再検証用としてSystemDKに登録する。

% ------------------------------------------------------------
\subsection{8) 最終FEM/FPGA再検証}
確定構造でFEMを再実行し,剛性・応力・共振をFPGA制御モデルへ再入力する。
PIDは安定性,FSMはモード一貫性を確認し,閉ループ収束を検証する。

% ------------------------------------------------------------
\subsection{9) SystemDK整合収束判定}
AITLの三層で最終整合を判定する。PIDは物理安定性,FSMは安全遷移,
LLMは設計論理の整合性を確認し,全条件成立時にASIC段階へ進む。

% ------------------------------------------------------------
\subsection{10) ASIC設計}
PDKを適用し,RTL設計,論理合成,配置配線(P\&R),CTS,タイミング検証を経て
GDSを生成する。設計データはSystemDKへ統合され,解析・制御情報と整合される。

% ------------------------------------------------------------
\subsection{11) 製造・ウエハテスト}
マスク作成~製造後,ウエハテストで電気/応答特性を評価し,
SystemDKへ登録してモデル再同定に用いる。

% ------------------------------------------------------------
\subsection{12) BR/IP/PKG製造・組立}
ASIC実装をBRDK/IPDK/PKGDKで実施し,熱・応力・ノイズ再評価を行う。
製造・組立情報もSystemDKで一元管理する。

% ------------------------------------------------------------
\subsection{13) SystemDK最終検証(AITLによる自律安定制御確認)}
統合データを用い,AITL三層で最終検証を実施する:
PIDは実時間安定化,FSMは安全遷移監督,LLMは形式整合検証と再同定を担当する。
これにより,本体系は\textbf{構造設計・制御理論・チップレット物理特性を統合最適化する自律型設計制御基盤}として確立される。
