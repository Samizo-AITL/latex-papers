% ============================================================
% 4. 統合設計フロー(Final++: Robot/Actuator × Chiplet Integration / 💯)
% ============================================================
\section{統合設計フロー}

SystemDK with AITL Core の設計フローは,
ロボット/アクチュエータの制御設計を起点に,
FPGA/ASIC,チップレット構造,および実時間検証までを
単一スキーマで結合する\textbf{閉ループ型最適化プロセス}である。
AITL(PID・FSM・LLM)を中核に,
FEMで得た剛性・慣性・熱・応力・ノイズ情報を制御系へ反映し,
位置・速度・力の安定性と長期信頼性を同時に保証する。
本章では,プロセス手順に加えて,
\textbf{収束判定,安全性保証,評価指標,および再現性}を形式的に定義する。

% --- 設計フロー図(1カラム幅に自動フィット) ---
\begin{figure}[t]
  \centering
  \resizebox{\columnwidth}{!}{%
  \begin{tikzpicture}[
    font=\small,
    flowstep/.style={
      rounded corners, draw, thick, align=center,
      minimum width=38mm, minimum height=7mm
    },
    flowarr/.style={-{Latex[length=3mm]}, thick}
  ]
  \node[flowstep] (spec) {① 仕様策定・全体アーキテクチャ\\モジュール選定};
  \node[flowstep, below=6mm of spec] (ctrl) {② 制御系設計\\MATLAB/Simulink(PID, FSM)\\LLM整合 $\rightarrow$ C $\rightarrow$ Verilog};
  \node[flowstep, below=6mm of ctrl] (fpga) {③ FPGA回路設計\\RTL/SDC・固定小数点・I/O定義};
  \node[flowstep, below=6mm of fpga] (asic) {④ ASIC回路設計\\PDK適用/合成・P\&R・CTS};
  \node[flowstep, below=6mm of asic] (struct) {⑤ 構造設計(BRDK/IPDK/PKGDK)\\電源・信号・サーマル経路定義};
  \node[flowstep, below=6mm of struct] (fem) {⑥ FEM/ノイズ解析\\熱・応力・PI/SI/EMI};
  \node[flowstep, below=6mm of fem] (fpga_test) {⑦ FPGA検証(HIL)\\実時間応答・安定性評価};
  \draw[flowarr] (spec) -- (ctrl);
  \draw[flowarr] (ctrl) -- (fpga);
  \draw[flowarr] (fpga) -- (asic);
  \draw[flowarr] (asic) -- (struct);
  \draw[flowarr] (struct) -- (fem);
  \draw[flowarr] (fem) -- (fpga_test);
  \draw[-{Latex[length=2mm]}, dashed]
    (fem.west) .. controls +(-18mm,0mm) and +(-18mm,0mm) .. (ctrl.west)
    node[midway, left, align=right, xshift=-1mm] {\footnotesize 物理定数\\反映};
  \draw[-{Latex[length=2mm]}, dashed]
    (fpga_test.east) .. controls +(18mm,0mm) and +(18mm,0mm) .. (fpga.east)
    node[midway, right, align=left, xshift=1mm] {\footnotesize 応答補正};
  \end{tikzpicture}}
  \caption{制御主導の統合フロー(1カラム幅フィット)}
  \label{fig:design_flow}
\end{figure}

% ------------------------------------------------------------
\subsection{1) 仕様策定}
対象のロボット/アクチュエータ仕様(位置分解能,応答時間,力/トルク,剛性,
熱上限,ノイズ制約)を定義し,SystemDKスキーマ最上位ノードに登録する。
以降の制御・構造・解析と双方向リンクされ,変更は即時伝播する。

% ------------------------------------------------------------
\subsection{2) 制御系設計(MATLAB/Simulink,FSM,LLM統合)}
AITLに基づき制御モデルを設計する。
PID層は位置・速度・力の閉ループ制御,FSM層はモード(NORMAL/HOLD/RETURN/FAULT)
監督,LLM層は仕様と制御モデルの論理整合検証と修正指令を担当する。
設計後,Simulink CoderでCを生成し,ツールチェーンでVerilogへ変換して
ハードウェア実装互換を確保する。

% ------------------------------------------------------------
\subsection{3) FPGA回路設計・実時間制御検証}
生成VerilogをFPGA実装し,センサ入出力とアクチュエータを閉ループ接続する。
HIL検証で振動応答・追従誤差・過渡安定性を計測し,
PIDゲイン・FSM閾値を更新,データはSystemDKへ格納する。

% ------------------------------------------------------------
\subsection{4) 仮構造定義(Pre-BRDK/IPDK/PKGDK)}
FPGA検証結果に基づき,実装可能な仮構造(BRDK/IPDK/PKGDK)を定義する。
初期の熱経路・剛性・配線制約をモデル化し,チップレット解析の前提を整える。

% ------------------------------------------------------------
\subsection{5) FEM/ノイズ解析(Chiplet-level Physical Analysis)}
ASICダイ,インターポーザ(2.5D/3D: TSV, $\mu$-bump),PKG配線を対象に,
熱伝導・応力集中・機械共振(FEM)と,SI/PI/EMI(ノイズ)を解析する。
結果はSystemDKへ統合され,PIDの時定数やFSM閾値へ反映される。

% ------------------------------------------------------------
\subsection{6) 制御・構造・解析の再設計ループ}
制御応答と物理解析の整合が得られるまで,
PIDゲイン/FSM閾値/構造パラメータを反復最適化する。
LLM層が差分を検知し,SystemDKスキーマへ自動反映する。

% ------------------------------------------------------------
\subsection{7) 構造確定設計(BRDK/IPDK/PKGDK Fix)}
収束結果から最終のボード/インターポーザ/パッケージ設計を確定し,
再解析・再検証用としてSystemDKに登録する。

% ------------------------------------------------------------
\subsection{8) 最終FEM/FPGA再検証}
確定構造でFEMを再実行し,剛性・応力・共振をFPGA制御モデルへ再入力する。
PIDは安定性,FSMはモード一貫性を確認し,閉ループ収束を検証する。

% ------------------------------------------------------------
\subsection{9) SystemDK整合収束判定}
AITLの三層で最終整合を判定する。PIDは物理安定性,FSMは安全遷移,
LLMは設計論理の整合性を確認し,全条件成立時にASIC段階へ進む。

% ------------------------------------------------------------
\subsection{10) ASIC設計}
PDKを適用し,RTL設計,論理合成,配置配線(P\&R),CTS,タイミング検証を経て
GDSを生成する。設計データはSystemDKへ統合され,解析・制御情報と整合される。

% ------------------------------------------------------------
\subsection{11) 製造・ウエハテスト}
マスク作成~製造後,ウエハテストで電気/応答特性を評価し,
SystemDKへ登録してモデル再同定に用いる。

% ------------------------------------------------------------
\subsection{12) BR/IP/PKG製造・組立}
ASIC実装をBRDK/IPDK/PKGDKで実施し,熱・応力・ノイズ再評価を行う。
製造・組立情報もSystemDKで一元管理する。

% ------------------------------------------------------------
\subsection{13) SystemDK最終検証(AITLによる自律安定制御確認)}
統合データを用い,AITL三層で最終検証を実施する:
PIDは実時間安定化,FSMは安全遷移監督,LLMは形式整合検証と再同定を担当する。
これにより,本体系は\textbf{構造設計・制御理論・チップレット物理特性を統合最適化する自律型設計制御基盤}として確立される。

% ------------------------------------------------------------
\subsection*{収束判定と安全性保証(Formal Criteria)}
設計変数ベクトル $x$,制御パラメータ $\theta=\{K_P,K_I,K_D\}$,
信頼性指標 $\rho$ に対し,以下を満たすとき収束と定義する:
\begin{align}
\|\Delta x_k\|_\infty &< \varepsilon_x,\quad
\|\Delta \theta_k\|_\infty < \varepsilon_\theta,\quad
|\Delta \rho_k| < \varepsilon_\rho, \label{eq:conv}\\
\tau_{\mathrm{PID}} &\ll \tau_{\mathrm{FSM}} \ll \tau_{\mathrm{LLM}}. \label{eq:timescales}
\end{align}
FSM安全性は不変条件で与える:
\begin{equation}
\mathcal{I}:\; T \le T_{\mathrm{crit}},\;\sigma \le \sigma_{\mathrm{crit}},\;V \le V_{\mathrm{crit}}
\;\Rightarrow\; \texttt{SAFE}(t)\ \text{保持}. \label{eq:invariant}
\end{equation}

% ------------------------------------------------------------
\subsection*{評価指標と報告形式(KPIs \& Reporting)}
本フローの成果は,以下の主要指標で報告する(単位/定義を脚注で固定):
\begin{itemize}
  \item \textbf{安定化時間} $t_{\mathrm{settle}}$(2\%帯域)\footnote{ステップ応答が目標の$\pm2\%$内に収まるまでの時間。},
        \textbf{最大オーバーシュート} $M_p$。
  \item \textbf{熱上昇} $\Delta T_{\mathrm{pk}}$,\textbf{応力ピーク} $\sigma_{\mathrm{pk}}$(FEM由来)。
  \item \textbf{追従誤差RMS} $e_{\mathrm{rms}}$(HIL測定)。
  \item \textbf{信頼性余裕} $R(t)$(NBTI/HCI/TDDBモデルに基づく余寿命比)。
\end{itemize}

\begin{table}[t]
\caption{評価指標の提出フォーマット(例)}
\label{tab:kpi}
\centering
\begin{tabular}{lccc}
\toprule
指標 & 従来MBD & SystemDK & 改善率[\%] \\
\midrule
$t_{\mathrm{settle}}$ [s] &  &  &  \\
$M_p$ [\%] &  &  &  \\
$\Delta T_{\mathrm{pk}}$ [°C] &  &  &  \\
$\sigma_{\mathrm{pk}}$ [MPa] &  &  &  \\
$e_{\mathrm{rms}}$ &  &  &  \\
$R(t)$ [--] &  &  &  \\
\bottomrule
\end{tabular}
\end{table}

% ------------------------------------------------------------
\subsection*{再設計アルゴリズム(Pseudo-code for Reconfiguration)}
\begin{algorithm}[H]
\caption{AITL-Driven Design Reconfiguration}
\label{alg:reconf}
\begin{algorithmic}[1]
\STATE $S \leftarrow$ SystemDKスキーマ,ログ,FEM/測定データ
\IF{$\mathrm{DetectInconsistency}(S)=\texttt{true}$}
  \STATE $\Delta \leftarrow \mathrm{Diagnose}(S)$
  \STATE $\Pi \leftarrow \mathrm{ProposeRedesign}(\Delta)$ \COMMENT{LLM層(RAG + 物理制約)}
  \STATE $\{\theta',x'\} \leftarrow \mathrm{Apply}(\Pi)$ \COMMENT{PID/FSMへAXI4-Lite更新}
  \STATE $\mathrm{Verify}(S,\theta',x')$ \COMMENT{HIL + FEM再実行}
\ENDIF
\STATE 収束条件(\ref{eq:conv})\&(\ref{eq:timescales})\&(\ref{eq:invariant})を満たすまで反復
\end{algorithmic}
\end{algorithm}

% ------------------------------------------------------------
\subsection*{再現性(Reproducibility Checklist)}
\begin{enumerate}
  \item \textbf{スキーマ:} JSON/YAML完全版,バージョン固定(ハッシュ付与)。
  \item \textbf{モデル:} FEMメッシュ・材料定数・境界条件・負荷ケースを明示。
  \item \textbf{制御:} サンプリング周期,固定小数点幅,飽和処理,アンチワインドアップ方式。
  \item \textbf{I/F:} AXI4-Liteレジスタマップ(PID/FSM/LLMコマンド領域)を表に提示。
  \item \textbf{計測:} センサ仕様,ノイズ整形,フィルタ係数,HIL台構成図。
  \item \textbf{公開物:} スクリプト(生成~合成~HIL自動化),図表用ノートブック,KPI再計算手順。
\end{enumerate}
