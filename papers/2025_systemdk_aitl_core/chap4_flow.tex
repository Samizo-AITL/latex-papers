% ============================================================
% 4. 統合設計フロー
% ============================================================
\section{統合設計フロー}

SystemDK with AITL Core の設計フローは,
仕様定義から制御実装・整合検証までを一貫して扱う
閉ループ型の統合設計プロセスである。
従来の順序的な設計フロー(Specification → Simulation → Control)とは異なり,
SystemDKでは各工程が同一データスキーマ上で双方向に接続され,
構造・解析・制御が常時同期される。

\vspace{1em}
\noindent フロー全体を以下に示す。

\begin{verbatim}
仕様策定
   ↓
FEM解析(熱 / 応力 / ノイズ)
   ↓
PID制御設計
   ↓
FSM状態遷移定義
   ↓
LLM整合検証
   ↓
SystemDK統合(閉ループ最適化)
\end{verbatim}

各ステップの役割を以下に説明する。

% ------------------------------------------------------------
\subsection{1) 仕様策定(Specification Definition)}
設計対象の目的関数(性能・信頼性・コストなど)と,
主要パラメータ(電圧・温度範囲・応力限界など)を定義する。
SystemDKでは,これら仕様情報が最上位ノードとして登録され,
後続の解析・制御データとリンクされる。
仕様層の更新はスキーマ全体に即時伝搬するため,
設計初期段階から動的連携が確立される。

% ------------------------------------------------------------
\subsection{2) FEM解析(Thermo–Mechanical Simulation)}
構造層のデータ(形状・材料・境界条件)を入力として,
熱伝導解析・応力解析・振動/ノイズ解析を実施する。
結果は物理応答テンソルとして挙動層に格納され,
PID層への入力データ(制御対象特性)として渡される。
SystemDKではFEMモデルの出力フォーマットも
共通スキーマ(JSON/YAML形式)に従うため,
解析と制御のデータ交換が直接行える。

% ------------------------------------------------------------
\subsection{3) PID制御設計(Dynamic Stabilization)}
FEM解析で得られた応答特性に基づき,
物理的安定化を担うPID制御器を設計する。
この段階では各制御対象に対して
\[
u(t) = K_P e(t) + K_I \int e(t)\,dt + K_D \frac{de(t)}{dt}
\]
のゲイン調整を行い,
システムが過渡応答および定常状態の両方で安定するように同定する。
PID層のゲインパラメータはSystemDKスキーマの「Control Node」に格納され,
FSM層からの動的モード切替命令を受け取る。

% ------------------------------------------------------------
\subsection{4) FSM状態遷移定義(Mode Supervision)}
PID層の安定性領域を監督し,
設計動作モードを安全に切り替えるための状態遷移モデルを定義する。
FSM(Finite State Machine)は,
設計条件・負荷条件・制御出力に基づいて,
各状態間の遷移条件を形式的に記述する。

代表的な例として,
\begin{center}
\texttt{NORMAL → SATURATE → COOLDOWN → NORMAL}
\end{center}
が挙げられる。
FSMモデルもスキーマ内で定義され,
PIDパラメータと連携して制御安定性を維持する。

% ------------------------------------------------------------
\subsection{5) LLM整合検証(Formal Consistency Checking)}
LLM(Logical Layer for Modeling)によって,
構造層・挙動層・制御層間のデータ整合性を形式的に検証する。
各ノードの依存関係・データ型・時系列が正しく対応しているかを自動チェックし,
不整合が検出された場合はスキーマ修正または再解析命令を発行する。
これにより,モデル更新や再設計時にも全階層の論理一貫性が保証される。

% ------------------------------------------------------------
\subsection{6) SystemDK統合と閉ループ最適化}
上記のすべてのプロセスをSystemDKの中核スキーマで統合し,
AITLの三層制御構造と連携させる。
PID層がリアルタイム安定性を維持し,
FSM層が安全動作遷移を管理し,
LLM層がデータ整合性を保証することで,
設計・解析・制御が常時連携する閉ループ最適化が実現される。

SystemDK with AITL Core の設計フローは,
静的な工程管理ではなく,
設計過程そのものを「自律的に安定化・最適化する動的プロセス」として定義する点に特徴がある。
