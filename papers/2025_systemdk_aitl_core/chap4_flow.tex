% ============================================================
% 4. 統合設計フロー(Revised: Control-driven Closed-loop Process)
% ============================================================
\section{統合設計フロー}

SystemDK with AITL Core の設計フローは,
制御設計を起点として構造・解析・検証までを統一的に結合する
閉ループ型の統合設計プロセスである。
従来の階層的フロー(Specification → Simulation → Verification)とは異なり,
SystemDKでは,制御・回路・構造・解析が同一データスキーマ上で連携し,
すべての情報が自律的に整合・最適化される。

\vspace{1em}
\noindent 本研究における設計フローを以下に示す。

\begin{verbatim}
仕様策定
   ↓
制御系設計(MATLAB/Simulink, FSM, LLM統合)
   ↓
回路設計(FPGA)
   ↓
回路設計(ASIC)
   ↓
構造設計(BRDK/IPDK/PKGDK)
   ↓
FEM/ノイズ解析
   ↓
FPGA検証(実時間閉ループ)
   ↓
SystemDK統合・閉ループ最適化
\end{verbatim}

各ステップの役割を以下に示す。

% ------------------------------------------------------------
\subsection{1) 仕様策定(Specification Definition)}
システムの性能目標(応答時間,温度上限,ノイズ許容量など)と
信頼性要求(寿命・応力限界・電源変動)を定義する。
これらの仕様は SystemDK スキーマの最上位ノードとして登録され,
後続の制御設計・回路実装・構造設計と常時リンクされる。

% ------------------------------------------------------------
\subsection{2) 制御系設計(Control System Design: MATLAB/Simulink, FSM, LLM Integration)}
本段階では AITL(Adaptive Intelligent Tri-Layer)構造に基づき,
制御モデルを MATLAB/Simulink 上で設計する。

PID層は物理量の閉ループ制御を行い,
FSM層は動作モードの状態遷移を管理し,
LLM層は設計整合と論理一貫性を監督する。

PIDおよびFSMモデルはSimulink Stateflowで定義し,
シミュレーションで応答安定性を検証する。
LLM層により仕様・解析・制御間の形式整合がチェックされ,
SystemDKスキーマへ統合登録される。

モデルはSimulink Coderを用いてCコード化され,
SystemDKツールチェーンによりVerilogへ変換される。
これにより,制御モデルはFPGA/ASIC実装に直接利用できる形となる。

% ------------------------------------------------------------
\subsection{3) 回路設計(FPGA Implementation)}
生成されたVerilogコードをもとに,
FPGA上でPID制御・FSM遷移を実装する。
固定小数点演算のビット幅,演算周期,I/O仕様を確定し,
SDC制約下でタイミング最適化を行う。
SystemDKはこれらの設計パラメータをスキーマとして保存し,
制御応答と物理環境を一貫して追跡する。

% ------------------------------------------------------------
\subsection{4) 回路設計(ASIC Implementation)}
FPGAで検証済みのロジックをASIC化する工程である。
PDKを適用し,論理合成(Synthesis),配置配線(P&R),
およびクロックツリー合成(CTS)を実施する。
消費電力・IR-drop・遅延などの制約条件を
SystemDKスキーマで管理し,
FEM/ノイズ解析と整合させる。

% ------------------------------------------------------------
\subsection{5) 構造設計(BRDK/IPDK/PKGDK)}
ASICロジックを実装可能な物理構造に展開する。
ボード(BRDK),インターポーザ(IPDK),パッケージ(PKGDK)の
階層間接続を定義し,電源/信号/熱経路を設計する。
幾何パラメータや配線長はSystemDKスキーマに記録され,
解析段階で直接利用される。

% ------------------------------------------------------------
\subsection{6) FEM/ノイズ解析(Physical Analysis)}
構造設計データをもとに熱・応力・ノイズ解析を実施する。
FEM解析により得られた温度分布・応力場・ノイズ応答を
SystemDKに登録し,制御設計条件(PID時定数,FSM閾値)と整合化する。
これにより,物理信頼性と制御安定性が同一データ構造上で保証される。

% ------------------------------------------------------------
\subsection{7) FPGA検証(Real-time Verification)}
解析結果をもとにFPGAで実時間検証を実施する。
制御応答(オーバーシュート,整定時間)および
FSM遷移動作を確認し,SystemDKに測定データを格納する。
AITLによってPID/FSMの安定性収束が判定される。

% ------------------------------------------------------------
\subsection{8) SystemDK統合と閉ループ最適化}
全階層の設計・解析・制御データをSystemDK中核スキーマで統合し,
LLM層が形式整合を最終確認する。
PID層が実時間安定性を,FSM層がモード監督を,
LLM層が設計整合を保証することで,
SystemDKは自律的に設計過程を閉ループ最適化する。
