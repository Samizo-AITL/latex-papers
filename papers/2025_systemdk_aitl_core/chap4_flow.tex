% ============================================================
% 4. 統合設計フロー(Final: Robot/Actuator × Chiplet Integration)
% ============================================================
\section{統合設計フロー}

SystemDK with AITL Core の設計フローは,
ロボットやアクチュエータといった物理運動系の制御設計を起点とし,
FPGA/ASIC/チップレット構造/実時間検証までを統合的に結合する
閉ループ型の設計最適化プロセスである。
本体系は,PID・FSM・LLMから成るAITL三層制御構造を中核に据え,
構造解析で得られた剛性・慣性・応力情報を制御系へ反映させることで,
位置・速度・力の安定性と信頼性を同時に保証する。

\begin{verbatim}
① 仕様策定・全体アーキテクチャ定義
   ↓
② 制御系設計(MATLAB/Simulink: PID/FSM/LLM統合)
   └ 統合Cコード生成 → Verilog変換
   ↓
③ FPGA設計・実時間制御検証(HIL)
   ↓
④ 仮構造定義(Pre-BRDK/IPDK/PKGDK)
   ↓
⑤ FEM/ノイズ解析(チップレット単位)
   ↓
⑥ 制御・構造・解析の再設計ループ
   ↓
⑦ 構造確定設計(BRDK/IPDK/PKGDK Fix)
   ↓
⑧ 最終FEM/FPGA再検証(閉ループ収束確認)
   ↓
⑨ SystemDK整合収束判定(AITL三層で整合)
   ↓
⑩ ASIC設計(RTL→合成→配置配線→GDS)
   ↓
⑪ マスク作成 → IC製造 → ウエハテスト
   ↓
⑫ BR/IP/PKG製造・組立
   ↓
⑬ SystemDK最終検証(AITLによる自律運動安定制御確認)
\end{verbatim}

% ------------------------------------------------------------
\subsection{1) 仕様策定}
対象となるロボットまたはアクチュエータの仕様を定義する。
位置分解能,応答時間,出力力,剛性,熱許容限界,ノイズ制約などを明確化し,
SystemDKスキーマの最上位ノードとして登録する。
この仕様情報は,制御・構造・解析の各層と双方向リンクされ,
変更が即座に他階層へ伝播する。

% ------------------------------------------------------------
\subsection{2) 制御系設計(MATLAB/Simulink, FSM, LLM統合)}
AITL(Adaptive Intelligent Tri-Layer)構造に基づき,
制御系モデルをMATLAB/Simulink上で設計する。
PID層はアクチュエータの位置・速度・力を対象とした閉ループ制御を行い,
FSM層は動作モード(NORMAL/HOLD/RETURN/FAULTなど)の状態遷移を監督する。
LLM層は設計仕様と制御モデルの論理整合性を検証し,
不整合がある場合には自動修正を指示する。

設計完了後,Simulink CoderによりCコードを生成し,
SystemDKツールチェーンを介してVerilogへ変換する。
これにより,制御アルゴリズムはFPGA/ASIC実装に適した形で統合される。

% ------------------------------------------------------------
\subsection{3) FPGA回路設計・実時間制御検証}
生成されたVerilogコードをFPGAに実装し,
センサ入力(位置・速度・力)とアクチュエータ出力を閉ループ接続する。
Hardware-in-the-Loop(HIL)検証を行い,
振動応答,追従誤差,過渡安定性を測定する。
PIDゲイン調整やFSM遷移条件を更新し,
SystemDKへ測定データを記録して設計情報と連携させる。

% ------------------------------------------------------------
\subsection{4) 仮構造定義(Preliminary BRDK/IPDK/PKGDK)}
FPGA検証結果をもとに,
回路構成を実装可能な仮構造モデル(BRDK/IPDK/PKGDK)として定義する。
これにより,構造設計の初期剛性・熱経路・信号配線をモデル化し,
後続のチップレット解析に使用する。
SystemDKはこの構造定義を制御モデルと同期させ,
制御と構造の双方向関係を保持する。

% ------------------------------------------------------------
\subsection{5) FEM/ノイズ解析(Chiplet-level Physical Analysis)}
本段階では,チップレット構造を対象に,
熱・応力・ノイズの物理解析を実施する。
解析対象はASICチップ,チップレット間インターポーザ(2.5D/3D TSV, μ-bump),
およびPKG配線経路である。
FEM解析では,熱伝導,応力集中,機械的共振を評価し,
ノイズ解析(SI/PI/EMI)では,
信号品質,電源安定性,クロストークを解析する。

解析結果はSystemDKスキーマへ統合され,
PID制御の時定数やFSM遷移閾値に反映される。
例えば,熱伝達の遅いチップレットはゲイン補償を行い,
高ノイズ経路はFSMで安全遷移モードを強化する。
この段階で,制御モデルと物理特性の動的整合が確立する。

% ------------------------------------------------------------
\subsection{6) 制御・構造・解析の再設計ループ}
制御応答と構造解析結果が一致するまで,
PIDゲイン・FSM閾値・構造パラメータを反復最適化する。
LLM層が整合性を常時監視し,
SystemDKスキーマ内で差分を自動反映する。

% ------------------------------------------------------------
\subsection{7) 構造確定設計(BRDK/IPDK/PKGDK Fix)}
再設計ループにより収束した構造パラメータをもとに,
ボード(BRDK),インターポーザ(IPDK),パッケージ(PKGDK)の
最終設計を確定する。
確定モデルは再解析および再検証用にSystemDKへ登録される。

% ------------------------------------------------------------
\subsection{8) 最終FEM/FPGA再検証(Closed-loop Convergence Verification)}
確定構造を用いて最終FEM解析を実施し,
得られた剛性・応力・共振特性をFPGA制御モデルに再入力する。
PID層が位置応答の安定性を,
FSM層が動作モードの一貫性を確認し,
全体システムの閉ループ収束性を検証する。

% ------------------------------------------------------------
\subsection{9) SystemDK整合収束判定}
AITL三層がSystemDK全体の整合状態を最終判定する。
PID層が物理的安定性を,
FSM層が状態遷移の安全性を,
LLM層が設計論理の整合性を検証し,
全条件が満たされた段階でASIC設計フェーズへ移行する。

% ------------------------------------------------------------
\subsection{10) ASIC設計}
FPGAで検証済みのロジックをASIC化する。
PDKを適用し,RTL設計,論理合成,配置配線(P\&R),
クロックツリー合成(CTS),タイミング検証を経てGDSを生成する。
SystemDKはこれらの設計データをスキーマに統合し,
物理解析および制御情報と整合させる。

% ------------------------------------------------------------
\subsection{11) 製造・ウエハテスト}
GDSからマスクを作成し,ICを製造する。
ウエハテストにより電気特性・応答特性を評価し,
SystemDKに測定結果を登録する。
実測データは制御モデルの再同定と整合検証に使用される。

% ------------------------------------------------------------
\subsection{12) BR/IP/PKG製造・組立}
ASICをボード(BRDK),インターポーザ(IPDK),パッケージ(PKGDK)へ実装し,
熱・応力・ノイズ分布を確認する。
SystemDKは各階層の製造・組立情報を統合管理し,
最終検証段階に備える。

% ------------------------------------------------------------
\subsection{13) SystemDK最終検証(AITLによる自律運動安定制御確認)}
SystemDK上で全階層(設計・製造・実測)の統合データを用い,
AITL三層により最終検証を実施する。

\begin{itemize}
  \item PID層: アクチュエータの位置・速度・力をリアルタイムで安定化し,外乱や振動に対して追従誤差を最小化する  
  \item FSM層: 動作モード(NORMAL/HOLD/RETURN/FAULTなど)の安全遷移を監督し,過負荷・異常状態からの保護を行う  
  \item LLM層: 設計・制御・実測データ間の形式整合性を検証し,モデル再同定や最適化に反映する  
\end{itemize}

この最終検証により,
SystemDK with AITL Core は,
ロボットやアクチュエータなどの運動制御系において,
構造設計・制御理論・チップレット物理特性を統合的に最適化する
自律型設計制御基盤として確立される。
