% ============================================================
% 5. 信頼性統合
% ============================================================
\section{信頼性統合}

SystemDK with AITL Core における信頼性統合とは,
物理的劣化現象を制御ループの一部として取り込み,
設計・解析・制御の各層で整合的に扱うことで,
長期安定性と自律補償性を同時に保証する仕組みである。

従来の信頼性解析は,設計完了後に独立して行われる「後工程解析」であり,
制御設計との整合が十分に取れないことが多かった。
これに対して本体系では,
PID層・FSM層・LLM層の三層が協調して
信頼性パラメータを監視・補償・検証する構造を採る。

% ------------------------------------------------------------
\subsection{PID層における信頼性モデルの組込み}
PID層では,物理量制御に加えて,
時間依存の劣化現象を明示的にモデル化する。
代表的な信頼性劣化要因として以下を考慮する。

\begin{itemize}
  \item NBTI(Negative Bias Temperature Instability)
  \item HCI(Hot Carrier Injection)
  \item TDDB(Time Dependent Dielectric Breakdown)
  \item 熱劣化および機械的疲労
\end{itemize}

これらのモデルは,FEM解析や加速寿命試験から得られる
応力・温度・電流密度などのパラメータを入力として,
次式のような劣化関数 $\Delta P(t)$ によって表される。

\begin{equation}
\Delta P(t) = f(T, V, \sigma, t)
\end{equation}

ここで $T$ は温度,$V$ は電圧,$\sigma$ は応力を示す。
PID制御器は,$\Delta P(t)$ に基づいて動的にゲインを調整し,
劣化進行に伴う制御性能の低下を補償する。
すなわち,PID層は「時間軸上の安定性」を保証する適応制御器として機能する。

% ------------------------------------------------------------
\subsection{FSM層による閾値監視とモード遷移制御}
FSM層は,PID層から送られる信頼性指標を継続的に監視し,
異常傾向や限界挙動を検出した場合に,
設計動作モードを自律的に切り替える。

例として,閾値監視変数を $\theta(t)$ とした場合,
FSM層は以下の遷移条件を管理する:

\begin{center}
\texttt{NORMAL} $\xrightarrow{\theta > \theta_{warn}}$ \texttt{DEGRADE}  
$\xrightarrow{\theta > \theta_{crit}}$ \texttt{SAFE\_SHUTDOWN}
\end{center}

- \texttt{NORMAL}:信頼性指標が設計範囲内で安定  
- \texttt{DEGRADE}:劣化傾向を検出し,補償ゲイン再調整を実施  
- \texttt{SAFE\_SHUTDOWN}:物理破壊リスクを検出し,安全動作に遷移  

この制御により,FSM層は物理信頼性の動的監督者として機能し,
PID層の安定化動作と連携して「信頼性制御ループ」を形成する。

% ------------------------------------------------------------
\subsection{LLM層による長期整合性と設計再帰}
LLM層は,信頼性データの蓄積と整合性検証を担当する。
劣化パラメータや寿命モデルの更新が発生した場合,
LLM層はSystemDKスキーマ全体を走査し,
構造層および制御層との整合性を再検証する。
この過程により,設計中・運用中に得られた信頼性知見が
次世代設計へ自動的に再帰的反映される。

また,LLM層は信頼性履歴を「知識ノード」として保存し,
条件付き信頼性関数 $R(T, \sigma, V)$ を
再学習データとしてSystemDK内に保持する。
これにより,設計段階で未知だった劣化条件に対しても,
将来的に自律的な設計修正が可能となる。

% ------------------------------------------------------------
\subsection{統合ループとしての信頼性–制御結合}
PID層・FSM層・LLM層が連携することにより,
信頼性と制御の統合ループが形成される。
このループでは,物理劣化の進行が制御応答に即時反映され,
FSMがモード遷移を制御し,
LLMが整合性検証を通じてシステム健全性を保証する。

従来は分離されていた
「信頼性解析」と「制御安定化設計」が,
SystemDK上で一つの自律的閉ループとして統合されることにより,
動作寿命・安定性・安全性の三要素を
同一設計理論内で同時に最適化できる。
