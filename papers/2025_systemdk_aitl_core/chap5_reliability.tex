% ============================================================
% 5. 信頼性統合
% ============================================================
\section{信頼性統合}

SystemDK with AITL Core における信頼性統合とは,
物理的劣化現象を制御ループの一部として取り込み,
設計・解析・制御の各層を整合的に結合することにより,
長期安定性と自律補償性を同時に保証する仕組みである。

従来の信頼性解析は,設計完了後に独立して行われる「後工程解析」であり,
制御設計との整合が十分に取れないという課題があった。
本体系では,AITLの三層構造(PID・FSM・LLM)が協調して,
信頼性パラメータを監視・補償・再定義することで,
「信頼性制御ループ」を閉じる構造を採る。

% ------------------------------------------------------------
\subsection{PID層:物理劣化への実時間補償}
PID層は,最内層として物理量(温度・電流・応力など)の
閉ループ制御を担うと同時に,
時間依存の劣化現象を制御対象に含める。

代表的な信頼性劣化要因は以下の通りである。
\begin{itemize}
  \item NBTI(Negative Bias Temperature Instability)
  \item HCI(Hot Carrier Injection)
  \item TDDB(Time Dependent Dielectric Breakdown)
  \item 熱劣化および機械的疲労
\end{itemize}

これらはFEM解析や加速寿命試験により得られた
温度 $T$,電圧 $V$,応力 $\sigma$,時間 $t$ の関数として,
次式の劣化モデル $\Delta P(t)$ により表される。
\begin{equation}
\Delta P(t) = f(T, V, \sigma, t)
\end{equation}

PID制御器はこの $\Delta P(t)$ を実時間で参照し,
ゲイン調整を動的に行うことで,
劣化進行による制御性能の低下を自律補償する。
すなわちPID層は,
「物理量+時間変動」を同時制御する
動的安定化層として機能する。

% ------------------------------------------------------------
\subsection{FSM層:信頼性状態の監督とモード制御}
FSM層は,PID層から送られる信頼性指標を監視し,
異常傾向や限界挙動を検出した場合に,
システム動作モードを安全側に遷移させる。

信頼性監視変数を $\theta(t)$ とした場合,
FSM層の典型的な状態遷移は次のように表される。
\begin{center}
\texttt{NORMAL}
$\xrightarrow{\theta > \theta_{warn}}$
\texttt{DEGRADE}
$\xrightarrow{\theta > \theta_{crit}}$
\texttt{SAFE\_SHUTDOWN}
\end{center}

\begin{itemize}
  \item \texttt{NORMAL}:信頼性指標が設計範囲内で安定。
  \item \texttt{DEGRADE}:劣化傾向を検出し,PIDゲインの再調整を指令。
  \item \texttt{SAFE\_SHUTDOWN}:物理破壊リスクを検出し,安全動作に遷移。
\end{itemize}

この制御によりFSM層は,
PID層が扱う「物理的閉ループ制御」の上位監督層として,
信頼性を動的に監視・介入する構造を持つ。

% ------------------------------------------------------------
\subsection{LLM層:長期信頼性整合と設計再帰}
LLM層は,SystemDK全体に対する外層的監督層として機能し,
設計・構造・制御データの整合を継続的に検証する。

FEMや運転実測から新しい劣化モデルや寿命パラメータが更新された場合,
LLM層はSystemDKスキーマ全体を再評価し,
設計パラメータ・制御条件・信頼性境界の整合を再確認する。
これにより,制御ループ内で検出された劣化知見が,
次世代設計へ自動的に再帰的反映される。

LLM層はさらに,信頼性履歴を「知識ノード」として保存し,
条件付き信頼性関数 $R(T, \sigma, V)$ を
SystemDK内に再学習データとして蓄積する。
これにより,未知環境下でも自律的な設計修正が可能となる。

% ------------------------------------------------------------
\subsection{統合ループ:信頼性-制御の自律結合}
三層が協調することで,
信頼性と制御の統合ループが形成される。
PID層は実時間で劣化補償を行い,
FSM層は状態モード制御により安全性を確保し,
LLM層は設計整合性を維持・更新する。

この階層連携により,
SystemDK上で「制御安定化」と「信頼性解析」が
同一スキーマ上で動的結合され,
動作寿命・安定性・安全性を同時に最適化できる
自律的設計理論体系が成立する。
