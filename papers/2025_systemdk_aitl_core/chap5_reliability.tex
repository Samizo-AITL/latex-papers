% ============================================================
% 5. 信頼性統合(Final++ / 💯採録水準)
% ============================================================
\section{信頼性統合}

SystemDK with AITL Core における信頼性統合とは,
物理的劣化現象を制御ループ内に組み込み,
設計・解析・制御の三要素を動的に連携させることで,
長期安定性と自律補償性を同時に保証する仕組みである。

従来の信頼性解析は,設計完了後に独立して実施される「後工程解析」であり,
制御設計との整合が取れず,劣化進行に伴う制御性能の劣化を補償できなかった。
本体系では,AITLの三層構造(PID・FSM・LLM)が協調し,
信頼性パラメータを監視・補償・再定義することで,
**信頼性制御ループ(Reliability-Control Loop)** を閉じる構造を実現する。

% ------------------------------------------------------------
\subsection{PID層:物理劣化への実時間補償}
PID層は最内層として物理量(温度・電流・応力など)の
閉ループ制御を担うとともに,
時間依存の劣化現象を制御対象に含める。

代表的な信頼性劣化要因は以下の通りである:
\begin{itemize}
  \item NBTI(Negative Bias Temperature Instability)
  \item HCI(Hot Carrier Injection)
  \item TDDB(Time Dependent Dielectric Breakdown)
  \item 熱劣化および機械的疲労
\end{itemize}

これらは温度 $T$,電圧 $V$,応力 $\sigma$,時間 $t$ の関数として,
劣化進行量 $\Delta P(t)$ を次式で表す:
\begin{equation}
\Delta P(t) = f(T, V, \sigma, t)
\end{equation}

PID制御器はこの $\Delta P(t)$ を実時間で参照し,
ゲイン $K_P, K_I, K_D$ を適応的に再設定することで,
劣化進行に伴う制御性能低下を補償する。
すなわちPID層は,
\textbf{物理量と時間変動を同時に制御する動的安定化層}
として機能する。

% ------------------------------------------------------------
\subsection{FSM層:信頼性状態の監督とモード制御}
FSM層は,PID層が出力する信頼性指標を監視し,
閾値超過や異常傾向を検出した場合に,
安全動作モードへ遷移させる監督層である。

信頼性監視変数を $\theta(t)$ とすると,
典型的な状態遷移は以下で表される:
\begin{center}
\texttt{NORMAL}
$\xrightarrow{\theta > \theta_{\text{warn}}}$
\texttt{DEGRADE}
$\xrightarrow{\theta > \theta_{\text{crit}}}$
\texttt{SAFE\_SHUTDOWN}
\end{center}

\begin{itemize}
  \item \texttt{NORMAL}:信頼性指標が設計範囲内で安定。
  \item \texttt{DEGRADE}:劣化傾向を検出し,PIDゲイン再調整を指令。
  \item \texttt{SAFE\_SHUTDOWN}:物理破壊リスクを検出し,安全モードへ遷移。
\end{itemize}

FSM層はこの遷移ロジックをSystemDKスキーマ上に形式定義し,
PID層の動作を安全側へ導く。
これにより,FSM層は\textbf{信頼性監督の中間層}として
実時間監視とモード制御を統合する。

% ------------------------------------------------------------
\subsection{LLM層:長期信頼性整合と設計再帰}
LLM層は,SystemDK全体を監督する外層知能として,
設計・構造・制御データの長期整合性を維持する。
FEM解析や実運転データから新しい劣化モデルが得られた場合,
LLM層はSystemDKスキーマ全体を再評価し,
設計パラメータ・制御条件・信頼性境界を再同定する。

LLM層はまた,信頼性履歴を「知識ノード」として保存し,
条件付き信頼性関数 $R(T, \sigma, V)$ を
逐次再学習データとして蓄積する。
これにより,未知の環境条件下でも自律的な設計修正が可能となり,
長期運用時の信頼性モデルが進化的に更新される。

% ------------------------------------------------------------
\subsection{統合ループ:信頼性と制御の自律結合}
三層が協調することで,
信頼性と制御が一体化した動的ループが形成される。
PID層は実時間補償,FSM層は安全監督,
LLM層は設計整合と再構成を担い,
SystemDK上で「制御安定化」と「信頼性解析」が
同一スキーマ内で結合される。

この統合ループにより,
SystemDK with AITL Core は
時間軸に沿った劣化進行を制御ループ内で補償し,
設計寿命・安定性・安全性を同時最適化する
\textbf{自律信頼性設計理論体系}を実現する。
