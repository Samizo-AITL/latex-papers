% ============================================================
% 5. 信頼性統合(Final++ / 💯採録水準)
% ============================================================
\section{信頼性統合}

SystemDK with AITL Core における信頼性統合とは,
物理的劣化現象を制御ループ内に組み込み,
設計・解析・制御を単一スキーマで同時最適化することで,
長期安定性と自律補償性を保証する仕組みである。
従来は設計完了後の「後工程解析」であり制御設計と非同期であったが,
本体系はAITL(三層:PID・FSM・LLM)により
\textbf{Reliability-Control Loop} を閉じる。

\subsection{PID層:劣化を考慮した実時間補償}
代表的な劣化要因(NBTI, HCI, TDDB, 熱・機械疲労)を
温度 $T$,電圧 $V$,応力 $\sigma$,時間 $t$ の関数として
\emph{健康状態} $h(t)$ に集約する:
\begin{equation}
h(t) = f(T,V,\sigma,t),\quad 0 \le h < 1,
\label{eq:health}
\end{equation}
ここで $h \!\uparrow$ は性能劣化の進行を表す。
例として,NBTIのしきい値シフトは
\begin{equation}
\Delta V_{th}(t) = A_{\mathrm{NBTI}}\,
\exp\!\left(-\frac{E_a}{k T}\right)\!
\left(\frac{V}{V_0}\right)^{m} t^{n},
\label{eq:nbti}
\end{equation}
HCIの劣化量は
\begin{equation}
\Delta g_m(t) = A_{\mathrm{HCI}}\,
\left(\frac{I_d}{I_0}\right)^{p}
\exp\!\left(-\frac{E_a}{k T}\right) t^{n},
\label{eq:hci}
\end{equation}
TDDBの破壊確率(Weibull)は
\begin{equation}
F_{\mathrm{TDDB}}(t)=1-\exp\!\left[-\left(\frac{t}{\eta(V,T)}\right)^{\beta}\right],
\quad \eta(V,T)=\eta_0 \exp\!\left(\frac{\gamma}{E}\right),
\label{eq:tddb}
\end{equation}
を用いる($E$は酸化膜電界,$\beta$形状,$\eta$尺度)。
これらを正規化し $h(t)$ に写像する。

劣化はプラントゲイン $G$・時定数 $\tau$ 等へ反映され,
周波数特性に影響する。PIDゲインは $h$ の関数として適応:
\begin{align}
K_P(t) &= K_{P0}\bigl(1 - \alpha_P\,h(t)\bigr),\quad
K_I(t) = K_{I0}\bigl(1 - \alpha_I\,h(t)\bigr), \label{eq:gain_sched}\\
K_D(t) &= K_{D0}\bigl(1 - \alpha_D\,h(t)\bigr),
\end{align}
ただし $\alpha_{\{\cdot\}}\!\ge 0$ はスケジューリング感度。
位相余裕 $\phi_m(t)$ が安全余裕 $\phi_{\min}$ を常に満たすよう
\begin{equation}
\phi_m\bigl(K_P(t),K_I(t),K_D(t);\,G(h)\bigr) \ge \phi_{\min}.
\label{eq:pm}
\end{equation}
アンチワインドアップとアクチュエータ飽和制限も $h$ 連動で更新する。

\subsection{FSM層:信頼性状態監督と安全モード制御}
信頼性監視変数 $\theta(t)$(例:$T_{\max},\,\sigma_{\max},\,I_{\mathrm{RMS}}$)から
統合指標 $\Theta(t)$ を構成し,統計的に設定した閾値で遷移:
\begin{equation}
\Theta(t)=w_T\frac{T}{T_{\mathrm{crit}}}+w_\sigma\frac{\sigma}{\sigma_{\mathrm{crit}}}
+w_V\frac{V}{V_{\mathrm{crit}}},\quad \sum w_i=1.
\end{equation}
\[
\texttt{NORMAL}\xrightarrow{\Theta>\Theta_{\mathrm{warn}}}\texttt{DEGRADE}
\xrightarrow{\Theta>\Theta_{\mathrm{crit}}}\texttt{SAFE\_SHUTDOWN}.
\]
閾値は実運転分布 $\mathcal{N}(\mu,\sigma)$ から
$\Theta_{\mathrm{warn}}\!=\mu+2\sigma$,
$\Theta_{\mathrm{crit}}\!=\mu+3\sigma$ 等で初期化し,
運用で逐次更新する。
安全不変条件は
\begin{equation}
\mathcal{I}:\; T\!\le\!T_{\mathrm{crit}},\ \sigma\!\le\!\sigma_{\mathrm{crit}},\ V\!\le\!V_{\mathrm{crit}}
\Rightarrow \texttt{SAFE}(t)\ \text{保持}.
\label{eq:invariant}
\end{equation}

\subsection{LLM層:長期整合・再同定・再設計}
LLM層は SystemDK スキーマとログ(FEM/HIL/製造試験)に対し
\emph{整合性検査} $\mathrm{ConsistencyCheck}(\cdot)$ を実施し,
劣化モデルの同定・入替え・ハイブリダイズを提案する。
RAG(スキーマ+物理モデル)を用いて
\[
\mathrm{ProposeRedesign}:\ \{f,\,G,\,\theta\}\ \mapsto\ \{f',\,G',\,\theta'\}
\]
を生成し,AXI4-Lite 経由で PID/FSM に反映する。
リアルタイム性は要求せず($\tau_{\mathrm{PID}}\!\ll\!\tau_{\mathrm{FSM}}\!\ll\!\tau_{\mathrm{LLM}}$),
オフライン〜準オンラインで運用する。

\subsection{ヘルス推定と同定(Observer/Estimator)}
$h$ は不可観測であるため拡張カルマン/UKFで同定する。
状態ベクトル $\xi=[x^\top\ h]^\top$,入力 $u$,観測 $y$ とすると
\begin{align}
\dot{x} &= f_x(x,u,h,t) + w,\qquad
\dot{h} = f_h(T,V,\sigma,t) + \nu, \\
y &= g(x,u,h,t) + v,
\end{align}
$w,\nu,v$ は雑音。推定 $\hat{h}$ を用いて
(\ref{eq:gain_sched}) をオンライン更新し,
(\ref{eq:pm}) と (\ref{eq:invariant}) を満たすようゲインを制約付き最適化する:
\begin{equation}
\min_{\theta} J = \int_0^{T_f}\!\!\left(q_e e^2 + q_u u^2 + q_h \hat{h}^2\right)dt
\quad \text{s.t.}\ \ (\ref{eq:pm}),(\ref{eq:invariant}).
\end{equation}

\subsection{信頼性関数と余寿命(RUL)}
要素寿命分布を Weibull とし,システムは直列構成の保守的近似:
\begin{align}
R_i(t) &= \exp\!\left[-\left(\frac{t}{\eta_i(T,V,\sigma)}\right)^{\beta_i}\right],\\
R_{\mathrm{sys}}(t) &= \prod_i R_i(t),\quad
\mathrm{RUL} = \inf\{t: R_{\mathrm{sys}}(t)\le R_{\min}\}.
\end{align}
$\eta_i$ には Arrhenius/E-model 等の加速を含める。
LLM層は $R_{\mathrm{sys}}$ が閾値 $R_{\min}$ を下回る予兆で
\texttt{DEGRADE} への遷移と設計変更案(熱経路改善,駆動電圧デレーティング等)を提示する。

\subsection{統合ループ:制御×信頼性の閉結合}
三層協調により,
PIDは実時間補償,FSMは安全監督,LLMは長期整合と再設計を担い,
SystemDK上で「制御安定化」と「信頼性解析」が同一スキーマ内で結合される。
Lyapunov 関数 $V(x)$ に対し
\[
\dot{V} \le -\lambda V + \gamma\,h(t)
\]
を満たすようゲインと操作を更新し,$h\!\downarrow$(補償)で
漸近安定性余裕を維持する。

\subsection*{パラメータ校正と再現性(実務テンプレ)}
\begin{enumerate}
\item \textbf{モデル校正:} $E_a,n,m,p,\beta,\eta_0,\gamma$ をベンチ試験で回帰。
\item \textbf{観測設計:} $T,\sigma,V,I_{\mathrm{RMS}}$ の計測系を仕様化(帯域・量子化・校正)。
\item \textbf{同定:} EKF/UKF の雑音共分散と初期化($\hat{h}(0)=0$ 等)。
\item \textbf{安全値:} $T_{\mathrm{crit}},\sigma_{\mathrm{crit}},V_{\mathrm{crit}},\phi_{\min}$ を設計審査で承認。
\item \textbf{I/F:} AXI4-Lite レジスタに $K_P,K_I,K_D,\Theta_{\mathrm{warn/crit}}$ を割当(表で公開)。
\item \textbf{報告:} KPI($t_{\mathrm{settle}},M_p,\Delta T_{\mathrm{pk}},\sigma_{\mathrm{pk}},e_{\mathrm{rms}},R(t)$)と改善率を表\&図で提示。
\end{enumerate}

\subsection*{記号一覧(抜粋)}
$T$: 温度,\ $V$: 電圧,\ $\sigma$: 応力,\ $h$: 健康状態,\ $R$: 信頼性関数,\
$\beta$: Weibull形状,\ $\eta$: 尺度,\ $\phi_m$: 位相余裕.
