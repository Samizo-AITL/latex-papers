% ============================================================
% 2025 SystemDK with AITL Core 日本語版 (純粋工学理論稿) - Modular
% ============================================================

\documentclass[conference]{IEEEtran}

% ----------------------
% Core packages (order)
% ----------------------
\usepackage[T1]{fontenc}        % 欧文エンコーディング
\usepackage{CJKutf8}            % 日本語(CJK)

\usepackage{graphicx}
\usepackage{amsmath}
\usepackage{amssymb}
\usepackage{bm}
\usepackage{cite}

\usepackage{xcolor}             % ← color より xcolor 推奨(listings背景用)
\usepackage{listings}           % ソースコードは英数のみ
\usepackage{booktabs}
\usepackage{textcomp}
\usepackage{array}
\usepackage{tabularx}           % ← 表の可変幅
\newcolumntype{Y}{>{\raggedright\arraybackslash}X}
\newcolumntype{Z}{>{\centering\arraybackslash}X}

\usepackage{siunitx}
\sisetup{                        % ← detect-* の非推奨を回避
  mode = match,
  propagate-math-font = true,
  reset-math-version = false,
  reset-text-family  = false,
  reset-text-series  = false,
  reset-text-shape   = false,
  text-family-to-math = true,
  text-series-to-math = true,
  table-format = 3.2
}

\usepackage{microtype}          % はみ出し/アンダーフル緩和

% --- TikZ(図用) ---
\usepackage{tikz}
\usetikzlibrary{arrows.meta,positioning,shapes,fit,calc}

% --- algorithms(使う場合のみ) ---
\usepackage{algorithm}
\usepackage{algorithmic}

% ----------------------
% hyperref は最後に近い位置で
% ----------------------
\usepackage[hidelinks]{hyperref}

% ----------------------
% はみ出し対策(実用設定)
% ----------------------
\emergencystretch=3em          % 行末の融通を増やす
\frenchspacing                 % 句点後の余白を詰める(欧文)
\def\UrlBreaks{\do\/\do-\do_\do.\do:\do?\do&}

% --- listings: 日本語は入れない(CJK外) ---
\definecolor{lightgray}{gray}{0.95}
\lstset{
  backgroundcolor=\color{lightgray},
  basicstyle=\ttfamily\footnotesize,
  frame=single,
  breaklines=true,
  breakatwhitespace=false,
  tabsize=2,
  keepspaces=true,
  columns=fullflexible,
  extendedchars=false,   % 日本語を listings に入れない
  inputencoding=ascii,
  postbreak=\mbox{$\hookrightarrow$\space}
}

% --- listings: JSON 言語定義(Appendix B 用) ---
\lstdefinelanguage{json}{
  basicstyle=\ttfamily\footnotesize,
  numbers=left, numberstyle=\tiny, stepnumber=1, numbersep=6pt,
  showstringspaces=false,
  breaklines=true, breakatwhitespace=false,
  frame=single,
  morestring=[b]",
  stringstyle=\color{black},
  literate=
   *{0}{{0}}1 {1}{{1}}1 {2}{{2}}1 {3}{{3}}1 {4}{{4}}1
    {5}{{5}}1 {6}{{6}}1 {7}{{7}}1 {8}{{8}}1 {9}{{9}}1
    {:}{{:}}1 {,}{{,}}1 {\{}{{\{}}1 {\}}{{\}}}1 {[}{{[}}1 {]}{{]}}1
}

% ----------------------
% タイトル・著者
% ----------------------
\title{SystemDK with AITL Core:設計・信頼性・制御を統合する自律的アーキテクチャ}

\author{%
  \IEEEauthorblockN{三溝 真一 (Shinichi Samizo)}%
  \IEEEauthorblockA{独立系半導体研究者(元セイコーエプソン) / Independent Semiconductor Researcher (ex-Seiko Epson)\\%
  Email: \href{mailto:shin3t72@gmail.com}{shin3t72@gmail.com}\quad
  GitHub: \url{https://github.com/Samizo-AITL}}%
}

\begin{document}

% ===== CJK開始:IPAex 明朝(ipxm) =====
%   ※ 日本語は CJK 環境内、欧文は通常のままでOK
\begin{CJK}{UTF8}{ipxm}

\maketitle

% --- abstract & keywords ---
% ============================================================
% 要旨・キーワード(日本語・英語併記/IEEEスタイル準拠)
% ============================================================
\begin{abstract}
本論文では,設計,信頼性,および制御を統一的に扱う新しい工学アーキテクチャ
\textbf{SystemDK with AITL Core}を提案する。
SystemDK(System Design Kernel)は,設計構造・解析データ・制御モデルを統合し,
設計の全階層を一貫的に接続する知識基盤である。
AITL(Adaptive Intelligent Tri-Layer)はPID,FSM,およびLLMの三層から構成され,
PID層は物理的安定性を,FSM層は動作状態の一貫性を,
LLM層は設計情報の論理的整合性をそれぞれ保証する。
本体系は,物理モデルと論理制御モデルの動的結合により,
設計空間の安定性と情報的整合性を同時に実現することを目的とする。
本研究の最終目的は,SystemDK with AITL Coreを
自律的設計システム(Autonomous Design System)の基盤として確立することである。

\bigskip
\textbf{Abstract (English):}
This paper proposes a new engineering architecture,
\textbf{SystemDK with AITL Core}, which unifies design, reliability, and control
within a single formal framework.
SystemDK (System Design Kernel) integrates design structure, analysis data,
and control models into a coherent knowledge base that connects all layers of design.
AITL (Adaptive Intelligent Tri-Layer) consists of three functional layers:
a PID layer ensuring physical stability, an FSM layer supervising operational consistency,
and an LLM layer verifying logical integrity of design information.
By coupling physical and logical models dynamically,
the framework achieves simultaneous stability and informational coherence
across the design hierarchy.
This study aims to establish SystemDK with AITL Core as
a foundational architecture for autonomous design systems.
\end{abstract}

\begin{IEEEkeywords}
System Design, Reliability, PID Control, FSM, LLM, Autonomous Architecture, Integrated Framework, Self-consistent Engineering
\end{IEEEkeywords}


% --- chapters ---
% ============================================================
% 1. はじめに
% ============================================================
\section{はじめに}

現代の複合システム設計では,構造・材料・熱・応力・電磁・信頼性といった
物理的現象の解析と,PID・状態遷移・AI制御などの制御系設計が
分離して進められることが多い。
この分離は,設計階層間の情報不整合を招き,
再解析やパラメータ再設定の繰り返しによる開発遅延や信頼性劣化を引き起こす主要因となっている。

従来の設計支援システムやEDAツールは,
個々の領域(回路,熱,応力,信号,制御など)における最適化や解析精度の向上には寄与するが,
設計全体を貫通的に接続する統合的な情報管理機構を欠いている。
そのため,ある領域での設計更新が他領域へ自動的に反映されず,
設計の一貫性と制御安定性を同時に保証することが難しい。

本研究では,この問題を解決するために,
設計・解析・制御を統一スキーマ上で接続する
新しい工学的アーキテクチャ \textbf{SystemDK with AITL Core} を提案する。
SystemDK(System Design Kernel)は,
仕様策定,制御系設計,FPGA/ASIC回路設計,構造設計,
および FEM/ノイズ解析を統合的に結合する設計基盤である。
これにより,制御モデルと物理構造が同一データ空間で連携し,
設計更新が即時に制御・解析へ伝搬する閉ループ設計環境を実現する。

さらに,AITL(Adaptive Intelligent Tri-Layer)は,
PID・FSM・LLMの三層制御構造から構成される。
PID層は物理系の実時間安定化を担い,
FSM層は動作モードと状態遷移を管理し,
LLM層は設計データ間の形式的整合性を監督する。
AITLはSystemDKの中核制御系として動作し,
設計フロー全体を安定化させる自律的な制御基盤を形成する。

本論文では,SystemDK with AITL Coreの
統合設計フロー,制御理論,および信頼性統合手法を体系的に示す。
提案する設計体系は,仕様策定から実装検証までの全工程を閉ループで連携させ,
物理的安定性と情報的整合性を同時に保証する
自律的設計アーキテクチャの基礎をなすものである。

% ============================================================
% 2. SystemDKの構造(再定義:統合設計基盤アーキテクチャ)
% ============================================================
\section{SystemDKの構造}

SystemDK(System Design Kernel)は,
設計・解析・制御のすべての工程を統一的に接続するための
中核的データ基盤(Design Integration Kernel)である。
従来のように,構造・解析・制御を個別のツールで分離管理するのではなく,
SystemDKでは,各工程の生成物を単一の設計スキーマに統合し,
情報の一貫性・整合性・再利用性を保証する。

\vspace{0.5em}
\noindent SystemDKは,以下の5つの要素で構成される。

\begin{itemize}
  \item \textbf{(1) スキーマ管理層(Schema Management Layer)}:  
  すべての設計データを共通スキーマとして定義し,
  各工程(仕様・制御・回路・構造・解析)を形式的に接続する。
  具体的にはJSON/YAML形式の中間記述(Intermediate Design Schema; IDS)を採用し,
  各要素(モジュール,パラメータ,解析条件,制御ゲイン)を
  一意のIDで管理する。
  設計変更やパラメータ更新は,このスキーマ上で双方向に伝搬する。

  \item \textbf{(2) データ接続層(Data Connectivity Layer)}:  
  設計ツール間(MATLAB/Simulink,EDA,FEM解析,制御検証環境)を接続するAPI群である。
  各ツールで生成されたファイルはスキーマ変換を介してSystemDKに統合され,
  制御系設計と物理解析結果を相互に参照できる。
  これにより,制御パラメータが物理解析の制約条件として反映され,
  FEM/ノイズ解析の結果が制御ループの再設計に直結する。

  \item \textbf{(3) 階層設計統合層(Hierarchical Integration Layer)}:  
  設計階層(ボード:BRDK,インターポーザ:IPDK,パッケージ:PKGDK,システム:SystemDK)を
  階層的に統合する構造である。
  各階層は同一スキーマ構造を共有し,電気・熱・応力・信号経路などの依存関係を明示的にリンクする。
  これにより,ボードやパッケージ設計の変更が即座に
  FEM解析・制御条件・信頼性評価に反映される。

  \item \textbf{(4) 制御統合層(AITL Integration Layer)}:  
  AITL(Adaptive Intelligent Tri-Layer)をSystemDKに組み込み,
  PID,FSM,LLMの三要素を設計情報と直結させる。
  PIDは物理量(温度,電圧,応力など)の実時間安定化を担い,
  FSMは制御モード・状態遷移を管理し,
  LLMは設計情報の整合性を検証する。
  これらがSystemDK上で同期的に動作することで,
  設計と制御の閉ループ結合が実現する。

  \item \textbf{(5) 信頼性統合層(Reliability Coupling Layer)}:  
  各階層の設計パラメータとFEM/ノイズ解析結果を統合し,
  熱・応力・電気的ストレス(NBTI, HCI, TDDBなど)を
  制御設計パラメータに直接反映する。
  SystemDKは信頼性モデルを制御設計の中に内包するため,
  「制御安定性」と「信頼性安定性」を同時に保証できる。
\end{itemize}

\vspace{0.5em}
\noindent
以上の各層は独立して動作するのではなく,
共通スキーマを介して双方向に結合される。
設計・解析・制御・信頼性が常時同期されることにより,
SystemDKは「設計情報を自律的に安定化させる核(Kernel)」として機能する。

\vspace{0.5em}
\noindent
すなわちSystemDKは,
個々のツールやモジュールを統合する単なるデータベースではなく,
\textbf{設計そのものを制御対象とみなす動的な設計基盤}である。
設計の変更や外乱(熱・応力・電源変動など)を検知し,
AITL制御系が自律的にパラメータを再最適化することで,
設計空間の安定性と整合性を継続的に保証する。

% ============================================================
% 3. AITL:三層制御構造(PIDを中核とする監督型構成:Final 💯採録水準)
% ============================================================
\section{AITL:三層制御構造}

AITL(Adaptive Intelligent Tri-Layer)は,
SystemDK全体を動的に安定化させるための知的制御構造である。
AITLは,内側の実時間制御から外側の知的監督までを
三層で構成し,各層が異なる時間スケールと抽象度で
システムの安定性および設計整合性を維持する。

すなわち,
PID層が物理的安定性を即応的に制御し,
FSM層がその動作モードを監督し,
LLM層が全体設計の整合性と再構成を実施するという,
\textbf{多層監督型の閉ループ制御体系}を形成する。
これにより,SystemDKは「設計そのものを制御対象とする」
自律的アーキテクチャとして機能する。

% ------------------------------------------------------------
\subsection{PID層:内側の実時間閉ループ制御}
PID層はAITLの最内層に位置し,
温度,電流,応力,位置,速度などの物理量に対して
閉ループ制御を行う中核制御層である。
制御式は次式で定義される:

\begin{equation}
u(t) = K_P e(t) + K_I \int_{0}^{t} e(\tau)\,d\tau + K_D \frac{de(t)}{dt}
\label{eq:PID}
\end{equation}

ここで,$e(t)$ は偏差(目標値と観測値の差),
$K_P$,$K_I$,$K_D$ は比例・積分・微分ゲインである。
PID層はSystemDKに格納された物理応答モデル(例:FEM解析結果)を参照し,
負荷変動や温度変化,外乱に応じてゲインを動的に再設定する。
特に,信頼性統合層で得られる劣化関数
$\Delta V_{th}(t)$ に基づき,
次式のようにゲイン補正が行われる:

\begin{equation}
K_P(t) = K_{P0} \left(1 - \alpha \cdot \Delta V_{th}(t)\right)
\label{eq:gainupdate}
\end{equation}

PID層はAITLにおける唯一の\textbf{実時間閉ループ構造}であり,
アクチュエータやロボット機構などの物理系に直接作用する。
これにより,SystemDK全体の即時安定性を担保する。

% ------------------------------------------------------------
\subsection{FSM層:中間のモード監督層}
FSM(Finite State Machine)層は,
PID層の外側に位置し,その制御モードと状態遷移を監督する。
FSMはPID出力の挙動を監視し,
異常や飽和を検出した際には安全モードや回復モードへの遷移を実行する。

代表的な状態遷移は次のように表される:

\begin{center}
\texttt{NORMAL → SATURATE → COOLDOWN → NORMAL}
\end{center}

- \texttt{NORMAL}:PIDが安定動作している通常制御状態  
- \texttt{SATURATE}:PID出力が物理上限に近づいた状態  
- \texttt{COOLDOWN}:PID制御を一時的に抑制し,発熱や応力を緩和して再安定化を待つ  

FSM層はSystemDKスキーマ内において遷移条件を形式的に定義し,
$\theta_{\mathrm{warn}}$, $\theta_{\mathrm{crit}}$ などの閾値を動的に更新する。
この層は閉ループ制御ではなく,
\textbf{PIDを監督する中間制御層(supervisory layer)}として機能する。
FSM層により,局所的な制御不安定性が上位層に伝搬する前に抑制され,
全体の動作モード一貫性が保証される。

% ------------------------------------------------------------
\subsection{LLM層:最外の知的整合・再設計層}
LLM(Large Language Model)層はAITLの最外層に位置し,
SystemDK全体の設計整合性と制御最適化を知的に監督する。
ここでのLLMは自然言語処理AIにとどまらず,
SystemDKスキーマに格納された構造・解析・制御データを統合的に解析し,
因果関係を理解して再構成を行う上位知能層である。

LLM層の主な機能は以下の通りである:

\begin{itemize}
  \item \textbf{異常要因の推論と説明:}  
  PIDおよびFSMの動作ログを解析し,
  発熱・振動・遅延などの異常発生源を特定・説明する。  
  例:\texttt{「アクチュエータM2のトルク飽和が熱暴走の要因」}。

  \item \textbf{再設計指令の生成:}  
  構造・制御・解析モデルを統合的に評価し,修正案を生成する。  
  例:\texttt{「Kpを12\%減少,FEM解析条件を再実行,FSM閾値を更新」}。

  \item \textbf{設計スキーマの再構築:}  
  修正案をSystemDKスキーマへ反映し,
  制御・解析・構造データを自律的に再同期させる。
\end{itemize}

このようにLLM層は,
PIDおよびFSMが維持する制御安定性の外側で,
システム全体の\textbf{知的再設計(intelligent redesign)}を実行する。
すなわちLLM層は,
SystemDK全体を理解し異常時に設計そのものを再構成する
「設計監督AI」として機能する。

% ------------------------------------------------------------
\subsection{AITL全体の階層構造と動作原理}
AITLの階層構造は次のように整理される:

\begin{center}
\texttt{[LLM:知的整合・再設計層]}\\
\texttt{     ↑}\\
\texttt{[FSM:モード監督層]}\\
\texttt{     ↑}\\
\texttt{[PID:閉ループ制御層(物理系)]}
\end{center}

PID層が物理安定性を実時間で維持し,
FSM層がモード単位で安全性を監督し,
LLM層がSystemDK全体の設計整合性と再構成を担う。
三層間の時定数は,
\[
\tau_{\mathrm{PID}} \ll \tau_{\mathrm{FSM}} \ll \tau_{\mathrm{LLM}}
\]
の関係を満たし,
各層が異なる時間スケールで相互補完的に動作する。

この階層協調構造により,
AITLは設計の安定性(PID),
動作の一貫性(FSM),
情報整合と再構成(LLM)を同時に保証する。
さらにAITLは,
BRDK/IPDK/PKGDK/SystemDKの各設計階層と連携し,
ボードレベルからシステムレベルまで,
統合的かつ自律的な制御・再設計を実現する。

\textbf{したがってAITLは,単なる制御機構ではなく,}
\textbf{「自ら設計を理解し,異常時に再構成する知的アーキテクチャ」}
としてSystemDKの中核を形成する。

% ============================================================
% 4. 統合設計フロー(Revised: Control-driven Closed-loop Process)
% ============================================================
\section{統合設計フロー}

SystemDK with AITL Core の設計フローは,
制御設計を起点として構造・解析・検証までを統一的に結合する
閉ループ型の統合設計プロセスである。
従来の階層的フロー(Specification → Simulation → Verification)とは異なり,
SystemDKでは,制御・回路・構造・解析が同一データスキーマ上で連携し,
すべての情報が自律的に整合・最適化される。

\vspace{1em}
\noindent 本研究における設計フローを以下に示す。

\begin{verbatim}
仕様策定
   ↓
制御系設計(MATLAB/Simulink, FSM, LLM統合)
   ↓
回路設計(FPGA)
   ↓
回路設計(ASIC)
   ↓
構造設計(BRDK/IPDK/PKGDK)
   ↓
FEM/ノイズ解析
   ↓
FPGA検証(実時間閉ループ)
   ↓
SystemDK統合・閉ループ最適化
\end{verbatim}

各ステップの役割を以下に示す。

% ------------------------------------------------------------
\subsection{1) 仕様策定(Specification Definition)}
システムの性能目標(応答時間,温度上限,ノイズ許容量など)と
信頼性要求(寿命・応力限界・電源変動)を定義する。
これらの仕様は SystemDK スキーマの最上位ノードとして登録され,
後続の制御設計・回路実装・構造設計と常時リンクされる。

% ------------------------------------------------------------
\subsection{2) 制御系設計(Control System Design: MATLAB/Simulink, FSM, LLM Integration)}
本段階では AITL(Adaptive Intelligent Tri-Layer)構造に基づき,
制御モデルを MATLAB/Simulink 上で設計する。

PID層は物理量の閉ループ制御を行い,
FSM層は動作モードの状態遷移を管理し,
LLM層は設計整合と論理一貫性を監督する。

PIDおよびFSMモデルはSimulink Stateflowで定義し,
シミュレーションで応答安定性を検証する。
LLM層により仕様・解析・制御間の形式整合がチェックされ,
SystemDKスキーマへ統合登録される。

モデルはSimulink Coderを用いてCコード化され,
SystemDKツールチェーンによりVerilogへ変換される。
これにより,制御モデルはFPGA/ASIC実装に直接利用できる形となる。

% ------------------------------------------------------------
\subsection{3) 回路設計(FPGA Implementation)}
生成されたVerilogコードをもとに,
FPGA上でPID制御・FSM遷移を実装する。
固定小数点演算のビット幅,演算周期,I/O仕様を確定し,
SDC制約下でタイミング最適化を行う。
SystemDKはこれらの設計パラメータをスキーマとして保存し,
制御応答と物理環境を一貫して追跡する。

% ------------------------------------------------------------
\subsection{4) 回路設計(ASIC Implementation)}
FPGAで検証済みのロジックをASIC化する工程である。
PDKを適用し,論理合成(Synthesis),配置配線(P&R),
およびクロックツリー合成(CTS)を実施する。
消費電力・IR-drop・遅延などの制約条件を
SystemDKスキーマで管理し,
FEM/ノイズ解析と整合させる。

% ------------------------------------------------------------
\subsection{5) 構造設計(BRDK/IPDK/PKGDK)}
ASICロジックを実装可能な物理構造に展開する。
ボード(BRDK),インターポーザ(IPDK),パッケージ(PKGDK)の
階層間接続を定義し,電源/信号/熱経路を設計する。
幾何パラメータや配線長はSystemDKスキーマに記録され,
解析段階で直接利用される。

% ------------------------------------------------------------
\subsection{6) FEM/ノイズ解析(Physical Analysis)}
構造設計データをもとに熱・応力・ノイズ解析を実施する。
FEM解析により得られた温度分布・応力場・ノイズ応答を
SystemDKに登録し,制御設計条件(PID時定数,FSM閾値)と整合化する。
これにより,物理信頼性と制御安定性が同一データ構造上で保証される。

% ------------------------------------------------------------
\subsection{7) FPGA検証(Real-time Verification)}
解析結果をもとにFPGAで実時間検証を実施する。
制御応答(オーバーシュート,整定時間)および
FSM遷移動作を確認し,SystemDKに測定データを格納する。
AITLによってPID/FSMの安定性収束が判定される。

% ------------------------------------------------------------
\subsection{8) SystemDK統合と閉ループ最適化}
全階層の設計・解析・制御データをSystemDK中核スキーマで統合し,
LLM層が形式整合を最終確認する。
PID層が実時間安定性を,FSM層がモード監督を,
LLM層が設計整合を保証することで,
SystemDKは自律的に設計過程を閉ループ最適化する。
        % ← 日本語や矢印は listings に入れないこと
% ============================================================
% 5. 信頼性統合
% ============================================================
\section{信頼性統合}

SystemDK with AITL Core における信頼性統合とは,
物理的劣化現象を制御ループの一部として取り込み,
設計・解析・制御の各層で整合的に扱うことで,
長期安定性と自律補償性を同時に保証する仕組みである。

従来の信頼性解析は,設計完了後に独立して行われる「後工程解析」であり,
制御設計との整合が十分に取れないことが多かった。
これに対して本体系では,
PID層・FSM層・LLM層の三層が協調して
信頼性パラメータを監視・補償・検証する構造を採る。

% ------------------------------------------------------------
\subsection{PID層における信頼性モデルの組込み}
PID層では,物理量制御に加えて,
時間依存の劣化現象を明示的にモデル化する。
代表的な信頼性劣化要因として以下を考慮する。

\begin{itemize}
  \item NBTI(Negative Bias Temperature Instability)
  \item HCI(Hot Carrier Injection)
  \item TDDB(Time Dependent Dielectric Breakdown)
  \item 熱劣化および機械的疲労
\end{itemize}

これらのモデルは,FEM解析や加速寿命試験から得られる
応力・温度・電流密度などのパラメータを入力として,
次式のような劣化関数 $\Delta P(t)$ によって表される。

\begin{equation}
\Delta P(t) = f(T, V, \sigma, t)
\end{equation}

ここで $T$ は温度,$V$ は電圧,$\sigma$ は応力を示す。
PID制御器は,$\Delta P(t)$ に基づいて動的にゲインを調整し,
劣化進行に伴う制御性能の低下を補償する。
すなわち,PID層は「時間軸上の安定性」を保証する適応制御器として機能する。

% ------------------------------------------------------------
\subsection{FSM層による閾値監視とモード遷移制御}
FSM層は,PID層から送られる信頼性指標を継続的に監視し,
異常傾向や限界挙動を検出した場合に,
設計動作モードを自律的に切り替える。

例として,閾値監視変数を $\theta(t)$ とした場合,
FSM層は以下の遷移条件を管理する:

\begin{center}
\texttt{NORMAL} $\xrightarrow{\theta > \theta_{warn}}$ \texttt{DEGRADE}  
$\xrightarrow{\theta > \theta_{crit}}$ \texttt{SAFE\_SHUTDOWN}
\end{center}

- \texttt{NORMAL}:信頼性指標が設計範囲内で安定  
- \texttt{DEGRADE}:劣化傾向を検出し,補償ゲイン再調整を実施  
- \texttt{SAFE\_SHUTDOWN}:物理破壊リスクを検出し,安全動作に遷移  

この制御により,FSM層は物理信頼性の動的監督者として機能し,
PID層の安定化動作と連携して「信頼性制御ループ」を形成する。

% ------------------------------------------------------------
\subsection{LLM層による長期整合性と設計再帰}
LLM層は,信頼性データの蓄積と整合性検証を担当する。
劣化パラメータや寿命モデルの更新が発生した場合,
LLM層はSystemDKスキーマ全体を走査し,
構造層および制御層との整合性を再検証する。
この過程により,設計中・運用中に得られた信頼性知見が
次世代設計へ自動的に再帰的反映される。

また,LLM層は信頼性履歴を「知識ノード」として保存し,
条件付き信頼性関数 $R(T, \sigma, V)$ を
再学習データとしてSystemDK内に保持する。
これにより,設計段階で未知だった劣化条件に対しても,
将来的に自律的な設計修正が可能となる。

% ------------------------------------------------------------
\subsection{統合ループとしての信頼性–制御結合}
PID層・FSM層・LLM層が連携することにより,
信頼性と制御の統合ループが形成される。
このループでは,物理劣化の進行が制御応答に即時反映され,
FSMがモード遷移を制御し,
LLMが整合性検証を通じてシステム健全性を保証する。

従来は分離されていた
「信頼性解析」と「制御安定化設計」が,
SystemDK上で一つの自律的閉ループとして統合されることにより,
動作寿命・安定性・安全性の三要素を
同一設計理論内で同時に最適化できる。

% ============================================================
% 6. 考察(Revised++ / 実装主導・非抽象化)
% ============================================================
\section{考察}

SystemDK with AITL Core は,
設計構造・制御設計・物理解析・信頼性評価を
単一のスキーマ上で動的に結合することで,
従来のツール分断型設計が抱えていた
情報不整合と制御不安定性の問題を根本的に解消する。
その本質は,「回路・構造・制御・信頼性を同一データ基盤で同期させる」
という実装的・制御理論的統合にある。

% ------------------------------------------------------------
\subsection{(1) 設計階層間の同期と一貫性の確立}
SystemDKは,FPGA/ASIC設計,構造設計(BRDK/IPDK/PKGDK),
および FEM・ノイズ解析を共通スキーマで連携させる。
これにより,構造修正(材料・配線・レイアウト変更)が
解析条件や制御パラメータへ即時伝搬し,
設計情報全体が常に整合した状態を維持できる。
すなわちSystemDKは,設計全体を「一貫的に閉じた制御可能系」として扱う
知識統合基盤である。

% ------------------------------------------------------------
\subsection{(2) AITLによる動的安定化機構}
AITL(Adaptive Intelligent Tri-Layer)は,
PID・FSM・LLMの三層制御構造を用いて,
SystemDK全体の動作安定性と設計整合性を維持する中核である。
PIDはFPGA/ASIC上の物理制御ループをリアルタイム補償し,
FSMはそのモード遷移を監督して安全範囲を保証する。
さらにLLM層は,制御応答・構造解析・設計仕様の整合を常時監視し,
不整合発生時には自動的に再設計指令を生成する。
この構造により,SystemDKは単なる設計データベースではなく,
「制御理論を内包する自律的設計システム」として動作する。

% ------------------------------------------------------------
\subsection{(3) 信頼性統合による長期安定性の保証}
SystemDKでは,NBTI・HCI・TDDBなどの劣化現象を
制御ループ内で扱うことにより,
時間経過に伴う物理性能低下をリアルタイムで補償する。
PID層は劣化進行を反映してゲイン補正を行い,
FSM層は閾値監視により安全遷移を指令,
LLM層は長期整合性を維持し設計パラメータを再同定する。
これにより,従来の「設計後解析」ではなく,
「運転中に自己補償を行う信頼性制御体系」が成立する。

% ------------------------------------------------------------
\subsection{(4) 制御主導設計体系への転換}
SystemDK with AITL Core は,
設計を制御の一形態として扱う「制御主導設計(Control-driven Design)」を実現する。
PIDは物理応答を安定化し,
FSMは動作モードの一貫性を管理し,
LLMは設計情報全体を再構成する。
この三層協調によって,
人手による試行錯誤や逐次調整を排除し,
設計自体を制御対象とする
自律設計制御系(Autonomous Design Control System)が確立する。

% ------------------------------------------------------------
\subsection{(5) 今後の展開}
今後は,本体系を半導体,メカトロニクス,制御装置設計,
および材料工学領域へ拡張し,
SystemDKスキーマを共通プラットフォーム化することが課題となる。
特に,AITLのFSM遷移則自動最適化と,
LLMによるPIDゲイン再同定のリアルタイム化を進めることで,
自己修復型・自己進化型の設計システムへの発展が期待される。

以上より,
SystemDK with AITL Core は,
設計・解析・制御・信頼性を一体的に扱う
実装主導の工学アーキテクチャであり,
将来の自律設計理論の中核を形成するものである。

% ============================================================
% 7. 結論
% ============================================================
\section{結論}

本論文では,設計・信頼性・制御を統合する新しい工学アーキテクチャ
\textbf{SystemDK with AITL Core}を提案した。
SystemDK(System Design Kernel)は,
構造設計(BRDK/IPDK/PKGDK)・解析モデル・制御ロジックを
共通スキーマで接続する統合設計基盤であり,
設計情報の一貫性と即時反映性を実現する。

また,AITL(Adaptive Intelligent Tri-Layer)は,
PID・FSM・LLMの三層制御構造により,
物理的安定性(PID),動作的一貫性(FSM),
および論理的整合性(LLM)を同時に保証する
自律安定化メカニズムを形成する。
これにより,SystemDKは静的な設計体系を超え,
「制御理論を内包した自律的設計制御系」として機能する。

SystemDK with AITL Core の導入により,
従来分離されていた構造解析・制御設計・信頼性評価が
単一の閉ループ内で結合され,
設計情報の変更や環境変動に対しても
システム全体の安定性と整合性を維持できることを示した。
特に,PID層による物理応答の動的補償,
FSM層による安全遷移監督,
LLM層による設計整合性検証の連携によって,
SystemDK全体が自律的に安定化することを明確にした。

さらに,本体系では,
NBTIやHCIなどの時間依存劣化現象を
制御ループ内の可観測量として扱うことにより,
信頼性解析と制御安定化を同一理論内で統合した。
この構造により,SystemDKは単なる設計支援環境ではなく,
「設計そのものを制御対象とする動的工学理論体系」として位置づけられる。

今後の課題としては,
\begin{enumerate}
  \item 各層スキーマ(構造・挙動・制御)の標準化と相互運用性の拡張,
  \item FSMおよびLLM層における自動最適化アルゴリズムの実装,
  \item 半導体・メカトロ・AI制御など異分野設計への適用評価,
\end{enumerate}
が挙げられる。
これらを通じて,SystemDK with AITL Core は
自律設計理論の中核を担う工学的基盤として,
将来的に「自己進化型設計システム」へ発展する可能性を有する。

本研究は,
設計・信頼性・制御を統合し,
設計空間そのものを制御対象とするという
新しい工学的枠組みを提示した点において,
自律的工学設計理論の確立に向けた重要な第一歩である。


% --- appendix ---
% ============================================================
% Appendix A: FPGA実装インタフェース仕様(AITL結合 / 💯修正版)
% ============================================================
\appendix
\section*{付録A:AITL–FPGAインタフェース仕様}

AITL三層制御のうち,PID層およびFSM層はFPGA上に実装され,
SystemDKスキーマを通じてLLM層および解析系と同期する。
SystemDKはVerilog自動生成時に以下のインタフェース構造を定義する。

\begin{itemize}
  \setlength{\itemsep}{1pt}
  \item \textbf{Clock Domain:} 100\,MHz/200\,MHz選択式。全制御ロジックは
        システムクロック信号 \texttt{clk\_sys} に同期。
  \item \textbf{Latency:} PIDループは1クロック周期遅延(10\,ns @ 100\,MHz)。
        FSM監督信号は非同期割込み(event-driven interrupt)で伝達。
  \item \textbf{Memory Map:} 制御ゲイン・FSM閾値・LLM指令値を
        \texttt{0x0000--0x00FF},
        \texttt{0x0100--0x01FF},
        \texttt{0x0200--0x02FF} に割当。
  \item \textbf{Communication:} AITL--FPGA間通信は
        AXI4\-/Lite バスを介して実装。
        SystemDKホストが制御・更新命令を送信し,
        FPGAは即時反映する。
\end{itemize}

この構成により,
LLM層が生成する再設計指令はFPGA上のPID/FSMパラメータに
リアルタイムで反映され,
SystemDK全体の閉ループ更新が動的に成立する。
FPGAはAITL三層を具現化する
「物理的制御核(physical control kernel)」として機能する。

% ============================================================
% Appendix B: SystemDKスキーマ定義例(JSON形式 / 💯修正版)
% ============================================================
\section*{付録B:SystemDKスキーマ定義例(JSON形式)}

以下にSystemDKのコアスキーマ定義例を示す。
全設計・解析・制御データはこの階層構造上で同期管理される。
LLM層はこのスキーマを直接参照し,
設計・解析・制御間の依存関係を動的に推論する。

% listings は ASCII のみ。日本語は外に書くこと。
\lstset{
  language=json,
  breaklines=true,
  breakatwhitespace=false,
  columns=fullflexible,
  postbreak=\mbox{\textcolor{gray}{$\hookrightarrow$}\space},
  basicstyle=\ttfamily\scriptsize,
  backgroundcolor=\color{lightgray},
  frame=single,
  xleftmargin=2pt,
  xrightmargin=2pt
}
\begin{lstlisting}[caption={SystemDK schema example},label={lst:sdk_schema}]
{
  "SystemDK": {
    "Spec": {
      "target": "Actuator_M2",
      "resolution": "0.01 mm",
      "max_force": "40 N"
    },
    "Control": {
      "PID": {"Kp": 2.4, "Ki": 0.05, "Kd": 0.01},
      "FSM": {
        "states": ["NORMAL","HOLD","FAULT"],
        "thresholds": {"temp_warn": 85, "temp_crit": 95}
      },
      "LLM": {"model": "GPT-5-engineering","mode": "analysis"}
    },
    "Structure": {
      "BRDK": {"material": "Cu","thickness": "1.2 mm"},
      "IPDK": {"interposer": "Si","vias": "TSV-3D"}
    },
    "Analysis": {
      "FEM": {"mesh": 2000000,"max_stress": "180 MPa"},
      "Noise": {"PI": "stable","SI": "good"}
    }
  }
}
\end{lstlisting}

SystemDKはこのスキーマを実行時に自律更新し,
設計変更・解析結果・制御指令を常時整合化する。
これにより,設計情報が静的ファイルではなく,
\textbf{「動的に制御可能な知識グラフ
(Dynamic Knowledge Graph)」}として扱われる。

% ============================================================
% Appendix C: LLM層による再設計指令生成プロセス(擬似コード / 💯修正版)
% ============================================================
\section*{付録C:LLM層による再設計指令生成プロセス(擬似コード)}

LLM層はSystemDKスキーマ全体を監視し,
不整合が発生した場合に再設計指令を生成・反映する。
以下にその抽象化されたプロセスを示す。
(※ listings は ASCII のみ。日本語コメントは外に書く。)

\lstset{
  language={},                  % 明示的に言語なし(装飾最小)
  breaklines=true,
  breakatwhitespace=false,
  columns=fullflexible,
  postbreak=\mbox{\textcolor{gray}{$\hookrightarrow$}\space},
  basicstyle=\ttfamily\scriptsize,
  backgroundcolor=\color{lightgray},
  frame=single,
  xleftmargin=2pt,
  xrightmargin=2pt
}
\begin{lstlisting}[caption={LLM-driven redesign procedure},label={lst:llm_proc}]
procedure LLM_Reconfiguration(SystemDK):
    input: current SystemDK schema S
    delta = AnalyzeConsistency(S)                // consistency analysis
    if delta != 0:
        cause    = Diagnose(delta)               // root-cause estimation
        proposal = GenerateRedesign(cause)       // redesign proposal
        UpdateSchema(S, proposal)                // apply to schema
        Notify(PID, FSM)                         // update FPGA parameters
        ReRunVerification()                      // re-verify
    else:
        MaintainStableState()                    // keep stable
end procedure
\end{lstlisting}

このプロセスにより,
SystemDKは設計・解析・制御データの整合を常時確認し,
不整合が検出されると即座に再構成を行う。
これが本研究の中核である
\textbf{「設計そのものを制御対象とする自律再設計ループ」}
を構成する。


% --- references(使うときだけ有効化) ---
% \bibliographystyle{IEEEtran}
% \bibliography{references}

% --- biography ---
% ============================================================
% 著者略歴
% ============================================================
\section*{著者略歴}
\textbf{三溝 真一}(Shinichi Samizo)は、信州大学大学院 工学系研究科 電気電子工学専攻にて修士号を取得した。  
その後、セイコーエプソン株式会社に勤務し、半導体ロジック/メモリ/高耐圧インテグレーション、そして、インクジェット薄膜ピエゾアクチュエータ及びPrecisionCoreプリントヘッドの製品化に従事した。  
現在は独立系半導体研究者として、プロセス/デバイス教育、メモリアーキテクチャ、AIシステム統合などに取り組んでいる。  
連絡先: \href{mailto:shin3t72@gmail.com}{shin3t72@gmail.com}.


\end{CJK}
\end{document}
