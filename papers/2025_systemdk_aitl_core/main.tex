% ============================================================
% 2025 SystemDK with AITL Core 日本語版 (純粋工学理論稿) - Modular
% ============================================================

\documentclass[conference]{IEEEtran}

% ----------------------
% Core packages (order)
% ----------------------
\usepackage[T1]{fontenc}        % 欧文エンコーディング
\usepackage{CJKutf8}            % 日本語(CJK)

\usepackage{graphicx}
\usepackage{amsmath}
\usepackage{amssymb}
\usepackage{bm}
\usepackage{cite}

\usepackage{xcolor}             % ← color より xcolor 推奨(listings背景用)
\usepackage{listings}           % ソースコードは英数のみ
\usepackage{booktabs}
\usepackage{textcomp}
\usepackage{array}
\usepackage{tabularx}           % ← 表の可変幅
\newcolumntype{Y}{>{\raggedright\arraybackslash}X}
\newcolumntype{Z}{>{\centering\arraybackslash}X}

\usepackage{siunitx}
\sisetup{                        % ← detect-* の非推奨を回避
  mode = match,
  propagate-math-font = true,
  reset-math-version = false,
  reset-text-family  = false,
  reset-text-series  = false,
  reset-text-shape   = false,
  text-family-to-math = true,
  text-series-to-math = true,
  table-format = 3.2
}

\usepackage{microtype}          % はみ出し/アンダーフル緩和

% --- TikZ(図用) ---
\usepackage{tikz}
\usetikzlibrary{arrows.meta,positioning,shapes,fit,calc}

% --- algorithms(使う場合のみ) ---
\usepackage{algorithm}
\usepackage{algorithmic}

% ----------------------
% hyperref は最後に近い位置で
% ----------------------
\usepackage[hidelinks]{hyperref}

% ----------------------
% はみ出し対策(実用設定)
% ----------------------
\emergencystretch=3em          % 行末の融通を増やす
\frenchspacing                 % 句点後の余白を詰める(欧文)
\def\UrlBreaks{\do\/\do-\do_\do.\do:\do?\do&}

% --- listings: 日本語は入れない(CJK外) ---
\definecolor{lightgray}{gray}{0.95}
\lstset{
  backgroundcolor=\color{lightgray},
  basicstyle=\ttfamily\footnotesize,
  frame=single,
  breaklines=true,
  breakatwhitespace=false,
  tabsize=2,
  keepspaces=true,
  columns=fullflexible,
  extendedchars=false,   % 日本語を listings に入れない
  inputencoding=ascii,
  postbreak=\mbox{$\hookrightarrow$\space}
}

% --- listings: JSON 言語定義(Appendix B 用) ---
\lstdefinelanguage{json}{
  basicstyle=\ttfamily\footnotesize,
  numbers=left, numberstyle=\tiny, stepnumber=1, numbersep=6pt,
  showstringspaces=false,
  breaklines=true, breakatwhitespace=false,
  frame=single,
  morestring=[b]",
  stringstyle=\color{black},
  literate=
   *{0}{{0}}1 {1}{{1}}1 {2}{{2}}1 {3}{{3}}1 {4}{{4}}1
    {5}{{5}}1 {6}{{6}}1 {7}{{7}}1 {8}{{8}}1 {9}{{9}}1
    {:}{{:}}1 {,}{{,}}1 {\{}{{\{}}1 {\}}{{\}}}1 {[}{{[}}1 {]}{{]}}1
}

% ----------------------
% タイトル・著者
% ----------------------
\title{SystemDK with AITL Core:設計・信頼性・制御を統合する自律的アーキテクチャ}

\author{%
  \IEEEauthorblockN{三溝 真一 (Shinichi Samizo)}%
  \IEEEauthorblockA{独立系半導体研究者(元セイコーエプソン) / Independent Semiconductor Researcher (ex-Seiko Epson)\\%
  Email: \href{mailto:shin3t72@gmail.com}{shin3t72@gmail.com}\quad
  GitHub: \url{https://github.com/Samizo-AITL}}%
}

\begin{document}

% ===== CJK開始:IPAex 明朝(ipxm) =====
%   ※ 日本語は CJK 環境内、欧文は通常のままでOK
\begin{CJK}{UTF8}{ipxm}

\maketitle

% --- abstract & keywords ---
% ============================================================
% 要旨・キーワード(日本語・英語併記/IEEEスタイル準拠)
% ============================================================
\begin{abstract}
本論文では,設計,信頼性,および制御を統一的に扱う新しい工学アーキテクチャ
\textbf{SystemDK with AITL Core}を提案する。
SystemDK(System Design Kernel)は,設計構造・解析データ・制御モデルを統合し,
設計の全階層を一貫的に接続する知識基盤である。
AITL(Adaptive Intelligent Tri-Layer)はPID,FSM,およびLLMの三層から構成され,
PID層は物理的安定性を,FSM層は動作状態の一貫性を,
LLM層は設計情報の論理的整合性をそれぞれ保証する。
本体系は,物理モデルと論理制御モデルの動的結合により,
設計空間の安定性と情報的整合性を同時に実現することを目的とする。
本研究の最終目的は,SystemDK with AITL Coreを
自律的設計システム(Autonomous Design System)の基盤として確立することである。

\bigskip
\textbf{Abstract (English):}
This paper proposes a new engineering architecture,
\textbf{SystemDK with AITL Core}, which unifies design, reliability, and control
within a single formal framework.
SystemDK (System Design Kernel) integrates design structure, analysis data,
and control models into a coherent knowledge base that connects all layers of design.
AITL (Adaptive Intelligent Tri-Layer) consists of three functional layers:
a PID layer ensuring physical stability, an FSM layer supervising operational consistency,
and an LLM layer verifying logical integrity of design information.
By coupling physical and logical models dynamically,
the framework achieves simultaneous stability and informational coherence
across the design hierarchy.
This study aims to establish SystemDK with AITL Core as
a foundational architecture for autonomous design systems.
\end{abstract}

\begin{IEEEkeywords}
System Design, Reliability, PID Control, FSM, LLM, Autonomous Architecture, Integrated Framework, Self-consistent Engineering
\end{IEEEkeywords}


% --- chapters ---
% ============================================================
% 1. はじめに(Revised Final)
% ============================================================
\section{はじめに}

現代の複合システム設計では,
構造・材料・熱・応力・電磁・信頼性といった物理現象の解析と,
PID・状態遷移・AI制御などの制御系設計が,
独立した領域として個別に進められることが多い。
この分離は,設計階層間の情報不整合を招き,
パラメータ再設定やモデル再解析の繰り返しによる
開発遅延や信頼性劣化を引き起こす主要因となっている。

従来のEDAツールや設計支援システムは,
各領域(回路,熱,応力,信号,制御)における
局所的な最適化や解析精度向上には寄与しているが,
設計全体を俯瞰的に連結する統合的情報基盤を欠いている。
そのため,一部の設計変更が他の解析・制御領域へ自動伝搬せず,
設計全体としての一貫性と制御安定性を同時に保証することが困難である。

本研究では,この課題を解決するために,
設計・解析・制御を単一のデータスキーマ上で連結する
新しい工学アーキテクチャ
\textbf{SystemDK with AITL Core} を提案する。
SystemDK(System Design Kernel)は,
仕様策定,制御系設計,FPGA/ASIC回路設計,構造設計,
および FEM/ノイズ解析を統合的に接続し,
設計情報の流れを閉ループ化する知識基盤である。
これにより,構造的変更が制御モデルや解析条件へ即時反映され,
設計全体が自律的に安定化・最適化される環境を実現する。

さらに,AITL(Adaptive Intelligent Tri-Layer)は,
SystemDKの中核制御構造として機能する知的制御フレームワークであり,
PID・FSM・LLMの三層から構成される。
PID層は物理量の実時間安定化を,
FSM層は動作モードと状態遷移の一貫性を,
LLM層は設計データ間の論理整合性を監督する。
この三層制御構造により,
SystemDK全体が設計変更や解析結果に応じて
自動的に再整合・再最適化を行う自律的設計基盤が形成される。

本論文では,
SystemDK with AITL Core の
統合設計フロー,制御理論,および信頼性統合手法を体系的に示す。
提案する設計体系は,
仕様策定から実装検証までの全工程を閉ループで接続し,
物理的安定性と情報的整合性を同時に保証する
次世代の自律設計アーキテクチャの基礎をなすものである。

% ============================================================
% 2. SystemDKの構造
% ============================================================
\section{SystemDKの構造}
SystemDK(System Design Kernel)は,
設計情報を「構造・挙動・制御」の三層に再定義する。
\begin{itemize}
  \item \textbf{構造層 (Structure Layer)}:材料・形状・配線などの物理構造定義
  \item \textbf{挙動層 (Behavior Layer)}:熱・応力・電磁・信頼性特性の解析結果
  \item \textbf{制御層 (Control Layer)}:PIDおよびFSMによる制御論理
\end{itemize}
これらは共通スキーマ(JSON/YAML形式)で接続され,
構造変更が制御設計へ即時反映される。

% ============================================================
% 3. AITL:三層制御構造(PIDを中核とする監督型構成:Final+)
% ============================================================
\section{AITL:三層制御構造}

AITL(Adaptive Intelligent Tri-Layer)は,
SystemDK全体を動的に安定化させるための知的制御構造である。
AITLは,内側の実時間制御から外側の知的監督までを
三層で構成し,それぞれが異なる時間スケールと抽象度で
システムの安定性と設計整合性を維持する。

すなわち,
PID層が物理的安定性を即応的に制御し,
FSM層がその動作モードを監督し,
LLM層が全体設計の整合性と再構成を行うという,
\textbf{多層監督型の閉ループ制御体系}を形成する。

% ------------------------------------------------------------
\subsection{PID層:内側の実時間閉ループ制御}
PID層はAITLの最内層に位置し,
温度,電流,応力,位置,速度などの物理量に対して
閉ループ制御を行う中核制御層である。
制御式は次式で定義される:

\begin{equation}
u(t) = K_P e(t) + K_I \int_{0}^{t} e(\tau)\,d\tau + K_D \frac{de(t)}{dt}
\end{equation}

ここで,$e(t)$ は偏差(目標値と観測値の差),
$K_P$,$K_I$,$K_D$ は比例・積分・微分ゲインである。
PID層はSystemDKに格納された物理応答モデル(例:FEM解析結果)を参照し,
負荷変動・温度変化・外乱に応じてゲインを動的に再設定する。
PIDはAITLにおける唯一の**実時間閉ループ構造**であり,
アクチュエータやロボット機構といった物理系に直接作用する。

% ------------------------------------------------------------
\subsection{FSM層:中間のモード監督層}
FSM(Finite State Machine)層は,
PID層の外側に位置し,その制御モードや状態遷移を監督する。
FSMはPID出力の状態を監視し,
異常や飽和を検出した際には安全モードや回復モードへの切替を実行する。

代表的な状態遷移は次のように表される:

\begin{center}
\texttt{NORMAL → SATURATE → COOLDOWN → NORMAL}
\end{center}

- \texttt{NORMAL}:PIDが安定動作している通常制御状態  
- \texttt{SATURATE}:PID出力が物理上限に近づいた状態  
- \texttt{COOLDOWN}:PID制御を一時抑制し,発熱や応力を緩和して再安定化を待つ  

FSM層はSystemDKスキーマ内で遷移条件を定義し,
制御モードと安全性の一貫性を形式的に保証する。
この層は閉ループ制御ではなく,
\textbf{PIDを監督する中間制御層(supervisory layer)}として機能する。

% ------------------------------------------------------------
\subsection{LLM層:最外の知的整合・再設計層}
LLM(Large Language Model)層はAITLの最外層に位置し,
SystemDK全体の設計整合性と制御最適化を知的に監督する。
ここでのLLMは,単なる自然言語AIではなく,
SystemDKスキーマに格納された構造・解析・制御データを統合的に解析し,
因果関係を理解した上で再構成を行う上位知能層である。

LLM層の主な機能は以下の通りである:

\begin{itemize}
  \item \textbf{異常要因の推論と説明:}  
  PIDやFSMの動作ログを解析し,発熱・振動・遅延などの異常発生源を特定・説明する。  
  例:\texttt{「アクチュエータM2のトルク飽和が熱暴走の要因」}。

  \item \textbf{再設計指令の生成:}  
  構造・制御・解析モデルを統合的に評価し,修正案を生成する。  
  例:\texttt{「Kpを12\%減少,FEM解析条件を再実行,FSM閾値を更新」}。

  \item \textbf{設計スキーマの再構築:}  
  修正案をSystemDKのスキーマ(データ構造)へ反映し,
  制御・解析・構造データを自律的に再同期させる。
\end{itemize}

このようにLLM層は,
PIDおよびFSMが維持する制御安定性の外側で,
システム全体の**知的再設計(intelligent redesign)**を実行する。
すなわち,LLMはSystemDK全体を理解し,
異常時に設計そのものを再構成する「設計監督AI」として機能する。

% ------------------------------------------------------------
\subsection{AITL全体の階層構造と動作原理}
AITLの階層構造は次のように整理される:

\begin{center}
\texttt{[LLM:知的整合・再設計層]}\\
\texttt{     ↑}\\
\texttt{[FSM:モード監督層]}\\
\texttt{     ↑}\\
\texttt{[PID:閉ループ制御層(物理系)]}
\end{center}

PID層が物理安定性を実時間で維持し,
FSM層がモード単位で安全性を監督し,
LLM層がSystemDK全体の設計整合性と再構成を担う。
これら三層は時間スケールと抽象度を異にしながら,
共通のSystemDKスキーマを介して常時同期している。

AITLはこの三層協調構造により,
設計の安定性(PID),
動作の一貫性(FSM),
情報整合と再構成(LLM)を同時に保証する。
さらにAITLは,
BRDK/IPDK/PKGDK/SystemDKの各階層と連携し,
ボードレベルからシステムレベルまで
統合的かつ自律的な制御・再設計を実現する。

\textbf{AITLはしたがって,単なる制御機構ではなく,}
\textbf{「自ら設計を理解し,異常時に再構成する知的アーキテクチャ」}
としてSystemDKの中核を形成する。

% ============================================================
% 4. 統合設計フロー(Final++: Robot/Actuator × Chiplet Integration / 💯)
% ============================================================
\section{統合設計フロー}

SystemDK with AITL Core の設計フローは,
ロボット/アクチュエータの制御設計を起点に,
FPGA/ASIC,チップレット構造,および実時間検証までを
単一スキーマで結合する\textbf{閉ループ型最適化プロセス}である。
PID・FSM・LLMから成るAITLを中核に,
FEMで得た剛性・慣性・熱・応力・ノイズ情報を制御へ反映し,
位置・速度・力の安定性と長期信頼性を同時に保証する。

% --- 設計フロー図(1カラム幅に自動フィット) ---
\begin{figure}[t]
  \centering
  \resizebox{\columnwidth}{!}{%
  \begin{tikzpicture}[
    font=\small,
    flowstep/.style={
      rounded corners, draw, thick, align=center,
      minimum width=38mm, minimum height=7mm
    },
    flowarr/.style={-{Latex[length=3mm]}, thick}
  ]

  % Nodes (top to bottom)
  \node[flowstep] (spec) {① 仕様策定・全体アーキテクチャ\\モジュール選定};
  \node[flowstep, below=6mm of spec] (ctrl) {② 制御系設計\\MATLAB/Simulink(PID, FSM)\\LLM整合 $\rightarrow$ C $\rightarrow$ Verilog};
  \node[flowstep, below=6mm of ctrl] (fpga) {③ FPGA回路設計\\RTL/SDC・固定小数点・I/O定義};
  \node[flowstep, below=6mm of fpga] (asic) {④ ASIC回路設計\\PDK適用/合成・P\&R・CTS};
  \node[flowstep, below=6mm of asic] (struct) {⑤ 構造設計(BRDK/IPDK/PKGDK)\\電源・信号・サーマル経路定義};
  \node[flowstep, below=6mm of struct] (fem) {⑥ FEM/ノイズ解析\\熱・応力・PI/SI/EMI};
  \node[flowstep, below=6mm of fem] (fpga_test) {⑦ FPGA検証(HIL)\\実時間応答・安定性評価};

  % Main arrows
  \draw[flowarr] (spec) -- (ctrl);
  \draw[flowarr] (ctrl) -- (fpga);
  \draw[flowarr] (fpga) -- (asic);
  \draw[flowarr] (asic) -- (struct);
  \draw[flowarr] (struct) -- (fem);
  \draw[flowarr] (fem) -- (fpga_test);

  % Feedbacks (dashed, short)
  \draw[-{Latex[length=2mm]}, dashed]
    (fem.west) .. controls +(-18mm,0mm) and +(-18mm,0mm) .. (ctrl.west)
    node[midway, left, align=right, xshift=-1mm] {\footnotesize 物理定数\\反映};

  \draw[-{Latex[length=2mm]}, dashed]
    (fpga_test.east) .. controls +(18mm,0mm) and +(18mm,0mm) .. (fpga.east)
    node[midway, right, align=left, xshift=1mm] {\footnotesize 応答補正};

  \end{tikzpicture}}
  \caption{制御主導の統合フロー(1カラム幅フィット)}
  \label{fig:design_flow}
\end{figure}

% ------------------------------------------------------------
\subsection{1) 仕様策定}
対象のロボット/アクチュエータ仕様(位置分解能,応答時間,力/トルク,剛性,
熱上限,ノイズ制約)を定義し,SystemDKスキーマ最上位ノードに登録する。
以降の制御・構造・解析と双方向リンクされ,変更は即時伝播する。

% ------------------------------------------------------------
\subsection{2) 制御系設計(MATLAB/Simulink,FSM,LLM統合)}
AITLに基づき制御モデルを設計する。
PID層は位置・速度・力の閉ループ制御,FSM層はモード(NORMAL/HOLD/RETURN/FAULT)
監督,LLM層は仕様と制御モデルの論理整合検証と修正指令を担当する。
設計後,Simulink CoderでCを生成し,ツールチェーンでVerilogへ変換して
ハードウェア実装互換を確保する。

% ------------------------------------------------------------
\subsection{3) FPGA回路設計・実時間制御検証}
生成VerilogをFPGA実装し,センサ入出力とアクチュエータを閉ループ接続する。
HIL検証で振動応答・追従誤差・過渡安定性を計測し,
PIDゲイン・FSM閾値を更新,データはSystemDKへ格納する。

% ------------------------------------------------------------
\subsection{4) 仮構造定義(Pre-BRDK/IPDK/PKGDK)}
FPGA検証結果に基づき,実装可能な仮構造(BRDK/IPDK/PKGDK)を定義する。
初期の熱経路・剛性・配線制約をモデル化し,チップレット解析の前提を整える。

% ------------------------------------------------------------
\subsection{5) FEM/ノイズ解析(Chiplet-level Physical Analysis)}
ASICダイ,インターポーザ(2.5D/3D: TSV, $\mu$-bump),PKG配線を対象に,
熱伝導・応力集中・機械共振(FEM)と,SI/PI/EMI(ノイズ)を解析する。
結果はSystemDKへ統合され,PIDの時定数やFSM閾値へ反映される。

% ------------------------------------------------------------
\subsection{6) 制御・構造・解析の再設計ループ}
制御応答と物理解析の整合が得られるまで,
PIDゲイン/FSM閾値/構造パラメータを反復最適化する。
LLM層が差分を検知し,SystemDKスキーマへ自動反映する。

% ------------------------------------------------------------
\subsection{7) 構造確定設計(BRDK/IPDK/PKGDK Fix)}
収束結果から最終のボード/インターポーザ/パッケージ設計を確定し,
再解析・再検証用としてSystemDKに登録する。

% ------------------------------------------------------------
\subsection{8) 最終FEM/FPGA再検証}
確定構造でFEMを再実行し,剛性・応力・共振をFPGA制御モデルへ再入力する。
PIDは安定性,FSMはモード一貫性を確認し,閉ループ収束を検証する。

% ------------------------------------------------------------
\subsection{9) SystemDK整合収束判定}
AITLの三層で最終整合を判定する。PIDは物理安定性,FSMは安全遷移,
LLMは設計論理の整合性を確認し,全条件成立時にASIC段階へ進む。

% ------------------------------------------------------------
\subsection{10) ASIC設計}
PDKを適用し,RTL設計,論理合成,配置配線(P\&R),CTS,タイミング検証を経て
GDSを生成する。設計データはSystemDKへ統合され,解析・制御情報と整合される。

% ------------------------------------------------------------
\subsection{11) 製造・ウエハテスト}
マスク作成~製造後,ウエハテストで電気/応答特性を評価し,
SystemDKへ登録してモデル再同定に用いる。

% ------------------------------------------------------------
\subsection{12) BR/IP/PKG製造・組立}
ASIC実装をBRDK/IPDK/PKGDKで実施し,熱・応力・ノイズ再評価を行う。
製造・組立情報もSystemDKで一元管理する。

% ------------------------------------------------------------
\subsection{13) SystemDK最終検証(AITLによる自律安定制御確認)}
統合データを用い,AITL三層で最終検証を実施する:
PIDは実時間安定化,FSMは安全遷移監督,LLMは形式整合検証と再同定を担当する。
これにより,本体系は\textbf{構造設計・制御理論・チップレット物理特性を統合最適化する自律型設計制御基盤}として確立される。
        % ← 日本語や矢印は listings に入れないこと
% ============================================================
% 5. 信頼性統合
% ============================================================
\section{信頼性統合}
PID層に信頼性モデル(NBTI, HCI, 熱劣化など)を組み込み,
FSM層が閾値を監視してモード切替を行う。
これにより,物理信頼性と制御安定性を同一ループで保証する。

% ============================================================
% 6. 考察
% ============================================================
\section{考察}

SystemDK with AITL Core は,構造・挙動・制御の三要素を
統一的に扱うことにより,従来の階層分断型設計が抱えていた
情報不整合と制御不安定性の問題を根本的に解消する枠組みを提供する。
その特徴は,単なる設計支援ツールや解析環境ではなく,
設計そのものを「制御可能な動的システム」として再定義した点にある。

% ------------------------------------------------------------
\subsection{(1) 階層統合による設計一貫性の確立}
SystemDKは,構造層・挙動層・制御層を共通スキーマで接続し,
構造変更が解析および制御設計に即時反映される動的更新機構を備える。
これにより,構造設計(BRDK/IPDK/PKGDK)と
制御設計(AITL層)が常に整合した状態で維持される。
また,構造—解析—制御の各要素が同一データ空間で管理されるため,
設計段階での局所最適や手戻りを最小化し,
設計全体を「整合性を保つ閉ループ系」として扱うことが可能になる。
すなわちSystemDKは,EDAやCAEを越えた
「設計知識基盤(Design Knowledge Kernel)」としての機能を持つ。

% ------------------------------------------------------------
\subsection{(2) AITLによる動的安定化機構}
AITL(Adaptive Intelligent Tri-Layer)は,
PID,FSM,LLMの三層制御構造によって,
SystemDK上の設計情報と物理挙動を動的に安定化させる中核である。
PID層は,FEM解析やノイズシミュレーションから得られた
時定数・応答データを基にリアルタイム制御を行う。
FSM層は,PID動作の状態遷移を監督し,
制御モードの安定性と安全性を確保する。
LLM層は,設計構造・解析モデル・制御条件の整合を常時検証し,
不整合検出時にはパラメータ再同定を指示する。
この三層協調により,SystemDKは単なる静的データベースではなく,
「制御理論を内包した自律設計系」として機能する。

% ------------------------------------------------------------
\subsection{(3) 信頼性統合による長期安定性の保証}
本体系では,物理的劣化(NBTI, HCI, TDDB, 熱疲労)を
制御ループ内の可観測変数として扱い,
PID層が補償,FSM層が閾値監視,LLM層が長期整合検証を行う。
従来の「設計後に行う信頼性解析」を
制御設計段階に統合することにより,
時間依存的な劣化進行を自律的に補償可能とした。
この「信頼性—制御統合」は,
SystemDK内で寿命・安定性・安全性を動的最適化する
新しい設計理論体系を形成している。

% ------------------------------------------------------------
\subsection{(4) 制御主導設計体系としての意義}
SystemDK with AITL Core は,
教育・運用・経営的要素を排除した純粋工学的アーキテクチャとして構築される。
その基本思想は,設計を人為的意思決定から切り離し,
制御理論(PID/FSM/LLM)によって
設計空間そのものを安定化させることである。
PID層が物理制御を担い,
FSM層が設計モードの一貫性を保証し,
LLM層が論理整合を保つという三階層制御構造により,
設計自体が制御対象となる「自律的設計制御系(Autonomous Design Control System)」が成立する。

% ------------------------------------------------------------
\subsection{(5) 今後の展開}
今後は,本体系を半導体・メカトロ・制御・材料工学など
異分野の設計プロセスへ適用し,
SystemDKスキーマを共通基盤化することが課題となる。
特に,AITL層におけるFSM遷移則の自動最適化と
LLM層によるゲイン再同定のリアルタイム化を進めることで,
自己修復型・自己進化型の設計システムへの発展が期待される。

本考察を通じて,
SystemDK with AITL Core は,
設計・信頼性・制御を統合する純粋工学的理論体系の中核を成し,
自律設計理論の基礎を構築するものである。

% ============================================================
% 7. 結論
% ============================================================
\section{結論}
本論文では,設計・信頼性・制御を統合する
\textbf{SystemDK with AITL Core}を提案した。
PID・FSM・LLMの三層制御構造により,
物理的安定性と論理的整合性を同時に保証できることを示した。
本体系は,自律的設計理論の中核をなす工学的基盤である。


% --- appendix ---
% ============================================================
% Appendix A: FPGA実装インタフェース仕様(AITL結合 / Final++ 💯)
% ============================================================
\appendix
\section*{付録A:AITL–FPGAインタフェース仕様}

AITL三層制御のうち,PID層およびFSM層はFPGA上に実装され,
SystemDKスキーマを通じてLLM層および解析系と同期する。
SystemDKはVerilog自動生成時に以下のインタフェース構造を定義する。

\begin{itemize}
  \item \textbf{Clock Domain:} 100 MHz/200 MHz選択式。全制御ロジックはシステムクロック信号 \texttt{clk\_sys} に同期。
  \item \textbf{Latency:} PIDループは1クロック周期遅延(10 ns @ 100 MHz)。
        FSM監督信号は非同期割込み(event-driven interrupt)により伝達。
  \item \textbf{Memory Map:} 制御ゲイン,FSM閾値,LLM指令値をそれぞれ
        \texttt{0x0000–0x00FF},\texttt{0x0100–0x01FF},\texttt{0x0200–0x02FF}に割当。
  \item \textbf{Communication:} AITL–FPGA間通信はAXI4-Liteバスを介して実装。
        SystemDKホストが制御・更新命令を送信し,FPGAは即時反映する。
\end{itemize}

この構成により,
LLM層が生成する再設計指令はFPGA上のPID/FSMパラメータに
リアルタイムで反映され,
SystemDK全体の閉ループ更新が動的に成立する。
すなわち,FPGAはAITL三層を具現化する「物理的制御核(physical control kernel)」として機能する。

---

% ============================================================
% Appendix B: SystemDKスキーマ定義例(JSON形式 / Final++ 💯)
% ============================================================
\section*{付録B:SystemDKスキーマ定義例(JSON形式)}

以下にSystemDKのコアスキーマ定義例を示す。
全ての設計・解析・制御データはこの階層構造上で同期管理される。
LLM層はこのスキーマを直接参照し,
設計・解析・制御間の依存関係を動的に推論する。

\begin{verbatim}
{
  "SystemDK": {
    "Spec": {
      "target": "Actuator_M2",
      "resolution": "0.01 mm",
      "max_force": "40 N"
    },
    "Control": {
      "PID": {"Kp": 2.4, "Ki": 0.05, "Kd": 0.01},
      "FSM": {
        "states": ["NORMAL", "HOLD", "FAULT"],
        "thresholds": {"temp_warn": 85, "temp_crit": 95}
      },
      "LLM": {
        "model": "GPT-5-engineering",
        "mode": "analysis"
      }
    },
    "Structure": {
      "BRDK": {"material": "Cu", "thickness": "1.2 mm"},
      "IPDK": {"interposer": "Si", "vias": "TSV-3D"}
    },
    "Analysis": {
      "FEM": {"mesh": 2e6, "max_stress": "180 MPa"},
      "Noise": {"PI": "stable", "SI": "good"}
    }
  }
}
\end{verbatim}

SystemDKはこのスキーマを実行時に自律更新し,
設計変更・解析結果・制御指令を常時整合化する。
これにより,設計情報が静的ファイルではなく,
\textbf{「動的に制御可能な知識グラフ(Dynamic Knowledge Graph)」}として扱われる。

---

% ============================================================
% Appendix C: LLM層による再設計指令生成プロセス(擬似コード / Final++ 💯)
% ============================================================
\section*{付録C:LLM層による再設計指令生成プロセス(擬似コード)}

LLM層はSystemDKスキーマ全体を監視し,
不整合が発生した場合に再設計指令を生成・反映する。
以下にその抽象化されたプロセスを示す。

\begin{verbatim}
procedure LLM_Reconfiguration(SystemDK):
    Input: Current SystemDK schema S
    Δ = AnalyzeConsistency(S)             // 整合性解析
    if Δ ≠ 0 then
        cause     = Diagnose(Δ)           // 不整合要因推定
        proposal  = GenerateRedesign(cause) // 再設計案生成
        UpdateSchema(S, proposal)         // スキーマ更新
        Notify(PID, FSM)                  // FPGAパラメータ即時更新
        Re-run Verification()             // 再検証実行
    else
        MaintainStableState()             // 安定状態維持
end procedure
\end{verbatim}

このプロセスにより,
SystemDKは設計・解析・制御データの整合を常時確認し,
不整合が検出されると即座に再構成を行う。
これが本研究の中核である
\textbf{「設計そのものを制御対象とする自律再設計ループ」}
を構成する。

---

\noindent
以上の3付録により,
SystemDK with AITL Core の全階層構造,
ハードウェア実装仕様,
および知的再設計アルゴリズムの全体像が
学術的・実装的に完全定義された。


% --- references(使うときだけ有効化) ---
% \bibliographystyle{IEEEtran}
% \bibliography{references}

% --- biography ---
% ============================================================
% 著者略歴
% ============================================================
\section*{著者略歴}
\textbf{三溝 真一}(Shinichi Samizo)は、信州大学大学院 工学系研究科 電気電子工学専攻にて修士号を取得した。  
その後、セイコーエプソン株式会社に勤務し、半導体ロジック/メモリ/高耐圧インテグレーション、そして、インクジェット薄膜ピエゾアクチュエータ及びPrecisionCoreプリントヘッドの製品化に従事した。  
現在は独立系半導体研究者として、プロセス/デバイス教育、メモリアーキテクチャ、AIシステム統合などに取り組んでいる。  
連絡先: \href{mailto:shin3t72@gmail.com}{shin3t72@gmail.com}.


\end{CJK}
\end{document}
