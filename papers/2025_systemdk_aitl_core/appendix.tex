% ============================================================
% Appendix A: FPGA実装インタフェース仕様(AITL結合 / 💯修正版)
% ============================================================
\appendix
\section*{付録A:AITL–FPGAインタフェース仕様}

AITL三層制御のうち,PID層およびFSM層はFPGA上に実装され,
SystemDKスキーマを通じてLLM層および解析系と同期する。
SystemDKはVerilog自動生成時に以下のインタフェース構造を定義する。

\begin{itemize}
  \setlength{\itemsep}{1pt}
  \item \textbf{Clock Domain:} 100\,MHz/200\,MHz選択式。全制御ロジックは
        システムクロック信号 \texttt{clk\_sys} に同期。
  \item \textbf{Latency:} PIDループは1クロック周期遅延(10\,ns @ 100\,MHz)。
        FSM監督信号は非同期割込み(event-driven interrupt)で伝達。
  \item \textbf{Memory Map:} 制御ゲイン・FSM閾値・LLM指令値を
        \texttt{0x0000--0x00FF},
        \texttt{0x0100--0x01FF},
        \texttt{0x0200--0x02FF} に割当。
  \item \textbf{Communication:} AITL--FPGA間通信は
        AXI4\-/Lite バスを介して実装。
        SystemDKホストが制御・更新命令を送信し,
        FPGAは即時反映する。
\end{itemize}

この構成により,
LLM層が生成する再設計指令はFPGA上のPID/FSMパラメータに
リアルタイムで反映され,
SystemDK全体の閉ループ更新が動的に成立する。
FPGAはAITL三層を具現化する
「物理的制御核(physical control kernel)」として機能する。

% ============================================================
% Appendix B: SystemDKスキーマ定義例(JSON形式 / 💯修正版)
% ============================================================
\section*{付録B:SystemDKスキーマ定義例(JSON形式)}

以下にSystemDKのコアスキーマ定義例を示す。
全設計・解析・制御データはこの階層構造上で同期管理される。
LLM層はこのスキーマを直接参照し,
設計・解析・制御間の依存関係を動的に推論する。

\lstset{
  language=json,
  breaklines=true,
  breakatwhitespace=false,
  postbreak=\mbox{\textcolor{gray}{$\hookrightarrow$}\space},
  basicstyle=\ttfamily\scriptsize,
  backgroundcolor=\color{lightgray},
  frame=single,
  xleftmargin=2pt,
  xrightmargin=2pt
}
\begin{lstlisting}[caption={SystemDKスキーマ例},label={lst:sdk_schema}]
{
  "SystemDK": {
    "Spec": {
      "target": "Actuator_M2",
      "resolution": "0.01 mm",
      "max_force": "40 N"
    },
    "Control": {
      "PID": {"Kp": 2.4, "Ki": 0.05, "Kd": 0.01},
      "FSM": {
        "states": ["NORMAL","HOLD","FAULT"],
        "thresholds": {"temp_warn": 85, "temp_crit": 95}
      },
      "LLM": {"model": "GPT-5-engineering","mode": "analysis"}
    },
    "Structure": {
      "BRDK": {"material": "Cu","thickness": "1.2 mm"},
      "IPDK": {"interposer": "Si","vias": "TSV-3D"}
    },
    "Analysis": {
      "FEM": {"mesh": 2e6,"max_stress": "180 MPa"},
      "Noise": {"PI": "stable","SI": "good"}
    }
  }
}
\end{lstlisting}

SystemDKはこのスキーマを実行時に自律更新し,
設計変更・解析結果・制御指令を常時整合化する。
これにより,設計情報が静的ファイルではなく,
\textbf{「動的に制御可能な知識グラフ
(Dynamic Knowledge Graph)」}として扱われる。

% ============================================================
% Appendix C: LLM層による再設計指令生成プロセス(擬似コード / 💯修正版)
% ============================================================
\section*{付録C:LLM層による再設計指令生成プロセス(擬似コード)}

LLM層はSystemDKスキーマ全体を監視し,
不整合が発生した場合に再設計指令を生成・反映する。
以下にその抽象化されたプロセスを示す。

\lstset{
  language=Python,
  breaklines=true,
  postbreak=\mbox{\textcolor{gray}{$\hookrightarrow$}\space},
  basicstyle=\ttfamily\scriptsize,
  backgroundcolor=\color{lightgray},
  frame=single,
  xleftmargin=2pt,
  xrightmargin=2pt
}
\begin{lstlisting}[caption={LLM層の再設計指令生成プロセス},label={lst:llm_proc}]
procedure LLM_Reconfiguration(SystemDK):
    Input: Current SystemDK schema S
    Δ = AnalyzeConsistency(S)             // 整合性解析
    if Δ != 0:
        cause    = Diagnose(Δ)            // 不整合要因推定
        proposal = GenerateRedesign(cause) // 再設計案生成
        UpdateSchema(S, proposal)         // スキーマ更新
        Notify(PID, FSM)                  // FPGAパラメータ更新
        ReRunVerification()               // 再検証実行
    else:
        MaintainStableState()             // 安定状態維持
end procedure
\end{lstlisting}

このプロセスにより,
SystemDKは設計・解析・制御データの整合を常時確認し,
不整合が検出されると即座に再構成を行う。
これが本研究の中核である
\textbf{「設計そのものを制御対象とする自律再設計ループ」}
を構成する。

---

\noindent
以上3付録により,
SystemDK with AITL Core の全階層構造,
ハードウェア実装仕様,
および知的再設計アルゴリズムの全体像が
学術的・実装的に完全定義された。
