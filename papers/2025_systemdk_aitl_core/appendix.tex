% ============================================================
% Appendix A: FPGA実装インタフェース仕様(AITL結合)
% ============================================================
\appendix
\section*{付録A:AITL-FPGAインタフェース仕様}

AITL三層制御のうち,PID層とFSM層はFPGA上に実装される。
SystemDKはVerilog生成時に以下のインタフェース構造を定義する。

\begin{itemize}
  \item \textbf{Clock domain:} 100 MHz / 200 MHz selectable,クロック同期信号 \texttt{clk\_sys} に統一。
  \item \textbf{Latency:} PID制御ループ1周期遅延(10 ns @ 100 MHz)。FSM監督信号は非同期イベント割込みで伝達。
  \item \textbf{Memory map:} 制御ゲイン,FSM閾値,LLM指令値をそれぞれ \texttt{0x0000–0x00FF}, \texttt{0x0100–0x01FF}, \texttt{0x0200–0x02FF} に割当。
  \item \textbf{Communication:} AITL–FPGA間はAXI4-Liteバスで接続し,SystemDKがホスト側から設定更新。
\end{itemize}

この構造により,AITL層の再設計指令(LLM層生成)はFPGA上のPID/FSMパラメータへ即時反映され,
SystemDK全体の閉ループ更新がリアルタイムに成立する。

---

% ============================================================
% Appendix B: SystemDKスキーマ例(JSON形式)
% ============================================================
\section*{付録B:SystemDKスキーマ例}

以下にSystemDKのコアスキーマ定義を示す。
すべての設計・解析・制御データはこのJSON構造上で同期される。

\begin{verbatim}
{
  "SystemDK": {
    "Spec": {
      "target": "Actuator_M2",
      "resolution": "0.01mm",
      "max_force": "40N"
    },
    "Control": {
      "PID": {"Kp": 2.4, "Ki": 0.05, "Kd": 0.01},
      "FSM": {"states": ["NORMAL","HOLD","FAULT"],
              "thresholds": {"temp_warn": 85, "temp_crit": 95}},
      "LLM": {"model": "GPT-5-engineering", "mode": "analysis"}
    },
    "Structure": {
      "BRDK": {"material": "Cu", "thickness": "1.2mm"},
      "IPDK": {"interposer": "Si", "vias": "TSV-3D"}
    },
    "Analysis": {
      "FEM": {"mesh": 2e6, "max_stress": "180MPa"},
      "Noise": {"PI": "stable", "SI": "good"}
    }
  }
}
\end{verbatim}

このスキーマはSystemDK内部で動的に更新され,
LLM層が制御・構造・解析間の因果整合性を監視・再構築する。

---

% ============================================================
% Appendix C: LLM再設計プロセス擬似コード
% ============================================================
\section*{付録C:LLM層による再設計指令生成プロセス(擬似コード)}

\begin{verbatim}
procedure LLM_Reconfiguration(SystemDK):
    Input: Current SystemDK schema S
    Δ = AnalyzeConsistency(S)
    if Δ != 0 then
        cause = Diagnose(Δ)
        proposal = GenerateRedesign(cause)
        UpdateSchema(S, proposal)
        Notify(PID, FSM)    // FPGA parameters update
        Re-run Verification()
    else
        MaintainStableState()
end procedure
\end{verbatim}

本プロセスにより,
SystemDKは設計・解析・制御の全データ整合を常時確認し,
不整合検出時には自動的に再構成を行う。
これが本研究の中核である
「\textbf{設計そのものを制御対象とする自律再設計ループ}」
を形成する。

---

\noindent
以上3付録により,
SystemDK with AITL Core の全階層構造・実装仕様・知的再設計アルゴリズムが
完全に定義された。
