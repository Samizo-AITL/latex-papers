% ============================================================
% 2. SystemDKの構造(統合設計基盤アーキテクチャ:Final 💯採録水準)
% ============================================================
\section{SystemDKの構造}

SystemDK(System Design Kernel)は,
設計・解析・制御・信頼性評価を統一的に結合するための
中核アーキテクチャである。
従来のように個別ツール間で情報が分断されるのではなく,
SystemDKはすべての工程を共通データスキーマ上で接続し,
設計情報の一貫性・整合性・再利用性を保証する。

本章では,SystemDKを構成する主要要素を
五つの層として整理し,
それぞれの機能と相互連携を明確に定義する。

% ------------------------------------------------------------
\subsection{(1) スキーマ管理層(Schema Management Layer)}
SystemDKの中心に位置する層であり,
設計に関わる全情報(仕様,制御パラメータ,
回路構成,構造モデル,解析条件)を
一元的に管理する。
このスキーマは「設計の設計図」に相当し,
各要素をノードとリンクで表現した
階層的データ構造として定義される。
\footnote{実装上はJSONやYAMLなどの汎用形式を採用するが,
利用者はこれを意識することなく,
GUI上で統合的に操作できる。}

設計変更が生じると,
対応ノードの属性値が自動更新され,
制御設計やFEM解析に即時反映される。
これにより,全工程が同期した
閉ループ設計環境が維持される。

% ------------------------------------------------------------
\subsection{(2) データ接続層(Data Connectivity Layer)}
設計ツール間のデータ伝達と整合を担う層である。
MATLAB/Simulink(制御系設計),EDAツール(回路設計),
FEM解析ツール(熱・応力解析),および
SystemDKスキーマがAPI群を介して相互接続される。
各ツールの出力は変換モジュールを通じて統合され,
制御パラメータと解析結果が双方向に参照可能となる。
これにより,
「制御が解析に影響し,解析が再び制御を更新する」
という**情報循環ループ(information feedback loop)**が形成され,
設計プロセス全体が動的整合を維持する。

% ------------------------------------------------------------
\subsection{(3) 階層設計統合層(Hierarchical Integration Layer)}
SystemDKは,ボード(BRDK),インターポーザ(IPDK),
パッケージ(PKGDK),およびシステム(SystemDK)の
各設計階層を共通スキーマで統合管理する。
電気・熱・応力・信号経路などの物理依存関係を
階層的にリンクすることで,
局所的な設計変更(例:配線長や材料特性の修正)が,
上位階層のFEM解析・ノイズ評価・制御条件へ
自動的に伝搬する。
この構造により,
チップレットからシステム全体まで一貫した
マルチスケール最適設計が可能となる。

% ------------------------------------------------------------
\subsection{(4) 制御統合層(AITL Integration Layer)}
AITL(Adaptive Intelligent Tri-Layer)は,
SystemDKに組み込まれた知的制御フレームワークであり,
PID,FSM,LLMの三層から構成される。

\begin{itemize}
  \item \textbf{PID層}: 実時間制御を担い,温度・応力・振動・電流などの物理量を安定化する。
  \item \textbf{FSM層}: 動作モードと状態遷移を監督し,安全領域内での一貫運転を保証する。
  \item \textbf{LLM層}: 大規模言語モデルを活用し,設計全体の論理整合を解析して再設計指令を生成する。
\end{itemize}

LLM層は単なる自然言語処理AIではなく,
SystemDKスキーマ全体の依存関係を解析し,
設計・制御・解析間の因果関係を推論する
\textbf{上位知能層(meta-intelligence layer)}として機能する。
たとえば,アクチュエータの発熱やノイズ上昇が
PIDおよびFSMの補償範囲を超過した場合,
LLM層は構造および制御パラメータを同時に再設計し,
SystemDKスキーマ全体を再構成することで,
**自律的再最適化(self-reconfiguration)**を実現する。

% ------------------------------------------------------------
\subsection{(5) 信頼性統合層(Reliability Coupling Layer)}
信頼性解析と制御設計を結合する層である。
FEMおよびノイズ解析結果(温度分布・応力集中・電圧降下)を
制御系に直接フィードバックし,
制御パラメータを信頼性指標に基づいて補正する。
さらに,NBTI, HCI, TDDB などの劣化モデルを組み込み,
時間依存の性能変動を予測した上で,
AITLによる補償を動的に適用する。
これにより,SystemDKは時間軸上でも
信頼性を動的に維持する閉ループ設計制御を実現する。

% ------------------------------------------------------------
\subsection*{まとめ}
以上の五層により,SystemDKは,
仕様から実装・解析・制御・信頼性評価に至るまでを
単一スキーマ上で統合し,
AITLによる知的制御機構を通じて
自律的に安定化・再最適化を行う。
すなわち,SystemDKは
\textbf{「設計そのものを制御対象とみなし,
知能的に再構成する自律設計基盤」}
として定義される。
