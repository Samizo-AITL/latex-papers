% ============================================================
% 2. SystemDKの構造(統合設計基盤アーキテクチャ:Final+)
% ============================================================
\section{SystemDKの構造}

SystemDK(System Design Kernel)は,
設計・解析・制御・信頼性評価を統一的に接続するための
中核アーキテクチャである。
従来のように個別ツール間で情報を分断するのではなく,
SystemDKはすべての工程を共通データ構造上で結合し,
設計情報の一貫性・整合性・再利用性を保証する。

本章では,SystemDKを構成する主要要素を
5つの層として整理する。

% ------------------------------------------------------------
\subsection{(1) スキーマ管理層(Schema Management Layer)}
SystemDKの中核を成す層であり,
設計に関わるすべての情報(仕様,制御パラメータ,
回路構成,構造モデル,解析条件)を
一元的に管理する。
このスキーマは,いわば「設計の設計図」であり,
各要素をノードとリンクで構成した
階層的データ構造として定義される。
\footnote{実装上はJSONやYAMLといった汎用形式を採用するが,
利用者はこれを意識する必要がなく,
GUI上で統合的に操作可能である。}

設計変更が発生すると,
対応するノードの属性値が自動的に更新され,
制御設計やFEM解析へ即時反映される。
これにより,手動での再設定を必要とせず,
全工程が常時同期した閉ループ構造を保つ。

% ------------------------------------------------------------
\subsection{(2) データ接続層(Data Connectivity Layer)}
設計ツール間のデータ連携を担う層である。
MATLAB/Simulink(制御系設計),EDAツール(回路設計),
FEM解析ツール(熱・応力解析),および
SystemDKスキーマがAPI群で相互接続される。
各ツールの出力は変換モジュールを介してSystemDK上に統合され,
制御パラメータと解析結果が双方向に参照される。
この構造により,
「制御が解析に影響し,解析が再び制御を更新する」
という**情報循環ループ**が形成される。

% ------------------------------------------------------------
\subsection{(3) 階層設計統合層(Hierarchical Integration Layer)}
SystemDKは,ボード(BRDK),インターポーザ(IPDK),
パッケージ(PKGDK),システム(SystemDK)の
各設計階層を共通スキーマで統合管理する。
電気・熱・応力・信号経路などの依存関係を
階層的にリンクすることで,
ある階層での設計変更(例:配線長や材料特性の修正)が,
上位階層のFEM解析・ノイズ評価・制御条件へ自動伝搬する。
この構造により,
チップレットからシステム全体までを貫く最適設計が可能となる。

% ------------------------------------------------------------
\subsection{(4) 制御統合層(AITL Integration Layer)}
AITL(Adaptive Intelligent Tri-Layer)は,
SystemDKに組み込まれた知的制御フレームワークであり,
PID,FSM,LLMの三層で構成される。

\begin{itemize}
  \item \textbf{PID層}: 実時間制御を担い,温度・応力・振動・電流などの物理量を安定化する。
  \item \textbf{FSM層}: 動作モードと状態遷移を監督し,安全領域内での運転を保証する。
  \item \textbf{LLM層}: 大規模言語モデルに基づき,システム全体の整合性を解析し,再設計指令を生成する。
\end{itemize}

特にLLM層は,単なる自然言語処理AIではなく,
SystemDKスキーマ全体を理解し,
設計・制御・解析間の因果関係を推論する「上位知能層」として動作する。
たとえば,アクチュエータの発熱やノイズ上昇が
PIDおよびFSMの補償範囲を超えた場合,
LLM層は構造・制御パラメータを同時に再設計し,
SystemDKスキーマ全体を再構成することで
**自律的再最適化(self-reconfiguration)**を実現する。

% ------------------------------------------------------------
\subsection{(5) 信頼性統合層(Reliability Coupling Layer)}
信頼性解析と制御設計を統合する層である。
FEMやノイズ解析による結果(温度分布・応力集中・電圧降下)を
直接制御系にフィードバックし,
制御パラメータを信頼性指標に基づいて補正する。
さらに,NBTI, HCI, TDDB などの劣化モデルを組み込み,
長期的な性能劣化を予測した上で,
AITL制御により補償を行う。
これにより,信頼性を時間軸上でも動的に維持できる。

% ------------------------------------------------------------
\subsection*{まとめ}
以上の5層により,SystemDKは
仕様から実装・解析・制御・信頼性評価に至るまでを
単一のスキーマ上で統合し,
AITLによる知的制御機構を通じて
自律的に安定化・再最適化を行う。
すなわちSystemDKは,
\textbf{「設計そのものを制御対象とみなし,知能的に再構成する自律設計基盤」}
として定義される。

% ------------------------------------------------------------
\begin{figure*}[t]
  \centering
  \begin{tikzpicture}[
    font=\small,
    box/.style={rounded corners, draw, thick, align=center, minimum width=33mm, minimum height=7mm},
    group/.style={rounded corners, draw, dashed, inner sep=2mm},
    arr/.style={-{Latex[length=3mm]}, thick}
  ]

  % Physical stack group (left)
  \node[box] (brdk) {BRDK\\(基板)};
  \node[box, above=5mm of brdk] (ipdk) {IPDK\\(インターポーザ)};
  \node[box, above=5mm of ipdk] (pkgdk) {PKGDK\\(パッケージ)};

  \node[group, fit=(brdk)(ipdk)(pkgdk), label={[align=left]above:\textbf{物理/実装階層}}] (phys) {};

  % SystemDK kernel (center)
  \node[box, right=13mm of ipdk, minimum width=40mm] (sdk) {\textbf{SystemDK Kernel}\\(統合スキーマ/設計知識基盤)};

  % Analysis (top right)
  \node[box, above=8mm of sdk, minimum width=40mm] (analysis) {解析モデル\\FEM/PI・SI・EMI/信頼性};

  % Control (bottom right)
  \node[box, below=8mm of sdk, minimum width=40mm] (control) {制御ロジック\\PID(閉ループ)/FSM(モード)/LLM(整合)};

  % Arrows between physical and SystemDK
  \draw[arr] (pkgdk.east) -- ($(sdk.west)+(0,4mm)$);
  \draw[arr] (ipdk.east) -- (sdk.west);
  \draw[arr] (brdk.east) -- ($(sdk.west)+(0,-4mm)$);

  % Arrows to analysis/control
  \draw[arr] (sdk.north) -- (analysis.south);
  \draw[arr] (sdk.south) -- (control.north);

  % Feedback from analysis/control
  \draw[arr] (analysis.west) .. controls +(-10mm,6mm) and +(10mm,6mm) .. (sdk.west);
  \draw[arr] (control.west) .. controls +(-10mm,-6mm) and +(10mm,-6mm) .. (sdk.west);

  % Legend (right small)
  \node[align=left, anchor=west, right=8mm of analysis.east] (legend) {%
    \textbf{連携の意味}\\
    \(\triangleright\) 物理設計 \(\to\) SystemDK:構造・幾何・配線\\
    \(\triangleright\) SystemDK \(\leftrightarrow\) 解析:FEM/ノイズ/信頼性\\
    \(\triangleright\) SystemDK \(\leftrightarrow\) 制御:PID/FSM/LLM
  };

  \end{tikzpicture}
  \caption{SystemDK with AITL の統合階層:BRDK/IPDK/PKGDK と設計知識基盤(SystemDK)、解析・制御の結合}
  \label{fig:systemdk_stack}
\end{figure*}
