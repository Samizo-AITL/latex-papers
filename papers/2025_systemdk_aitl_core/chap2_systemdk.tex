% ============================================================
% 2. SystemDKの構造(統合設計基盤アーキテクチャ)
% ============================================================
\section{SystemDKの構造}

SystemDK(System Design Kernel)は,
設計・解析・制御・信頼性評価を統一的に接続するための
中核アーキテクチャである。
従来のように個別ツール間で情報を分断するのではなく,
SystemDKはすべての工程を共通データ構造上で結合し,
設計情報の一貫性・整合性・再利用性を保証する。

本章では,SystemDKを構成する主要要素を5つの層として整理する。

% ------------------------------------------------------------
\subsection{(1) スキーマ管理層(Schema Management Layer)}
SystemDKの中心となる層であり,
設計に関わるすべてのデータ(仕様,制御パラメータ,
回路構成,構造モデル,解析条件)を
一元的に管理する。
このスキーマは,いわば「設計の設計図」であり,
各要素をノードとリンクで関連付けた
階層的データ構造として構成される。
\footnote{実装上はJSONやYAMLといった汎用形式を用いて,
ツール間で読み書きできるようにしているが,
利用者はデータ形式を意識せず,
同一GUI上で設計全体を操作できる。}

設計変更が発生すると,
対応するノードの属性値が自動的に更新され,
制御設計やFEM解析へ即時反映される。
これにより,手動での再設定を必要とせず,
全工程が常時同期した状態を保つ。

% ------------------------------------------------------------
\subsection{(2) データ接続層(Data Connectivity Layer)}
設計ツール間のデータ連携を担う層である。
MATLAB/Simulink(制御系),EDAツール(回路設計),
FEM解析ツール(熱・応力解析),および
SystemDK自身のデータスキーマをAPI群で接続する。
各ツールが生成した結果は変換を介してSystemDK上に統合され,
制御パラメータと物理解析結果を相互に参照できる。
この構造により,
「制御が解析に影響し,解析が再び制御を更新する」
という双方向の閉ループ連携が実現される。

% ------------------------------------------------------------
\subsection{(3) 階層設計統合層(Hierarchical Integration Layer)}
SystemDKは,ボード(BRDK),インターポーザ(IPDK),
パッケージ(PKGDK),システム(SystemDK)の
各階層を共通スキーマで統合管理する。
電気・熱・応力・信号経路などの依存関係を
階層的にリンクすることで,
ある階層での設計変更(例:配線長や材料特性の修正)が
上位のFEM解析・ノイズ評価・制御条件に自動伝搬する。
この階層統合構造によって,
システム全体を通した最適設計が可能になる。

% ------------------------------------------------------------
\subsection{(4) 制御統合層(AITL Integration Layer)}
AITL(Adaptive Intelligent Tri-Layer)は,
SystemDKに組み込まれた知的制御構造であり,
PID,FSM,LLMの三層で構成される。

\begin{itemize}
  \item PID層: 実時間制御を担い,温度・応力・振動・電流などの物理量を安定化する。
  \item FSM層: 動作モード遷移を監督し,安全領域内での運転を保証する。
  \item LLM層: Large Language Model によって,システム全体の整合性を監査・再設計する。
\end{itemize}

特にLLM層は,単なる自然言語AIではなく,
SystemDKの全データ構造を解析して設計意図を理解し,
異常検出時に再構成案を生成する「上位知能」として動作する。
たとえば,あるアクチュエータの発熱やノイズ上昇を検知した際,
PIDやFSMだけでは補償できない場合,
LLMが構造・制御パラメータを再設計し,
SystemDKのスキーマ全体を更新することで
自律的な再最適化を実現する。

% ------------------------------------------------------------
\subsection{(5) 信頼性統合層(Reliability Coupling Layer)}
信頼性解析と制御設計を統合する層である。
FEMやノイズ解析の結果(温度分布・応力集中・電圧降下)を
直接制御系に反映させ,
制御パラメータを信頼性の観点から補正する。
NBTI, HCI, TDDB などの劣化モデルを組み込み,
長期的な性能劣化を予測しながら,
制御側で補償を行うことで
設計寿命を延ばすことができる。

% ------------------------------------------------------------
\subsection*{まとめ}
以上のように,SystemDKは
仕様から実装・解析・制御までのすべての情報を
単一の設計スキーマ上で統合し,
AITLによる知的制御機構を通じて
自律的に安定化・最適化を行う。
すなわちSystemDKは,
単なるデータベースではなく,
\textbf{「設計そのものを制御対象とみなす自律的知能設計基盤」}
として定義される。
