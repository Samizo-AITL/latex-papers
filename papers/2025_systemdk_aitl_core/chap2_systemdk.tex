% ============================================================
% 2. SystemDKの構造(再定義:統合設計基盤アーキテクチャ)
% ============================================================
\section{SystemDKの構造}

SystemDK(System Design Kernel)は,
設計・解析・制御のすべての工程を統一的に接続するための
中核的データ基盤(Design Integration Kernel)である。
従来のように,構造・解析・制御を個別のツールで分離管理するのではなく,
SystemDKでは,各工程の生成物を単一の設計スキーマに統合し,
情報の一貫性・整合性・再利用性を保証する。

\vspace{0.5em}
\noindent SystemDKは,以下の5つの要素で構成される。

\begin{itemize}
  \item \textbf{(1) スキーマ管理層(Schema Management Layer)}:  
  すべての設計データを共通スキーマとして定義し,
  各工程(仕様・制御・回路・構造・解析)を形式的に接続する。
  具体的にはJSON/YAML形式の中間記述(Intermediate Design Schema; IDS)を採用し,
  各要素(モジュール,パラメータ,解析条件,制御ゲイン)を
  一意のIDで管理する。
  設計変更やパラメータ更新は,このスキーマ上で双方向に伝搬する。

  \item \textbf{(2) データ接続層(Data Connectivity Layer)}:  
  設計ツール間(MATLAB/Simulink,EDA,FEM解析,制御検証環境)を接続するAPI群である。
  各ツールで生成されたファイルはスキーマ変換を介してSystemDKに統合され,
  制御系設計と物理解析結果を相互に参照できる。
  これにより,制御パラメータが物理解析の制約条件として反映され,
  FEM/ノイズ解析の結果が制御ループの再設計に直結する。

  \item \textbf{(3) 階層設計統合層(Hierarchical Integration Layer)}:  
  設計階層(ボード:BRDK,インターポーザ:IPDK,パッケージ:PKGDK,システム:SystemDK)を
  階層的に統合する構造である。
  各階層は同一スキーマ構造を共有し,電気・熱・応力・信号経路などの依存関係を明示的にリンクする。
  これにより,ボードやパッケージ設計の変更が即座に
  FEM解析・制御条件・信頼性評価に反映される。

  \item \textbf{(4) 制御統合層(AITL Integration Layer)}:  
  AITL(Adaptive Intelligent Tri-Layer)をSystemDKに組み込み,
  PID,FSM,LLMの三要素を設計情報と直結させる。
  PIDは物理量(温度,電圧,応力など)の実時間安定化を担い,
  FSMは制御モード・状態遷移を管理し,
  LLMは設計情報の整合性を検証する。
  これらがSystemDK上で同期的に動作することで,
  設計と制御の閉ループ結合が実現する。

  \item \textbf{(5) 信頼性統合層(Reliability Coupling Layer)}:  
  各階層の設計パラメータとFEM/ノイズ解析結果を統合し,
  熱・応力・電気的ストレス(NBTI, HCI, TDDBなど)を
  制御設計パラメータに直接反映する。
  SystemDKは信頼性モデルを制御設計の中に内包するため,
  「制御安定性」と「信頼性安定性」を同時に保証できる。
\end{itemize}

\vspace{0.5em}
\noindent
以上の各層は独立して動作するのではなく,
共通スキーマを介して双方向に結合される。
設計・解析・制御・信頼性が常時同期されることにより,
SystemDKは「設計情報を自律的に安定化させる核(Kernel)」として機能する。

\vspace{0.5em}
\noindent
すなわちSystemDKは,
個々のツールやモジュールを統合する単なるデータベースではなく,
\textbf{設計そのものを制御対象とみなす動的な設計基盤}である。
設計の変更や外乱(熱・応力・電源変動など)を検知し,
AITL制御系が自律的にパラメータを再最適化することで,
設計空間の安定性と整合性を継続的に保証する。
