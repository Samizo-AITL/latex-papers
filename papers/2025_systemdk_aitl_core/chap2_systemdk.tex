% ============================================================
% 2. SystemDKの構造
% ============================================================
\section{SystemDKの構造}

SystemDK(System Design Kernel)は,設計空間を統合的に管理するための
知識構造(Knowledge Kernel)であり,
設計情報を「構造層」「挙動層」「制御層」の三層に階層化して定義する。
各層は異なる抽象度と目的を持ちながらも,
共通スキーマによって動的に連結され,
設計構造の変更が制御ロジックへ即時に反映される仕組みを備える。

\begin{itemize}
  \item \textbf{構造層 (Structure Layer)}:  
  材料特性,形状パラメータ,レイアウト構造などの静的情報を保持する層である。  
  各要素はノード・エッジ・属性からなるネットワーク構造として定義され,
  デバイス・パッケージ・システムレベルまでの設計階層を統一的に記述できる。  
  構造層は,FEMやSPICEなどの解析モデルの基盤データとして機能する。
  
  \item \textbf{挙動層 (Behavior Layer)}:  
  構造層で定義された形状・材料条件をもとに,
  熱・応力・電磁場・信頼性劣化などの物理現象を解析する層である。  
  各解析結果は多次元テンソルデータとして表現され,
  時間・温度・負荷条件に対する応答関数として保存される。  
  挙動層の出力は制御層におけるPIDゲインやFSM遷移条件の動的入力となる。
  
  \item \textbf{制御層 (Control Layer)}:  
  挙動層からの解析出力を用いて,
  システムの安定性・性能・信頼性を動的に維持するための制御ロジックを実装する層である。  
  主にPID制御,状態遷移(FSM),および形式的整合性検証(LLM)の3モジュールで構成される。  
  この層は,SystemDK全体の「動的安定化ループ」を形成し,
  解析結果の変化に応じて最適な制御パラメータを再同定する役割を担う。
\end{itemize}

これら三層は共通のデータスキーマによって定義され,
実装上はJSONまたはYAML形式の中間記述(Intermediate Design Schema; IDS)を用いる。
このIDSには,構造ID・解析モデルID・制御パラメータIDが一意に関連付けられており,
設計変更が発生すると,対応するノードが自動的に更新される。

SystemDKの特徴は,従来の静的設計モデルと異なり,
「構造と制御が双方向に影響し合う動的閉ループ」を形成する点にある。
これにより,例えば応力集中や熱暴走などの解析結果が
制御層のFSMに即時伝達され,制御ゲインが自律的に再設定される。
すなわち,SystemDKは設計そのものを「制御可能なシステム」として再定義するものである。
