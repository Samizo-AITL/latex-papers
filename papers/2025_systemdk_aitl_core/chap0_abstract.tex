% ============================================================
% Final Abstract (Japanese & English, IEEE style, 💯 version)
% ============================================================

\begin{abstract}
本論文では,設計・信頼性・制御を統合的に扱う新しい工学アーキテクチャ
\textbf{SystemDK with AITL Core}を提案する。
SystemDK(System Design Kernel)は,設計構造,解析データ,および制御モデルを
共通スキーマ上で統合し,設計階層全体を一貫的に接続する知識基盤である。
AITL(Adaptive Intelligent Tri-Layer)は,PID・FSM・LLMの三層で構成され,
PID層は物理的安定性を維持し,FSM層は動作モードの整合性を保証し,
LLM層は設計情報の論理的整合性を監督し再設計を指令する。
これらの層を動的に結合することにより,
SystemDK with AITL Coreは設計構造と制御論理の整合をリアルタイムに最適化する。
提案手法は,制御対象(ロボットやアクチュエータなど)の物理安定化と,
チップレットを含む電子構造設計の整合管理を同一の枠組みで実現し,
自律的かつ自己整合的な設計システムの新たな基盤を形成する。
\bigskip

\textbf{Abstract (English):}
This paper proposes a novel engineering architecture,
\textbf{SystemDK with AITL Core}, that unifies design, reliability, and control
within a single adaptive framework.
SystemDK (System Design Kernel) integrates design structures,
analytical data, and control models under a shared schema,
ensuring consistent connectivity across all hierarchical layers.
AITL (Adaptive Intelligent Tri-Layer) comprises three layers:
a PID layer maintaining physical stability,
an FSM layer ensuring operational consistency,
and an LLM layer supervising logical coherence and issuing redesign commands.
By dynamically coupling these layers,
SystemDK with AITL Core achieves real-time optimization between structural design and control logic.
The proposed architecture establishes a common foundation
for both physical stabilization in robotic and actuator systems
and structural consistency management in chiplet-based electronic design,
thereby advancing toward an autonomous and self-consistent engineering paradigm.
\end{abstract}

\begin{IEEEkeywords}
System Design, Reliability Integration, Control Architecture,
PID Control, Finite-State Machine, Large Language Model,
Chiplet Design, Robotic Systems, Autonomous Engineering,
Self-consistent Architecture
\end{IEEEkeywords}
