% ============================================================
% 1. はじめに(Revised Final++ / IEEE採録水準)
% ============================================================
\section{はじめに}

現代の複合システム設計では,
構造・材料・熱・応力・電磁・信頼性といった物理解析領域と,
PID制御・状態遷移・AI補償などの制御系設計領域が,
依然として個別に進められることが多い。
この分離は,設計階層間の情報不整合を引き起こし,
パラメータ再設定やモデル再解析の繰り返しによる
開発遅延および信頼性劣化の主要因となっている。

従来のEDAツールやCAE解析環境は,
各領域(回路,熱,応力,信号,制御)における
局所的最適化や解析精度向上には大きく貢献してきた。
しかし,これらは設計全体を俯瞰的に統合する情報基盤を欠いており,
ある領域の設計変更が他領域(例:制御・構造・信号)へ
自動的に伝搬する仕組みが存在しない。
その結果,設計全体としての一貫性と制御安定性を
同時に保証することが困難であった。

本研究では,この課題を根本的に解決するために,
設計・解析・制御を単一のデータスキーマ上で連結し,
情報整合性と自律最適化を同時に実現する
新しい工学アーキテクチャ
\textbf{SystemDK with AITL Core} を提案する。
SystemDK(System Design Kernel)は,
仕様策定,制御系設計,FPGA/ASIC回路設計,構造設計,
および FEM/ノイズ解析を統合的に接続する知識基盤であり,
構造変更が制御モデルや解析条件へ即時反映される
閉ループ型設計環境を実現する。

さらに,SystemDKの中核を成す
AITL(Adaptive Intelligent Tri-Layer)は,
PID・FSM・LLMの三層から構成される知的制御フレームワークである。
PID層はアクチュエータやロボット機構の物理量を実時間で安定化し,
FSM層はその動作モードと状態遷移を監督し,
LLM層は設計・解析・制御データの論理整合を検証し,
必要に応じて再設計を指令する。
この三層構造により,SystemDK全体は
設計変更や実測データの変化に応じて自律的に再整合・再最適化を行う,
いわば「自己修復型設計基盤」として機能する。

提案する SystemDK with AITL Core は,
制御対象(ロボット・アクチュエータなど)の物理安定化と,
チップレットを含む電子構造設計の整合管理を,
同一スキーマ上で統合的に扱うことを可能にする。
本論文では,
その統合設計フロー,制御理論,および信頼性統合手法を体系的に示し,
仕様策定から実装検証までの全工程を閉ループで接続する,
次世代の自律設計アーキテクチャの基礎を確立することを目的とする。
