% ============================================================
% 1. はじめに
% ============================================================
\section{はじめに}

現代の複合設計では、構造・材料・熱・応力・電磁・信頼性などの物理現象と、
PID・状態遷移・AI制御などの制御設計が、異なる領域として個別に扱われることが多い。
その結果、設計階層間で情報の不整合が生じ、
パラメータの再設定や解析モデルの再構築が繰り返されることで、
開発の遅延や信頼性の低下を招く要因となっている。

従来の設計支援システムやEDAツールは、
個別分野における最適化や解析精度の向上においては有効であるが、
設計全体を俯瞰した構造・解析・制御の統合管理機構を欠いている。
そのため、各要素のモデル更新が他領域に波及しにくく、
結果としてシステム全体としての安定性を保証することが困難である。

本研究では、この課題を解決するために、
設計構造・解析モデル・制御ロジックを統一的に表現し、
動的に連携させるための新しい工学アーキテクチャ
\textbf{SystemDK with AITL Core}を提案する。
SystemDK(System Design Kernel)は、構造層・挙動層・制御層の三階層で
設計情報を体系化し、それぞれの関係をデータスキーマにより形式的に結合することで、
構造的変更が即座に制御層へ伝搬する仕組みを実現する。

さらに、AITL(Adaptive Intelligent Tri-Layer)は
PID・FSM・LLMの三層から構成され、
PID層が物理的安定性を、FSM層が状態遷移の一貫性を、
LLM層が設計情報の論理的整合性を保証する。
この三層制御構造をSystemDK上に統合することで、
設計空間の安定性と情報整合性を同時に確保できる
自律的な設計基盤の構築を目指す。

本論文では、このSystemDK with AITL Coreの
アーキテクチャ構成・制御理論・信頼性統合手法を体系的に示し、
自律的かつ整合的に設計を最適化するための基礎モデルを提示する。
