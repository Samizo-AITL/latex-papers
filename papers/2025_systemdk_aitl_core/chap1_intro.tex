% ============================================================
% 1. はじめに(Revised Final)
% ============================================================
\section{はじめに}

現代の複合システム設計では,
構造・材料・熱・応力・電磁・信頼性といった物理現象の解析と,
PID・状態遷移・AI制御などの制御系設計が,
独立した領域として個別に進められることが多い。
この分離は,設計階層間の情報不整合を招き,
パラメータ再設定やモデル再解析の繰り返しによる
開発遅延や信頼性劣化を引き起こす主要因となっている。

従来のEDAツールや設計支援システムは,
各領域(回路,熱,応力,信号,制御)における
局所的な最適化や解析精度向上には寄与しているが,
設計全体を俯瞰的に連結する統合的情報基盤を欠いている。
そのため,一部の設計変更が他の解析・制御領域へ自動伝搬せず,
設計全体としての一貫性と制御安定性を同時に保証することが困難である。

本研究では,この課題を解決するために,
設計・解析・制御を単一のデータスキーマ上で連結する
新しい工学アーキテクチャ
\textbf{SystemDK with AITL Core} を提案する。
SystemDK(System Design Kernel)は,
仕様策定,制御系設計,FPGA/ASIC回路設計,構造設計,
および FEM/ノイズ解析を統合的に接続し,
設計情報の流れを閉ループ化する知識基盤である。
これにより,構造的変更が制御モデルや解析条件へ即時反映され,
設計全体が自律的に安定化・最適化される環境を実現する。

さらに,AITL(Adaptive Intelligent Tri-Layer)は,
SystemDKの中核制御構造として機能する知的制御フレームワークであり,
PID・FSM・LLMの三層から構成される。
PID層は物理量の実時間安定化を,
FSM層は動作モードと状態遷移の一貫性を,
LLM層は設計データ間の論理整合性を監督する。
この三層制御構造により,
SystemDK全体が設計変更や解析結果に応じて
自動的に再整合・再最適化を行う自律的設計基盤が形成される。

本論文では,
SystemDK with AITL Core の
統合設計フロー,制御理論,および信頼性統合手法を体系的に示す。
提案する設計体系は,
仕様策定から実装検証までの全工程を閉ループで接続し,
物理的安定性と情報的整合性を同時に保証する
次世代の自律設計アーキテクチャの基礎をなすものである。
