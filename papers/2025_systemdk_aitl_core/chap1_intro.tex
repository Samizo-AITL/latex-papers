% ============================================================
% 1. はじめに(Revised Final+)
% ============================================================
\section{はじめに}

現代の複合システム設計では,
構造・材料・熱・応力・電磁・信頼性といった物理現象の解析と,
PID・状態遷移・AI制御などの制御系設計が,
依然として独立した領域として個別に進められることが多い。
この分離は,設計階層間の情報不整合を招き,
パラメータ再設定やモデル再解析の繰り返しによる
開発遅延や信頼性劣化を引き起こす主要因となっている。

従来のEDAツールや設計支援システムは,
各領域(回路,熱,応力,信号,制御)における
局所的な最適化や解析精度向上には寄与しているが,
設計全体を俯瞰的に接続する統合的情報基盤を欠いている。
そのため,一部の設計変更が他の解析・制御領域へ自動伝搬せず,
設計全体としての一貫性と制御安定性を同時に保証することが困難である。

本研究では,この課題を解決するために,
設計・解析・制御を単一のデータスキーマ上で連結し,
情報の一貫性と自律最適化を同時に実現する
新しい工学アーキテクチャ
\textbf{SystemDK with AITL Core} を提案する。
SystemDK(System Design Kernel)は,
仕様策定,制御系設計,FPGA/ASIC回路設計,構造設計,
および FEM/ノイズ解析を統合的に接続する知識基盤である。
これにより,構造的変更が制御モデルや解析条件へ即時反映され,
設計全体が閉ループ的に安定化・最適化される環境を実現する。

さらに,SystemDKの中核を成す
AITL(Adaptive Intelligent Tri-Layer)は,
PID・FSM・LLMの三層から構成される知的制御フレームワークである。
PID層はアクチュエータやロボット機構の物理量を実時間で安定化し,
FSM層はその動作モードと状態遷移を監督し,
LLM層は設計・解析・制御データの論理整合を検証し,
必要に応じて再設計指令を発する。
この三層構造により,SystemDK全体は
設計変更や実測データの変化に応じて自律的に再整合・再最適化を行う
「自己修復型設計基盤」として機能する。

提案する SystemDK with AITL Core は,
制御対象(ロボット・アクチュエータなど)の
物理的安定化と,
チップレットを含む電子構造設計の
整合管理を同一スキーマ上で統合することを可能にする。
本論文では,
その統合設計フロー,制御理論,および信頼性統合手法を体系的に示し,
仕様策定から実装検証までの全工程を閉ループで接続する,
次世代の自律設計アーキテクチャの基礎を確立することを目的とする。
