% ============================================================
% 1. はじめに
% ============================================================
\section{はじめに}

現代の複合システム設計では,構造・材料・熱・応力・電磁・信頼性といった
物理的現象の解析と,PID・状態遷移・AI制御などの制御系設計が
分離して進められることが多い。
この分離は,設計階層間の情報不整合を招き,
再解析やパラメータ再設定の繰り返しによる開発遅延や信頼性劣化を引き起こす主要因となっている。

従来の設計支援システムやEDAツールは,
個々の領域(回路,熱,応力,信号,制御など)における最適化や解析精度の向上には寄与するが,
設計全体を貫通的に接続する統合的な情報管理機構を欠いている。
そのため,ある領域での設計更新が他領域へ自動的に反映されず,
設計の一貫性と制御安定性を同時に保証することが難しい。

本研究では,この問題を解決するために,
設計・解析・制御を統一スキーマ上で接続する
新しい工学的アーキテクチャ \textbf{SystemDK with AITL Core} を提案する。
SystemDK(System Design Kernel)は,
仕様策定,制御系設計,FPGA/ASIC回路設計,構造設計,
および FEM/ノイズ解析を統合的に結合する設計基盤である。
これにより,制御モデルと物理構造が同一データ空間で連携し,
設計更新が即時に制御・解析へ伝搬する閉ループ設計環境を実現する。

さらに,AITL(Adaptive Intelligent Tri-Layer)は,
PID・FSM・LLMの三層制御構造から構成される。
PID層は物理系の実時間安定化を担い,
FSM層は動作モードと状態遷移を管理し,
LLM層は設計データ間の形式的整合性を監督する。
AITLはSystemDKの中核制御系として動作し,
設計フロー全体を安定化させる自律的な制御基盤を形成する。

本論文では,SystemDK with AITL Coreの
統合設計フロー,制御理論,および信頼性統合手法を体系的に示す。
提案する設計体系は,仕様策定から実装検証までの全工程を閉ループで連携させ,
物理的安定性と情報的整合性を同時に保証する
自律的設計アーキテクチャの基礎をなすものである。
