% ============================================================
% 7. 結論
% ============================================================
\section{結論}

本論文では,設計・信頼性・制御を統合する新しい工学アーキテクチャ
\textbf{SystemDK with AITL Core}を提案した。
SystemDK(System Design Kernel)は,構造層・挙動層・制御層を共通スキーマで連結し,
設計情報の一貫性と即時反映性を実現する統合設計基盤である。
また,AITL(Adaptive Intelligent Tri-Layer)は,
PID,FSM,LLMの三層制御構造により,
物理的安定性,動作的一貫性,および論理的整合性を同時に保証する
自律安定化メカニズムを形成する。

SystemDK with AITL Core の導入により,
従来分離されていた構造解析・制御設計・信頼性評価が
一体的に動作する閉ループが実現され,
設計情報の更新や環境変動に対しても
システム全体の安定性を維持できることを示した。
特に,PID層による物理応答の動的補償,
FSM層による安全遷移監督,
LLM層によるデータ整合性検証の連携により,
設計空間の安定性と情報整合性を自律的に維持できることを明確にした。

さらに,本体系では,
NBTIやHCIなどの時間依存劣化現象を制御ループ内に取り込むことで,
信頼性解析と制御安定化を同一理論内で扱うことを可能にした。
これにより,SystemDKは単なる設計支援ツールを超え,
「設計そのものを制御対象とする動的工学理論」として位置づけられる。

今後の課題としては,
(1) 各層スキーマの標準化と相互運用性の拡張,  
(2) FSMおよびLLM層における自動最適化アルゴリズムの実装,  
(3) 半導体・メカトロ・AI制御など多分野設計への適用評価  
が挙げられる。
これらを通じて,SystemDK with AITL Core は
自律設計理論の中核をなす工学的基盤として,
将来の「自己進化型設計システム」へ発展する可能性を有する。
