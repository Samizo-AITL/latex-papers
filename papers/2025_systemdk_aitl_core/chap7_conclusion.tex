% ============================================================
% 7. 結論(Final++ / 💯採録水準)
% ============================================================
\section{結論}

本論文では,設計・制御・信頼性を統合する新しい工学アーキテクチャ
\textbf{SystemDK with AITL Core} を提案した。
SystemDK(System Design Kernel)は,
構造設計(BRDK/IPDK/PKGDK),回路設計(FPGA/ASIC),
および解析モデル(FEM/ノイズ/信頼性)を
共通スキーマで統合管理する知識基盤であり,
設計情報の一貫性・即時反映性・再利用性を保証する。
従来の分断型設計プロセスでは困難であった
「構造・制御・解析・信頼性の同時最適化」を,
単一の閉ループ内で実現した。

AITL(Adaptive Intelligent Tri-Layer)は,
PID・FSM・LLMの三層制御構造を中核に据え,
PID層が物理安定性を維持し,
FSM層がモード安全を監督し,
LLM層が設計論理の整合性と再構成を司る。
これにより,SystemDKは静的な設計環境を超え,
\textbf{「設計そのものを制御対象とする自律設計制御系」}
として動作することを明確に示した。

さらに,NBTI,HCI,TDDB などの時間依存劣化現象を
制御ループ内に取り込み,
PIDゲインの適応補償,FSMの信頼性遷移制御,
LLMによる長期整合性維持を協調させることで,
\textbf{信頼性と制御の統合理論}を確立した。
この構造により,SystemDKは
設計・制御・信頼性を同一理論内で動的に結合し,
長期安定運用と進化的最適化を同時に実現する。

提案手法の学術的貢献は以下の三点に要約される。
\begin{enumerate}
  \item \textbf{統合基盤の確立:}  
  設計・解析・制御・信頼性を単一スキーマで統合する
  \emph{System Design Kernel} の概念を初めて実装的に定義した。

  \item \textbf{知的制御構造の導入:}  
  PID・FSM・LLMによる三層監督型制御体系を提案し,
  設計情報の整合性を動的に監督・再構成する
  \emph{Adaptive Intelligent Tri-Layer (AITL)} を構築した。

  \item \textbf{信頼性制御理論の拡張:}  
  物理劣化モデル $f(T,V,\sigma,t)$ を制御ループ内に閉じ込め,
  信頼性解析と制御安定化を同一数理体系で結合した。
\end{enumerate}

これらにより,SystemDK with AITL Core は,
構造設計・制御設計・信頼性評価を動的に結合した
\textbf{実装主導・自律安定化型工学体系}を実現したといえる。

今後の課題としては,
\begin{enumerate}
  \item FPGA/ASIC/構造解析/制御スキーマの標準化と
        異ツール間の相互運用性拡張,
  \item FSMおよびLLM層における
        自動最適化およびゲイン再同定アルゴリズムの高速化,
  \item 半導体・メカトロニクス・AI制御など
        異分野設計への展開・実証,
\end{enumerate}
が挙げられる。
これらを発展させることで,
SystemDK with AITL Core は
設計から運用に至る全ライフサイクルを自律安定化する
\textbf{自己進化型設計システム(Self-Evolving Design System)}へと発展する。

総じて,本研究は,
設計空間を制御対象とみなし,
信頼性を内包した動的制御理論として設計を再定義することで,
\textbf{自律設計理論(Autonomous Design Theory)}への
新しい道を開いたものである。
