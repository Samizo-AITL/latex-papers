% ============================================================
% 7. 結論
% ============================================================
\section{結論}

本論文では,設計・信頼性・制御を統合する新しい工学アーキテクチャ
\textbf{SystemDK with AITL Core}を提案した。
SystemDK(System Design Kernel)は,
構造設計(BRDK/IPDK/PKGDK)・解析モデル・制御ロジックを
共通スキーマで接続する統合設計基盤であり,
設計情報の一貫性と即時反映性を実現する。

また,AITL(Adaptive Intelligent Tri-Layer)は,
PID・FSM・LLMの三層制御構造により,
物理的安定性(PID),動作的一貫性(FSM),
および論理的整合性(LLM)を同時に保証する
自律安定化メカニズムを形成する。
これにより,SystemDKは静的な設計体系を超え,
「制御理論を内包した自律的設計制御系」として機能する。

SystemDK with AITL Core の導入により,
従来分離されていた構造解析・制御設計・信頼性評価が
単一の閉ループ内で結合され,
設計情報の変更や環境変動に対しても
システム全体の安定性と整合性を維持できることを示した。
特に,PID層による物理応答の動的補償,
FSM層による安全遷移監督,
LLM層による設計整合性検証の連携によって,
SystemDK全体が自律的に安定化することを明確にした。

さらに,本体系では,
NBTIやHCIなどの時間依存劣化現象を
制御ループ内の可観測量として扱うことにより,
信頼性解析と制御安定化を同一理論内で統合した。
この構造により,SystemDKは単なる設計支援環境ではなく,
「設計そのものを制御対象とする動的工学理論体系」として位置づけられる。

今後の課題としては,
\begin{enumerate}
  \item 各層スキーマ(構造・挙動・制御)の標準化と相互運用性の拡張,
  \item FSMおよびLLM層における自動最適化アルゴリズムの実装,
  \item 半導体・メカトロ・AI制御など異分野設計への適用評価,
\end{enumerate}
が挙げられる。
これらを通じて,SystemDK with AITL Core は
自律設計理論の中核を担う工学的基盤として,
将来的に「自己進化型設計システム」へ発展する可能性を有する。

本研究は,
設計・信頼性・制御を統合し,
設計空間そのものを制御対象とするという
新しい工学的枠組みを提示した点において,
自律的工学設計理論の確立に向けた重要な第一歩である。
