% ============================================================
% 7. 結論(Final++ / 💯採録水準)
% ============================================================
\section{結論}

本論文では,設計・制御・信頼性を統合する新しい工学アーキテクチャ
\textbf{SystemDK with AITL Core} を提案した。
SystemDK(System Design Kernel)は,
構造設計(BRDK/IPDK/PKGDK),回路設計(FPGA/ASIC),
および解析モデル(FEM/ノイズ/信頼性)を
共通スキーマで接続する統合設計基盤であり,
設計情報の一貫性・即時反映性・再利用性を保証する。

AITL(Adaptive Intelligent Tri-Layer)は,
PID・FSM・LLMの三層制御構造を中核とし,
物理安定性(PID),動作一貫性(FSM),
論理整合性(LLM)を同時に維持する
自律安定化メカニズムを構成する。
これにより,SystemDKは静的な設計環境を超え,
\textbf{「設計そのものを制御対象とする自律設計制御系」}
として機能することを示した。

SystemDK with AITL Core により,
従来分離されていた構造解析・制御設計・信頼性評価が
単一の閉ループ内で動的に結合され,
設計変更や外部環境変動に対しても
全体システムの安定性と整合性を保てることを明らかにした。
特に,PID層による物理応答の補償,
FSM層によるモード安全監督,
LLM層による設計再構成の協調によって,
SystemDK全体が自律的に安定化する過程を明確に示した。

さらに,本体系では,
NBTI,HCI,TDDB などの時間依存劣化現象を
制御ループ内の可観測変数として組み込み,
信頼性解析と制御安定化を同一理論内で統合した。
これにより,SystemDKは単なる設計支援環境ではなく,
\textbf{「設計・制御・信頼性を統合した動的工学理論体系」}
として再定義される。

今後の展開としては,
\begin{enumerate}
  \item FPGA/ASIC/構造解析/制御スキーマの標準化と相互運用性の拡張,
  \item FSMおよびLLM層における自動最適化・ゲイン再同定アルゴリズムの実装,
  \item 半導体・メカトロ・AI制御など異分野設計への適用評価,
\end{enumerate}
が挙げられる。
これらを通じて,SystemDK with AITL Core は,
設計から運用に至る全工程を自律的に安定化させる
\textbf{自己進化型設計システム(Self-Evolving Design System)}
への発展が期待される。

本研究は,
設計・信頼性・制御を統一理論として結合し,
「設計空間を制御対象として扱う」新しい工学的概念を提示した点で,
自律設計理論の確立に向けた重要な第一歩である。
