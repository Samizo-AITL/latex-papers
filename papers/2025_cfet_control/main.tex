\documentclass[conference]{IEEEtran}

% --- Preamble ---
\usepackage[utf8]{inputenc}
\usepackage[T1]{fontenc}
\usepackage{amsmath,amssymb}
\usepackage{graphicx}
\usepackage{cite}
\usepackage{url}
\usepackage{hyperref}

\title{Time-Response-Aware Design of CFET Interconnect Delay, Self-Heating, and Stress Coupling via PID+FSM+LLM Supervision}

\author{
  \IEEEauthorblockN{Shinichi Samizo}
  \IEEEauthorblockA{Independent Semiconductor Researcher\\
  Project Design Hub, Samizo-AITL\\
  \textit{Email:} \href{mailto:shin3t72@gmail.com}{shin3t72@gmail.com}\quad
  \textit{GitHub:} \href{https://github.com/Samizo-AITL}{Samizo-AITL}}
}

\begin{document}
\maketitle

\begin{abstract}
Gate-all-around (GAA) nanosheet FETs can be designed under static assumptions, where parasitics and thermal effects are treated as fixed values. However, complementary FETs (CFETs) with stacked n/p channels suffer from strong vertical self-heating and stress coupling. These effects vary dynamically, leading to RC delay shifts that static design cannot capture. This paper introduces a time-response-aware design paradigm: proportional--integral--derivative (PID) feedback regulates delay deviation, finite-state machine (FSM) guards ensure safety under hotspots, and large language model (LLM) supervision adapts controller gains under workload drift. Compact RC--thermal--stress networks were simulated in SystemDK across sweeps of via resistance, inter-tier capacitance, coupling factor, and burst power. Results show more than two orders of magnitude suppression of delay deviation, reducing peak error from $\sim$8\% to $2.6\times 10^{-3}$ and steady-state error below $10^{-6}$. FSM guarantees bounded actuation, while LLM retunes thresholds dynamically. This reframes CFET optimization from static prediction to dynamic compensation, addressing self-heating and stress-induced variability in sub-2\,nm integration.
\end{abstract}

\section{Introduction}
Until the GAA generation, device and circuit design could rely on static analysis: resistance, capacitance, and temperature rise were treated as fixed values. However, as we move to CFET integration, where nFET and pFET are vertically stacked, two challenges dominate: (1) \emph{self-heating}, where the top tier's heat propagates to the bottom tier, raising resistance and delay; and (2) \emph{stress coupling}, where vertical stacking and thermal expansion generate asymmetric strain, modulating threshold voltage and carrier mobility. Both effects are strongly time-dependent and interact with RC delay.

Conventional static design optimizes for a snapshot condition, but fails to account for how delay and temperature evolve over time. This limitation motivates a new paradigm: \emph{time-response-aware design}, where stability and convergence under dynamic workloads become first-class design targets. We incorporate control theory---PID feedback, FSM guards, and LLM supervision---to stabilize delay and temperature in CFET stacks. Unlike prior studies that only modeled parasitics~\cite{yakimets2020cfet,irds2023}, we demonstrate runtime compensation.

\section{Modeling}
The FO1 delay is expressed as
\begin{equation}
T_{FO1} = (R_{wire}+R_{via})(C_{load}+C_{inter}).
\end{equation}
Resistance increases with temperature:
\begin{equation}
R(T) = R_0 \left(1 + \alpha (T-25^\circ C)\right).
\end{equation}
Thermal dynamics follow
\begin{equation}
C_{th}\frac{dT}{dt} = P\cdot R_{th} - (T - T_{amb}).
\end{equation}
Stress-induced mobility degradation is modeled by
\begin{equation}
\mu_{eff} = \mu_0 (1 - \gamma \sigma_{eff}),
\end{equation}
where $\sigma_{eff}$ is the thermo-mechanical stress proportional to $\Delta T$ and vertical confinement. Delay therefore couples to both $T$ and $\sigma$.

\section{Control Architecture}
\textbf{PID:} regulates delay deviation $\varepsilon_d$ through DVFS actuation $u$.  
\textbf{FSM:} enforces HOT mode when $T_{top}>85^\circ$C, bounding $u\leq u_{max}$ to avoid runaway.  
\textbf{LLM:} supervises by retuning $(K_p,K_i,K_d)$ and FSM thresholds when overshoot or oscillation exceed tolerance.  

This layered design provides stability (PID), safety (FSM), and adaptability (LLM).

\section{Experimental Setup}
Simulations used SystemDK 2025, step size $1$\,ns, horizon $1.5$\,s. Parameters: $R_{via}=1$--$10~\Omega$, $C_{inter}=1$--$5$\,fF, $k_c=0.3$--0.9, burst power $0.1$--$1.0$\,W. Stress coupling coefficient $\gamma$ was swept 0.05--0.2. Thermal RC constants were calibrated from CFET compact models.

\section{Results}
\subsection{Transient Response}
Without control, delay error rises $\sim$8\% under burst heating. PID reduces error but overshoots. PID+FSM clamps actuation at hotspots. PID+FSM+LLM converges smoothly with error $<10^{-6}$.

\subsection{Stress Impact}
Fig.~\ref{fig:stress} shows delay deviation vs stress factor $\gamma$. Static design cannot compensate, but controlled design maintains stability across $\gamma=0.05$--0.2.

\subsection{Metrics}
\begin{table}[b]
\renewcommand{\arraystretch}{1.1}
\caption{CONTROLLER PERFORMANCE}
\centering
\begin{tabular}{|l|c|c|c|}
\hline
Metric & No Ctrl & PID+FSM & PID+FSM+LLM \\
\hline
Peak deviation & $\sim$8\% & $10^{-3}$ & $2.6\times 10^{-3}$ \\
Steady error   & $10^{-2}$ & $\sim$0   & $<10^{-6}$ \\
Overshoot      & Large     & Suppressed & Minimal \\
Stress tolerance & None    & Limited   & Wide \\
\hline
\end{tabular}
\end{table}

\begin{figure}[t]
\centering
\includegraphics[width=\columnwidth]{figs/stress_response.png}
\caption{Delay deviation vs stress factor $\gamma$: static vs controlled design.}
\label{fig:stress}
\end{figure}

\section{Stability Analysis}
The closed-loop $T(j\omega)=L/(1+L)$ with $L=C\cdot G$ was analyzed. PID gains satisfy phase margin $>45^\circ$. FSM bounds control effort under thermal runaway. LLM maintains margins when $\{k_c,\gamma,P\}$ drift.

\section{Discussion and Limitations}
Our controller reduces CFET delay deviation by over 100$\times$, robust across thermal and stress variations. Limitations: compact-model abstraction, omission of full 3D parasitics, simplified LLM adaptation. Future work: chip-in-loop validation, extension to forksheet and 3D sequential CFET, and integration with cooling-aware policies.

\section{Conclusion}
We proposed a time-response-aware design for CFETs, addressing both self-heating and stress coupling. By integrating PID, FSM, and LLM into the design methodology, delay and thermal behavior stabilize under dynamic workloads. This reframes CFET design from static prediction to dynamic compensation.

\bibliographystyle{IEEEtran}
\bibliography{refs}

\section*{Author Biography}
\noindent\textbf{Shinichi Samizo} received the M.S. degree in Electrical and Electronic Engineering from Shinshu University, Japan. He worked at Seiko Epson Corporation on semiconductor memory and mixed-signal devices, and contributed to inkjet MEMS and PrecisionCore printhead technology. He is now an independent researcher focusing on device physics, memory, and AI-integrated systems.\\
Contact: \href{mailto:shin3t72@gmail.com}{shin3t72@gmail.com}, \href{https://github.com/Samizo-AITL}{Samizo-AITL}

\end{document}
