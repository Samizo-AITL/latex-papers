\documentclass[conference]{IEEEtran}

% ===== 日本語(LuaLaTeX想定) =====
\usepackage{luatexja}
\usepackage{luatexja-fontspec}
\setmainjfont{IPAexMincho}
\setsansjfont{IPAexGothic}

% ===== 図表・数式 =====
\usepackage{amsmath,amssymb}
\usepackage{graphicx}
\usepackage{booktabs}
\usepackage{cite}
\usepackage{siunitx}

\title{0.25\,\textmu m HV CMOSにおけるTiシリサイド相転移不安定性と\\
埋込みSRAM信頼性:携帯電話向けLCDドライバICの歴史的事例}

\author{\IEEEauthorblockN{三溝 真一}
\IEEEauthorblockA{Independent Semiconductor Researcher / Project Design Hub, Samizo-AITL\\
Email: shin3t72@gmail.com}
}

\begin{document}
\maketitle

\begin{abstract}
1990年代後半、富士見6インチラインで開発されたLOCOSベースの高耐圧技術を起点に、1998年には酒田8インチFabにおいて0.35\,\textmu m CMOSへ高耐圧デバイスを混載し、モノクロLCDドライバICが量産化された。2000年代初頭の携帯電話のカラーパネル化(aTFT)により1\,Mbit級の埋込みSRAM需要が顕在化し、酒田では0.25\,\textmu m HV CMOSが採用された。しかしTiSi$_2$のC49$\to$C54相転移不完全性とhalo B吸収が局所高抵抗スポットを生み、1\,Mbit SRAMでランダム単ビット不良として顕在化した。本稿は、プロセス選定、故障機構、暫定・恒久対策、および事業的帰結を整理する。
\end{abstract}

\section{序論}
(ここに背景を書く)

\section{プロセス選定の背景}
(ここに0.35µm/0.25µm/0.18µmの比較を書く)

\section{技術的背景}
(TiSi$_2$相転移、B拡散など)

\section{故障解析}
(1Mbit SRAMでの単ビット不良)

\section{対策}
(暫定・恒久対策)

\section{ビジネス的帰結}
(市場シェアとタイミング)

\section{結論}
(まとめ)

\bibliographystyle{IEEEtran}
\bibliography{refs}
\end{document}
