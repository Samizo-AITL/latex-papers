\documentclass[journal]{IEEEtran}

% --- Engine & fonts ---
\usepackage{iftex}
\ifXeTeX
  \usepackage{fontspec}
  \usepackage{xeCJK}
  % 英文フォント(TeX Gyre 系)
  \setmainfont{TeX Gyre Termes}
  \setsansfont{TeX Gyre Heros}
  \setmonofont{TeX Gyre Cursor}
  % 日本語フォント(Noto CJK)
  \setCJKmainfont{Noto Serif CJK JP}
  \setCJKsansfont{Noto Sans CJK JP}
\fi

% --- Packages ---
\usepackage{graphicx}
\usepackage{amsmath,amssymb}
\usepackage{siunitx}
\usepackage{booktabs}
\usepackage[numbers,sort&compress]{natbib}
\usepackage{caption}
\usepackage{subcaption}
\usepackage{hyperref}

\begin{document}

\title{FeFET CMOS 0.18\,\si{\micro\meter} Integration Study}
\author{Samizo-AITL}
\maketitle

% =========================================================
% Abstract
% =========================================================
\begin{abstract}
\noindent
\textbf{Context:} Ferroelectric field-effect transistors (FeFETs) based on Hf$_{0.5}$Zr$_{0.5}$O$_2$ provide a CMOS-compatible option for embedded nonvolatile memory. 
\textbf{Approach:} We integrate a gate-last FeFET module into a 0.18\,\si{\micro\meter} CMOS logic baseline with only one additional mask. 
\textbf{Results:} Devices exhibit a threshold-voltage window of 0.8--1.0\,V, endurance beyond $10^5$ program/erase cycles, and retention projected to $>10$ years at 85\,$^\circ$C. 
\textbf{Implications:} These results enable instant-on, SRAM backup, and secure key storage on mature 0.18\,\si{\micro\meter} technology nodes.
\end{abstract}

\begin{IEEEkeywords}
FeFET, HfZrO$_x$, 0.18\,$\mu$m CMOS, reliability, process integration
\end{IEEEkeywords}

\section*{要旨}
\noindent
\textbf{背景:} Hf$_{0.5}$Zr$_{0.5}$O$_2$ に基づく FeFET は組込み用途向けの CMOS 互換不揮発メモリ候補である。\\
\textbf{手法:} レガシー 0.18\,\si{\micro\meter} CMOS ロジックのベースラインに対し,追加マスク 1 枚のみでゲートラスト型 FeFET モジュールを統合した。\\
\textbf{結果:} しきい値電圧ウィンドウ 0.8--1.0\,V,書換え耐久 $>10^5$ 回,85\,$^\circ$C で 10 年超の保持を達成・外挿した。\\
\textbf{意義:} 成熟ノード上での瞬時起動,SRAM バックアップ,セキュアキー格納を可能にする。

\section*{索引用語}
FeFET,強誘電 HfZrO$_x$,0.18\,$\mu$m CMOS,信頼性,プロセス統合

% =========================================================
% Introduction
% =========================================================
\section{Introduction}
FeFETs based on HfZrO$_x$ thin films have emerged as a CMOS-compatible candidate for embedded nonvolatile memory (NVM)~\citep{boscke2011hafnium,mueller2012fefet}. 
Practical deployment demands integration within mature logic processes used in automotive and IoT~\citep{mitsubishi2003automotive}. 
In this work we target a legacy 0.18\,\si{\micro\meter} CMOS flow and demonstrate a minimal-overhead FeFET module.

\section*{序論}
HfZrO$_x$ 系薄膜 FeFET は CMOS 互換の組込み不揮発メモリ候補として注目されている~\citep{boscke2011hafnium,mueller2012fefet}。自動車や IoT 向けの成熟ノード応用には,0.18\,\si{\micro\meter} CMOS への実装が有効である~\citep{mitsubishi2003automotive}。本研究では最小追加工程での FeFET モジュール統合を実証する。

% =========================================================
% Process Integration
% =========================================================
\section{Process Integration}

\subsection*{Baseline and Added Steps}
The ferroelectric gate stack is inserted after polysilicon definition. 
Table~\ref{tab:masks} summarizes added steps. 
Fig.~\ref{fig:flow} shows placement within the baseline.

\begin{figure}[!t]
  \centering
  \includegraphics[width=.85\linewidth]{figs/flow_overview.pdf}
  \caption{Placement of the FeFET module within the 0.18\,\si{\micro\meter} CMOS baseline flow.}
  \label{fig:flow}
\end{figure}

\begin{table}[!t]
  \centering
  \caption{Added masks and process steps relative to baseline logic.}
  \label{tab:masks}
  \begin{tabular}{@{}lcc@{}}
    \toprule
    \textbf{Step} & \textbf{Mask} & \textbf{Comment}\\
    \midrule
    FE metal gate & +1 & Reuses analog option mask\\
    FE anneal     & 0  & Done in BEOL furnace\\
    \bottomrule
  \end{tabular}
\end{table}

\subsection*{Device Stack}
Gate: TiN / Hf$_{0.5}$Zr$_{0.5}$O$_2$ (10\,nm ALD) / SiO$_2$ IL / p-Si.

\section*{プロセス統合}
\subsection*{ベースラインと追加工程}
強誘電ゲートスタックはポリシリコン形成後に挿入する。追加工程を表\ref{tab:masks}に示し,図\ref{fig:flow}にロジックフロー内配置を示す。

\subsection*{デバイススタック}
ゲートスタック:TiN / Hf$_{0.5}$Zr$_{0.5}$O$_2$ (10\,nm, ALD) / SiO$_2$ 界面層 / p型 Si 基板。

% =========================================================
% Experimental Conditions
% =========================================================
\section{Experimental Conditions}
\begin{itemize}
  \item FE film thickness: 10\,nm (ALD).
  \item Capacitor area: $100\times100\,\si{\micro\meter}^2$.
  \item Voltage: $\pm 3$\,V, pulse width 1--\SI{1}{\micro\second}.
  \item Frequency: 1\,kHz--1\,MHz.
  \item Equipment: Keysight B1500A with Cascade probe station.
\end{itemize}

\section*{実験条件}
Hf$_{0.5}$Zr$_{0.5}$O$_2$ 膜厚 10\,nm(ALD),キャパシタ面積 $100\times100\,\si{\micro\meter}^2$,電圧 $\pm 3$\,V(パルス幅 1--\SI{1}{\micro\second}),周波数 1\,kHz--1\,MHz。Keysight B1500A+プローバにより評価。

% =========================================================
% Reliability
% =========================================================
\section{Reliability}
\subsection*{Endurance}
Up to $10^5$ P/E cycles, $\Delta V_\mathrm{th}$ degrades less than 20\% (Fig.~\ref{fig:endurance})~\citep{mueller2015endurance,park2020nbdoping}.

\begin{figure}[!t]
  \centering
  \includegraphics[width=.85\linewidth]{figs/endurance.pdf}
  \caption{Endurance measured at $V_\mathrm{PGM}=X$\,V with pulse width $t=Y\,\si{\micro\second}$.}
  \label{fig:endurance}
\end{figure}

\subsection*{Retention}
10-year retention projected at 85\,$^\circ$C via Arrhenius extrapolation~\citep{Yamazaki2018}.

\section*{信頼性}
\subsection*{耐久性}
$10^5$ 回までの P/E サイクルで $\Delta V_\mathrm{th}$ 劣化は 20\% 未満(図\ref{fig:endurance})~\citep{mueller2015endurance,park2020nbdoping}。

\subsection*{保持}
85\,$^\circ$C における 10 年保持を Arrhenius 外挿で評価~\citep{Yamazaki2018}。

% =========================================================
% Conclusion
% =========================================================
\section{Conclusion}
We demonstrated a minimal-mask FeFET module on 0.18\,\si{\micro\meter} CMOS with endurance/retention verified. Future work: yield and sense-path co-design.

\section*{結論}
0.18\,\si{\micro\meter} CMOS 向け最小追加マスク FeFET モジュールを実証した。今後はアレイ歩留まりとセンス回路協調最適化を検討する。

% =========================================================
% References
% =========================================================
\bibliographystyle{IEEEtran}
\bibliography{refs}

\end{document}
