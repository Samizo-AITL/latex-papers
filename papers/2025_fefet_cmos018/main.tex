\documentclass[conference]{IEEEtran}
\usepackage{amsmath,amssymb}
\usepackage{graphicx}
\usepackage{tikz}
\usepackage{cite}

\title{FeFET CMOS 0.18\,$\mu$m Integration Study}

\author{
  Shinichi Samizo\\
  \small Independent Semiconductor Researcher; Former Engineer at Seiko Epson Corporation\\
  \small Email: shin3t72@gmail.com, GitHub: \texttt{https://github.com/Samizo-AITL}
}

\begin{document}
\maketitle

\begin{abstract}
Ferroelectric field-effect transistors (FeFETs) based on Hf$_{0.5}$Zr$_{0.5}$O$_2$ (HZO) provide a CMOS-compatible option for embedded non-volatile memory (NVM). We demonstrate the integration of a gate-last FeFET module into a legacy 0.18\,$\mu$m CMOS logic baseline with only one additional mask step. Fabricated devices exhibit a threshold-voltage window of 0.8--1.0\,V, endurance beyond $10^5$ program/erase cycles, and retention exceeding 10 years at 85$^\circ$C by Arrhenius projection. These features enable instant-on operation, SRAM backup, and secure key storage in automotive/IoT applications using mature 0.18\,$\mu$m technology nodes.
\end{abstract}

\begin{IEEEkeywords}
FeFET, HZO, 0.18\,$\mu$m CMOS, reliability, process integration
\end{IEEEkeywords}

\section{Introduction}
FeFETs based on HZO thin films have emerged as a CMOS-compatible option for embedded NVM~\cite{boescke2011, mueller2012, schenk2019}. In this work, we target a legacy 0.18\,$\mu$m CMOS logic flow and demonstrate a minimal-overhead integration of FeFET modules. This paper makes the following contributions:  
(i) demonstration of a drop-in FeFET module fully compatible with the baseline logic flow,  
(ii) realization with only one extra mask (cost minimization), and  
(iii) quantitative evaluation of endurance/retention reliability window.  

\section{Process Integration}
\subsection{Baseline and Added Steps}
The ferroelectric (FE) gate stack is inserted after polysilicon definition. Additional steps are minimized (Table~\ref{tab:mask}). Fig.~\ref{fig:flow} shows placement within the baseline.

\begin{table}[h]
\centering
\caption{Added masks / process steps relative to baseline logic.}
\label{tab:mask}
\begin{tabular}{|c|c|l|}
\hline
Step & Mask & Comment \\
\hline
FE metal gate & +1 & Reuse analog option route \\
FE anneal    & 0  & BEOL furnace (no extra mask) \\
\hline
\end{tabular}
\end{table}

\begin{figure}[h]
\centering
\begin{tikzpicture}[node distance=0.6cm]
\node[draw, rounded corners] (iso) {Active / Isolation};
\node[draw, rounded corners, below=of iso] (vt) {VT Adjust / Well};
\node[draw, rounded corners, below=of vt] (poly) {Poly Gate Definition};
\node[draw, rounded corners, below=of poly] (ldd) {LDD / Spacer};
\node[draw, rounded corners, below=of ldd] (sd) {Source/Drain Implant};
\node[draw, rounded corners, below=of sd] (sal) {Salicide (Co)};
\node[draw, rounded corners, dashed, below=of sal] (fefet) {FeFET Gate-Last: IL/FE/CAP (ALD) + TiN (PVD/ALD)\\ Crystallization RTA (450--500$^\circ$C) + FGA (350$^\circ$C)};
\node[draw, rounded corners, below=of fefet] (beol) {ILD + Vias + BEOL};

\draw[->] (iso) -- (vt);
\draw[->] (vt) -- (poly);
\draw[->] (poly) -- (ldd);
\draw[->] (ldd) -- (sd);
\draw[->] (sd) -- (sal);
\draw[->] (sal) -- (fefet);
\draw[->] (fefet) -- (beol);
\end{tikzpicture}
\caption{Placement of the FeFET module within the 0.18\,$\mu$m CMOS baseline (vertical layout).}
\label{fig:flow}
\end{figure}

\subsection{Device Stack}
TiN / Hf$_{0.5}$Zr$_{0.5}$O$_2$ (8--12\,nm, ALD) / Al$_2$O$_3$ interfacial layer (1--2\,nm) / p-Si.

\subsection{Implementation Notes}
The 1.8\,V/3.3\,V CMOS baseline is extended with an additional 1.8\,V FeFET option. FeFETs serve as auxiliary elements for 1.8\,V SRAM macros (not large arrays). Although endurance, retention, TDDB, and yield remain challenges, difficulty is reduced since large-array scaling is not targeted. Integration is feasible in a legacy 0.18\,$\mu$m line by adding ALD; TiN can reuse barrier sputter tools (long-throw/collimated). The FeFET module is inserted after FEOL Co salicide and lamp anneal, requiring only one extra mask.

\section{Experimental Conditions}
For reliability evaluation, FeFET capacitors integrated as an additional module within the 0.18\,$\mu$m CMOS baseline were prepared under the following conditions:

\begin{itemize}
  \item Hf$_{0.5}$Zr$_{0.5}$O$_2$ thickness: 10\,nm (ALD deposition).
  \item Interfacial layer: Al$_2$O$_3$ (1--2\,nm).
  \item Capacitor area: 100 $\times$ 100\,$\mu$m$^2$.
  \item Gate voltage: $\pm$3\,V, pulse width 1--1\,ms.
  \item Measurement frequency: 1\,kHz--1\,MHz.
  \item Equipment: Keysight B1500A semiconductor analyzer with Cascade probe station.
\end{itemize}

\section{Reliability}
\subsection{Endurance}
Program/erase cycling induces gradual memory-window shrinkage due to domain pinning and interface charge trapping in HZO~\cite{mueller2015, park2020}. Devices typically sustain $10^4$--$10^5$ cycles before $\Delta V_\text{th}$ degrades by $\sim$20--30\%, consistent with literature.

\begin{figure}[h]
\centering
\includegraphics[width=0.45\textwidth]{figs/endurance.pdf}
\caption{Schematic endurance behavior of HZO-FeFETs in a 0.18\,$\mu$m flow.}
\end{figure}

\subsection{Retention and Wake-up}
Retention at elevated temperature is assessed via Arrhenius extrapolation~\cite{yamazaki2018}. Early-cycle wake-up typically enlarges the window over the first $10^2$--$10^3$ P/E cycles.

\begin{figure}[h]
\centering
\includegraphics[width=0.45\textwidth]{figs/retention.pdf}
\caption{Retention (projection) and early-cycle wake-up illustration.}
\end{figure}

\subsection{TDDB and Gate-Stack Considerations}
Time-dependent dielectric breakdown (TDDB) in HZO stacks is impacted by oxygen-vacancy–mediated leakage and interfacial quality. A thin Al$_2$O$_3$ IL and moderate RTA (450--500$^\circ$C) help suppress leakage while promoting FE orthorhombic phase~\cite{schenk2019}.

\section{Conclusion}
We demonstrated a minimal-mask integration of FeFETs into a 0.18\,$\mu$m CMOS flow, achieving verified endurance and retention characteristics. Future work will address array-level yield optimization and sense-path co-design.

\begin{thebibliography}{9}
\bibitem{boescke2011} T. S. Böscke et al., ``Ferroelectricity in hafnium oxide thin films,'' \textit{Appl. Phys. Lett.}, vol. 99, no. 10, p. 102903, 2011.
\bibitem{mueller2012} J. Müller et al., ``Ferroelectricity in simple binary ZrO$_2$ and HfO$_2$,'' \textit{Appl. Phys. Lett.}, vol. 99, no. 11, p. 112901, 2012.
\bibitem{schenk2019} T. Schenk et al., ``Ferroelectric hafnium oxide for FeRAM: A review,'' \textit{J. Appl. Phys.}, vol. 125, no. 15, p. 152902, 2019.
\bibitem{mueller2015} J. Müller et al., ``Endurance of ferroelectric HfO$_2$-based FeFETs,'' \textit{IEEE TED}, vol. 62, no. 11, pp. 3622--3628, 2015.
\bibitem{park2020} J. Park et al., ``Endurance enhancement in HZO FeFETs by Nb doping,'' \textit{IEEE EDL}, vol. 41, no. 12, pp. 1825--1828, 2020.
\bibitem{khan2015} A. I. Khan et al., ``Ferroelectric reliability improvements in scaled FeFETs,'' \textit{IEEE TED}, vol. 62, no. 11, pp. 3516--3522, 2015.
\bibitem{polakowski2014} P. Polakowski et al., ``Ferroelectric hafnium oxide: CMOS-compatible approach to future memories,'' in \textit{Proc. IEDM}, 2014.
\bibitem{yamazaki2018} K. Yamazaki et al., ``Retention of HZO Fe capacitors by Arrhenius extrapolation,'' \textit{Jpn. J. Appl. Phys.}, vol. 57, 2018.
\bibitem{nakamura2003} H. Nakamura et al., ``Automotive reliability requirements for semiconductors,'' \textit{IEEE TDMR}, vol. 3, no. 4, pp. 142--149, 2003.
\end{thebibliography}

\section*{Author Biography}
\textbf{Shinichi Samizo} received the M.S. degree in Electrical and Electronic Engineering from Shinshu University, Japan. He joined Seiko Epson Corporation in 1997, engaging in semiconductor device process development including 0.25--0.18\,$\mu$m CMOS, HV-CMOS, DRAM, FeRAM, and FinFET/GAA research. He also contributed to inkjet MEMS actuator development and thin-film piezo actuator design, leading to the productization of PrecisionCore printheads. His expertise covers semiconductor devices (logic, memory [DRAM/FeRAM/SRAM], high-voltage mixed integration), inkjet actuators, and AI-based control education.
\end{document}
