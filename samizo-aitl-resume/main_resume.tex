%==========================================
% 三溝 真一 職務経歴書(講師応募対応版/全般用)
%==========================================

\PassOptionsToPackage{margin=18mm}{geometry}

\RequirePackage{iftex}
\ifluatex
  \typeout{>> LuaLaTeX detected. Build continues.}
\else
  \errmessage{This document must be built with LuaLaTeX. (lualatex main_resume.tex)}
\fi

\documentclass[lualatex,ja=standard,11pt]{bxjsarticle}

\usepackage{geometry}
\usepackage{luatexja}
\usepackage{luatexja-fontspec}
\setmainjfont{HaranoAji Mincho}
\setsansjfont{HaranoAji Gothic}
\setmainfont{TeX Gyre Termes}
\setsansfont{TeX Gyre Heros}
\renewcommand{\familydefault}{\sfdefault}

\usepackage{titlesec}
\usepackage{enumitem}
\setlist[itemize]{leftmargin=*, itemsep=1.5pt, topsep=2pt}
\titleformat{\section}{\Large\bfseries}{\thesection}{6pt}{}
\titleformat{\subsection}{\bfseries}{\thesubsection}{6pt}{}

\usepackage{tabularx}
\usepackage{array}
\usepackage{xcolor}
\usepackage{hyperref}
\hypersetup{colorlinks=true, linkcolor=black, urlcolor=blue}

\newcommand{\cvsection}[1]{\section*{#1}\vspace{-0.3em}\hrule\vspace{0.5em}}

\begin{document}

\begin{center}
{\Huge \textbf{職務経歴書}} \\[1.2em]
{\Large 三溝 真一(さみぞ しんいち)} \\[0.5em]
1972年4月23日生(53歳) \\[0.3em]
〒390-0221 長野県松本市里山辺1729 \\[0.3em]
Tel:090-4316-8054 Email:\href{mailto:sin3t72@gmail.com}{sin3t72@gmail.com} \\[0.3em]
GitHub:\href{https://samizo-aitl.github.io/}{https://samizo-aitl.github.io/}
\end{center}

\vspace{1em}
\hrule
\vspace{1em}

\cvsection{職務要約}
半導体プロセス・デバイス(CMOSロジック/DRAM/高耐圧統合)、インクジェットMEMS(薄膜PZTアクチュエータ)、プリントヘッド電装、BOM運用、品質保証の分野で25年以上の実績を有する。量産トラブルの根因解析から工程最適化・歩留まり改善を主導。近年はBOM電子化、BCP二拠点生産、コスト合理化(Auめっき厚最適化で年約10億円削減)を推進。  
信州大学 山沢清人研究室(後の学長)にて磁性薄膜デバイス研究に従事し、電気学会 MAG研究会(1997年)で発表。現在は独立系半導体研究者・教育者として活動、オープン教材「Edusemi」「EduController」「AITL-H」をGitHubで公開し、教育・顧問・講師業に注力。

\cvsection{学歴}
信州大学 工学部 電気電子工学専攻 修士(工学修士)\\
指導教員:山沢 清人 教授(後の信州大学学長)\\
研究テーマ:磁性薄膜マイクロリアクトルの特性解析と低抵抗導体膜形成\\
研究成果:電気学会マグネティックス研究会(1997年)にて発表

\cvsection{職務経歴}
\textbf{セイコーエプソン株式会社(1997–2024)}

\begin{itemize}
  \item \textbf{IC基盤技術部/Tプロジェクト/IC製品技術部(1997–2001)}:酒田8inchライン立上げ、0.35µm配線モジュール開発/0.35–0.25µm DRAM立ち上げ量産化、VSRAM品質向上。Pause Refresh不良のn$^+$/p$^-$リークを特定し、歩留まり65\%→80\%台後半。
  \item \textbf{IC技術開発部(2002–2006)}:0.25µm高耐圧統合、aTFT製品化。SRAM単ビット不良をTiSi$_2$相転移不完全・B拡散で解明、RTA最適化。
  \item \textbf{研究開発本部 NVプロジェクト(2007)}:FeRAM特性・信頼性解析。水素還元起因インプリントを特定し、N$_2$パージ保管運用を導入。
  \item \textbf{研究開発本部 Pプロジェクト(2008–2011)}:薄膜PZTアクチュエータ評価。クラック要因特定、酢酸プレウェットで恒久対策。
  \item \textbf{IJ要素技術部(2012–2020)}:PrecisionCore向けCOF新規導入、ACF+Sn–Ag二重接合、ドライバIC BCP(二拠点互換)、AuめっきCpk設計(0.425±0.125µm)。BOM電子化、HCS緊急対応、ISO教育。
  \item \textbf{IJS品質保証部(2021–2024)}:Fコスト低減、品質教育、計測器管理、社内講師。
\end{itemize}

\cvsection{スキル}
\begin{itemize}
  \item 半導体:0.35–0.18µm CMOS/DRAM/HV-CMOS/FeRAM/TiSi$_2$工程
  \item MEMS/インクジェット:薄膜PZT、静電アクチュエータ、COF/ACF接合
  \item 品質・信頼性:不良解析、SPC、スクリーニング設計、ISO内部監査
  \item 教育:教育訓練体系設計、講師実務、教材開発(LaTeX/Markdown/Mermaid)
  \item AI×制御:AITL(PID×FSM×LLM)自律制御設計、SystemDK統合研究
\end{itemize}

\cvsection{現在の活動(2024–)}
独立系半導体研究者・教育者として活動。  
ポータル「Samizo-AITL」(\href{https://samizo-aitl.github.io/}{https://samizo-aitl.github.io/})を通じ、  
半導体・制御・システム設計教育をテーマに教材を公開中:  
\begin{itemize}
  \item \textbf{Edusemi-v4x}:半導体物性/プロセス/設計教育教材
  \item \textbf{EduController}:PID/適応制御/状態遷移制御教材
  \item \textbf{AITL-H}:PID×FSM×LLMによる自律制御アーキテクチャPoC
\end{itemize}
これらは教育・顧問・大学講義に直結する知的教育資産として整備済み。

\cvsection{研究業績・学会発表}
\begin{itemize}
  \item 三溝 真一, 池田 慎治, 佐藤 敏郎, 山沢 清人:\\
  「積層マイクロリアクトルの磁性膜とコイル導体の抵抗率に関する一考察」,\\
  電気学会マグネティックス研究会資料, MAG-97-14–27, pp.41–46 (1997年1月31日).\\
  J-GLOBAL ID: \href{https://jglobal.jst.go.jp/detail?JGLOBAL_ID=200902109670399960}{200902109670399960}.\\
  \textit{※信州大学 山沢清人教授(当時・後の学長)指導のもと実施された研究であり、
  工学部電気電子工学専攻における磁性薄膜デバイス開発の成果。}
\end{itemize}

\cvsection{希望条件}
勤務形態:非常勤/常勤/リモートいずれも可\\
希望職種:大学講師・技術顧問・教育研究職・技術コンサルタント\\
勤務地:在宅または長野県・首都圏を中心に柔軟対応可能

\end{document}
