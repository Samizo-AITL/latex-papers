%==========================================
% 三溝 真一 職務経歴書(日本語版)
%==========================================
\documentclass[11pt]{bxjsarticle}

\usepackage{fontspec}
\usepackage{luatexja}
\usepackage{luatexja-fontspec}
\setmainjfont{HaranoAji Mincho}
\setsansjfont{HaranoAji Gothic}
\setmainfont{TeX Gyre Termes}
\setsansfont{TeX Gyre Heros}
\renewcommand{\familydefault}{\sfdefault}

\usepackage[margin=18mm]{geometry}
\usepackage{titlesec}
\usepackage{enumitem}
\setlist[itemize]{left=0pt .. 1em, itemsep=1.5pt, topsep=2pt}
\titleformat{\section}{\Large\bfseries}{\thesection}{6pt}{}
\titleformat{\subsection}{\bfseries}{\thesubsection}{6pt}{}

\usepackage{tabularx}
\usepackage{array}
\usepackage{xcolor}
\usepackage{siunitx}
\sisetup{detect-all=true}

\usepackage{hyperref}
\hypersetup{colorlinks=true, linkcolor=black, urlcolor=blue}

\newcommand{\cvsection}[1]{\section*{#1}\vspace{-0.3em}\hrule\vspace{0.5em}}
\newcommand{\um}{\SI{}{\micro\metre}} % 例: 0.25\,\um → \SI{0.25}{\micro\metre}

\begin{document}

\begin{center}
{\Huge \textbf{職務経歴書}} \\[1.2em]
{\Large 三溝 真一(みみぞ しんいち)} \\[0.5em]
1972年4月23日生(53歳) \\[0.3em]
〒390-0221 長野県松本市里山辺1729 \\[0.3em]
Tel:090-4316-8054 Email:\href{mailto:sin3t72@gmail.com}{sin3t72@gmail.com} \\[0.3em]
GitHub:\href{https://samizo-aitl.github.io/}{https://samizo-aitl.github.io/}
\end{center}

\vspace{1em}
\hrule
\vspace{1em}

\cvsection{職務要約}
半導体プロセス(DRAM/VSRAM, HV-CMOS, TiSi$_2$)、インクジェットMEMS(薄膜PZT)、電装BOM運用、品質保証の一連業務で27年以上の実績。量産トラブルの根因解析から工程最適化・歩留まり改善(例:DRAM 65\%→80\%台後半、VSRAM 30\%台→80\%超)を主導。近年はBOM電子化、BCP二拠点生産、コスト合理化(Auめっき厚最適化で年約10億円削減)を推進。2024年以降は独立し、教育教材(Edusemi, EduController, AITL-H)をGitHubで公開。

\cvsection{学歴}
信州大学 工学部 電気電子工学専攻 修士(工学修士)

\cvsection{職務経歴}
\textbf{セイコーエプソン株式会社(1997–2024)}

\begin{itemize}
  \item \textbf{IC基盤技術部(1997–2001)}:酒田8inchライン立上げ、\SIlist{0.35;0.25}{\micro\metre}配線/DRAM評価。Pause Refresh不良のn$^+$/p$^-$リーク(プラズマダメージ)特定、歩留まり65\%→80\%台後半。
  \item \textbf{IC技術開発部(2002–2006)}:\SI{0.25}{\micro\metre}高耐圧統合、aTFT製品化。SRAM単ビット不良をTiSi$_2$ C49→C54相転移不完全・B拡散で解明しRTA最適化。
  \item \textbf{NVプロジェクト(2007)}:FeRAM特性・信頼性解析。水素還元起因インプリント特定、N$_2$パージ保管運用導入。
  \item \textbf{Pプロジェクト(2008–2011)}:薄膜PZTアクチュエータ評価。クラック要因特定、酢酸プレウェットで恒久対策。
  \item \textbf{IJ要素技術(2012–2020)}:PrecisionCore向け新規COF、ACF+Sn–Ag二重接合導入、ドライバIC BCP(完全互換二拠点)、AuめっきCpk設計(\SI{0.425 \pm 0.125}{\micro\metre})。HCS緊急対応・BOM電子化・ISO教育。
  \item \textbf{IJS品質保証(2021–2024)}:Fコスト低減、品質教育、計測器管理。
\end{itemize}

\cvsection{主要実績}
\begin{itemize}
  \item DRAM/VSRAM歩留まり改善:65\%→80\%台後半、30\%台→80\%超
  \item Auめっき厚のCpk設計(\SI{0.425 \pm 0.125}{\micro\metre})で年約10億円コスト削減
  \item ACF+Sn–Ag二重接合で鉛フリー化と事業継続を確保
  \item ドライバIC BCP(プロセス移植型)で完全互換の二拠点量産
  \item HCSチップ実装と多拠点4M統合で目的外使用防止
\end{itemize}

\cvsection{スキル}
\begin{itemize}
  \item 半導体プロセス:DRAM/VSRAM, HV-CMOS, FeRAM, TiSi$_2$, RTA最適化, 高耐圧統合
  \item MEMS/インクジェット:薄膜PZT, 静電アクチュエータ, COF/ACF実装
  \item 品質・信頼性:不良解析, スクリーニング設計, 統計的工程管理(Cpk設計)
  \item 組織運用:BOM/図面/4M統合、BCP二拠点、ISO教育・内部監査
  \item 教育・公開:Markdown/LaTeX/Mermaid、GitHub Pages、教材開発
\end{itemize}

\cvsection{現在の活動(2024–)}
独立系半導体研究者として活動中。  
「Samizo-AITL」ポータルサイトを通じ、教育教材(Edusemi-v4x, EduController, AITL-H)およびPoCを公開。  
研究テーマは以下の3領域:
\begin{enumerate}
  \item バイオインクジェットMEMS技術(Pbフリー静電薄膜アクチュエータ)
  \item 先端半導体構造進化(FinFET~GAA~CFETのスケーリング臨界と信頼性設計)
  \item SystemDK with AITL Core(PID×FSM×LLMの自律制御アーキテクチャ)
\end{enumerate}

\cvsection{希望条件}
勤務形態:常勤/非常勤/リモートいずれも可  
職種:大学講師、技術顧問、研究・教育職、技術コンサルタント

\end{document}
