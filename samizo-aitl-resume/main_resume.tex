%==========================================
% 三溝 真一 職務経歴書(日本語版)
%==========================================

%==== ページ設定 ====
%\PassOptionsToPackage{margin=18mm}{geometry}
\PassOptionsToClass{engine=luatex}{bxjsarticle}

%==== LuaLaTeX 強制チェック ====
\RequirePackage{iftex}
\ifluatex
  \typeout{>> LuaLaTeX detected. Build continues.}
\else
  \errmessage{This document must be built with LuaLaTeX. (lualatex main_resume.tex)}
\fi

%==== クラス指定(engine=luatex は不要:古いTeXLive対策) ====
%\documentclass[ja=standard,11pt]{bxjsarticle}
\documentclass[engine=luatex,ja=standard,11pt]{bxjsarticle}

%==== 日本語/欧文フォント設定 ====
\usepackage{fontspec}
\usepackage{luatexja}
\usepackage{luatexja-fontspec}
\setmainjfont{HaranoAji Mincho}
\setsansjfont{HaranoAji Gothic}
\setmainfont{TeX Gyre Termes}
\setsansfont{TeX Gyre Heros}
\renewcommand{\familydefault}{\sfdefault}

%==== 書式関連 ====
\usepackage{titlesec}
\usepackage{enumitem}
\setlist[itemize]{leftmargin=*, itemsep=1.5pt, topsep=2pt}
\titleformat{\section}{\Large\bfseries}{\thesection}{6pt}{}
\titleformat{\subsection}{\bfseries}{\thesubsection}{6pt}{}

\usepackage{tabularx}
\usepackage{array}
\usepackage{xcolor}
\usepackage{siunitx}
\sisetup{detect-all=true}

%==== リンク設定 ====
\usepackage{hyperref}
\hypersetup{colorlinks=true, linkcolor=black, urlcolor=blue}

%==== カスタムセクション ====
\newcommand{\cvsection}[1]{\section*{#1}\vspace{-0.3em}\hrule\vspace{0.5em}}

%==========================================
% 本文
%==========================================
\begin{document}

\begin{center}
{\Huge \textbf{職務経歴書}} \\[1.2em]
{\Large 三溝 真一(さみぞ しんいち)} \\[0.5em]
1972年4月23日生(53歳) \\[0.3em]
〒390-0221 長野県松本市里山辺1729 \\[0.3em]
Tel:090-4316-8054 Email:\href{mailto:sin3t72@gmail.com}{sin3t72@gmail.com} \\[0.3em]
GitHub:\href{https://samizo-aitl.github.io/}{https://samizo-aitl.github.io/}
\end{center}

\vspace{1em}
\hrule
\vspace{1em}

\cvsection{職務要約}
半導体プロセス(DRAM/VSRAM, HV-CMOS, TiSi$_2$)、インクジェットMEMS(薄膜PZT)、電装BOM運用、品質保証の一連業務で27年以上の実績。量産トラブルの根因解析から工程最適化・歩留まり改善を主導。近年はBOM電子化、BCP二拠点生産、コスト合理化(Auめっき厚最適化で年約10億円削減)を推進。2024年以降は独立し、教育教材(Edusemi, EduController, AITL-H)をGitHubで公開。

\cvsection{学歴}
信州大学 工学部 電気電子工学専攻 修士(工学修士)

\cvsection{職務経歴}
\textbf{セイコーエプソン株式会社(1997–2024)}

\begin{itemize}
  \item \textbf{IC基盤技術部/Tプロジェクト/IC製品技術部(1997–2001)}:酒田8inchライン立上げ、0.35/0.25 µm配線/DRAM評価。Pause Refresh不良のn$^+$/p$^-$リーク(プラズマダメージ)特定、歩留まり65\%→80\%台後半。
  \item \textbf{IC技術開発部(2002–2006)}:0.25 µm高耐圧統合、aTFT製品化。SRAM単ビット不良をTiSi$_2$ C49→C54相転移不完全・B拡散で解明しRTA最適化。
  \item \textbf{NVプロジェクト(2007)}:FeRAM特性・信頼性解析。水素還元起因インプリント特定、N$_2$パージ保管運用導入。
  \item \textbf{Pプロジェクト(2008–2011)}:薄膜PZTアクチュエータ評価。クラック要因特定、酢酸プレウェットで恒久対策。
  \item \textbf{IJ要素技術(2012–2020)}:PrecisionCore向け新規COF、ACF+Sn–Ag二重接合導入、ドライバIC BCP(完全互換二拠点)、AuめっきCpk設計(0.425±0.125 µm)。HCS緊急対応・BOM電子化・ISO教育。
  \item \textbf{IJS品質保証(2021–2024)}:Fコスト低減、品質教育、計測器管理。
\end{itemize}

\cvsection{スキル}
\begin{itemize}
  \item 半導体プロセス:DRAM/VSRAM, HV-CMOS, FeRAM, TiSi$_2$, RTA最適化, 高耐圧統合
  \item MEMS/インクジェット:薄膜PZT, 静電アクチュエータ, COF/ACF実装
  \item 品質・信頼性:不良解析, 統計的工程管理, スクリーニング設計
  \item 組織運用:BOM/図面/4M統合、BCP二拠点、ISO教育・内部監査
  \item 教育・公開:Markdown/LaTeX/Mermaid, GitHub Pages, 教材開発
\end{itemize}

\cvsection{現在の活動(2024–)}
独立系半導体研究者として活動中。  
ポータル「Samizo-AITL」を通じ、教育教材(Edusemi-v4x, EduController, AITL-H)を公開。  
研究テーマは以下の3領域:
\begin{enumerate}
  \item バイオインクジェットMEMS技術(Pbフリー静電薄膜アクチュエータ)
  \item 先端半導体構造進化(FinFET~GAA~CFETのスケーリング臨界と信頼性設計)
  \item SystemDK with AITL Core(PID×FSM×LLMの自律制御アーキテクチャ)
\end{enumerate}

\cvsection{希望条件}
勤務形態:常勤/非常勤/リモートいずれも可  
職種:大学講師、技術顧問、研究・教育職、技術コンサルタント

\end{document}
