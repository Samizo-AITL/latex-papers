%==========================================
% 三溝 真一 職務経歴書(講師+顧問対応/PrecisionCore版)
%==========================================

\PassOptionsToPackage{margin=18mm}{geometry}

\RequirePackage{iftex}
\ifluatex
  \typeout{>> LuaLaTeX detected. Build continues.}
\else
  \errmessage{This document must be built with LuaLaTeX. (lualatex main_resume.tex)}
\fi

\documentclass[lualatex,ja=standard,11pt]{bxjsarticle}

\usepackage{geometry}
\usepackage{luatexja}
\usepackage{luatexja-fontspec}
\setmainjfont{HaranoAji Mincho}
\setsansjfont{HaranoAji Gothic}
\setmainfont{TeX Gyre Termes}
\setsansfont{TeX Gyre Heros}
\renewcommand{\familydefault}{\sfdefault}

\usepackage{titlesec}
\usepackage{enumitem}
\setlist[itemize]{leftmargin=*, itemsep=1.5pt, topsep=2pt}
\titleformat{\section}{\Large\bfseries}{\thesection}{6pt}{}
\titleformat{\subsection}{\bfseries}{\thesubsection}{6pt}{}

\usepackage{tabularx}
\usepackage{array}
\usepackage{xcolor}
\usepackage{hyperref}
\hypersetup{colorlinks=true, linkcolor=black, urlcolor=blue}

\newcommand{\cvsection}[1]{\section*{#1}\vspace{-0.3em}\hrule\vspace{0.5em}}

\begin{document}

\begin{center}
{\Huge \textbf{職務経歴書}} \\[1.2em]
{\Large 三溝 真一(さみぞ しんいち)} \\[0.5em]
1972年4月23日生(53歳) \\[0.3em]
〒390-0221 長野県松本市里山辺1729 \\[0.3em]
Tel:090-4316-8054 Email:\href{mailto:sin3t72@gmail.com}{sin3t72@gmail.com} \\[0.3em]
GitHub:\href{https://samizo-aitl.github.io/}{https://samizo-aitl.github.io/}
\end{center}

\vspace{1em}
\hrule
\vspace{1em}

\cvsection{職務要約}
半導体・MEMS・インクジェット分野で25年以上の開発・量産経験を有する。  
薄膜PZTアクチュエータの信頼性設計を確立し、Epson PrecisionCoreヘッドの基盤技術を形成。  
量産トラブルの根因解析から歩留まり改善、品質教育、BOM電子化、二拠点BCP生産を推進。  
信州大学 山沢清人研究室(後の学長)にて磁性薄膜マイクロリアクトル研究に従事、電気学会 MAG研究会(1997年)で発表。  
現在は独立系半導体研究者・教育者として活動し、オープン教材(Edusemi, EduController, AITL-H)を通じ教育・顧問業務を展開中。

\cvsection{学歴}
信州大学 工学部 電気電子工学専攻 修士(工学修士)\\
指導教員:山沢 清人 教授(後の信州大学学長)\\
研究テーマ:磁性薄膜マイクロリアクトルの特性解析と低抵抗導体膜形成\\
研究成果:電気学会マグネティックス研究会(1997年)にて発表

\cvsection{職務経歴}
\textbf{セイコーエプソン株式会社(1997–2024)}

\begin{itemize}
  \item \textbf{IC基盤技術部/Tプロジェクト/IC製品技術部(1997–2001)}:
  酒田8inchライン立上げ、0.35µm配線モジュール開発/0.35–0.25µm DRAM立上げ量産化、VSRAM品質向上。
  Pause Refresh不良のn$^+$/p$^-$リーク(プラズマダメージ)特定、歩留まり65\%→80\%台後半。

  \item \textbf{IC技術開発部(2002–2006)}:
  0.25µm高耐圧統合、aTFT製品化。SRAM単ビット不良をTiSi$_2$相転移不完全・B拡散で解明しRTA最適化。

  \item \textbf{研究開発本部 NVプロジェクト(2007)}:
  FeRAM特性・信頼性解析。水素還元起因インプリント特定、N$_2$パージ保管運用導入。

  \item \textbf{研究開発本部 Pプロジェクト(2008–2011)}:
  薄膜PZTアクチュエータ電気特性・信頼性評価を担当。
  酢酸プレウェット処理によるクラック恒久対策を確立し、高周波駆動時の絶縁破壊リスクを半減。
  本技術は後のPrecisionCoreヘッドへ継承され、Epsonインクジェットの中核技術として量産化された。

  \item \textbf{IJ要素技術部(2012–2020)}:
  PrecisionCore向けCOF新規導入、ACF+Sn–Ag二重接合、ドライバIC BCP(二拠点互換)、AuめっきCpk設計(0.425±0.125µm)。
  BOM電子化、HCS緊急対応、ISO教育。年約10億円のコスト削減を達成。

  \item \textbf{IJS品質保証部(2021–2024)}:
  Fコスト低減、品質教育、計測器管理、社内講師。
\end{itemize}

\cvsection{スキル}
\begin{itemize}
  \item 半導体:0.35–0.18µm CMOS/DRAM/HV-CMOS/FeRAM/TiSi$_2$工程最適化
  \item MEMS/インクジェット:薄膜PZT・静電アクチュエータ設計、COF/ACF接合、PrecisionCore技術継承
  \item 品質・信頼性:不良解析、SPC、スクリーニング設計、ISO内部監査
  \item 教育・公開:Markdown/LaTeX/Mermaid教材開発、GitHub Pages運用
  \item AI×制御:AITL(PID×FSM×LLM)自律制御設計、SystemDK統合研究
\end{itemize}

\cvsection{現在の活動(2024–)}
独立系半導体研究者・教育者として活動。  
ポータル「Samizo-AITL」(\href{https://samizo-aitl.github.io/}{https://samizo-aitl.github.io/})を通じ、  
半導体・制御・システム設計教育をテーマに教材を公開中:  
\begin{itemize}
  \item \textbf{Edusemi-v4x}:半導体物性/プロセス/設計教育教材
  \item \textbf{EduController}:PID/適応制御/状態遷移制御教材
  \item \textbf{AITL-H}:PID×FSM×LLMによる自律制御アーキテクチャPoC
\end{itemize}
研究・教材成果を通じ、大学・企業向け講義・顧問業務を展開中。

\cvsection{研究業績・学会発表}
\begin{itemize}
  \item 三溝 真一, 池田 慎治, 佐藤 敏郎, 山沢 清人:\\
  「積層マイクロリアクトルの磁性膜とコイル導体の抵抗率に関する一考察」,\\
  電気学会マグネティックス研究会資料, MAG-97-14–27, pp.41–46 (1997年1月31日).\\
  J-GLOBAL ID: \href{https://jglobal.jst.go.jp/detail?JGLOBAL_ID=200902109670399960}{200902109670399960}.\\
  \textit{※信州大学 山沢清人教授(当時・後の学長)指導のもと実施された研究であり、
  工学部電気電子工学専攻における磁性薄膜デバイス開発の成果。}
\end{itemize}

\cvsection{希望条件}
勤務形態:非常勤/常勤/リモートいずれも可\\
希望職種:大学講師・技術顧問・教育研究職・技術コンサルタント\\
勤務地:在宅または長野県・首都圏を中心に柔軟対応可能

\end{document}
