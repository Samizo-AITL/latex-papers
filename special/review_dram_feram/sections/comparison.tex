Table~\ref{tab:comparison} summarizes representative literature values for DRAM and FeRAM. DRAM offers ultra-high endurance (effectively \mbox{$\geq 10^{16}$} refresh cycles) and low energy per bit, but limited retention that mandates refresh. FeRAM provides non-volatility with retention $\geq 10^{5}$ s and endurance \mbox{$10^{12}$--$10^{13}$} cycles, at the cost of higher write energy \cite{noheda2023,martin2020}.

\begin{table}[!t]
\caption{Representative metrics from literature. Values are order-of-magnitude guides.}
\label{tab:comparison}
\centering
\begin{tabular}{@{}lccccc@{}}
\toprule
Tech. & Speed (ns) & Retention (s) & Endurance (cycles) & Energy/bit (fJ) & Cell \\
\midrule
DRAM  & $\sim 10$     & $\sim 6.4\times 10^{-2}$ & $\ge 10^{16}$  & $10$--$100$   & $6F^{2}$ \\
FeRAM & $\lesssim 10$ & $\ge 10^{5}$             & $10^{12}$--$10^{13}$ & $10^{2}$--$10^{3}$ & 1T (FeFET) \\
\bottomrule
\end{tabular}
\end{table}

Figure~\ref{fig:svr} and Fig.~\ref{fig:evs} are placeholders that illustrate where speed--retention and energy--speed plots can be inserted later.

\begin{figure}[!t]
\centering
\fbox{\rule{0pt}{1.8in}\rule{2.8in}{0pt}}
\caption{Speed vs. retention (placeholder).}
\label{fig:svr}
\end{figure}

\begin{figure}[!t]
\centering
\fbox{\rule{0pt}{1.8in}\rule{2.8in}{0pt}}
\caption{Write energy per bit vs. write speed (placeholder).}
\label{fig:evs}
\end{figure}
