Hybrid memory hierarchies aim to combine DRAM performance with FeRAM persistence. Placing FeRAM near the memory controller or as a near-memory store (for example, an NVDIMM-like role) can reduce refresh energy, enable instant-on features, and accelerate checkpointing and recovery \cite{noheda2023,kim2021_dram}.

\subsection*{Benefits}
\begin{itemize}
  \item Reduced refresh overhead: cold pages can be parked in FeRAM, cutting DRAM refresh and standby power.
  \item Fast persistence: OS and application state can be checkpointed to FeRAM with microsecond-scale latency.
  \item Data resilience: FeRAM can provide crash-consistent buffers and metadata persistence.
\end{itemize}

\subsection*{Constraints and trade-offs}
\begin{itemize}
  \item Endurance and variability: FeRAM endurance ($10^{12}$--$10^{13}$ cycles) is high but below DRAM refresh activity; variability and aging must be monitored.
  \item Write cost: per-bit write energy and sometimes latency exceed DRAM; policies should bias read-mostly or cold data to FeRAM.
  \item Integration cost: adding ferroelectric layers or FeFETs introduces process compatibility and reliability risks that require qualification.
\end{itemize}

\subsection*{System directions}
\begin{itemize}
  \item Tiering policies using intensity and retention awareness to migrate cold or persistent objects to FeRAM.
  \item Refresh co-optimization that shrinks DRAM refresh in regions shadowed by FeRAM.
  \item Controller and OS support for wear tracking, retention-aware placement, and error telemetry.
\end{itemize}
