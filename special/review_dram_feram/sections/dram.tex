DRAM technology progressed via high-k dielectrics, deep trench or stacked capacitors, and continual process innovations. A key challenge is to maintain sufficient capacitance while suppressing leakage and variability at extremely small dimensions \cite{choi2022}. Device and array proposals investigate capacitor aspect-ratio increases and layout optimizations, yet refresh power and timing margins remain system concerns \cite{kim2021_dram}.

To extend scaling, the community studies directions analogous to 3D NAND, such as stacking capacitor arrays or exploring 3D DRAM architectures. These ideas may relieve planar density pressure but add significant integration complexity and do not remove the need for refresh \cite{iedm2023_dram}. From a metric viewpoint, DRAM offers sub-ns class access, very low read energy per bit, and effectively unlimited endurance (refresh-limited rather than wear-out limited), but short data retention that mandates periodic refresh \cite{choi2022}.
