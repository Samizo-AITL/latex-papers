Memory hierarchies are central to modern computing systems. DRAM remains the dominant volatile memory owing to speed, density, and scalability \cite{choi2022,kim2021_dram}. However, DRAM scaling faces limits as cell capacitors shrink below the deep sub-20 nm regime, and refresh overheads persist \cite{kim2021_dram,iedm2023_dram}.

In parallel, ferroelectric memories based on doped HfO$_2$ (FeRAM and FeFET) have re-emerged as promising non-volatile options thanks to CMOS compatibility and fast polarization switching \cite{boscke2011,mueller2012,noheda2023}. This review contrasts DRAM and FeRAM from device trends to system implications, and outlines hybrid uses that combine DRAM performance with FeRAM persistence.

Historically, mainstream DRAM traces to the 1T1C concept of Dennard, while ferroelectric memory concepts were reported decades earlier \cite{scott1998}. Recent results in hafnia-based ferroelectrics renewed interest in embedded non-volatile memories \cite{boscke2011,mueller2012}.
