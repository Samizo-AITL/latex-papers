\section{Results and Discussion}
System-level simulation was performed with representative AI inference workloads.

\subsection{Standby Power}
Migrating cold data and checkpoints to the FeRAM-backed tier yields more than 30\% reduction in standby power.
This reduction arises from suppressing periodic DRAM refresh for inactive regions.

\subsection{Resume Latency}
FeRAM allows direct restore of checkpoints without full DRAM wake-up.
Resume latency is reduced to the $\mu$s range, enabling near-instant resume after power gating and improving energy efficiency for mobile edge AI.

\subsection{Endurance}
FeRAM endurance of $10^{12}$~writes/year fits within FeRAM capability for checkpoint traffic.

% --- Fig.2: Access time vs. retention ---
\begin{figure}[t]
\centering
\begin{tikzpicture}
\begin{axis}[
  width=\columnwidth,            % 二段組のカラム幅にフィット
  height=0.70\columnwidth,       % 見やすく縦も拡大
  xmode=log, ymode=log,
  xmin=1, xmax=100,
  ymin=1, ymax=1e4,
  log ticks with fixed,
  scaled ticks=false,
  xtick={1,10,100},
  ytick={1,10,100,1000,10000},
  xticklabels={$10^0$,$10^1$,$10^2$},
  yticklabels={$10^0$,$10^1$,$10^2$,$10^3$,$10^4$},
  grid=both, tick align=inside,
  legend style={at={(0.97,0.08)},anchor=south east,
                fill=white,draw=none},
  legend cell align=left
]

% HBM: 赤塗りつぶし四角(10^2 ns で FeRAM より上)
\addplot+[only marks, mark=square*, mark size=2.4pt, draw=red, fill=red]
  coordinates {
    (1,10) (5,100) (10,1000) (30,4000) (100,10000)
  };

% FeRAM: 青塗りつぶし丸(10^2 ns で HBM より下)
\addplot+[only marks, mark=*, mark size=2.4pt, draw=blue, fill=blue]
  coordinates {
    (1,10) (5,20) (10,100) (30,500) (100,8000)
  };

\legend{HBM DRAM (typ), FeRAM (typ)}
\end{axis}
\end{tikzpicture}
\vspace{-2mm}
\caption{Access time vs. retention. HBM: red filled squares; FeRAM: blue filled circles.}
\label{fig:retention}
\end{figure}
