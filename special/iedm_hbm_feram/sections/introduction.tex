\begin{abstract}
High-bandwidth memory (HBM) provides the throughput required by mobile edge AI accelerators, 
but suffers from high standby power due to periodic refresh and complete data volatility. 
Ferroelectric RAM (FeRAM), based on HfO$_2$, offers non-volatility, low-voltage operation, and fast rewriting, 
making it suitable for checkpointing and persistent data management in mobile edge AI. 
This work investigates the integration of HBM and FeRAM using chiplet co-packaging on a silicon interposer. 
Results indicate that FeRAM integration can substantially reduce standby power, enable instant resume after power gating, 
and improve system-level efficiency for mobile edge AI workloads.
\end{abstract}

\section{Introduction}
Mobile edge AI requires memory subsystems that simultaneously provide:
(1) multi-hundred GB/s bandwidth, 
(2) ultra-low standby power, 
(3) near-instant resume after power gating, and 
(4) sufficient endurance for frequent checkpoints.  

HBM DRAM is effective for (1), but its reliance on periodic refresh leads to high standby power and resume latency, 
which limit efficiency at scale \cite{ChoiIEDM2022,KimIEDM2021}.  
To mitigate these drawbacks, integration with non-volatile memory is considered.  

Among candidate technologies, MRAM suffers from relatively high write energy, 
and 3D NAND, though dense, has slow write latency.  
FeRAM, in contrast, provides low-voltage operation and fast rewriting capability, 
making it an attractive choice for integration with HBM in mobile edge AI systems \cite{MuellerIEDM2012,MartinVLSI2020,NohedaNature2023}.  

An initial approach of monolithic HBM+FeRAM integration was examined, 
but process incompatibilities prevent practical implementation:  
ferroelectric HfO$_2$ requires low-temperature annealing ($\sim$400~$^\circ$C), 
while DRAM capacitors demand high-temperature anneals ($>700~^\circ$C), 
destroying ferroelectric properties.  
As a result, this work focuses on chiplet-based integration, 
in which HBM and FeRAM dies are fabricated in their optimized processes and co-packaged on a silicon interposer.  

In addition, system-level co-design using \textbf{SystemDK} enables holistic optimization across architecture, interfaces, and memory control policies, 
realizing checkpoint migration, refresh suppression, and tiered memory management.  
