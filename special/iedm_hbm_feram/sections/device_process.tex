\section{Device and Process Integration}
HBM DRAM stacks are typically fabricated with high-temperature capacitor anneals ($>700~^\circ$C), 
whereas FeRAM/FeFET devices require lower-temperature processing ($\sim$400~^\circ$C) to stabilize the ferroelectric o-phase in HfO$_2$. 
This thermal budget mismatch currently hinders monolithic integration.

\subsection{Chiplet-based Integration (Practical Solution)}
The most practical near-term approach is chiplet-based integration:  
HBM stacks and FeRAM/FeFET dies are fabricated in their respective optimized flows and co-integrated on a silicon interposer using $\mu$-bump connections.  
This architecture enables:
\begin{itemize}
  \item High-bandwidth operation from HBM ($>$300~GB/s),
  \item Persistent storage of checkpoints, metadata, and cold data in FeRAM,
  \item Reduction of refresh-induced traffic in DRAM.
\end{itemize}

\subsection{Monolithic Integration (Research Challenge)}
A longer-term research direction is embedding FeFET arrays within the HBM logic base die.  
In principle, DRAM capacitor HfO$_2$ and FeFET gate-stack HfO$_2$ could coexist; however, their annealing requirements are incompatible.  
Potential enablers include selective/dual annealing, dopant modulation, or stress engineering.  
At present, monolithic HBM+FeFET integration remains an open challenge for device and process research.

% ===== Fig.1: System-level schematic with SystemDK (TikZ) =====
\begin{figure}[!t]
\centering
\begin{tikzpicture}[font=\footnotesize, >=Stealth, node distance=1.0cm]
  % Styles
  \tikzset{
    blk/.style={draw=black, rounded corners, fill=black!6, minimum width=3.4cm, minimum height=9mm, align=center},
    note/.style={draw=black, rounded corners, fill=black!2, align=left, inner sep=3pt, text width=7.8cm},
    arrow/.style={->, thick},
    sysdk/.style={draw=black, thick, rounded corners, fill=black!4, align=center, inner sep=4pt, minimum width=11.2cm}
  }

  % SystemDK box
  \node[sysdk] (sdk) { \textbf{SystemDK Co-Design Framework}\\
  \scriptsize (Architecture / Interfaces / Package / OS Policies) };

  % Main blocks
  \node[blk, below=1.1cm of sdk, xshift=-4.0cm] (cpu) {CPU / Accelerator\\(Power-gated)};
  \node[blk, right=2.5cm of cpu] (hbm) {HBM (DRAM)\\High Bandwidth};
  \node[blk, right=2.5cm of hbm] (feram) {FeRAM Chiplet\\Persistent Tier};

  % Controller
  \node[blk, below=1.3cm of hbm, minimum width=8.6cm] (ctrl) {Memory Controller \& Policy Engine};

  % Policies note
  \node[note, below=0.9cm of ctrl] (pol) {Policies: tiering (Hot/Warm/Cold), checkpoint $\rightarrow$ FeRAM, 
  refresh reduction for FeRAM-backed regions, wear/ECC/telemetry.};

  % Data arrows
  \draw[arrow] (cpu) -- node[above]{data/requests} (hbm);
  \draw[arrow] (hbm) -- node[above]{ckpt/metadata} (feram);
  \draw[arrow] (ctrl.north) -- ++(-3.4,0.0) |- (cpu.south);
  \draw[arrow] (ctrl.north) -- (hbm.south);
  \draw[arrow] (ctrl.north) -- ++(3.4,0.0) |- (feram.south);

  % SystemDK supervision arrows
  \draw[arrow] (sdk.south) |- (cpu.north);
  \draw[arrow] (sdk.south) -- (hbm.north);
  \draw[arrow] (sdk.east) .. controls +(1.0,-0.2) and +(0.8,0.6) .. (feram.north);
\end{tikzpicture}
\caption{Chiplet-level integration supervised by \textbf{SystemDK}: CPU/Accelerator with HBM and FeRAM chiplet co-integrated on an interposer.}
\label{fig:system_schematic_sdk}
\end{figure}
