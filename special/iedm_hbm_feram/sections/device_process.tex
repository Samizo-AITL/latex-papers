\section{Device and Process Integration}
HBM DRAM stacks are typically fabricated with high-temperature capacitor anneals ($>700~^\circ$C),
whereas FeRAM/FeFET devices require lower-temperature processing ($\sim$400~^\circ$C$) to stabilize the ferroelectric o-phase in HfO$_2$.
This thermal budget mismatch currently hinders monolithic integration.

\subsection{Chiplet-based Integration (Practical Solution)}
The most practical near-term approach is chiplet-based integration:
HBM stacks and FeRAM/FeFET dies are fabricated in their respective optimized flows and co-integrated on a silicon interposer using $\mu$-bump connections.
This architecture enables:
\begin{itemize}
  \item High-bandwidth operation from HBM ($>$300~GB/s),
  \item Persistent storage of checkpoints, metadata, and cold data in FeRAM,
  \item Reduction of refresh-induced traffic in DRAM.
\end{itemize}

\subsection{Monolithic Integration (Research Challenge)}
A longer-term research direction is embedding FeFET arrays within the HBM logic base die.
In principle, DRAM capacitor HfO$_2$ and FeFET gate-stack HfO$_2$ could coexist; however, their annealing requirements remain incompatible.
Potential enablers include selective or dual-step annealing, dopant modulation, or stress engineering.
At present, monolithic HBM+FeFET integration remains an open challenge for device and process research.

% ===== Fig.1: Minimal chiplet integration (clean layout, no overlaps) =====
\begin{figure}[t]
\centering
\begin{tikzpicture}[
  font=\footnotesize, >=latex,
  node distance=10mm and 12mm,
  box/.style={draw, rounded corners=2pt, fill=black!6,
              minimum width=28mm, minimum height=7mm, align=center},
  ctrl/.style={draw, rounded corners=2pt, fill=black!10,
               minimum width=70mm, minimum height=7mm, align=center}
]

% Top: controller
\node[ctrl] (mc) {Memory Controller \& Policy Engine};

% Bottom row: three boxes (even spacing, no overlaps)
\node[box, below left =of mc]  (cpu)   {CPU / Accelerator};
\node[box, below       =of mc] (hbm)   {HBM (DRAM)};
\node[box, below right =of mc] (feram) {FeRAM Chiplet};

% Data paths inside the tier (labels kept small and centered on the arrows)
\draw[->, thick] (cpu) -- node[midway, below, inner sep=1pt]{\scriptsize bandwidth} (hbm);
\draw[->, thick] (hbm) -- node[midway, below, inner sep=1pt]{\scriptsize persistent data} (feram);

% Supervision from controller (no labels to keep clean)
\draw[->, thick] (mc.south) -- (cpu.north);
\draw[->, thick] (mc.south) -- (hbm.north);
\draw[->, thick] (mc.south) -- (feram.north);

\end{tikzpicture}

\vspace{2pt}
\caption{Minimal chiplet integration. Bandwidth traffic flows CPU$\rightarrow$HBM; 
persistent data and checkpoints reside on the FeRAM chiplet. 
The controller/policy engine orchestrates tiering and checkpointing.}
\label{fig:chiplet_min}
\end{figure}
