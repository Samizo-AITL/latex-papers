\section{Device and Process Integration}
HBM DRAM stacks are typically fabricated with high-temperature capacitor anneals ($>700~^\circ$C),
whereas FeRAM/FeFET devices require lower-temperature processing ($\sim$400~^\circ$C$) to stabilize the ferroelectric o-phase in HfO$_2$.
This thermal budget mismatch currently hinders monolithic integration.

\subsection{Chiplet-based Integration (Practical Solution)}
The most practical near-term approach is chiplet-based integration:
HBM stacks and FeRAM/FeFET dies are fabricated in their respective optimized flows and co-integrated on a silicon interposer using $\mu$-bump connections.
This architecture enables:
\begin{itemize}
  \item High-bandwidth operation from HBM ($>$300~GB/s),
  \item Persistent storage of checkpoints, metadata, and cold data in FeRAM,
  \item Reduction of refresh-induced traffic in DRAM.
\end{itemize}

\subsection{Monolithic Integration (Research Challenge)}
A longer-term research direction is embedding FeFET arrays within the HBM logic base die.
In principle, DRAM capacitor HfO$_2$ and FeFET gate-stack HfO$_2$ could coexist; however, their annealing requirements remain incompatible.
Potential enablers include selective or dual-step annealing, dopant modulation, or stress engineering.
At present, monolithic HBM+FeFET integration remains an open challenge for device and process research.

% ===== Fig.1: Minimal chiplet view (TikZ) =====
\begin{figure}[!t]
\centering
\begin{tikzpicture}[font=\footnotesize, >=Stealth, node distance=0.9cm]
  % Styles
  \tikzset{
    blk/.style={draw=black, rounded corners, fill=black!6, minimum width=3.4cm, minimum height=9mm, align=center},
    arrow/.style={->, thick},
    frame/.style={draw=black, rounded corners, fill=black!2, minimum width=10.5cm, minimum height=11mm, align=center}
  }

  % --- Top: memory controller/policy block (だけ上に持ち上げる) ---
  % 以前は HBM の下側に置いていたが、上に 0.5cm シフトして配置
  \node[blk, yshift=+0.5cm] (ctrl) {Memory Controller \& Policy Engine};

  % --- Middle: three main blocks in a row ---
  \node[blk, below=1.1cm of ctrl, xshift=-3.6cm] (cpu)  {CPU / Accelerator};
  \node[blk, right=2.4cm of cpu]                  (hbm)  {HBM (DRAM)};
  \node[blk, right=2.4cm of hbm]                  (feram){FeRAM Chiplet\\Persistent Tier};

  % --- Outer frame with simple label ---
  \node[frame, below=0.6cm of ctrl, minimum height=1.0cm, yshift=+0.3cm] (frame) { };

  % --- Data/policy arrows ---
  \draw[arrow] (cpu) -- (hbm);
  \draw[arrow] (hbm) -- node[above, yshift=1pt]{ckpt/metadata} (feram);

  % Controller supervision (上から三者へ)
  \draw[arrow] (ctrl.south) |- (cpu.north);
  \draw[arrow] (ctrl.south) -- (hbm.north);
  \draw[arrow] (ctrl.south) |- (feram.north);

  % Caption text inside the frame (うすい注釈)
  \node[font=\scriptsize, align=center] at ($(frame.north)+(0,0.35)$) {Minimal chiplet integration view};
  \node[font=\scriptsize, align=center] at ($(frame.south)+(0,-0.25)$) {CPU connects to HBM for bandwidth; FeRAM holds persistent data; policies manage tiering/checkpoints.};
\end{tikzpicture}
\caption{Minimal chiplet integration schematic. Only the \emph{Memory Controller \& Policy Engine} is lifted above the data path; CPU/Accelerator, HBM, and FeRAM chiplet are on one row.}
\label{fig:system_schematic_sdk}
\end{figure}
