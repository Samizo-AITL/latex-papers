\section{Device and Process Integration}
HBM DRAM stacks are typically fabricated with high-temperature capacitor anneals ($>700~^\circ$C), 
whereas FeRAM/FeFET devices require lower-temperature processing ($\sim$400~^\circ$C) to stabilize the ferroelectric o-phase in HfO$_2$. 
This incompatibility between high- and low-temperature requirements currently hinders monolithic integration.

\subsection{Chiplet-based Integration (Practical Solution)}
The most practical near-term approach is chiplet-based integration:  
HBM stacks and FeRAM/FeFET dies are fabricated in their respective optimized flows and co-integrated on a silicon interposer using $\mu$-bump connections.  
This architecture enables:
\begin{itemize}
  \item High-bandwidth operation from HBM ($>$300~GB/s),
  \item Persistent storage of checkpoints, metadata, and cold data in FeRAM,
  \item Reduction of refresh-induced traffic in DRAM.
\end{itemize}

\subsection{Monolithic Integration (Research Challenge)}
A longer-term research direction is embedding FeFET arrays within the HBM logic base die.  
In principle, DRAM capacitor HfO$_2$ and FeFET gate-stack HfO$_2$ could coexist; however, their annealing requirements remain incompatible.  
Possible enablers include selective or dual-step annealing, dopant modulation, or stress engineering.  
At present, monolithic HBM+FeFET integration remains an open challenge for device and process research.

\begin{figure}[!t]
\centering
\begin{tikzpicture}[font=\footnotesize, >=Stealth, node distance=2.0cm]
  \tikzset{blk/.style={draw, rounded corners, minimum width=2.6cm, minimum height=8mm, align=center}}
  \node[blk] (cpu) {CPU / Accelerator};
  \node[blk, right=of cpu] (hbm) {HBM (DRAM)};
  \node[blk, right=of hbm] (feram) {FeRAM Chiplet};
  \draw[->] (cpu) -- (hbm) node[midway,above]{high BW};
  \draw[->] (hbm) -- (feram) node[midway,above]{ckpt/metadata};
\end{tikzpicture}
\caption{Minimal chiplet-level view: CPU connects to HBM for bandwidth, FeRAM holds persistent data.}
\label{fig:minimal_chiplet}
\end{figure}
