\section{Device and Process Integration}
HBM DRAM stacks are fabricated with high-temperature capacitor anneals ($>700~^\circ$C), 
while FeRAM/FeFET requires lower-temperature processing ($\sim$400~^\circ$C) to preserve ferroelectric o-phase in HfO$_2$.
This incompatibility makes monolithic integration difficult at present.

\subsection{Chiplet-based Integration (Practical Solution)}
The most feasible near-term path is chiplet integration:
HBM stacks and FeRAM/FeFET dies are fabricated in their respective optimized flows and co-integrated on a silicon interposer with $\mu$-bump connections.
This allows:
\begin{itemize}
  \item HBM to deliver bandwidth $>$300 GB/s,
  \item FeRAM to hold checkpoints, metadata, and cold data persistently,
  \item Refresh traffic reduction in DRAM.
\end{itemize}

\subsection{Monolithic Integration (Research Challenge)}
A future direction is embedding FeFET arrays inside the HBM logic base die.
Both DRAM capacitor HfO$_2$ and FeFET gate-stack HfO$_2$ coexist, but require conflicting anneals.
Possible research enablers include selective annealing, dopant modulation, or stress engineering.
Today, monolithic HBM+FeFET remains a research challenge.

% ===== Fig.1: System-level schematic with SystemDK (TikZ) =====
\begin{figure}[!t]
\centering
\begin{tikzpicture}[font=\footnotesize, >=Stealth, node distance=1.2cm]
  \tikzset{
    blk/.style={draw=black, rounded corners, fill=black!6, minimum width=3.2cm, minimum height=8mm, align=center},
    note/.style={draw=black, rounded corners, fill=black!3, align=left, inner sep=3pt},
    arrow/.style={->, thick},
    sysdk/.style={draw=black, rounded corners, fill=black!2, align=center, inner sep=4pt, minimum width=10.4cm}
  }
  % SystemDK box on top
  \node[sysdk] (sdk) { \textbf{SystemDK Top-down Co-Design}\\
  \scriptsize (Architecture / Interfaces / Package / OS Policies) };
  % main blocks
  \node[blk, below=1.0cm of sdk, xshift=-4.0cm] (cpu) {CPU / Accelerator\\(power-gated)};
  \node[blk, right=2.2cm of cpu] (hbm) {HBM (DRAM)\\High Bandwidth};
  \node[blk, right=2.2cm of hbm] (feram) {FeRAM Chiplet\\Persistent Tier};
  \node[blk, below=1.2cm of hbm, minimum width=7.6cm] (ctrl) {Memory Controller \& Policy Engine};
  % policies note
  \node[note, below=0.9cm of ctrl, text width=7.9cm] (pol) {Policies: tiering (Hot/Warm/Cold), checkpoint $\rightarrow$ FeRAM, refresh reduction for FeRAM-backed regions, wear/ECC/telemetry.};
  % data arrows
  \draw[arrow] (cpu) -- node[above]{data/requests} (hbm);
  \draw[arrow] (hbm) -- node[above]{ckpt/metadata} (feram);
  \draw[arrow] (ctrl.north) -- ++(-3.4,0.0) |- (cpu.south);
  \draw[arrow] (ctrl.north) -- (hbm.south);
  \draw[arrow] (ctrl.north) -- ++(3.4,0.0) |- (feram.south);
  % SystemDK supervision arrows
  \draw[arrow] (sdk.south) |- (cpu.north);
  \draw[arrow] (sdk.south) -- (hbm.north);
  % 「右側から回す」スタイルの矢印
  \draw[arrow] (sdk.east) .. controls +(1.0,-0.1) and +(0.8,0.6) .. (feram.north);
\end{tikzpicture}
\caption{Chiplet-level integration supervised by \textbf{SystemDK}: CPU/Accelerator with HBM and FeRAM chiplet on an interposer.}
\label{fig:system_schematic_sdk}
\end{figure}
